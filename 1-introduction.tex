\chapter{Introduction}

%The story  I want to tell explains the Casimir force via its historical origins which serve to introduce the big results, while sidestepping weird irrelevant bullshit.  I want to hit the important context for modern scientists and emphasize quantities realted to current experiments.  I also want to outline the currently available most powerful methods to set context for what else is possible, and what its limitations are.  

%I think I can do this with a partial historical introduction.

\section{Casimir Forces in general and physical interpretation}

The Casimir force is an attractive force that arises due to fluctuations in quantum fields.  

\begin{itemize}
\item Cite Casimir/Casimir-Polder and Lifshitz
\item Cite Books - Milonni and Bordag
\item Physical size of energies and length-scale.  Perfect electrical conductor, and atom-wall
\item Casimir vs Van Der Waals vs London Forces
\item Renormalization
\item Non-additivity of forces
\item Search for repulsive forces
\item Dynamical Casimir effect/Unruh Effect?
\end{itemize}

\section{Physical relevance and experimental relevance}

\section{Experiments}

\begin{itemize}
\item Lamoreaux - for 1997 measurements, and also recent thermal work by Sushkov.
\item Capasso
\item Mohideen  
\item Cornell - atoms near wall
\item Kimble atoms near toroidal resonators.  
\item Painter?
\item Atom-chips
\item Cronin
\end{itemize}

\section{Other Computational methods}

\subsection{Proximity Force Approximation}

\begin{itemize}
\item Find first use?
\item Note problem with non-additivity
\item 
\end{itemize}

\subsection{Green function methods}

\begin{itemize}
\item Cite Barton
\item Cite Philbin(?)
\end{itemize}

\subsection{Reflection Matrix}

\begin{itemize}
\item Cite Balian and Duplantier
\item Cite Lambrecht and French collaborators
\item Cite Milton
\end{itemize}

\subsection{Scattering Matrix Path Integral Methods}

\begin{itemize}
\item Physical Picture based on generalized Green theorem from SIE
\item Cite Emig,Jaffe,  and others for initial analytical techniques.  Relies on Green theorem.
\item Cite Johnson/Reid/ for numerical progress.  Note use of existent analytical methods and similarities to existent numerical FTDT techniques on earlier papers.  
\item Note success,applicability
\item Estimates on scaling.
\end{itemize}

\subsection{Worldlines}

\begin{itemize}
\item Cite Kirsten, Strassler on general worldline
\item Cite QED Worldline paper on numerics?
\item Cite Gies papers (all of them!) note work on thermal/geometry.
\item Cite Schaden applying
\item Shortcomings: 
\end{itemize}

\section{Thesis outline}


Our work is based on the worldline method developed by Gies~\textit{et.~al}\cite{Gies03}


%%% Local Variables: 
%%% mode: latex
%%% TeX-master: "thesis_master"
%%% End: 
