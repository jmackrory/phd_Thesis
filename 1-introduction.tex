\chapter{Introduction}
\label{ch:introduction}

The Casimir effect is one of the more surprising consequences of quantum electrodynamics (QED), the quantum theory
describing the interaction of matter and photons.
\citet{Casimir1948} showed that according to QED a pair of  electrically neutral, conducting bodies will be attracted to one another due to their mutual interaction
with the quantized electromagnetic (EM) field, even if the EM field is in its vacuum state.  
In brief, the total energy of in the field is the sum of the energies of each of the field modes, 
where each mode has vacuum energy proportional to its frequency.   
The presence of the conducting bodies restricts the electric field on the plates.
The allowed modes must have a half-integer number of wavelengths fit between the plates.  
Changing the plates separation changes the allowed modes, and thus the total energy. 
While the total vacuum energy is a badly divergent quantity, the difference in energy for two configurations
is well defined.  
The vacuum energy is compared between the cases when the plates are a finite distance apart, and when
they are removed to arbitrarily far apart.  
As the plates are brought closer together, this vacuum energy difference is reduced, leading to an attractive force
between the plates.  

While the Casimir effect is a generic consequence of quantum field theory subject to boundary conditions,
the electromagnetic Casimir effect is the most important example.  
This is because photons (the quanta of the EM field) are massless which makes their interaction long-ranged, 
and the coupling of electromagnetism to matter is much stronger than gravity, the only other long-ranged fundamental force.  
In the electromagnetic Casimir effect, one can think of the electrons in a body emitting and absorbing virtual photons that can in turn
interact with electrons in other bodies. %  The fluctuating electric dipoles in the bodies then become correlated with one
% another, which leads to an attractive force.
The interacting bodies can be pairs of atoms~\citep{CasimirPolder1948}, 
macroscopic bodies such as metallic planes and dielectric slabs~\citep{Lifshitz1956}, or any combination.

Alternatively, the Casimir effect can be attributed to the attraction between instantaneous dipoles forming in the bodies.  
This interpretation is intimately related to the van der Waals force, where 
molecules with fluctuating dipole moments are attracted to one another~\citep{vanderWaals}.
While the emphasis on fluctuating EM fields or fluctuating dipoles may differ between Casimir and van der Waals forces,
they ultimately describe the same phenomenon.%, albeit at slightly different distance scales.  

Despite the prediction of the Casimir force in 1948, precise measurements of the Casimir force were only
carried out in the late 1990s~\citep{Lamoreaux1997,Mohideen1998}.  
These first modern experiments measured the Casimir force between conducting spheres and plates, 
rather than between parallel conducting plates, since a sphere and plate are easier to align and control. 
Experiments have also been carried out to measure the forces between atoms and surfaces~\citep{Sukenik1993,Perreault2005,Harber2005}.

Beyond their importance as observable consequences of quantum field theory, Casimir effects are also 
important to a range of modern experiments and developing technologies.
They are important for microelectromechanical systems (MEMS), where the Casimir attraction between components 
leads to stiction, which causes the pieces to be permanently stuck together~\citep{Buks2001}.  
Casimir forces are also important for technologies using atoms near dielectric surfaces~\citep{Folman2000,Alton2011, Hung2013}.
In these experiments, the attractive Casimir potential sets a lower bound for how
close the atoms can be brought to the surface.  Consequently, the Casimir effect must be considered in the engineering 
of these devices.

The advent of these experiments in complicated arrangements of bodies, has spurred the development
of a number of theoretical and computational methods for computing Casimir effects~\citep{Dalvit2011,Bordag2009}. 
To model these experiments, it is necessary to be able to compute the Casimir effect between arbitrarily shaped bodies, with 
realistic material properties---which in general requires a numerical approach~\citep{Johnson2011}.
The most important of these modern methods are the so-called ``scattering method''~\citep{Lambrecht2006,Rahi2009,Reid2009}
 and ``worldline methods''~\citep{Gies2003}.
To date, the scattering method is the only general purpose numerical method available for computing
Casimir forces in arbitrary arrangements of bodies~\citep{Reid2009,Reid2011,Reid2013}.  
This method considers fluctuating currents confined to the surfaces of the interacting bodies,
where the currents at each patch of surface interact with one another via emitting and reabsorbing photons.
The Casimir energy is calculated by evaluating the determinant of the (large) scattering matrix for all of these patches~\citep{Reid2011}.  
The worldline method is another promising method for calculating Casimir energies~\citep{Gies2003},
which considers an ensemble of closed Brownian paths propagating through space.
These paths can be intuitively thought of as the space-time trajectory of a virtual particle.
As the paths propagate, they accumulate a weight based on whether the path intersects any of the 
bodies.  The Casimir energy is found by summing up the contributions from paths at all starting points
and sizes.   
% The presence of bodies is encoded in a potential which is
% accumulated along the path based on whether the Brownian paths intersect any of the bodies.
The worldline method has only been developed for scalar fields interacting with idealized surfaces.
In contrast to the scattering method, the worldline method is a Monte Carlo method, which makes it 
easy to parallelize.  

It is the goal of this thesis to extend the worldline method to computing electromagnetic Casimir effects.
This requires accounting for the vector nature of the EM field, and including realistic coupling to 
material properties, while attempting to retain as many of its appealing properties as possible.    
In Chapter~\ref{ch:EM_quantization}, we will discuss quantization of the EM field, and introduce
two versions of the worldline path integral: one in terms of the vector and scalar potentials, and another 
in terms of two scalar fields.  The scalar field description is specialized for planar geometries, 
but it does account for the magnetic and dielectric properties of the medium.
We will develop analytical and numerical methods in chapters~\ref{ch:analytical}, \ref{ch:numerical},
and \ref{ch:force}, 
and show agreement with known electromagnetic Casimir results.

The rest of this chapter will cover simple examples of Casimir effects, and expand on background material.
In Section~\ref{sec:casimir}, we will introduce some simple calculations for Casimir effects such as
the Casimir--Polder potential between an atom and a conducting wall, and the Lifshitz formula for the Casimir
effect between dielectric half-spaces (both of which we will use as checks on our later work).
In Section~\ref{sec:expt_review} we will briefly survey recent experiments on the Casimir effect.
In Section~\ref{sec:numerical_review} we will cover the most prominent numerical methods for computing the Casimir effect.
In particular we will introduce the Feynman path integral and the scalar worldline method at length.  
Finally, in Section~\ref{sec:thesis_outline} we will review the rest of the thesis, and present
some of the key equations.

% At the outset, we note that there have been a number of books published on the subject of Casimir forces 
% in recent years.  This reflects the importance of Casimir forces to both physicists and chemists, as 
% well as the advent of new experiments which have spurred the development of new theoretical methods.  
% Milonni's clear text on QED sets the Casimir force in context as one manifestation of the quantized theory of light~\citep{Milonni1994}.
% while Milton's text emphasizes using Green function methods for computing a number of consequences for 
% the Casimir effect for various fields, dimensions~\citep{Milton2001}.
% There have been more recent where the state-of-the-art in both theory and experiments is discussed by practitioners --- 
% notably in the volumes by \citet{Bordag2009}, 
% and the collection editted by \citet{Dalvit2011}.  
%%\todo{Cite Vogel and Welsch~\cite{VogelWelsch2006}.  Also cite \citet{Buhmann2012-vol1,Buhmann2012-vol2}}
% While this thesis emphasizes simple physical applications, the application of Casimir or 
% van der Waals forces is also important to problems in surface physics and chemistry, as discussed in the texts by 
% \citet{Parsegian2006} and \citet{Israelachvili2011}.


\section{Casimir Effect}
\label{sec:casimir}
Casimir and van der Waals forces are intimately related, and describe the same basic quantum mechanical force. 
Van der Waals forces were first discovered as deviations from ideal gas behavior, which can be attributed 
to the atoms possessing a finite size and inter-atomic forces~\citep{vanderWaals,Parsegian2006}.
\citet{London1930} gave these interatomic forces a theoretical underpinning in terms of fluctuating,
 induced dipoles using quantum mechanical perturbation theory.
  This work leads to an interatomic potential with a characteristic $d^{-6}$ scaling.
While this scaling is similar to the other possible dipole-dipole interactions, 
these dispersion forces are often the dominant contribution to inter-molecular forces~\citep{Israelachvili2011}.

London's perturbation theory assumes that the dipole interact with another instantaneously, which
ignores the finite speed of light.  \citet{CasimirPolder1948} extended this calculation to quantum electrodynamics,
where they accounted for the retardation due to the finite speed of light. 
The retardation causes the potential to decay more quickly at larger distances.  

\citet{Casimir1948} then showed that uncharged, conducting plates would be attracted by their mutual
interaction with the quantized EM field.
% This arises since the total energy in the quantized EM field is reduced due to the field's interaction
% with the conducting plates.
This calculation emphasizes the role played by the fields, and suggests
a global interaction due to the effective boundary conditions imposed by the plates.
This interpretation stands in contrast to the van der Waals picture which emphasizes the pair-wise interactions between induced dipoles.

This was followed by later work by Lifshitz and co-workers on the Casimir effect between dielectric bodies~\citep{Lifshitz1956,Dzyaloshinskii1961}.
This is particularly relevant since some of their later work showed that the Casimir effect can be derived 
from the pair-wise van der Waals interactions of all of the constituent parts~\citep{Dzyaloshinskii1961}.  

In this thesis, we will follow the quantum optics convention and use 
``Casimir effect'' as an umbrella term for all of these vacuum fluctuation forces.  
Near-field forces where the dipole interactions can be considered as instantaneous
will be referred to as ``van der Waals forces''.  The forces between atoms and microscopic bodies 
will be referred to as ``Casimir--Polder forces'',  in distinction to the Casimir 
forces between macroscopic bodies.    

% Given that background, we will typically refer to the Casimir effect as the umbrella term for all of these
% vacuum fluctuation forces.  Near-field forces where the interactions can be considered as instantaneous
% dipole interactions will be referred to as van der Waals forces.  In honor of Casimir and Polder's work
% we will refer to forces between atoms and macroscopic bodies as Casimir--Polder forces, in distinction to the Casimir 
% forces between macroscopic bodies.    

We will start our development from Casimir and Polder's work, since that was framed in the more modern language of quantum field theory, 
and naturally encompasses all of the limiting cases.  

\subsection{Casimir--Polder Forces}

\citet{CasimirPolder1948} computed the energy between pairs of atoms, and for atoms and conducting walls 
using non-relativistic quantum electrodynamics to account for the retardation due to the finite speed of light. 
They found that in the far-field 
(where the transition wavelengths $\lambda=2\pi c/\omega_A$ exceed the separation of the atoms $d$, $d\gg 2\pi c/\omega_A$)
the inter-atomic potential decays more rapidly as $d^{-7}$, instead of the typical $d^{-6}$ scaling for London forces.
 The change in power law can be 
attributed to the induced dipoles decorrelating over the time of flight of the virtual photon, 
and thus having a weaker effective interaction.
  
%\todo{Is this from Milonni?}
They also found an attractive force between an atom and perfectly conducting wall, with a $d^{-3}$ scaling
in the near field regime, that passes over to $d^{-4}$ scaling in the far field.
In this case the atom can be thought of as interacting with its negative image in the wall,   
which leads to an attractive potential for the atom,
\begin{equation}
  V\subCP(d) =-\frac{3\hbar c\alpha_0}{32\pi^2\epsilon_0 d^4},\label{eq:Casimir_Polder}
\end{equation}
where $\hbar$ is the reduced Planck's constant, $c$ is the speed of light, $\alpha_0$ is the atom's static polarizability,
and $\epsilon_0$ is the permittivity of free space.  

\subsubsection{Derivation of the Atom-Perfect Conductor Potential}
\label{sec:CP_calc}
The Casimir--Polder potential between an atom in its ground state and a surface can be derived via perturbation theory
in the coupling of the atom to the EM field.  We will consider the attraction between
an atom and a perfectly conducting wall.\footnote{
This derivation is adapted from Chapter~13 and Section~14.3 of \citet{SteckNotes} and \citet[Section~3.12]{Milonni1994}.}
The Hamiltonian for the whole atom-field system is
\begin{equation}
  H = H_{\text{atom}} + H_{\text{field}} + H_{\text{int}}
\end{equation}
where we have split the energy into energy for atom, the EM field, and a term describing their
interaction.  
The atomic Hamiltonian is 
\begin{gather}
  H_{\text{atom}} = \frac{\hat{\vect{p}}^2}{2m} + \sum_j\hbar\omega_{j}\sigma_j^\dag\sigma_j 
  \label{eq:Hatom}
\end{gather}
where the first term is the kinetic energy, and the second term gives the atom's internal 
electronic energy. The atom's quantized energy levels are given by $E_j=\hbar\omega_j$, 
and $\sigma_j=|g\rangle\langle e_j|$ is the lowering operator for the atom's internal state.  
The field Hamiltonian is
\begin{equation}
  H_{\text{field}} = \sum\subkz \hbar\omega_k\left(a^\dag\subkz a\subkz + \frac{1}{2}\right),
  \label{eq:Hfield}
\end{equation}
where $a\subkz$ is the annihilation operator for the EM field mode with wavenumber $k$, and polarization $\zeta$,
which has spatial mode function $\vect{f}\subkz(x).$
This field energy also includes the zero-point energy, $\sum_k\hbar\omega_k/2$, which will be important 
in the Casimir effect.
For the Casimir--Polder calculation, this zero-point-energy is a divergent constant which drops out when
considering the energy differences when the atom is moved close to the surface from arbitrarily far away.  
Finally, the interaction Hamiltonian couples the internal state of the atom to the quantized light field,
\begin{equation}
H_{\text{int}} = -\vect{d\cdot E} = \sum_j\sum\subkz
  \sqrt{\frac{\hbar\omega_k}{2\epsilon_0}}(\sigma_j+\sigma_j^\dag)
  \vect{d}_j\cdot[a\subkz \vect{f}^*_k(\hat{x})+a^\dag\subkz \vect{f}\subkz(\hat{x})].
  \label{eq:Hint}
\end{equation}
The interaction Hamiltonian is written in the dipole approximation, which assumes that the atom is much smaller than
the relevant wavelengths.  In this case, the dominant wavelengths are typically on the order of the separation between the atom
and the wall.  
In this calculation the mode functions $\vect{f}\subkz$ must satisfy the EM boundary conditions
on the surfaces of the bodies.  The resulting EM mode functions (and thus the potential)
are then sensitive to the arrangements of the bodies.  
%This is distinct from the case in most field theory calculations where the modes are described by plane waves.  

Note that some of the terms in the interaction Hamiltonian violate energy conservation, in the sense that 
$[H_{\text{atom}}+H_{\text{field}},H_{\text{int}}]\ne 0$, so $H_{\text{int}}$ causes transitions between the eigenstates of the 
non-interacting Hamiltonians $H_{\text{atom}}+H_{\text{field}}$. 
For example,
$\sigma_j^\dag a^\dag\subkz$ creates a photon with frequency $\omega_k$ and raises the atom from the ground state to an excited state
with energy $\hbar\omega_j$.
These terms are normally dropped in the rotating-wave-approximation, since they oscillate quickly
in time as $e^{i(\omega_j+\omega_k)t}$, and average down to zero on typical atomic timescales.  
However, these energy non-conserving terms lead to observable effects at higher order in perturbation theory.
The first order energy shift $\langle E_n|H_{\text{int}}|E_n\rangle=0$, since the mean value of the electric
field in vacuum is zero. 
The Casimir--Polder potential emerges when computing the energy shift for the atom from $H_{\text{int}}$ to second order.
For an atom in its ground state, the Casimir--Polder potential is given by
\begin{equation}
  V\subCP = -\sum_{n\ne 0,\vect{k},\zeta} \frac{
    \langle E_0, 0|  H_{\text{int}}|E_n, 1\subkz\rangle\langle E_n, 1\subkz|  H_{\text{int}}|E_0, 0\rangle}{\hbar(\omega_k+\omega_{n0})},
\end{equation}
where $|E_n,1\subkz\rangle$ denotes the state with the atom in energy level $n$ and one photon in mode $\vect{k}$,
$|0\rangle$ denotes the vacuum state of the EM field, and the transition frequency $\omega_{n0}=(E_n-E_0)/\hbar$.
The shift $V\subCP$ can be understood as two virtual transitions---one from the atomic 
ground state with no photons to an atomic excited state with one photon,
followed by a return transition.  The total energy shift is found by summing over all possible intermediate states.
This process is represented schematically via the Feynman diagram in Figure~\ref{fig:feynman_CP}, where 
an atom in the ground-state emits a virtual photon, and re-absorbs it.  This transition
is a ``virtual'' one, since these intermediate transitions violate energy conservation, and these 
intermediate states are not directly physically observable on a detector.
\begin{figure}
  \centering
\begin{fmffile}{atom-loop}
  \begin{fmfgraph*}(60,30)
    \fmfleft{i}
    \fmfright{o}
    \fmftop{t}
    \fmf{plain}{i,v1}
    \fmf{plain}{v2,o}
    \fmf{plain,label=$|e_j\rangle$}{v1,v2}
    \fmf{photon,left=0.5,tension=.4}{v1,vt,v2}
    \fmf{phantom,tension=1}{t,vt}
    \fmflabel{$|g\rangle$}{i}
    \fmflabel{$|g\rangle$}{o}
    \fmflabel{$|1_{\vect{k}}\rangle$}{vt}
  \end{fmfgraph*}
\end{fmffile}
\caption[Feynman diagram for Casimir--Polder energy]
{Feynman diagram representing an atom interacting with EM field via emitting and absorbing photons.  
  The wavy line represents the EM Green function in the presence of boundaries---as opposed to the usual plane 
  waves exploited in field theory computations.  The atom is excited into intermediate states.
}
\label{fig:feynman_CP}
\end{figure}

After substituting in $H_{\text{int}}$, the Casimir--Polder energy can be written as
\begin{align}
  V\subCP(\vect{r}) 
&= -\sum_{m\ne 0,\vect{k},\zeta} \frac{\hbar\omega_k}{6\epsilon_0}
    \frac{|\langle E_0|\vect{d}|E_n\rangle|^2 |\vect{f}_{\vect{k},\zeta,i}(\vect{r})|^2}
    {\hbar(\omega_k+\omega_{n0})},
\end{align}
where we substituted the form of $H_{\text{int}}$ and assumed a spherically symmetric atom, 
$|\langle E_0|d_i|E_n\rangle|^2=|\langle E_0|\vect{d}|E_n\rangle|^2/3$,
which corresponds to assuming that all components of dipole matrix elements are equal.

The mode functions for the electric field near a perfectly conducting plane 
can be substituted into the energy.
Following \citet[Section~3.12]{Milonni1994} % (Section~3.12 in Ref.~\citep{Milonni1994})
, we assume the atom is close to one wall 
of a perfectly conducting box, but far from all other walls.
The EM mode functions for a perfectly conducting box are given by 
\begin{align}
  \vect{f}\subkz(\vect{r}) = \sqrt{\frac{8}{V}}\bigg(&
  \hat{x}(\hat{\varepsilon}\subkz\cdot\hat{x})\cos(k_xx)\sin(k_y y)\sin(k_zz)\nonumber\\
  &+\hat{y}(\hat{\varepsilon}\subkz\cdot\hat{y})\sin(k_xx)\cos(k_y y)\sin(k_zz)\nonumber\\
  &+\hat{z}(\hat{\varepsilon}\subkz\cdot\hat{z})\sin(k_xx)\sin(k_y y)\cos(k_zz)\bigg),
\end{align}
where $\hat{\varepsilon}\subkz$ are the polarization unit vectors, and the wavenumbers are given by $k_i=n_i\pi/L$,
with $n_i$ integers~\citep[Section~ 8.4.1]{SteckNotes}.
% The Casimir--Polder energy for an atom and a perfectly conducting wall is then 
% \begin{align}
%  V\subCP(\vect{r})=-\frac{2\hbar}{\epsilon_0V}\alpha_0\sum_{\vect{k},\zeta} \big[ &
%   (\hat{\varepsilon}\subkz\cdot\hat{x})^2\cos^2(k_xx)\sin^2(k_y y)\sin^2(k_zz)\nonumber\\
%   &+(\hat{\varepsilon}\subkz\cdot\hat{y})^2\sin^2(k_xx)\cos^2(k_y y)\sin^2(k_zz)\nonumber\\
%   &+(\hat{\varepsilon}\subkz\cdot\hat{z})^2\sin^2(k_xx)\sin^2(k_y y)\cos^2(k_zz)\bigg].
% \end{align}
The square-modulus $|\vect{f}\subkz(\vect{r})|^2$ can be simplified under a couple limits.  
If the atom is close to the $z=0$ plane, but very far from the other walls, the squared-sinusoids in $x$ and $y$
can be replaced by their average value of $1/2$, since they will be quickly oscillating.   
In addition, the polarization vectors form a resolution of the transverse identity (since Gauss's law 
implies that the electric field is transverse, $\nabla\cdot\vect{E}=0$),
\begin{equation}
  \sum_{\zeta} \hat{\varepsilon}\subkz^{i}\hat{\varepsilon}\subkz^{j} = \delta_{ij}-\frac{k_ik_j}{k^2}.
\end{equation}
%The Casimir--Polder potential can now be factored into atomic and field pieces~\cite{McLachlan1963}.
The sum over dipole matrix elements can also be rewritten in terms of the atom's ground-state polarizability,  
%The first is the polarizability for the atom in its ground state is
\begin{equation}
  \alpha_{ij}(\omega) = \sum_n 
  \frac{2\omega_{n0}\langle E_0|d_i|E_n\rangle\langle E_n| d_j|E_0\rangle}{\hbar(\omega_{n0}^2-\omega^2)}.
\end{equation}
If we assume the atom's distance from the surface $d$ is larger than the atom's dominant emission wavelength, $\omega_{n0}$
then the dominant contribution to the sum will come from frequencies for which $\omega_k\sim \frac{c}{d}\ll \omega_{n0}$.
In that limit, the atomic polarizability can be replaced by the atom's static (zero frequency) polarizability 
\begin{equation}
  \alpha_0=\lim_{\omega\rightarrow 0}\alpha(\omega) = \sum_n
  \frac{2|\langle E_0|\vect{d}|E_n\rangle|^2}{3\hbar\omega_{n0}}.
\end{equation}
In this far-field limit, the Casimir--Polder energy can be approximated as 
\begin{equation}
  V\subCP(\vect{r})= -\frac{\hbar}{4\epsilon_0}\alpha_0\sum_{\vect{k},\zeta}\omega_k |\vect{f}\subkz(\vect{r})|^2.
\end{equation}
With these simplifications, and taking the limit of a large box to convert the sum over wavevectors into an
integral, the Casimir--Polder potential is given by 
\begin{align}
 V\subCP(\vect{r})=-\frac{\hbar\alpha_0}{8\pi^3\epsilon_0}\int_{k_z>0} d^3k\,\omega_k\bigg[ &
  \bigg(1-\frac{k_x^2}{k^2}\bigg)[1-\cos(2k_zz)]
  +\bigg(1-\frac{k_y^2}{k^2}\bigg)[1-\cos(2k_zz)]\nonumber\\
  &+\bigg(1-\frac{k_z^2}{k^2}\bigg)[1+\cos(2k_z z)]\bigg],
\end{align}
where the sinusoids have been rewritten using double-angle formulae. 
The $z$-independent parts lead to a constant, divergent contribution to the energy.  
In order to extract a finite energy shift it is essential to renormalize the energy by subtracting
off this constant energy.  
This corresponds to considering the energy change as the atom is brought
close to the surface from arbitrarily far away.  
Throughout this thesis, this simple energy subtraction is the only renormalization that will be required. 

The renormalized Casimir--Polder energy between an atom and a conducting plane can be evaluated in spherical
coordinates, although some care is required to regularize these oscillatory integrals---this can be done by introducing an exponential
convergence factor $e^{-a k}$ and taking the limit $a\rightarrow 0$ at the end of the computation:
\begin{align}
 V\subCP(\vect{r})- V\sup0&=\lim_{a\rightarrow 0}-\frac{\hbar\alpha_0}{8\pi^3\epsilon_0}\int d^3k\,\omega_k 
  (-1)\frac{2k_z^2}{k^2}\cos(2k_zz) e^{-ka}\\
&=\lim_{a\rightarrow 0}\frac{\hbar c\alpha_0}{4\pi^2\epsilon_0}\int_0^\infty dk \int_0^{\pi/2} d\theta\,k^3\sin\theta 
  \cos^2\theta\cos(2kz\cos\theta) e^{-ka}\\
  &=-\frac{3\hbar c\alpha_0}{32\pi^2\epsilon_0 z^4}.\label{eq:CP_conductor}
\end{align}
While we have carried out the calculation for the case of perfectly-conducting planar interfaces, similar computations
can be carried out for dielectric interfaces and more general shapes of macroscopic bodies.  
In the case of an atom near a planar, dielectric interface the Casimir--Polder potential is 
\begin{equation}
  V\subCP(\vect{r})-V\sup0=-\frac{3\hbar c\alpha_0}{32\pi^2\epsilon_0z^4}\eta(\chi),\label{eq:VCP_eta}
\end{equation}
where $\chi$ is the static susceptibility of the medium, and the efficiency factor $\eta$ approaches zero
as $\chi\rightarrow 0$ and unity as $\chi\rightarrow\infty$.  
Note that this type of calculation relies on having analytical expressions available for the mode functions,
which limits this approach to simple geometries.  

% The Casimir--Polder energy can be further factored in terms of atomic polarizabilities,
% and Green functions for the EM field~\cite{McLachlan1963}. 
% The atomic polarizability measures the linear response of an atom to the EM field,
% and is given by 
% where $d_i$ is the $i^\text{th}$ component of the dipole operator $\vect{d}=e\vect{r}$.
% %\todo{What does polarizability look like for a simple atom?}
% At this order of perturbation theory, the EM Green tensor is given by its classical counterpart,

% To linear order in the coupling, the energy can be written on the imaginary frequency $is$ as 
% \begin{equation}
%   V\subCP(\vect{x}) = -\frac{\hbar c}{2}\int_0^\infty ds \alpha_{jk}(is)G^{(S)}_{kj}(\vect{r},\vect{r},is),
% \end{equation}
% where $\alpha_{ij}$ is the polarizability tensor, $G^{(S)}$ is the scattering part of the EM Green
% tensor~\cite{McLachlan1963, McLachlan1963a}.
% The scattering part of the Green function essentially subtracts off the vacuum 
% EM Green function $G^{(S)}=G-G^{(0)}$ --- this is essentially renormalizing the energy
% by subtracting off contributions that are independent of the bodies.  
% This has the same from as a linear-response calculation~\cite{Altland2011}, finding the linear-response of the 
% atom to it's interaction with the EM field.

% While we have only given the linear energy shift, there are a number of higher loop contributions
% which can be resummed analytically due to the relative simplicity of the expansion~\cite{Rosa2011}.  
% %\todo{Cite Dzyaloshinksii or papers book for resummation?}
% The result of which is 
% \begin{equation}
%   V\subCP(\vect{r}) = \frac{\hbar c}{2\pi}\int_0^\infty ds \tr \log[I - \alpha(is) G^{(S)}(\vect{r},\vect{r})].
% \end{equation}
% This method of computing Casimir--Polder energies via classical Green functions has been expanded upon
% in Ch. 7 of Milonni~\cite{Milonni1994}, Vogel and Welsch~\cite{VogelWelsch2006} and Buhmann~\cite{Buhmann2012-vol1,Buhmann2012-vol2}.

% \begin{enumerate}
% \item Cite Schwinger~\cite{Schwinger1978, Milton1978}  Scalar green functions.  
% \item Cite Barton
% \end{enumerate}

\subsection{Forces between Bodies: Casimir Energy}

We now turn our attention to the Casimir effect where macroscopic bodies are attracted to 
one another via their interaction with the quantized EM field~\citep{Casimir1948}.  
The presence of the bodies restricts the allowed modes of the electric field, 
which is illustrated for perfectly conducting planes in Figure~\ref{fig:Casimir_sketch}.
Each quantized mode of the EM field contributes
to the energy, even in the ground state with zero photons.
The total ground state energy in the EM field is 
\begin{equation}
  E=\sum_\alpha\frac{\hbar\omega_{\alpha}}{2},
\end{equation}
where $\alpha$ indexes all possible modes.  
While the total energy is divergent, a finite answer can be found by considering 
the energy difference between two different configurations of bodies.  
% The subtraction to eliminate divergences is a simple form of field theoretic renormalization, 
% and will be essential in computations of Casimir quantities.  
In this case, the energy is renormalized by subtracting the energy when the bodies are moved arbitrarily far apart
from one another.  
For example, the renormalized energy between two perfectly conducting plates is
\begin{equation}
  E-E_0 = -\frac{\hbar c}{720\pi^2 d^3},\label{eq:Casimir_energy}
\end{equation}
where $d$ is the distance between the plates.  % Note that in this expression there is no
% mention of the material properties of the media --- this is an artifact of imposing Dirichlet
% boundary conditions on the surfaces.  

\begin{figure}
\center
\includegraphics[width=6cm]{fig/intro/twoplanes_wave}
\caption[Allowed modes between parallel plates]
{Sketch of allowed modes between perfectly conducting parallel plates. 
 Only waves with a half-integer number of wavelengths are allowed between the plates.
The blue modes are only allowed outside the plates, while green modes are allowed inside
and outside.  The modes have been vertically offset for clarity.  }
\label{fig:Casimir_sketch}
\end{figure}

The theory was extended by Lifshitz and coworkers to describe forces between dielectric half-spaces~\citep{Lifshitz1956,
Dzyaloshinskii1959,Dzyaloshinskii1961}.  This can recover the Casimir force between 
perfect conductors, and Casimir--Polder forces between atom's and dielectric surfaces.  
We will sketch the derivation of the Lifshitz formula (which will be used later),
since this can naturally also compute the Casimir energy.

%  As intimidating as this expression is, it can be heuristically derived with
% relative ease via the ``argument principle''~\cite{vanKampen1968}.  
% In essence, since the Casimir energy is the sum of the vacuum energy over all modes, the energy can
% be written as a complex contour integral of the energy against a function with poles at the allowed 
% energies.  For two plates the allowed modes satisfy the Fabry-Perot condition,
% \begin{equation}
%   \Delta = 1 - r_1r_2 e^{2ik d}.
% \end{equation}


\subsubsection{Derivation of Lifshitz Formula}
\label{sec:lifshitz}
The Lifshitz formula for the Casimir energy can be found with an argument due to \citet{vanKampen1968}.
In its full generality, the Lifshitz formula gives the total energy for two planar dielectric bodies with dielectric constants 
$\epsilon_1$ and $\epsilon_2$,
separated by a medium with dielectric constant $\epsilon_3$.  
\footnote{This derivation parallels those in \citet[Section~7.2]{Milonni1994}, and \citet[Ch.~12]{Bordag2009}.
A similar result emerges from the scattering approach as discussed by \citet{Lambrecht2011}.}
The energy for the EM field in its ground state is 
\begin{equation}
  E = \sum_{\zeta}\sum_{k_x,k_y,\omega} \frac{\hbar\omega_k}{2},
\end{equation}
where the sum runs over all of the allowed modes for the particular arrangement of bodies.  
In this case, the  non-zero contribution to the Casimir effect comes from surface plasmon modes, which propagate along the interfaces
and decay exponentially away from the bodies.  
The mode sum can be converted into an integral with respect to the transverse wavenumber $k_T:=\sqrt{k_x^2+k_y^2}$.
The sum over frequencies can be recast as a contour integral over complex frequency $\xi$, against a function $\Delta^{(\zeta)}$,
 whose poles occur at the allowed frequencies, with residue $\omega_k$  
\begin{equation}
  E = \frac{L^2}{(2\pi)^2}\sum_{\zeta}\int_0^\infty dk_T\,k_T\oint d\xi\, 
  \frac{\hbar \xi}{2} \frac{1}{(2\pi i)\Delta^{(\zeta)}(\xi)}\frac{d\Delta^{(\zeta)}(\xi)}{d\xi}.
\end{equation}
The function $[2\pi i\Delta(\xi)]^{-1}d\Delta(\xi)/d\xi$ is designed to have unit residue at the zeroes of $\Delta(\xi)$, 
The factor of $L^2$ is accounted for by considering the energy per unit area.  
The energy can be simplified by integrating by parts, leading to 
\begin{equation}
  \frac{E}{A} = \frac{\hbar}{16\pi^3 i}\sum_{\zeta}\int_0^\infty dk_Tk_T\oint d\xi \, \ln\Delta^{(\zeta)}(\xi).
  \label{eq:lifshitz_logDelta}
\end{equation}
The most important modes are the surface modes, since these modes are sensitive to the position of the other body, where
these modes exponentially decay between the bodies.
The allowed frequencies for these modes must satisfy 
\begin{equation}
  r^{(\zeta)}_{13}r^{(\zeta)}_{23} e^{-2k_z d}=1,
\end{equation}
where $r^{(\zeta)}_i$ are the reflection coefficients for surface $i$ and polarization $\zeta$, and\
the wavenumber is given by $k_z=\sqrt{k_T^2-\epsilon(\omega)\omega^2/c^2}$.  The reflection coefficients
are given by 
\begin{equation}
  r\supTE_{13} = \frac{k_{z,1}-k_{z,3}}{k_{z,1}+k_{z,3}} \qquad 
  r\supTM_{13} = \frac{\epsilon_3k_{z,1}-\epsilon_1k_{z,3}}{\epsilon_3k_{z,1}+\epsilon_1k_{z,3}}.
\end{equation}
The frequency condition is suggestive of the requirement that accumulated round-trip phase 
including reflections from both walls, is unity for
allowed modes [this interpretation is bolstered by considering the Casimir force between realistic mirrors~\citep{Genet2003}].
This suggests choosing $\Delta(\xi) = 1-r^{(\zeta)}_{13}r^{(\zeta)}_{23} e^{-2k_z d}$, where the wavenumber $k_z$
and the reflection coefficients are functions of $\xi$.  

The contour integral~(\ref{eq:lifshitz_logDelta}) can be split into two pieces. The integral over the right semi-circle 
is independent of $d$, and decays to zero when the semicircular contour is taken to infinity.
  This leaves the integral along the imaginary frequency axis, $\xi=is$ where $s$ is real.  
Casimir effects are most naturally discussed along the imaginary frequency axis.  
Due to the causal nature of the dielectric response functions, the dielectric function is a 
smooth, real function on the imaginary axis.  This also means that the $z$-wavenumber is also real,
with $k_z=\sqrt{k_T^2+\epsilon(is)s^2/c^2}$, 
and that oscillatory functions like plane wave factors are replaced with real, decaying exponentials.  
Both of these features are extremely attractive for numerical methods, and so numerical methods also
work with imaginary frequency~\citep{Johnson2011}.

The Casimir energy between two dielectric half-spaces of permittivities $\epsilon_1$ and $\epsilon_2$
separated by a gap of thickness filled with permittivity $\epsilon_3$ is given by 
\begin{align}
\frac{E}{L^2} =& -\frac{\hbar}{2\pi^2c^3}\int_0^\infty ds\, s^2 \epsilon_3
\int_1^\infty dp\,p\sum_{\zeta=\text{TE, TM}}\log\left(1 - r^{(\zeta)}_{13}r^{(\zeta)}_{23}e^{-2\sqrt{\epsilon_3}ps d/c}\right),
\label{eq:lifshitz}
\end{align}
where the EM reflection coefficients are given by 
\begin{align}
  r\supTE_{ij}  = \frac{\kappa_i-\kappa_j}{\kappa_i+\kappa_j} \quad
  r\supTM_{ij}  = \frac{\epsilon_j \kappa_i - \epsilon_i \kappa_j}{\epsilon_j \kappa_i + \epsilon_i \kappa_j},
\end{align}
and
\begin{equation}
  \kappa_i = \sqrt{p^2 + \epsilon_i/\epsilon_3-1},
\end{equation}
following \citet{Zhou1995}.
The variables have been adjusted to agree with the Lifshitz calculation, by defining 
\begin{align}
  k_T &= \frac{s}{c}\sqrt{\epsilon_3(p^2-1)}.
\end{align}
In general, this integral form is the simplest expression for the Casimir energy between two dielectrics planes. 
% As intimidating as this expression may be, it is a necessary touchstone for later comparison, and it allows 
% us to examine the different scaling regimes at different distances.  
The perfect-conductor Casimir energy result (\ref{eq:Casimir_energy}) can be found by taking the strong-coupling limit,
 $r^{(\zeta)}_i\rightarrow 1$, setting $\epsilon_3=1$,  and evaluating the integrals using 
\begin{equation}
  \int_0^\infty ds\,s^2 \int_1^{\infty} dp\,p \log\left(1 - e^{-2 s p d/c}\right) = -\frac{c^3 \pi^4}{360 d^3}.
\end{equation}
The Casimir--Polder results for interacting atoms can be recovered from the Lifshitz formula by taking the limit of dilute bodies,
$\epsilon \approx 1+\alpha_0n$, where $n\ll 1$ is the density, and $\alpha_0$ is the static polarizability.

The Lifshitz theory can be extended to account for dispersion and finite temperature.  
Some care is required in quantizing the EM field within dielectric media,
since according to the Kramers-Kr\"onig relations, the presence of dispersion implies dissipation.
However, it has been observed that one gets the correct answers by a direct substitution
$\epsilon(\vect{x})\rightarrow \epsilon(\omega,\vect{x})$.  
\citet{Barash1975} and \citet{Rosa2010} investigated this more carefully by
in terms of the total thermodynamic energy and the work done on the microscopic details of the medium.   

At non-zero temperature it is necessary to also include the effects of real, thermally excited photons.
At inverse temperature $\beta=(\kB T)^{-1}$, for a 
frequency $\omega$, the mean number of photons is $\bar{n}(\omega) = \coth(\beta\hbar \omega/2)$.  
Note that $\coth(i x)$ has simple poles at $x=\pm n \pi$ for $n$  integer. 
Exactly the same style of argument can be used to find the thermal Casimir energy between dielectrics, 
but now due to the presence of $\coth(is)$, the integral over imaginary frequency picks up the residues of the integrand at the 
Matsubara frequencies $s_n:= 2\pi n/(\beta\hbar)$.
The resulting free energy per unit area is 
\begin{align}
  \frac{\mathcal{F}}{L^2} =& \frac{\kB T}{2\pi}{\sum_{n=0}^\infty}' s_n^2 \epsilon_3
  \int_1^\infty dp\,p\sum_{\zeta=\text{TE, TM}}\log\left[1 - r^{(\zeta)}_{13}(is_n)r^{(\zeta)}_{23}(is_n)e^{-2\sqrt{\epsilon_3}ps_n d/c}\right],
  \label{eq:lifshitz_finite_temp}
\end{align}
where the primed sum weights the $n=0$ term by $1/2$, and all functions of frequency are evaluated at $s_n.$ 
At zero temperature, this result passes over to the previous one, 
by transforming the sum over frequencies into an integral.  This is the most general form of the Lifshitz
formula, and can recover all of the limiting behaviors in the near and far field, and perfect conducting
media, and rarefied media.

\subsubsection{Physical Interpretation}

Casimir's original calculation vividly shows the importance of vacuum fields, and is said to show the reality 
of the vacuum field~\citep{Jaffe2005}.  This is due to the emphasis given to the imposed boundary conditions,
which are emphasized over the matter that created the boundary conditions.  
It is the author's opinion that the Casimir effect is best thought of as a long-ranged interaction between dielectric bodies 
mediated via the EM field~\citep{Jaffe2005, Rahi2009}.  This picture is also
analogous to the intuitive photon exchange picture used to explain the Casimir--Polder potential.
% This view makes the clearest connection to the underlying fundamental physics, rather than emphasizing idealized
% boundary conditions.
  Figure~\ref{fig:electron-effective-interaction} shows a term contributing to the Casimir effect where 
electrons on different bodies interact with one another via the EM field.
  The solid lines should
be understood as the current operators for the electrons bound to a particular, separate media, while the 
wavy lines are the EM green functions describing the photon.  In fact, if summed over all such dipole ``bubbles,''
one can recover the full Casimir force results---as was done by \citet{Dzyaloshinskii1961} in re-summing a field 
theoretic expansion.  
In that case the closed electron loops should be understood as current-current correlation functions,
$\langle j_\mu j_\nu\rangle$, where under linear response theory, this correlation function is related to the conductivity
tensor $\sigma_{\mu\nu}$~\citep{Kubo1957,Altland2011}.  The conductivity tensor is in turn related to the dielectric tensor, $\epsilon_{ij}$
via Ohm's Law $j_i=\sigma_{ij}E_j,$ which makes the connection between the underlying fundamental physics,
and the material functions used in the Casimir effect.\footnote{
  This relationship was pointed out by \citet{Rahi2009}, as a justification for their starting point
  in quantizing the EM field in media.}


 \begin{figure}
 \centering
 \begin{fmffile}{wall-wall}
% \begin{fmfgraph}(50,30)
%  \fmftop{t0,t1,t2,t3}
%  \fmfbottom{b0,b1,b2,b3}
%  \fmf{fermion,tension=0.5}{t1,v1}
%  \fmf{fermion,tension=0.5}{t2,v3}
%  \fmf{fermion,tension=0.5}{b1,v2}
%  \fmf{fermion,tension=0.5}{b2,v4}
%  \fmffreeze
% \fmf{photon,tension=0}{v1,v3}
% \fmf{photon,tension=0}{v2,v4}
% \fmf{fermion,tension=0}{v1,v2}
% %\fmf{fermion,tension=0,left}{v2,v1}
% \fmf{fermion,tension=0}{v3,v4}
% %\fmf{fermion,tension=0,right}{v4,v3}
% \end{fmfgraph}
\begin{fmfgraph}(50,30)
 \fmftop{t0,t1,t2,t3}
 \fmfbottom{b0,b1,b2,b3}
 \fmf{phantom}{t1,v1}
 \fmf{phantom}{t2,v3}
 \fmf{phantom}{b1,v2}
 \fmf{phantom}{b2,v4}
 \fmffreeze
\fmf{photon}{v1,v3}
\fmf{photon}{v2,v4}
\fmf{fermion,tension=0}{v1,v2}
\fmf{fermion,tension=0,left}{v2,v1}
\fmf{fermion,tension=0}{v3,v4}
\fmf{fermion,tension=0,right}{v4,v3}
\end{fmfgraph}
\end{fmffile}
\caption[Casimir energy in terms of fundamental QED processes. ]
 {Casimir Energy in terms of fundamental QED processes.  The electrons are considered bound within their respective media,
 but still interact with electrons on other bodies by exchanging photons.  Any self-interactions are removed 
by renormalization via considering energy differences.  The effective interaction of the 
electron current with the field is described by the dielectric constant.}
\label{fig:electron-effective-interaction}
\end{figure}

\subsubsection{Different Distance Scaling Regimes}

The Casimir effect is important at distances around the resonant wavelengths of the atom or medium,
 which are typically on the order of a micron for optical transitions.  
The Casimir effect is typically computed in a long-wavelength or low energy limit where the constituents of the bodies can be treated 
as a continuum.
%This long-wavelength limit also corresponds to energies small compared to the binding energies of the bodies.
This approximation starts to break down when the distances between bodies approach an angstrom.
That is the separation of the constituent atoms of the bodies, the distance scale where exchange effects and 
other quantum physics becomes important.
At the other extreme, for distances beyond a hundred microns, the Casimir effect becomes too weak to detect.  

The distance scaling of the Casimir energy for bodies separated by a distance $d$, can be found by approximating 
the Lifshitz integral~(\ref{eq:lifshitz_finite_temp}) in certain limits.
%The differing scaling emerges by approximating the integrals in the Lifshitz formula~(\ref{eq:lifshitz_temperature}) in various limits.  
% The distance scaling depends on the nature of the bodies, and the how large the separation is in comparison to 
% relevant wavelengths in the problem.  
% For pairs of macroscopic slabs the energy scales as $d^{-2}$--$d^{-3}$, depending on the separation, 
% while the energy scales as $d^{-6}$--$d^{-7}$ for pairs of atoms.  
In particular, one must compare the separation of the bodies $d$ to the resonant wavelengths or frequencies 
 of the interacting media.  
This requires some knowledge of the peak frequencies $\omega_A$ of the atomic polarizabilities $\alpha(\omega)$ or the dielectric
 function $\epsilon(\omega)$.  At nonzero temperature, there is another distance scale given by the thermal wavelength,
 $\omega_T=\kB T/\hbar$.  One can estimate the most important frequencies by examining $\alpha(is),\epsilon(is),e^{-2pd\xi/c}$
and approximating the integral in various limits. 

 In the near field or van der Waals regime, the separation of the bodies is less than any of the 
 resonant wavelengths for the bodies $d\ll \omega_A/c$.  In that limit all of the frequencies 
 contribute, weighted by $\alpha(i\omega)$ and $\epsilon(i\omega)$.  The exponential factor $e^{-2pd\xi/c}$
 is also constant for all relevant frequencies.  In essence, the interaction is an instantaneous 
 dipole interaction between the bodies.  For example the atom-wall potential shows a $d^{-3}$ scaling. 

 In the retarded or Casimir--Polder regime, the atoms are much further than a resonant wavelength 
 $d\gg \omega_A/c$.  In that case the dominant contributions come at zero frequency, 
and the functions can be approximated with their static limit.
 This typically occurs for distances greater than a micron.
 In this far field regime, the potential typically decays more quickly.  For example, the  atom-wall
potentials shows a $d^{-4}$ scaling in this limit.

At even greater separations between the bodies is the thermal regime $d\sim \omega_T/c$, where the real photons excited by the 
 thermal field contribute significantly to polarizing the atom.  At room temperature, the thermal wavelength is
 $\lambda_T\sim 10\mu$m.  In this regime the potential falls off more slowly as $E\sim d^{-3}$,
 the same as the near-field van der Waals regime.  

\section{Overview of Casimir Experiments}
\label{sec:expt_review}

The Casimir effect has been measured in experiments, both for macroscopic bodies and atoms.
This section provides a brief overview of the broad categories of experiments where the Casimir effect is relevant,
and the challenges these experiments provide to theoretical and computational methods.    
The following is intended as a broad survey, since the full literature on the Casimir and Casimir--Polder 
effects is quite large.  

%Lamoreaux
\subsection{Experiments on Casimir Forces}

Despite its prediction in 1948, the Casimir effect proved quite difficult to directly measure.
Some early confirmations used the Casimir effect to explain the thickness of liquid helium film on 
the wall of its container~\citep{Sabisky1973,Dzyaloshinskii1961}.
In that case, helium satisfies the repulsive Casimir criterion and it is energetically favorable
to have a thin film of helium between the vacuum and the walls.  

The Casimir effect was only precisely measured in 1997 by \citet{Lamoreaux1997}.   
This experiment measured the Casimir force between a sphere above a metal plate
 via a torsion pendulum.  
This landmark experiment was closely followed by \citet{Mohideen1998},
who used an atomic force microscope in a closer distance regime to measure the force in a sphere-plate geometry.  
The Casimir force has also been directly measured in a nanoelectromechanical (NEMS) system 
by \citet{Chan2001}.  In this case, the Casimir force is detected by the modification it
makes to the frequency of a torsional oscillator suspended above a plate.  
The sphere-plate and oscillator geometries have the experimental advantage of removing the need to carefully
align the parallel metal plates. % Given the strength of the Casimir force it is hard to keep the plates exactly parallel,
% and separate, which is something that dogged early attempts to measure the Casimir force.
Despite the aforementioned difficulties, the Casimir force between parallel plates was measured precisely by \citet{Bressi2002}.  

The Casimir force is also important in applications of microelectromechanical systems (MEMS), 
as a source of stiction~\citep{Tas1996, Serry1998, Buks2001}.  This is particularly important
in free standing structures such as nano-oscillators.  
Given that the Casimir force is an attractive potential, if parts of the device get too close to the substrate
they will permanently stick to one another, leading to device failure.  

Precisely measuring such a small force requires careful calibration of the measurements 
and removing systematic effects.  Reviews of these and other difficulties are available~\citep{Lamoreaux2011, vanZwol2011, Bordag2009}.
Two of the primary experimental errors are due to 
patch potentials, and surface roughness.  
The patch potentials are randomly distributed, localized surface charge distributions on the surface of a conductor.  
Their Coulombic interaction leads to a $d^{-1}$ power law contribution to the total force, 
which must be subtracted off to extract the weaker Casimir force, which decays as $d^{-3}$ or $d^{-4}$~\citep{Sushkov2011}.  

% The patch potentials are localized surface 
% charge distributions, which due to the longer range Coulomb interaction must be subtracted off to extract the weaker
% Casimir force~\cite{Sushkov2011}.  
% For conducting bodies, such stray potentials can emerge from variations in the work function
% of the metal~\cite{Lamoreaux2011}.
% If the Fermi surface dips below the surface of the metal, a localized electric charge can build up.
% This is one of the dominant sources of the background force in Casimir experiments, 
% and must be subtracted off to extract the Casimir effect.
 While these electrostatic forces can be mitigated and in principle removed, the Casimir effect is a fundamental 
 effect and must be taken into account in engineering applications.  

The fact that the thin metallic films and surfaces used in these 
experiments are not perfectly smooth is referred to as surface roughness, and is one of 
the main theoretical sources of error in these experiments.  
In addition, the optical properties of the surface must also be carefully characterized, since the 
optical properties of a coating can vary significantly.  Another difficulty in predicting the size of the 
Casimir effect is that the optical properties must be interpolated from data for other experiments~\citep{vanZwol2011}.

\subsection{Experiments on Casimir--Polder Forces}

Van der Waals and Casimir--Polder forces were first observed experimentally in molecules, 
which prompted the further development of theory to explain the effects.  
Beyond those early experiments, Casimir--Polder forces have also been measured precisely in more modern experiments  
using isolated atoms in experiments using atomic beams, cavity QED, and Bose-Einstein condensates.  

The first modern attempts at directly measuring the Casimir--Polder force used atomic beams 
near surfaces.  \citet{Sukenik1993} made the first modern measurement of the Casimir--Polder force.
Their experiment passed a hot beam of atoms through an optical cavity and detected
the Casimir--Polder force by measuring the fraction of the atoms that passed through the cavity undeflected. 
More recent experiments by \citet{Perreault2005}, and \citet{Lonij2009} measured 
the Casimir--Polder force by passing an atomic beam through a grating and detecting the phase-shift via atom interferometry.  

The Casimir--Polder effect has also been observed in the context of a Bose--Einstein Condensate (BEC)
of ultra-cold atoms~\citep{Harber2005,Obrecht2007}.  % The BEC allows for precise distance control,
% and can be used in atom interferometry to detect small phase shifts.    
The atoms are confined to a harmonic trap, and can be brought near to a surface to probe the Casimir
force, where the Casimir--Polder force shifts the oscillation frequency of the harmonic trap.
  % These experiments were able to measure the Casimir--Polder force in the thermal regime,
% vary the distance of the atoms from the surface from $1\mu m$ up to 10 $\mu m$
% This technique was able to measure the Casimir force in the thermal regime, which
% is often difficult since the force is weak at those distance, but was accessible here due to the stability of the BEC.
% In this case it was ``easy'' to  to observe
% the cross over between the Casimir--Polder and thermal regimes.

The Casimir--Polder force is also important in developing atomic technologies.  
Atoms are an attractive platform for a number of reasons:
Each atom of the same species is identical; 
atoms have readily accessible, well-defined transitions that can be used to control their motion,
internal state and interactions; and atoms
have internal states that are long-lived, which would be important in storing information.

In recent years there has been a concerted push to develop technology that retains the appealing features 
of cold atoms in an architecture that can be scaled up to having large numbers of addressable 
atoms~\citep{Kimble2008}.  
% There is a large effort across many fields to harness quantum technologies to develop scalable 
% quantum devices for computation and simulation of quantum systems
The desire to get strong coupling between the atom and light fields, addressable qubits, and a scalable
architecture has pushed groups towards developing traps that hold atoms close to dielectric surfaces.  
In this limit, the Casimir--Polder force is the dominant force, which can only be partially mitigated
by using laser fields to generate repulsive potentials.
In designing these new devices it is essential to compute and account for the Casimir--Polder force
the atoms experience when brought close to the dielectric surface.  

One direction that has been pursued is the atom chip~\citep{Folman2000,Schneider2003,Salem2010},
where atoms are trapped within a few microns of the surface via a combination of lasers and magnetic fields from wires embedded in
the surface.  %The atoms are typically trapped within a micron of the surface.  
In most applications the Casimir effect imposes a lower bound on how close bodies can be brought 
to each other, which in turn limits the coupling strength, as well as how small devices can be made.
In the atom-chip example, bringing the atoms closer than a micron lead to most of the atoms escaping the 
trap~\citep{Lin2004}.

Another direction that has been pursued is strong coupling of atoms to light via cavity QED.  
Kimble's group is developing microscopic dielectric waveguides to allow trapping, addressing and strongly interacting with  
single atoms in a scalable manner~\citep{Alton2011, Hung2013, Goban2014}.  In more recent work,
the Casimir--Polder potential is explicitly accounted for as part of the trapping potential~\citep{Goban2014},
and must be precisely computed.

\subsection{Current Experimental Directions}

Beyond directly detecting the Casimir effect, experiments are also moving in some directions worth highlighting,
since they are quite challenging for the theory to handle.
There is a continued effort to find repulsive Casimir effects, via material properties or geometric efforts.
In addition, some experiments search for new forces on the micron scale, where any deviation from the predicted
Casimir effect may be a new force.  In that case it is essential to be able to precisely calculate
Casimir forces, and carefully remove all known backgrounds. 

% \subsubsection{Thermal Force and Material Model}

% There is a long-standing dispute between experimentalists on the best model to describe the 
% metals used in Casimir experiments.  In particular, should the metallic bodies be described by 
% the Drude or Plasma models?  These models provide the following dielectric constants for metals,
% \begin{align}
%   \epsilon_{\text{Drude}} &= 1+\frac{\omega_p^2}{\omega(\omega+i\gamma)}\\
%   \epsilon_{\text{plasma}} &= 1-\frac{\omega_p^2}{\omega^2},
% \end{align}
% where $\omega_p$ is the plasma frequency and $\gamma$ is the dissipation rate.
% The Drude model is consistent with Maxwell's equations, but some earlier experiments fit the plasma model 
% better.  This is also entangled in another debate over the role of the thermal Casimir effect, in particular what happens
% to the zero-frequency term from the TE polarization?  The Drude model has a simple pole at zero frequency,
% while the plasma model 
% These mattes are more a problem of reasoning than lacking theoretical or computational technology.  

% \subsubsection{Controversies over model}

% There are a couple as-yet unresolved issues in Casimir physics.
% Two of the leading experimental groups disagree on the appropriate model for the dielectric constant
% of a realistic metal.  
% Furthermore, there is also disagreement


%  Thermal casimir force
% Sushkov\cite{Sushkov2011}.
% Fight in literature over exact model used to describe metals at finite temperature.
% Drude vs plasma model.  
%  Lamoreaux favors Drude model, Capasso/Mohideen favours plasma model.

\subsubsection{Repulsive Casimir Effects}

Given that Casimir effects tend to enforce lower bounds for how close bodies can approach each other,
there has been a search for repulsive Casimir effects.  This would open the possibility
of trapping particles, and potentially allow smaller devices to be constructed.  
Unfortunately, these prospects are somewhat limited, due to requiring unusual material properties.
From the Lifshitz formula, the force is repulsive if $r_{12}r_{23}<0$.
This implies Casimir repulsion should be possible if $\epsilon_1<\epsilon_3<\epsilon_2$ over a broad range of frequencies.
This was experimentally demonstrated for a gold sphere immersed in bromobenzene above a silica plate
by \citet{Munday2009}.  However, this is method is little help for Casimir forces between
identical materials or cold atoms in vacuum.  

Alternatively, the Casimir force is also repulsive for combinations of dielectric and magnetic materials~\citep{Boyer1974}.  
Given the strength of electric interactions over magnetic interactions in atoms, this spurred interest
in exploiting materials with strong magnetic responses~\citep{Kenneth2002}.  
Since these are relatively rare, there was some interest in exploiting metamaterials (arrays of micropatterned circuits with
effective magnetic response at certain wavelengths~\citep{Pendry1999}).  However, this was shown to be ineffective
for Casimir applications since the underlying metallic dielectric response dominates for the most important long wavelengths.
Since the metallic response implies an attractive potential, the overall Casimir effect is attractive~
\citep{Ianuzzi2003comment,Rosa2008,Pirozhenko2008,Yannopapas2009}.  

While the preceding discussion emphasized varying materials for Casimir applications, it may 
be possible to exploit similar ideas for repulsive Casimir--Polder effects~\citep{Milton2011,Milton2012},
since the atom responds to a narrower range of frequencies.  In the far field, the attractive dielectric
response would dominate over any repulsive response, so it might be possible to engineer a trap.  
These proposals require an anisotropic response from the atom, which might be possible in the excited state.

Another method of generating repulsive Casimir effects is by varying the geometry of the bodies.  
For example, the Casimir effect is repulsive in certain regimes for an elongated needle above a hole in a conducting plate~
\citep{Levin2010,Rodriguez2013}.  However, in this example the repulsion is unstable. 
%In fact, it seems that it is impossible to get stable levitation for similar bodies via Casimir forces alone~\citep{Rahi2010}.
A stability criterion can be derived for the Casimir energy from within the scattering approach to Casimir theory~\citep{Rahi2010,Rahi2011}.
(The scattering approach which will be briefly discussed in Section~\ref{sec:scattering}.)  Only bodies
composed of media for which the planar Casimir force is repulsive can be stably levitated~\citep{Rahi2011}.

\citet{Boyer1968} found that the renormalized Casimir self-stress of a conducting shell 
is repulsive.  In this case, a great deal of care is required in isolating 
divergent terms to find a finite result.
This surprising result has been verified multiple times [the issue of self-stresses
 is reviewed in \citet{Milton2011}, and chapters 5 and 6 of \citet{Milton2001}].
There is also a similar repulsive Casimir effect for a dielectric sphere.  
While the stress on a spherical shell is repulsive, the force between two spherical half-shells is attractive.  
Thus it is not clear how this self-stress could be measured.  

\subsubsection{Searches for New Physics}

The Casimir force is also important for speculative searches for new physics on the millimeter to micron
scale~\citep{Dimopoulos2003, Bezerra2011}.  Since the new physics must be relatively short-ranged, 
it is typically modeled with a Yukawa potential, $V_{\text{Yuk}}=\alpha e^{-\lambda r}/r$,
which models the interaction with a new massive particle.    
On the micron scale however, the Casimir effect is the dominant interaction between neutral bodies,
 and must be carefully subtracted in any experimental procedure.
 Experiments then look for deviations from the expected Casimir effect, which means that the 
theory and experiment must be in good agreement with one another.  
This approach has already been used to exclude regions of the parameter space for the hypothetical
Yukawa interaction~\citep{Obrecht2007,Bezerra2011,Sushkov2011a,Chen2016}.  
Experiments searching for modifications of gravity typically employ a thin gold layer over
a density modulation.  The gold layer provides a common short-ranged Casimir interaction, while the 
a density modulation allows measuring variations due to gravity~\citep{Sorrentino2009, Geraci2015}.
Given the difficulties in cleanly measuring the Casimir force, this even more ambitious program has yet 
to yield results.  

% \subsection{Chemistry/Helium/Geckos?}

% \begin{enumerate}
% \item Geckos use the Casimir force \cite{Autumn2002}.
% \item Military applications to mimic at human scale. Cite 2015 paper.   
% \end{enumerate}

\section{Computational Methods for Casimir Effects}
\label{sec:numerical_review}

Modern experiments require theoretical and computational methods
for the Casimir force that can account for a wide variety of material responses, anisotropies
and the ability to handle arbitrary shapes.  
% Although the Casimir force was explored by theorists before precision experiments were available,
% the advent of precision experiments and new technologies has spurred developments in 
% theoretical and computational methods.
% While early work focused on highly symmetric geometries 
% of bodies such as parallel planar bodies~\cite{Casimir1948,Lifshitz1956} or spheres~\cite{Boyer1968}, 
% current experiments require calculations for arbitrary shapes and arrangements of bodies, 
% which have realistic material properties.  
% This is a difficult task since the Casimir effect is a broadband phenomenon, depending on the whole 
% range of frequencies.  Furthermore, it depends sensitively on the geometry of the bodies involved, 
% and one must carefully renormalize the results to avoid infinite answers.  
For a simple, symmetric geometry (like the perfectly conducting planes we used in Section~\ref{sec:CP_calc})
it is possible to write down tractable analytical expressions for
the Casimir energy based on expanding the field in mode functions.
However, for completely general geometries these requirements force one to adopt a  
numerical approach to computing Casimir forces~\citep{Johnson2011}.
We will discuss three of these methods: the proximity-force approximation (PFA), the scattering
or fluctuating surface current approach, and the worldline method.

\subsection{Proximity Force Approximation}
\label{sec:PFA}
The proximity force approximation (PFA) or Derjaguin approximation, is an uncontrolled approximation to
the Casimir force between arbitrarily shaped objects~\citep{Derjaguin1934, Blocki1977}.  
The PFA treats each infinitesimal patch of the surfaces as if they were planar bodies,
and sums up the pairwise interactions between different patches.
The PFA is assumed to be valid if the radius of curvature of the bodies $R$ is large relative to 
their separation $d$.  
For many years the PFA was the only practical general method of estimating Casimir forces in arbitrary geometries.
The PFA has the advantage of being straightforward to implement, and functions as an order of magnitude
estimate for the Casimir force for arbitrary geometries.

However, it has some prominent limitations.  First, it is only valid for vanishing curvature.
Second, the PFA assumes that the force can be found by integrating up
the pair-wise Casimir forces between each pair of surface patches.  This ignores the non-additivity
of the Casimir force.  Unlike the potential between electric charges where the total potential is
the sum of the pair wise potential energies, the Casimir force for an arrangement
of bodies is not just the sum of the pair wise energies. [This is discussed further in Section~{8.2} and Section~{8.4} by \citet{Milonni1994}.]
As a crude justification, the Casimir energy involves a sum over the frequencies for mode functions 
of the systems.  Since the mode functions are changed in a global, nonlinear fashion by introducing another
body, the sum over frequencies also changes in a nonlinear fashion as more bodies are added.  


\subsection{Scattering Approach}
\label{sec:scattering}
The scattering approach is currently the only general method of computing 
EM Casimir forces between media.\footnote{The following is an extremely short introduction to the scattering
method.  The book chapters by \citet{Lambrecht2011}, \citet{Rahi2011} and \citet{Johnson2011} 
provide a varied introduction to the topic from some of the main contributors.}
The scattering method is based on techniques from classical EM theory and quantum mechanics. %.~\citep{Rahi2009}.
%The roots of the scattering method in Casimir physics go back a number of decades.  
This method has been developed by a number of groups as an analytical method for general geometries~\citep{Emig2004, Lambrecht2006,
Kenneth2006, Emig2007, MaiaNeto2008,Canaguier-Durand2012,Rahi2009}.  
The Casimir energy can be written as 
\begin{equation}
  E = \frac{\hbar c}{2\pi}\int_{0}^\infty d\xi \log\det[\mathbb{M}\mathbb{M}^{-1}_{\infty}]
  \label{eq:scattering}
\end{equation}
where $\mathbb{M}$ is the scattering matrix describing scattering between the free modes of the EM
field induced by the presence of bodies, and $\xi$ is the imaginary frequency~\citep{Rahi2009}.
The energy is renormalized via $\mathbb{M}^{-1}_\infty$,
which is the scattering matrix as the bodies are moved arbitrarily far apart; this renormalization removes any
self-coupling of the bodies to themselves. 
The indices of these matrices run over the labels of the possible modes (such as wavelength, polarization, mode origin for different bodies).
%Since different bodies have different shapes, it is necessary t be able to transform basis.  
Derivations similar to the argument principle used in Section~\ref{sec:lifshitz} can be applied to describe the scattering 
between modes---instead of reflection coefficients for a surface, one considers the full scattering matrix for each body.
%where matrix indices run over the labels (such as wavenumber and polarization) for the modes.
This version of the scattering method has been applied to two-body systems such as realistic mirrors~\citep{Lambrecht2006}, 
and spheres and planes with investigations of surface roughness~\citep{Canaguier-Durand2012}.   
This subclass of these methods rely on scattering between mode functions suited to analytical expansions,
 and while they in principle offer a general purpose numerical method, 
the simulations may be slow to converge if the choice of basis functions is poorly suited to the actual
geometry required.  

The Johnson group at MIT has developed a formulation of the scattering method that is better suited to numerical 
applications for piecewise constant media~\citep{Rodriguez2007,Rodriguez2007a, Rodriguez2009,Reid2009,Reid2011, Reid2013}.  
% Instead of describing the scattering from one mode into another, one can describe the scattering 
% locally at each patch of a surface.  
% The scattering between objects is described by coupling different surface patches together via the
% free-space Green functions.  
% This has justification from the Surface Integral Equations used in classical electromagnetism 
% ~\cite{Stratton1941}, and a variant on Green's theorem~\cite{Emig2004},
% where one can describe the field in each region by its free-space analogue, and couple it along the surface.
% (At the level of perturbation theory where the Casimir effect is computed, the photon Green function 
% can be described by its classical counterpart.)
% This method has also been able to leverage the development of classical EM solvers to the Casimir problem~\cite{Johnson2011}.
In particular, \citet{Reid2009,Reid2011, Reid2013} developed the fluctuating-surface-current formulation as a general method for computing Casimir
energies for piecewise continuous linear dielectric and magnetic media.  
In essence the method calculates the interaction between electric and magnetic surface currents 
on different bodies, mediated by the EM field.  Mathematically this is derived 
from a path integral for the EM field, where the fields are restricted to obeying EM boundary conditions at the 
surfaces via functional delta functions [simpler boundary conditions were handled in this fashion by
\citet{Bordag1985} and \citet{Li1991}].  The delta functions introduce fields 
bound to the surfaces, which can be interpreted as surface currents flowing to enforce boundary conditions.
After integrating out the EM field in the interior and exterior regions, 
these surface currents interact with one another via the EM Green functions.
Since the method assumes piecewise, homogeneous media and enforces EM boundary
conditions, it is the relatively simple homogeneous EM Green function that appears in these expressions.
These surface integrals are then discretized by splitting the surface into a finite number of patches.
All of the surface currents can then be integrated over, leaving a functional determinant analogous to Eq.~(\ref{eq:scattering})
where now the matrix elements $\mathbb{M}$ describe the coupling between different surface-patches induced
by the EM Green functions.  

Numerically, this method comes down to computing the determinant of a large matrix, which is 
an intensive operation.  If a matrix has $N$ non-zero entries, the determinant for a dense matrix requires $\order(N^3)$ operations.
While it is possible to parallelize computing the determinant~\citep{Beliakov2013}, this is difficult.
However, for a sparse matrix system, it may be possible to make this relatively efficient 
and only require $\order(N\log N)$ operations~\citep{Reid2009}.
Since each frequency $\xi$ contributes independently, the integral over $\xi$ could be
trivially parallelized, but this may only offer relatively little parallelization for some problems.

The fluctuating-surface-current method has been used to 
describe the energy dependence of tetrahedral nanoparticles, capsules, and other  
geometries~\citep{Reid2009,Reid2011,Rodriguez2010}.  It has also been used to find cases 
where the Casimir force is repulsive due to geometric effects~\citep{Levin2010,Rodriguez2013}.  
The scattering method has also been used in the design of atomic traps near dielectric waveguides, where the Casimir--Polder
force is an essential component of the trap~\citep{Hung2013}.  

As we noted, the scattering method is the only available general method for computing Casimir effects.
However, it is useful to have multiple methods with different computational properties
and biases, particularly when extending calculations to unexplored domains.  We now turn to the worldline
method, which offers a very different picture and numerical method.  

\section{Path Integrals}
\label{sec:feynman_path_integral}
In order to discuss the modern methods of computing the Casimir effect it is necessary to introduce
the path integral.  The path integral was originally developed by \citet{Feynman1948} as an alternative 
formulation of quantum mechanics~\citep{Feynman1965}.
In the path integral, the probability amplitude for a particle to propagate from one position to another,
is given by the sum over \emph{all} possible paths between the points.
[In fact the path integral can be derived as the propagator from more traditional operator quantum mechanics~\citep{Sakurai1994}.]
Each path is weighted with a phase $e^{iS[x(t)]/\hbar}$ where $S[x(t)]$ is the classical action for the path.

    Path integrals have been used extensively in a wide range of theoretical physics~\citep{Kleinert2012}.
    While offering an intuitive picture of quantum mechanics, they are much harder to use 
    than typical operator mechanics for anything other than the simplest problems~\citep{Feynman1965}.
    However, path integrals form a natural basis for quantum field theories, where they offer a relativistically covariant
    quantization procedure that naturally accounts for the gauge symmetries 
    that underlie the Standard Model of particle physics~\citep{Brown1994,Srednicki2008}.
    % In these field theoretic path integrals, the integral runs over
    % all possible field configurations connecting the initial and final states, and some care is required
    % to handle the redundant degrees of freedom implied by gauge invariance~\cite{Faddeev1991}.

    Path integrals have also been used in mathematics and statistics to describe stochastic 
    processes~\citep{Kac1949,Durrett1996, Karatzas1991}.  Rather than solving the Schr\"odinger equation, 
    this path integral solves a diffusion equation---this effectively passes over to ``imaginary time,''
    since after the Wick rotation $t\rightarrow -i \tau$, the Schr\"odinger equation is a diffusion equation.
    This mathematical path integral weights each path by $e^{-S_{E}[x]}$, where $S_E$ is the real-valued, imaginary time action for the path.
    In this form the path integral has clearer convergence properties, 
    since the paths are weighted by real, decaying exponentials, as opposed to the oscillatory integrals
    in Feynman's path integral.  

    Path integrals underlie most of the work carried out in this thesis: we will use path integrals
    to quantize the EM field, and the worldline method relies heavily on path integrals.
    In addition, we will use the connection between path integrals and diffusion equations
    to verify analytically that the worldline path integral gives the correct results, and enhance our numerical
    calculations.  Considering their importance to this thesis, we will 
    now derive Feynman's path integral, which will serve as a prototype for all of the path integrals
    that follow.  [Our derivation follows the simple one given in \citet{Sakurai1994}.]

    \subsection{Derivation of Feynman's Path Integral}

    Let us consider the quantum mechanical treatment of a particle moving in a $D$-dimensional space time, 
    in a time-independent potential $V(\vect{x})$. 
    The particle is described by the following Hamiltonian:
    \begin{equation}
      \op{H} =  \frac{\op{\vect{p}}^2}{2m} + V(\op{\vect{x}}).
    \end{equation}
    The position and momentum operators obey the following commutation relations,
    \begin{gather}
      [\op{x}_i,\op{p}_j] = i\hbar\delta_{ij}\qquad      [\op{x}_i,\op{x}_j] = [\op{p}_i,\op{p}_j]=0,
      \label{eq:commutation}
    \end{gather}
    and have the following resolutions of the identity,
    \begin{gather}
      I = \int d\vect{x} |\vect{x}\rangle\langle\vect{x}| = \int \frac{d\vect{p}}{(2\pi\hbar)^D} |\vect{p}\rangle\langle\vect{p}|.
      \label{eq:identity}
    \end{gather}
    The overlap between position and momentum eigenstates is
    \begin{equation}
      \langle \vect{x}|\vect{p}\rangle = e^{i\vect{p}\cdot\vect{x}/\hbar}.
      \label{eq:overlap}
    \end{equation}
    In quantum mechanics, the amplitude for a particle starting at $\vect{x}_0$ at time $t_0=0$, and propagating
    to $\vect{x}_N$ at time $t$ is given by 
    \begin{equation}
      \langle \vect{x}_N,t| \vect{x}_0, t_0\rangle = \langle \vect{x}_N| e^{-i\op{H}t/\hbar}|\vect{x}_0\rangle.
    \end{equation}
    The amplitude to propagate from $\vect{x}_0$ to $\vect{x}_N$ can be developed into a path integral in a number of steps.
    First,  the evolution operator is split into $N$ pieces, and $(N-1)$ resolutions of the $\vect{x}$-identity 
    and $N$ resolutions of the $\vect{p}$-identity are inserted between  the pieces
    \begin{align}
      \langle \vect{x}_N,t| \vect{x}_0, t_0\rangle % &= \int \prod_{k=1}^{N-1} d\vect{x}_k 
      % \langle \vect{x}_f|e^{-i\op{H}\Delta t/\hbar}|\vect{x}_{N-1}\rangle
      % \langle \vect{x}_{N-1}|e^{-i\op{H}\Delta t/\hbar}|\vect{x}_{N-2}\rangle
      % \cdots \langle \vect{x}_1|e^{-i\op{H}\Delta t/\hbar}|\vect{x}_{0}\rangle\\
      &=\int \prod_{k=1}^{N-1} d\vect{x}_k \prod_{j=1}^{N-1} \frac{d\vect{p}_j }{(2\pi\hbar)^D}
      \langle \vect{x}_N|\vect{p}_{N}\rangle\langle\vect{p}_{N}|e^{-i\op{H}\Delta t/\hbar}|\vect{x}_{N-1}\rangle\nonumber\\
      & \hspace{0.5cm}\times\langle \vect{x}_{N-1}|\vect{p}_{N-1}\rangle\langle\vect{p}_{N-1}|e^{-i\op{H}\Delta t/\hbar}|\vect{x}_{N-2}\rangle
      \cdots \langle \vect{x}_1|\vect{p}_1\rangle\langle \vect{p}_1|e^{-i\op{H}\Delta t/\hbar}|\vect{x}_{0}\rangle
    \end{align}
    where $\Delta t:=t/N$. 
    At this point we can note the basic structure: The total amplitude for 
    the particle to propagate from $\vect{x}_0$ to $\vect{x}_N$ is the product of the amplitudes to propagate 
    from one point $\vect{x}_k$ to the next $\vect{x}_{k+1}$, with the total amplitude being the sum over all
    such paths.  
    Each infinitesimal time evolution operator can factored into a kinetic and potential piece, 
    \begin{equation}
      e^{-i\op{H}\Delta t/\hbar} = \exp\left(-i\frac{\op{\vect{p}}^2}{2m\hbar}\Delta t\right)
      \exp\left(-\frac{i}{\hbar}V(\op{\vect{x}})\Delta t\right)+\order(\Delta t^2),
    \end{equation}
    where the corrections due to splitting and factorizing the exponential operator contribute at $\order(\Delta t^2)$.
    [In general, it is crucial to consistently carry out all expansions in path integrals to $\order(\Delta t)$.]  
    % For example, in path integrals in curved space this often requires working to high order in $\Delta x$, 
    % and exploiting the equivalent of the Ito rule $dx^2=dt$~\cite{deWitt1957,Kleinert2012,Grosche1998}.
    The position and momentum operators can then be replaced by their eigenvalues, and the
    state-overlap can be used to write,
    \begin{align}
      \langle \vect{x}_N,t| \vect{x}_i, t_0\rangle 
      % &= \int \prod_{k=1}^{N-1} d\vect{x}_k \prod_{j=0}^{N-1} \frac{d\vect{p}_j }{(2\pi\hbar)^D}
      % \nonumber\\
      % \prod_{n=0}^{N-1}\bigg[\langle \vect{x}_{n+1}|\vect{p}_{n+1}\rangle
      % \langle\vect{p}_{n+1}|e^{-i\vect{p}^2_{n+1}\Delta t/(2m\hbar)}
      %   e^{-iV(\vect{x}_n)\Delta t/\hbar }|\vect{x}_{n}\rangle\bigg]\\
        &= \int \prod_{k=1}^{N-1} \frac{d\vect{x}_k d\vect{p}_k }{(2\pi\hbar)^D}% \nonumber\\
      % &\times
\bigg(\prod_{n=0}^{N-1}  e^{-i\vect{p}^2_{n+1}\Delta t/(2m\hbar)-iV(\vect{x}_n)\Delta t/\hbar 
      +i(\vect{x}_{n+1}-\vect{x}_n)\cdot\vect{p}_{n+1}/\hbar}\bigg).
    \end{align}
    Since the momentum integrals are Gaussian, they can be straightforwardly evaluated,
    with the result
    \begin{align}
      \langle \vect{x}_N,t| \vect{x}_i, t_0\rangle 
      % &= \int \prod_{k=1}^{N-1} d\vect{x}_k \prod_{k=0}^{N-1} \frac{d\vect{p}_k }{(2\pi\hbar)^D}
      %   \prod_{n=0}^{N-1}\bigg\{  \exp\left[-\frac{i\Delta t}{2m\hbar}\left(\vect{p}_{n+1}  
      %       -\frac{m}{\Delta t}(\vect{x}_{n+1}-\vect{x}_n)\right)^2\right]\nonumber\\
      %   &\hspace{6cm}        \times \exp\left[ \frac{i m }{2\hbar\Delta t}(\vect{x}_{n+1}-\vect{x}_n)^2-\frac{i\Delta t}{\hbar}V(\vect{x}_n)\right]\bigg\}\\
        &= \int \prod_{k=1}^{N-1} d\vect{x}_k 
        \prod_{n=0}^{N-1}\bigg[\bigg(\frac{ m }{2\pi i\hbar\Delta t}\bigg)^{D/2}
        e^{i m (\vect{x}_{n+1}-\vect{x}_n)^2/(2\Delta t)}e^{ -iV(\vect{x}_n)\Delta t/\hbar}\bigg]\\
        &= \int D\vect{x} 
        \exp\left[\frac{i}{\hbar}\int_{0}^{t} dt'\,\left( \frac{m}{2} \dot{\vect{x}}^2-V[\vect{x}(t')]\right)\right].
    \end{align}
    In the final line we have taken the continuum limit, replacing $(\vect{x}_{n+1}-\vect{x}_n)/\Delta t
    \rightarrow \dot{\vect{x}}, \sum_n\Delta t f(n\Delta t) \rightarrow \int dt f(t)$, and introducing 
    $D\vect{x} = \prod_{k=1}^{N-1}d\vect{x}_k\Big[m /(2\pi i\hbar\Delta t)\Big]^{D/2}$.  The phase in exponent is the classical action for a particle
    in a potential.  Paths with the same phase will add together constructively, while 
    paths in regions where the phase is quickly varying will cancel.  
    This leads to a natural description for the classical limit ($\hbar\rightarrow 0$) 
    where only the paths of stationary phase where $\delta S[x(t)]=0$ contribute.  
    Quantizing field theories via the path integral is seen as 
    a more relativistically covariant process than the canonical quantization procedure, which must 
    single out a particular time.  
    The symmetries of the field are also naturally taken into account due to the presence of the action.

    % One can think of the integral as summing over all possible paths $\vect{x}(t)$
    % between points $x_0$ and $x_f$ weighted by their classical action.  This also suggests a number 
    % of semi-classical expansions, such as considering the limit when $\hbar\rightarrow 0$ to derive 
    % the classical limit, or in work on chaotic systems~\cite{Gutzwiller1990}.

    In this thesis, this simple type of derivation will be all that is required.  We 
    will often work with the imaginary time version, which replaces the oscillating exponentials with
    decaying exponentials.  
    The extension to field path integrals over fields is straightforward: the field $\phi(\vect{x})$ 
    is described by its value at finitely many points $\phi(\vect{x}_k)$, where the field at each point
    varies independently.  The field path integral involves an integral over the field values at all of these 
    points.  At the end of the calculation, the spacing between grid points goes to zero, 
    and the size of the grid is taken to be arbitrarily large.  
    We will also only need to consider Gaussian path integrals, of the type considered here.  
    This derivation will extended to include sources in Chapter~\ref{ch:feynman_kac}.

\section{Scalar Worldline Casimir Energies}
\label{sec:dirichlet_worldline}
The worldline method is an alternative method for computing Casimir energies~\citep{Gies2003}.
The worldline method is a descendant of the scalar electrodynamics 
discussed by \citet{Feynman1950}, where the dynamics of a scalar field 
is described in terms of a particle propagating in an artificial proper-time through a fixed background potential. 
The worldline method was later developed as an alternative method for 
carrying out general quantum field theory calculations in terms of single particle 
quantum mechanics~\citep{McKeon1993, Strassler1992,Schubert2001}.  
% The worldline method is heavily based on Feynman's path integral method, where the amplitude
% for a particle to move from one position to another is the sum over all paths between the positions,
% where each path acquires a phase proportional to the classical action for that path~\cite{Feynman1948,Feynman1965}.
The basic insight of the worldline method is that for one-loop effective actions\footnote{
One loop order in quantum field theory corresponds to processes such as the Casimir effect,
where the field emits and absorbs a virtual particle, such as in Figure~\ref{fig:feynman_CP}.  
These are the first correction from quantum effects.  For example, the Lamb shift and Casimir--Polder 
effect involve one loop, since the electron emits and reabsorbed a photon.  Higher loop orders 
corresponds to more virtual processes.  
The effective action, gives the equations of motion in the absence of any external driving on the system~\citep[Ch. 16]{WeinbergQFT2}).  
The one-loop effective action then gives calculates action accounting for the lowest order contributions from quantum fluctuations
around the classical solution.}, 
the field path integral calculation can be recast in terms of the path
 integral for particles traveling in closed space-time paths.
  Higher order loop calculations can also be carried out with more particles, 
and gauge fields can also be treated~\citep{Schubert2001}.
For example, the worldline method has been used to compute relativistic
field effects for QED such as the Lamb shift~\citep{Schmidt1995}.  
It has also been used as a numerical algorithm for computing these relativistic 
QED effects~\citep{Mazur2014}---however, these methods were developed for free-space interactions at high energy, rather than the 
low energy Casimir phenomena we seek to describe.  
The worldline method is also closely related to the Heat Kernel~[which is reviewed in \citet{Vassilevich2003}].
The Heat Kernel examines the divergence structure of a field theory by examining the short time behavior of the 
worldline.  


The worldline method was first used to compute scalar Casimir energies by Gies\etal\citep{Gies2003,Gies2006, Gies2006a}.
The scalar worldline method has been extended to nonzero temperatures \citep{Klingmueller2008},
 used to study the torsion of inclined planes~\citep{Weber2009},
and forces in the sphere-plane and cylinder-plane geometries~\citep{Weber2010, Weber2010a}.  
In these non-trivial geometries the worldline method has also been used to examine the failure of the proximity force approximation.
More recent work has focused on computing the stress-energy tensor~\citep{Schafer2012, Schafer2016},
with a view to exploring how the Casimir energy violates certain energy conditions (violations of which are required for certain exotic physics).

The scalar worldline is also related to some semiclassical expansions for the Casimir energy.  
In particular, it is a direct numerical method for computing the so-called optical path integral
discussed by \citet{Scardicchio2005, Scardicchio2006}.  The sum over intersecting paths is also reminiscent 
of the semiclassical approach to the Casimir force by \citet{Schaden1998}, which evaluates 
the Casimir energy by summing over all periodic orbits of light.  This latter work is particularly 
related to other work on the semiclassical limits of path integrals involving chaos~\citep{Gutzwiller1990}.
Both of these approximate techniques rely on a path integral expression for the Casimir energy that models electromagnetism as a
scalar field. The worldline provides a general way of evaluating those path integrals.   

The worldline method has also been applied to the Casimir piston, where there are interesting geometric effects
based on the geometry of the piston~\citep{Schaden2009,Schaden2009a}.
Most of this work is for idealized surfaces that imposed Dirichlet boundary conditions, but 
there has also been some effort to extend the worldline method to account for Neumann boundary
conditions~\citep{Fosco2010}.  To date there has only been speculation on how to extend the 
worldline method to electromagnetism~\citep{Aehlig2011}, which only considered perfect conductors,
and did not have concrete, correct results.    

% \begin{enumerate}
%   \item Semi-classical approach to Casimir force via trace formula~\cite{Schaden1998}
%   \item Also the optical path integral.
% \end{enumerate}

% We consider a scalar field coupled to a background potential $V(\vect{x},t)$.  This potential
% embodies the location of the bodies we are considering.  % Starting from the classical action,
% we will derive the Hamiltonian for the fields, and then compute the quantum partition function.  
% The partition function can be written as a path integral, which is readily evaluated as a functional
% determinant.  Ultimately we want the free energy, which can be further converted into a path integral
% for a fictitious single-particle.  This single-particle path integral forms the basis of the numerical
% world line method.   

\subsection[Derivation of the Scalar Casimir Worldline Path\\ Integral]{Derivation of the Scalar Casimir Worldline Path Integral}
\label{sec:dirichlet_worldline_derivation}
We now introduce the basic scalar worldline method, to discuss its positive features and limitations. 
We will use terminology and scaling of dimensions in common with our later work, rather than the 
choices used in the original papers by \citet{Gies2003}.
[See also chapter 20 of Steck's Quantum Optics notes for an alternative perspective on this work, including
some of the analytical techniques will be used in later chapters~\citep{SteckNotes}.]

Consider a scalar field $\phi(\vect{r},t)$, interacting with a background potential $V(\vect{r})$.  
As a matter of convention we will distinguish between $\vect{r}$ as a position label or parameter
 and $\vect{x}$, the coordinate of a path integral.
The action for the field $\phi$ is given by the time integral of the Lagrangian density $\cL$, 
\begin{equation}
  S = \int_0^T dt \int d\vect{r}\, \cL = \int_0^T dt \int d\vect{r} 
  \left( \frac{1}{2c^2}(\partial_t\phi)^2-\frac{1}{2}|\nabla\phi|^2-V(\vect{r})\phi^2\right).
\end{equation}
The potential $V(\vect{r})$ defines the surfaces of the interacting objects
\begin{equation}
  V(\vect{r}) := \lambda \sum_r \delta[\sigma_r(\vect{r}-\vect{R}_r)],
\end{equation}
where $\lambda$ is the coupling constant, $\sigma_r(\vect{r})=0$ defines the surfaces, 
and $\vect{R}_r$ marks the center location of each body.
In most work on scalar worldlines, the coupling constant $\lambda$ is taken to infinity, 
which corresponds to imposing Dirichlet boundary conditions on the surfaces. 
For planar geometries, this recovers electromagnetic Casimir results for idealized perfect conductors.  

From the Lagrangian, one can find the Hamiltonian and quantize the theory.
% (While this is a somewhat length procedure, it is the clear formal procedure for quantization,
%  and useful to follow in cases where there may be ambiguities).
The momentum conjugate to $\phi$ is given by
\begin{equation}
  \Pi(\vect{r},t) := \frac{\partial \cL}{\partial(\partial_t\phi)} = \frac{1}{c^2}\partial_t\phi(\vect{r},t).
\end{equation}
%where $\frac{\delta}{\delta f(t)}$ denotes the functional derivative with respect to $f(t)$.    
The Hamiltonian is then given by
\begin{align}
  H &:= \int d\vect{r}\,(\Pi\partial_t\phi -  L)
= \int d\vect{r}\,\bigg(\frac{\Pi^2}{2} + \frac{1}{2}(\nabla\phi)^2 +V(\vect{r})\phi^2\bigg).  
\end{align}
The theory can now be quantized by promoting the classical fields to quantum operators, 
$\phi\rightarrow \op{\phi},\, \Pi\rightarrow\op{\Pi}$, with equal-time commutation relations
\begin{equation}
  [\op{\phi}(\vect{r},t),\op{\Pi}(\vect{r'},t)] = i\hbar \delta(\vect{r}-\vect{r'}).
\end{equation}
In exactly analogous fashion to quantum mechanics, the overlap between states is given by 
\begin{equation}
  \langle \phi|\Pi\rangle = \exp\bigg(\frac{i}{\hbar}\int d\vect{r}\, \phi(\vect{r})\Pi(\vect{r})\bigg).
\end{equation}
Physical quantities of interest such as Casimir energies and forces can be computed
by taking suitable derivatives of the field partition function. 
The quantum partition function for the field is 
\begin{equation}
  Z = \tr\left( e^{-\beta\op{H}}\right) = \int d\phi \langle \phi| e^{-\beta \op{H}}|\phi\rangle,\label{eq:Zphi}
\end{equation}
and the trace is evaluated over the complete set of field states.  
It is actually more useful to carry out calculations with the free energy $\mathcal{F}=-\kB T \log Z$.
As in Section~\ref{sec:feynman_path_integral}, the exponential operator can be split into $N$ pieces, and resolutions of the identity
in both fields and conjugate-momentum fields can be inserted between each piece.  
% \begin{align}
%   Z &= \int d\phi_0\prod_{n=1}^N d\phi_n \langle \phi_n| e^{-\Delta \beta \op{H}}|\Pi_n\rangle
%   \langle\Pi_n| \phi_{n-1}\rangle
% \end{align}
After integrating out the momentum fields, the partition function can be written as a
Euclidean path integral
\begin{equation}
  Z = \int D\phi \exp\left[-\int_0^{\hbar\beta c} d\tau \int d\vect{r}
    \left( \frac{1}{2}(\partial_\tau\phi)^2+\frac{1}{2}(\nabla\phi)^2+V(\vect{r})\phi^2\right)\right],
\end{equation}
where $\tau=\beta\hbar c$.  The partition function can be cast into a more suggestive form
by integrating by parts in the exponential integrand, 
\begin{equation}
  Z = \int D\phi \exp\left[-\int_0^{\hbar\beta c} d\tau \int d\vect{r}\,\phi(\vect{r},\tau)
    \left(-\frac{1}{2}\partial_\tau^2-\frac{1}{2}\nabla^2+V(\vect{r})\right)\phi(\vect{r},\tau)\right].
\end{equation}
The surface terms from integrating by parts were discarded by assuming the fields tend to zero at spatial (and temporal)
infinity.% \comment{Actually need frequencies for periodic functions.  Especially for finite temperature,
% dispersion.  Will discuss more carefully later.}

The functional integral over $\phi$ is Gaussian and can be formally evaluated as a 
functional determinant, since the differential operator is positive operator.  
Some care is required in regularizing such infinite determinants.
This is done in analogy with finite dimensional Gaussian integrals.  
The fields can be considered as only being evaluated on a finite lattice of space-time points,
with the lattice also having a finite extent which bounds all bodies.  
The field at each point is treated as an independent variable from the others.   
The gradient operators can be treated via their finite difference approximations, 
which can be thought of as sparse matrices.
For example, $\partial_x^2\phi(x_k) \approx [\phi(x_k+\Delta)-2\phi(x_k)+\phi(x_k-\Delta)]/\Delta^2$.
In that case the partition function is a large, finite Gaussian integral of the form, 
\begin{equation}
  Z_{\text{reg}} = \int d\phi_k\exp\left(-\sum_{j,k}\Delta \tau (\Delta x)^{D-1}\phi_k A_{jk}\phi_j\right).
\end{equation}
where the fields have been labeled with position indices and the matrix $A$ represents the differential
operator.  
This regularized expression can be integrated, under the assumption that the eigenvectors of $A$ can be found,
where $\sum_kA_{jk}\psi^{(n)}_k=\lambda_{(n)}\psi^{(n)}_j$.  In that case, each Gaussian integral decouples and the 
regularized partition function can be written as 
\begin{equation}
   Z_{\text{reg}} = C \prod_n \lambda_{(n)}^{-1/2} = C \det(A)^{-1/2},
\end{equation}
where the determinant is understood to be the product of the eigenvalues of the operator $A$.\footnote{
In fact, there is an approach to computing van der Waals energies based on directly 
evaluating a functional determinant for electric fields on a discrete spatial grid~\citep{Maggs2006,Pasquali2008}. This approach 
omits any time evolution of the fields, but it does offer a direct method of trying to evaluate the field path integral.
That work relied on direct spatial discretization to evaluate the functional determinants, which limits
the size of medium that can be considered.
}
The limit of an arbitrarily large volume, and lattice resolution can be taken after integration.

In an analogous fashion, one can formally evaluate the partition function path integral as a 
functional determinant, 
\begin{equation}
  Z \propto {\det}^{-1/2}\left(-\frac{1}{2}\partial_\tau^2-\frac{1}{2}\nabla^2+V(\vect{r})\right).
\end{equation}
% The original computations for the worldline method stressed computing the quantum effective
% action for the scalar field.  This yields essentially the same expression.  This expression has
% also retained a factor of $2$ on the gradients --- this will simplify the representation of the 
% worldline path integral.
The proportionality is due to an additional (infinite) normalization constant which will
be canceled in the renormalization process.  
The free energy for the interacting field can be written as 
\begin{equation}
  \mathcal{F} = -\kB T\log Z = \frac{1}{2}\kB T 
\log\det\bigg(-\frac{1}{2}\partial_\tau^2-\frac{1}{2}\nabla^2+V(\vect{r})\bigg)+C,
  \label{eq:free-energy-det}
\end{equation}
where $C$ is a divergent constant.  
As it stands this functional determinant is divergent, but finite results can be found by subtracting off the 
free energy $\cF_0$ when the bodies are removed arbitrarily far apart.  
The renormalized free energy can now be written in terms of a single-particle path integral via some formal 
manipulations.  First, we will use the identity $\log\det A=\tr\log A$, which can be 
verified for positive finite matrices,  
\begin{align}
  \log\det A &= \log\prod_j \alpha_j
  =\sum_j \log\alpha_j
  = \tr\log A,\label{eq:log-det}
\end{align}
where we used the facts that the trace and determinant of a matrix $A$ are given by the sum
and product of its eigenvalues $\alpha_j$ respectively. 
Second, the logarithm can be rewritten in an integral representation,
\begin{equation}
  \log A -\log B= -\int_0^\infty \frac{d\cT}{\cT} (e^{-A\cT} - e^{-B\cT}),\label{eq:integral_log}
\end{equation}
where $A$ and $B$ are positive operators (i.e. $A$ and $B$ have strictly positive eigenvalues).
This expression also relies on a difference of terms to cancel out divergent terms at $\cT=0$.  The 
earlier renormalization by subtracting off the vacuum energy when the bodies are far apart provides exactly this subtraction. 

By applying Eqs.~(\ref{eq:log-det}) and (\ref{eq:integral_log}) to the free energy~(\ref{eq:free-energy-det}),
 the renormalized free energy can be rewritten as
\begin{equation}
  \mathcal{F}-\mathcal{F}_0 = -\frac{\kB T}{2}\int_0^\infty \frac{d\cT}{\cT}
  \tr\Big(e^{[(\partial_\tau^2+\nabla^2)/2-V(\vect{x})]\cT}-e^{(\partial_\tau^2+\nabla^2)\cT/2}\Big).
\end{equation}
The trace can be evaluated by introducing a $D$-dimensional auxiliary Hilbert space, where 
$\langle \vect{x},x_{\tau}| \op{p}_i|\psi\rangle = -i \partial_i\langle \vect{x},x_{\tau}|\psi\rangle$,
$[\op{x}_i,\op{p}_j]=i\delta_{ij}$.  Note that $\hbar=1$ in this auxiliary Hilbert space.  The free energy is then
\begin{equation}
  \mathcal{F}-\mathcal{F}_0 = -\frac{\kB T}{2}\int_0^\infty \frac{d\cT}{\cT}
  \int d\vect{x}_0d\tau_0\, \langle \vect{x}_0,\tau_0|e^{-(\op{p}_\tau^2+\op{\vect{p}}^2)\cT/2 -\cT V(\op{\vect{x}})}
-e^{-(\op{p}_\tau^2+\op{\vect{p}}^2)\cT/2}  |\vect{x}_0,\tau_0\rangle.
\end{equation}
The free energy is now in the form of the imaginary-time transition amplitude for a quantum particle
in $D$ space-time dimensions, in a potential $V$.
In the same fashion as in Section~\ref{sec:feynman_path_integral},
this can be converted into a single-particle path integral, although 
there are some minor differences.  First, the starting and ending points are the same,
so the paths form closed loops.  
Second, the parameter $\cT$ has dimension of $L^2$.  
It governs the spatial extent of the path, rather than the proper time between events.
The resulting worldline path integral for the free energy at zero temperature  is
  \begin{align}
    \mathcal{F}-\mathcal{F}_0 
    =&  -\frac{\kB T}{2}\int_0^\infty \frac{d\cT}{\cT}
    \int d\vect{x}_0  d\tau_0 \int \prod_{k=1}^Nd\vect{x}_k d\tau_k \nonumber\\
    &\times\prod_{k=0}^{N-1}\bigg(\frac{1}{(2\pi\Delta\cT)^{D/2}}
    e^{-(\vect{x}_{k+1}-\vect{x}_k)^2/(2\Delta \cT)}e^{-(\tau_{k+1}-\tau_k)^2/(2\Delta \cT)}\bigg)\nonumber\\
    &\times \bigg(\prod_{j=1}^Ne^{-\Delta\cT V(\vect{x}_j)}-1\bigg)\delta(\vect{x}_N-\vect{x}_0)
    \delta(\tau_N-\tau_0).
  \end{align}
The intermediate Gaussian integrals over $\tau_k$ can be carried out, since the potential
is independent of $\tau$.  
The final integral $\int d\tau_0 = \beta\hbar c$, since $\tau_0\in[0,\beta\hbar c]$.  
% (A more careful derivation for non-zero temperature handles the $\tau$ direction via a Fourier transform,
% which introduces the Matsubara frequencies, $s_n = 2\pi n/(\beta \hbar)$.  
There is also a normalization constant of $(2\pi\cT)^{-1/2}$ for each dimension due to the loop 
closure condition.  This can be thought of as the total normalization for $N$ Gaussian steps of length 
$\Delta \cT= \cT/N$, subject to the loop-closure requirement $\vect{x}_0=\vect{x}_N$.
% The normalization can be understood from manipulations on the delta function, where 
% we require the multidimensional version of 
% \begin{equation}
%   \int dx \delta[h(x)]f(x) = \sum_{f(x)=0}\frac{f(x)}{|h'(x)|},
% \end{equation}
% which is given by
% \begin{equation}
%   \int \prod_{k=1}^Ndq_k \delta[h(\vect{q})]f(\vect{q}) = \oint_{h^{-1}(0)} dS\, 
%   \frac{1}{\sqrt{|\nabla_qh(\vect{q})|^2}}f(\vect{q}),
% \end{equation}
% where $\nabla_q h(\vect{q}) = \sum_i \frac{\partial h}{\partial q_i}$, and 
% the surface integral runs over coordinates satisfying the $h(\vect{q})=0$ condition~\citep{Hormander1983}. 
% If we change variable to $\Delta \tau_k = \tau_{k+1}-\tau_k$, then the loop closure condition is 
% \begin{equation}
%   \tau_0 - \tau_N = \sum_{k}\Delta \tau_k = 0.
% \end{equation}
The free energy can be written in a more intuitive form, better suited to numerical calculations,
if we consider the coupled Gaussians as the probability distribution for paths through space-time.
Each path increment $\Delta \vect{x}_k=\vect{x}_{k+1}-\vect{x}_k$ is Gaussian with zero mean and variance $\Delta \cT$.
In addition, the resulting paths must close on themselves. 

The resulting paths are a specific form of Brownian motion.
 A Brownian motion (or Wiener process) is a continuous random process $W(t)$, that starts at the origin ${W(t=0)=0}$, 
and has increments $\Delta W(t):=W(t+\Delta t)-W(t)$, that are Gaussian random variables with $\dlangle \Delta W(t)\drangle=0$,
$\dlangle \Delta W(t) \Delta W(t')\drangle = 0$ for $t\ne t'$ and $\dlangle [\Delta W(t)]^2\drangle=0$~\citep{Gardiner2009}.  
(Note that we are using $\dlangle\cdots\drangle$ to denote the ensemble average.)
A Brownian bridge is a Brownian motion with fixed end points at times $t=0$ and $t=T$, where 
$B(t=0)=0$, and $B(t=T)=c$, and its increments obey the same statistics as the Wiener process~\citep{Karatzas1991}.  
Throughout this thesis, we will most often use the discrete form of these processes where $W_j=W(t_j)=W(j\Delta t)$.
Brownian motion can be straightforwardly generalized to multiple-dimensions.  

The result of these manipulations is 
\begin{align}
  \mathcal{F}-\mathcal{F}_0 =& -\frac{\hbar c}{2}\int \frac{d\cT}{(2\pi\cT)^{D/2}\cT} \int d\vect{x}_0
  \dlangle e^{-\cT\langle V\rangle} - 1\drangle_{\vect{x}(t)},
  \label{eq:scalar_worldline}
\end{align}
 where $\dlangle\cdots\drangle_{\vect{x}(t)}$ denotes an ensemble average over closed Brownian bridges $\vect{x}(t)$
starting at $\vect{x}_0$ and returning with $\vect{x}_N=\vect{x}_0$, and 
\begin{equation}
  \langle V\rangle := \frac{1}{\cT}\int_0^\cT dt\,V[\vect{x}(t)] = \frac{1}{N}\sum_{k=0}^{N-1}V(\vect{x}_k)
\end{equation}  
is the path-averaged value of the potential. 
The worldline method relies on generating an ensemble of closed Brownian bridges, and evaluating
the path-averaged potential for each path.  The total Casimir energy then requires further integrals over the starting point $\vect{x}_0$
of the paths, and the total path time $\cT$.  The path time $\cT$ governs the spatial extent of the paths, 
where the typical extent of a path is given by $x\sim\sqrt{\cT}$.
The renormalization against vacuum ensures that only paths that touch one of the bodies contribute.  
In order to extract interaction energies between two bodies (such as the two-body Casimir energy), the single body
energies for each body must also be subtracted from the total energy.  As a result, only
paths that touch both bodies contribute.  This is depicted in Figure~\ref{fig:strong_coupling_cartoon},
where the upper path would contribute to the Casimir energy, while the lower path would not.  
At small times $\cT$, both paths would shrink down around their starting points, 
and since the paths would not touch both bodies, neither would contribute.
This is a direct result of the energy renormalization---subtracting off
the vacuum energy cuts off the divergent $\cT$ integral as small $\cT$.  
At later times $\cT$, these paths would have larger extent, and both would contribute, but due 
to the $\cT^{-(1+D/2)}$ dependence, the lower path would have a smaller contribution.  

\begin{figure}
\center
\includegraphics[width=10cm]{fig/intro/hit_strong_coupling}
\caption[Schematic of worldline paths interacting with plane and sphere]
{Schematic of worldline paths interacting with a plane and a sphere.  
  Only paths which touch \emph{both} bodies will contribute at a given path time $\cT$.  
  The upper path touches both objects and will contribute to Casimir energy,
  while the lower path only touches one body, and does not contribute to Casimir energy.}
\label{fig:strong_coupling_cartoon}
\end{figure}

\subsection{Worldline Distance Dependence}

The distance dependence can also be read off from Eq.~(\ref{eq:scalar_worldline}).
  The typical extent of a Brownian motion of total path time $\cT$ is $\sqrt{\cT}$. 
A typical path will touch a surface a distance $d$ away, at total path time $\cT\sim d^2$.
Since the integrand~(\ref{eq:scalar_worldline}) is either zero or one, depending on whether any points
on the path intersect the bodies,
the energy density at a point $d$ from the surface is approximately $\int_{d^2}^\infty d\cT\, \cT^{1+D/2}\sim d^{-D}$.  
After integration over the starting point $\vect{x}_0$, the Casimir energy scales as $d^{-3}$ in four
 dimensions.

The worldline method has also been extended to nonzero temperatures \citep{Klingmueller2008}.
The generalization is straightforward---in essence the fields must be periodic on $\tau\in[0,\beta\hbar c]$,
since $\phi(0)=\phi(\beta\hbar c)$ due to the nature of the trace in Eq.~(\ref{eq:Zphi}).  This motivates
expanding the fields in a Fourier series, with the Matsubara frequencies $s_n=(2\pi n)/(\beta \hbar)$,
where each Fourier component contributes independently of the others.  The same sort of manipulations
can be carried out with the result 
\begin{align}
  \mathcal{F}-\mathcal{F}_0 =& -\kB T {\sum_{n=0}^\infty}'\int_0^\infty \frac{d\cT}{(2\pi\cT)^{(D-1)/2}\cT} \int d\vect{x}_0\,
  e^{-s_n^2\cT/(2c^2)}\dlangle e^{-\cT\langle V\rangle} - 1\drangle,
\end{align}
where the prime on the sum means that the $n=0$ term is multiplied by a $1/2$.  
Since the $\cT$ dependence differs, there is also a different distance dependence.
Since the effective dimension has been reduced by one, the energy density now scales as $d^{-(D-1)}$,
which means the renormalized energy density scales as $d^{-3}$, and the total energy scales as $d^{-2}$ in four dimensions.

\label{sec:worldline_distance_dep}

\subsection{Numerical Method}

In order to numerically evaluate the worldline Casimir energy, it is necessary to generate 
an ensemble of closed, Brownian paths.  Given the probability for a free Brownian motion $W(t)$ to close 
on itself after $N$ steps is negligible, it is essential to force the closure constraint when
constructing the paths.  

The simplest method generates a free Brownian motion, and then forces the path to close by subtracting
off a pro-rated fraction of the final position from each increment.  So if we have a free random walk,
\begin{equation}
  W_k = \sum_{j=1}^k \Delta W_k,
\end{equation}
where $\dlangle \Delta W_k \drangle =0$ and $\dlangle \Delta W_k\Delta W_j\drangle = \delta_{jk}\Delta \cT$,
then a closed Brownian bridge can be constructed as 
\begin{equation}
  B_k = \sum_{j=1}^k \Delta W_k -\frac{k}{N}W_N.\label{eq:prorate-loop}
\end{equation}
This algorithm has the virtue of simplicity, but it does require that knowledge the whole Brownian path 
in order to construct the closed version.
\citet{Gies2003} developed an improved algorithm, the so-called ``v-loop'' algorithm for generating
Brownian paths. 
A Brownian bridge can be constructed as 
\begin{equation}
  B_k = c_k B_{k-1} + \sqrt{c_k} \Delta W_k,
\end{equation}
where 
\begin{equation}
  c_k = \frac{N-k}{N-k+1}, \quad k=1,\cdots,N-1.
\end{equation}
Since we will use the v-loop algorithm in our own simulations, 
we will discuss this algorithm further in Chapter~\ref{ch:numerical}. 

Having constructed a path, it is then necessary to compute the worldline integrand $e^{-\cT\langle V\rangle}-1$ along that path.
If any point along the path intersects one of the surfaces, then in the strong coupling limit the 
potential $V=\lambda\delta[\sigma(\vect{x})]$ is nonzero, and in the $\lambda\rightarrow \infty$ limit, 
the worldline integrand goes to negative one.  
If however, no points on the path intersect a surface, then the potential is zero, and the renormalized
worldline integrand is also zero.  

Once a particular random path has been constructed, it is necessary to integrate the contributions from each starting
point $\vect{x}_0$, and path time $\cT$.  
Thus the worldline algorithm relies on finding the times $\cT$ when at least one path point intersects
the bodies, and integrating over those times.  This must further be integrated over every possible path starting point.
For simple geometries, these touching times can be found analytically for a particular random path,
which simplifies the method further~\citep{Weber2009,Weber2010}.

% \begin{itemize}
%   % \item Generate ensemble of closed Brownian paths.
%   % \item Can be done by decoupling Gaussian integrals - so-called ``v-loop'' algorithm.
%   % \item Can also create open Brownian walk, then force to close by pro-rating final position around walk.
%   \item Corresponds to IR vs BT?

\subsection[Advantages and Shortcomings of the Scalar \\ Method]{Advantages and Shortcomings of the Scalar Method}

The worldline method has a number of attractive features.  
First, it offers an intuitive picture of Casimir energies emerging from the spatial paths 
of virtual particles.   In this picture, the random paths explore all of space
and accumulate a contribution to the Casimir energy based on the potential $V$ they encounter.

Second, it offers a geometry independent method of handling Casimir forces.  The paths are 
created without reference to the underlying body geometry or a spatial discretization, so the method can be easily applied to arbitrarily
complicated arrangements of bodies.  The only requirement is that the paths are fine enough
to resolve the structure of the surfaces.  

Third, since each path is independent, the algorithm is trivially parallelizable: each path
can be handled by a separate computing process, without any requirement that the processes communicate
with one another, except when accumulating results.  This has the advantage of exploiting the growth of computing clusters with many nodes,
where that power can be harnessed with minimal effort: once the algorithm works on a single computer,
it can be easily extended to arbitrarily many computers to increase the size of the ensemble sampled
from, or reduce the time required to reach a given accuracy.  

% \begin{enumerate}
%   % \item Advantages

%   % \begin{enumerate}
%   %   \item Intuitively appealing picture of fluctuations moving through space-time.
%   %     However - fictious time, no relativistic speed limits.  
%   %   \item Algorithm is geometry independent, and no spatial or temporal grid.
%   %     Instead have path resolution.  
%   %     Generate paths, and see if they touch.
%   \item Trivially parallelizable since each trajectory is independent. 
%       Computation time scales as one /resources.  
%   \end{enumerate}

%  \item Shortcomings

However, the worldline method has some prominent shortcomings.  First of all, it only applies 
to scalar fields.
  The most important Casimir effects are due to EM radiation field, 
which is a transverse vector field.
  Second, it has only been applied for idealized potentials that effectively impose 
Dirichlet boundary conditions on the surfaces.  As a result it is missing any 
coupling of the fields to media with realistic properties.
Finally, the development has been focused on Casimir energies, with no simple way to 
extract Casimir--Polder energies for atoms near surfaces 
(although it may be possible to extract these from the stress-energy tensor). 
Thus far, there has only been speculation on how to extend the worldline method to 
electromagnetism, without any concrete results~\citep{Aehlig2011}.  

    % \begin{enumerate}
    %   \item No coupling of photons to medium.
    %   \item A scalar, not vector electromagnetism.
    %   \item Idealized boundary conditions.  
    %   \item No way to extract atomic energies.  (The simplest guess of suppressing spatial integral turns 
    %     out to be wrong direction).
    % \end{enumerate}
%\end{enumerate}

\subsection{Motivation and Goal for Thesis Project}

The goal of this thesis is to extend the scalar worldline method to vector electromagnetism.
Ideally we would retain the attractive features of the method, such as geometry independence of the paths,
and only needing an ensemble of simple Brownian motions.
In addition, we aim to improve the method to account for the two physical polarization
states of the EM field, and properly account for the material properties of the medium.  
Finally, the method must agree with known results in simple geometries.  

As later chapters in this thesis will show, we have partially met those goals.  We have developed analytical
and numerical techniques that can be applied to improving existing worldline algorithms.
This thesis focuses primarily on solving the planar problem---although this is well-studied,
it is a good platform for exploring and testing worldline methods.  
The methods we develop here could be used as uncontrolled approximations in general geometries, but with
no guarantee of correctness.  

% \section{Thesis outline}

% The rest of this thesis is laid out as follows: 
% In Chapter~\ref{ch:EM_quantization} we formally quantize
% the EM field in media, and derive the worldline expressions for the electromagnetic Casimir energy.
% We will attempt to carry this out in two ways. 
% First, we will develop the full vector path integral for the EM field.  Unfortunately, this path integral has a number of worrisome features.
% Second, we will split the EM field into two noninteracting scalar polarizations,
% the transverse-electric (TE) and transverse-magnetic (TM) polarizations.
% %We will also review others work on quantizing the EM field in media.

% In Chapter~\ref{ch:feynman_kac} we will discuss the analytical methods for solving single-particle path 
% integrals.  In particular, this relies on the Feynman-Kac formula, which states the path integral
%  is the solution to a diffusion equation.  In simple cases, the diffusion equation can be solved.
% We will derive the relevant path integrals solutions for the TE and TM polarizations in a planar geometry. 

% In Chapter~\ref{ch:analytical} we will use the results from the previous chapter to derive analytical 
% results showing agreement between the worldline method and prior results, at least for planar media.
% We will also discuss the transition between high and low temperature, and near field limits within the
% worldline context.  

% In Chapter~\ref{ch:numerical} we discuss the numerical methods for evaluating worldline path integrals.
% This will include efficient methods for computing derivatives, as well as Monte Carlo sampling methods
% for generating paths and sampling $\vect{x}_0$ and $\cT$.  
% We will examine the convergence of the methods as the path resolution is increased.

% In Chapter~\ref{ch:force} we develop general expressions for the force, torque and derivative of the force 
% for the TE Casimir energy between arbitrarily shaped, piecewise constant dielectric media.  
% We will then discuss how to efficiently sample these expressions, and numerically test them in the case of planar media.  

% Finally, we will summarize our findings and directions for future work in the conclusion.  

\section{Thesis outline}
\label{sec:thesis_outline}
The rest of this thesis is laid out as follows: 
In Chapter~\ref{ch:EM_quantization} we formally quantize
the EM field in media characterized by their relative permeability $\epsr$ and permittivity $\mur$.
We then develop two sets of worldline expressions for the electromagnetic Casimir energy.
Initially, we will develop the full vector path integral for the EM field partition function~(\ref{eq:Z_4pot}).  
This can be further developed into a worldline path integral expression~(\ref{eq:vector_path_integral}).
Unfortunately, it is not clear how to extract the two physical degrees of freedom from that complicated
expression, which involves the cancellation of a number of degrees of freedom.
Instead of that approach, we will split the EM field into two noninteracting scalar polarizations:
the transverse-electric (TE) and transverse-magnetic (TM) polarizations.
The worldline path integral for the free energy in the TE polarization 
at zero temperature in dispersion-free media is [Eq.~(\ref{eq:TE_worldline})]
\begin{align}
    \cF\subTE-\cF\sup0 &= -\frac{\hbar c}{2}\int_0^\infty\frac{d\cT}{(2\pi\cT)^{D/2}\cT}\int d\vect{x}_0
    \biggdlangle
    \frac{e^{-\langle V\subTE(z)\rangle\cT}}{\sqrt{\langle \epsr(z)\mur(z)\rangle}} -1
    \biggdrangle_{\vect{x}(t)},\label{eq:FTE_loc}
\end{align}
where 
\begin{equation}
  V\subTE(z) := \frac{1}{2}\big[(\partial_z\log\sqrt{\mur})^2-\partial_z^2\log\sqrt{\mur}\big].
\end{equation}
The equivalent TM polarization is recovered by exchanging $\epsr$ and $\mur$.  

In Chapter~\ref{ch:feynman_kac} we will discuss the analytical methods for solving single-particle path 
integrals.  The central expression underlying this analytical approach is the Feynman-Kac formula, which states the path integral
 is the solution to a diffusion equation.
As shown in Section~\ref{sec:derive_feynman_kac}, the diffusion equation~(\ref{eq:diffusion_equation}),
\begin{equation}
  \partial_t f = \frac{1}{2}\nabla^2 f  - [V(\vect{x})+\lambda]f +\delta(\vect{x-c}),
\end{equation}
can be solved by the following path integral [Eq.~(\ref{eq:f_soln})],
\begin{align}
  f(\vect{x}) = \int_0^{\infty} ds\,
  \biggdlangle \delta[\vect{x}+\vect{W}(s)-\vect{c}] \exp\bigg(-\lambda s -\int_0^s du\, V[\vect{x}+\vect{W}(s-u)]\bigg)\biggdrangle,
\end{align}
where we have substituted $g=\delta(\vect{x}-\vect{c})$ as the ``source'' term.  
In simple cases, the diffusion equation~(\ref{eq:diffusion_equation}) can be solved directly.
Most importantly, in Section~\ref{sec:TM boundary condition} we will show that a regularized form of the TM potential [Eq.~(\ref{eq:VsubTM})]
\begin{equation}
  V\subTM(z) := \frac{1}{2}\big[(\partial_z\log\sqrt{\epsr})^2-\partial_z^2\log\sqrt{\epsr}\big],
\end{equation}
leads to an effective boundary condition in the diffusion equation.
For a dielectric step characterized by $\epsr = 1+\chi\Theta(x-d)$, the effective boundary condition is given by~[Eq.~(\ref{eq:TM_boundary_conditions})],
\begin{align}
  f(d+0_+) =& e^{-\Xi}f(d-0_+) \qquad
  \partial_xf(d+0_+)= e^{\Xi}\partial_xf(d-0_+),
\end{align}
where $e^{\Xi}=\sqrt{1+\chi}$.
Using that boundary condition, Section~\ref{sec:TM_potential} shows that the path-averaged analytical solution for paths from $x=0$ to $x=c$ in time $t$,
interacting with potential $V\subTM$ is [Eq.~\ref{eq:TM_potential}],
\begin{align}
  \dlangle e^{-\int_0^t dt'\, V\subTM(x-d)}\drangle 
  &=\left\{ \begin{array}{lc} 
      1   + \sgn(d)\tanh\Xi\, e^{-2d(d-c)/t} & d(d-c)>0\\
      \sech\Xi & d(d-c)<0.
    \end{array}
  \right.  
\end{align}
This result is crucial to regularizing the TM potential and allowing the numerical calculations 
with the TM polarization to proceed.  
The path integral solutions for the step potentials $V=\chi\Theta(x-d)$ are given in Eqs.~(\ref{eq:Feynman-Kac TE one step})
and (\ref{eq:Feynman-Kac TE two step}), and are particularly relevant for TE calculations.  
The equivalent results also including TM boundary conditions are given in Eqs.~(\ref{eq:Feynman-Kac TM one step})
and (\ref{eq:Feynman-Kac TM two step}).

In Chapter~\ref{ch:analytical} we will use the results from the previous chapter to derive analytical 
results showing agreement between the worldline method and other calculations.
In Section~\ref{sec:casimir-polder_worldline}, the worldline Casimir--Polder energy is derived 
by treating the atom as a small perturbation to the material functions, where the 
atom is located at $\rA$,    with static polarizability $\alpha_0$.
We will typically simplify these expressions by assuming that the media are non-magnetic ($\mur=1$),
 and the atom's magnetic response can also be neglected.
In that case, the worldline expressions for the Casimir--Polder energy are given by [Eqs.~(\ref{eq:TE_Casimir_Polder})--(\ref{eq:TM_Casimir_Polder})]:
\begin{align}
    V\supTE\subCP(\vect{\rA}) &= \frac{\hbar c\alpha_0}{4\epsilon_0(2\pi)^{D/2}}\int_0^\infty\frac{d\cT}{\cT^{1+D/2}}
    \biggdlangle
      \frac{1}{\langle \epsr\rangle^{3/2}} -       \frac{1}{[\epsr(\rA)]^{3/2}}
    %   \right) \nonumber\\
    % &\hspace{1cm}
      \biggdrangle_{\vect{x}(t),\vect{x}(0)=\rA}\label{eq:VTE_CP_loc}
      \\
    V\supTM\subCP(\vect{\rA}) &= \frac{\hbar c\alpha_0}{4\epsilon_0(2\pi)^{D/2}}\int_0^\infty\frac{d\cT}{\cT^{1+D/2}}
    \biggdlangle
      \frac{e^{-\langle V\subTM\rangle\cT}}{\langle \epsr\rangle^{3/2}}-  \frac{1}{[\epsr(\rA)]^{3/2}}\nonumber\\
      &\hspace{5cm}       -\frac{\cT}{2\epsr(\rA)} \nabla^2 \frac{ e^{-\langle V\subTM\rangle\cT}}{\langle\epsr\rangle^{1/2}}
      \biggdrangle_{\vect{x}(t),\vect{x}(0)=\rA}.\label{eq:VTM_CP_loc}
\end{align}
Note that these Casimir--Polder worldline path integrals only involve paths emanating from the atom's position.  
The analytical results from Chapter~\ref{ch:feynman_kac} can used in the worldline path integrals 
after transforming the path integrals by using the Laplace--Mellin theorem~(\ref{eq:Laplace--Mellin}) and the inverse moment theorem~(\ref{eq:moment_theorem}).
The worldline method can then be shown to recover prior results such as the Casimir--Polder energy~(\ref{eq:VCP_eta}), 
and the Lifshitz expression for the energy~(\ref{eq:lifshitz}).
We will also discuss the transition between high and low temperature, and near field limits within the
worldline context.  

In Chapter~\ref{ch:numerical} we discuss the numerical methods for evaluating worldline path integrals.
This will typically involve using Monte Carlo sampling methods for generating paths and sampling $\vect{x}_0$ and $\cT$.  
These methods are discussed for the TE polarization in Section~\ref{sec:TE_Casimir_numerics}.
In Section~\ref{sec:TE_results}, we will present the numerical results for the TE polarization,
 and examine the convergence of the TE results as the path resolution is increased.
Numerical calculations involving the TM polarization are more challenging than their TE counterparts,
 even after regularization and analytical averaging.
As discussed in Section~\ref{sec:TM_scaling}, the numerically estimated TM path integral 
shows growing statistical errors as the number of steps $N$ increases.  
The ``birth-death path swarm'' is introduced in Section~\ref{sec:birth_death}, and is essential
for reducing the statistical errors associated with accumulating the product of the path-averaged
potential along the path.  
The derivatives required for the TM Casimir--Polder energy are estimated using a ``partial-averaging method'' discussed in Section~\ref{sec:partial_averaging}.
The numerical results for the TM polarization are presented in Section~\ref{sec:TM_results}, where
they show some agreement with the analytical results.

In Chapter~\ref{ch:force} we develop general expressions for the force, torque and derivative of the force 
for the TE worldline path integral.  These expressions emphasize that the Casimir force emerges from paths that
start on the surfaces of the bodies.  
For example, the ``pinning'' expression for the Casimir force on the second body is
is [Eq.~(\ref{eq:pinning_force})]:
\begin{align}
  \vect{F}_{2}&=
  -\frac{a\chi_2\hbar c}{2(2\pi)^{D/2}}
\int\limits_0^\infty \!\frac{d\cT}{\cT^{1+D/2}}    
\hspace{-3ex}
 \oint\limits_{\sigma_2(\vect{x}_0-\vect{R}_2)=0}^{}
  \hspace{-4ex} dS\hspace{1ex} 
  \hat{n}_2(\vect{x}_0) % \nonumber\\
  % &\hspace{0.5cm}\times 
  \Biggdlangle\frac{1}{\langle\epsilon_{\mathrm{r},12}\rangle^{a+1}}-\frac{1}{\langle\epsilon_{\mathrm{r},2}\rangle^{a+1}}
  \Biggdrangle_{\vect{x}(t)}.
\end{align}
Here $\sigma_2=0$ defines the surface of the second body, $\hat{n}_2$ is the surface normal of the body, while $\epsilon_{\mathrm{r},12}$ 
and $\epsilon_{\mathrm{r},2}$ are the dielectric functions involving both bodies, and only the second body, respectively.
Unfortunately, the pinning method fails to recover the correct answer for strong-coupling cases where $\chi/N\gg 1$,
which prompts developing the ``occupation'' method in Sec.~\ref{sec:occupation}.  
Some care is still required when using the occupation method numerically, since in the strong-coupling
limit, only rare paths that just touch the surfaces will contribute.  
A hybrid approach designed to capture the weak coupling and strong-coupling cases is discussed further in sections~\ref{sec:generic_coupling} and \ref{sec:softened_delta}.

Finally, we will summarize our findings and directions for future work in the conclusion.  

% \begin{figure}
% \centering
% \tikzstyle{chap} = [rectangle, rounded corners, minimum width=3cm, 
% minimum height=1cm,text centered, draw=black, fill=gray!30]
% \tikzstyle{arrow} = [thick,->,>=stealth]

%   \begin{tikzpicture}[node distance=2cm]
%     \node (intro) [chap] {Ch I: Introduction};
%     \node (EM) [chap, below of=intro] {Ch II: EM Quantization};
%     \node (Feynman) [chap, below of=EM] {Ch III: Feynman--Kac Formulae};
%     \node (Analytical) [chap, below of=Feynman, xshift=-4cm] {Ch IV: Analytical Worldlines};
%     \node (Numerical) [chap,  right of=Analytical,xshift=+6cm ] {Ch V: Numerical Worldlines};
%     \node (Force) [chap, below of=Numerical] {Ch VI: TE Forces and Torques};
%     \draw [arrow] (intro)--(EM);
%     \draw [arrow] (EM)--(Feynman);
%     \draw [arrow] (Feynman)--(Analytical);
%     \draw [arrow] (Feynman)--(Numerical);
%     \draw [arrow] (Numerical)--(Force);
%   \end{tikzpicture}
%   \caption[Chapter Dependencies]{Chapter Dependencies in this thesis.}
% \end{figure}


%%% Local Variables: 
%%% mode: latex
%%% TeX-master: "thesis_master"
%%% End: 