\chapter{Introduction}

%The story  I want to tell explains the Casimir force via its historical origins which serve to introduce the big results, while sidestepping weird irrelevant bullshit.  I want to hit the important context for modern scientists and emphasize quantities realted to current experiments.  I also want to outline the currently available most powerful methods to set context for what else is possible, and what its limitations are.  

%I think I can do this with a partial historical introduction.

Quantum mechanics is an inherently statistical theory - we can only make predictions for the probabilities of events.  
\comment{Cite Dirac/Sakurai as general textbook}.
\begin{itemize}
\item Add complex amplitudes together.  Take modulus squared to get probabilities.  
\item Feynman path integral construction of quantum mechanics- 
to get amplitude to get from one position to another, add up \emph{all} possible paths,
 with a suitable phase, $e^{-i S[x]/\hbar}$, \cite{Feynman1942, Feynman1965}
\item Found common use in field theory particularly for covariantly quantizing gauge field theories,
 such as the Standard Model (cite Weinberg).  Integrate over all field configurations.  
\item  Also related to stochastic processes, like Brownian motion \cite{Karatzas1991}.
   Can mathematically formulate a path integral as very high-dimensional integral.
  Can then use Monte-Carlo methods, to randomly sample from most important regions of integral.  
\end{itemize}

\begin{itemize}
\item Quantum measurements are inherently probabilitistic.  
\item Thus, sequences of measurements are also inherently random.  
\item Greater experiemtnal control means we can now probe isolated quantum systems repeatedly and see how they evolve.  
\item Move beyond simple projective measurements, which cannot describe things like position measurements.
\item To describe continuously weakly probing a system, also use stochastic processes as part of
 numerical simulation strategy.
\item Quantum tajectories correspond to monitoring a quantum system (like an atom) via a probe,
 for example shining light on the atom. \cite{Carmichael1993}
\item Simulated trajectories are possible trajectories a system could trace out as we observe.
  Also given a particular measurement record, they correspond to our best estimate of the current state of the system.  
\end{itemize}

\comment{There are two roles for the randomness here.
  From one point of view, the randomness is an intrinsic part of quantum mechanics,
 and thus naturally shows up in fluctuation phenomena like the Casimir effect, or in quantum measurements.
  Alternatively, if we are considering these calculations for energy or evolution as large integrals,
 then we can efficiently compute these integrals by randomly sampling from their dominant regions.
  In this sense, the randomness shows up as a calculational tool.}

This thesis will cover a two different computational projects in quantum optics.
The first project relates to computing Casimir forces via the worldline method.
  Most of the thesis is devoted to discussing the background, 
and discussing the analytical and numerical techniques developed.
  In brief, the worldline method is a numerical method for computing Casimir energies via path integrals.
  The numerical simulations

The second theme will be using quantum trajectories to simulate continupus positions measurements on atoms.
Both of these projects are related to fundamental quantum optical phenomena, 
that we will explore using stochastic calculus, and Monte-Carlo methods for numerical simulations.  

\begin{itemize}
\item General thesis about quantum physics of atoms using computation.  
\item Thesis covers numerical monte-carlo techniques, where random numbers used in simulation.
\item First new method for computing electromagnetic Casimir energies.
\item Second, continuous position measurements of atoms taking into account experimental constraints.  
\item United in perspective of numerical work relying on Monte-Carlo techniques,
 to explore quantum phenomena relevant to modern experiments and pushes toward future technology  
\end{itemize}



\section{Casimir Forces in general and physical interpretation}

\begin{itemize}
\item Casimir force arises due to fluctuations in quantum fields. 
\item Forces between macroscopic bodies, atoms and bodies and between atoms.  
\item Interesting for theory reasons as purely quantum field theory phenomena, 
  and relevance to current experimental physics.  
\end{itemize}


\begin{itemize}
\item General references Cite Books - Milonni~\cite{Milonni1994}, Milton~\cite{Milton2001}.
  Bordag~\cite{Bordag2009}, Dalvit~\cite{Dalvit2011} recent developments.  
\item More general books on Casimir and van der Waals forces as they relate to chemistry, 
  Parsegian~\cite{Parsegian2006}, Israelachivili~\cite{Israelachvili2011}.
\end{itemize}


\subsection{Casimir energy}

The Casimir force is an force that arises between bodies due to fluctuations in quantum fields.
  It was first predicted in 1948 by Henrik Casimir~\cite{Casimir1948}.
  If we consider two planar perfect conductors, we find that there is an 
attractive force between them due to the zero point energy of the electromagnetic field.
  In a quantum electrodynamics the ground state of each mode of the 
electromagnetic field contributes $\hbar\omega/2$, where $\hbar$ is Planck's
 constant, and $\omega$ is the frequency of the mode.
  The presence of the conductors forces the electric field to vanish on the surfaces,
 which restricts the allowed modes of the electromagnetic field.  
If we add up the energy from all modes of the electromagnetic field, 
and compare it to the case when the plates are infinitely far apart from one another, 
we find the energy is reduced as the plates are brought closer together.

  The energy between the plates is then
\begin{equation}
  E = -\frac{\hbar c}{240\pi^2 d^3},
\end{equation}
where $c$ is the speed of light in vacuum, $\hbar$ is the reduced Planck constant,
and $d$ is the distance between the plates.  

\subsubsection{Calculation for perfect conductors}

\comment{Give simple derivation of force? Cite Bordag/Dalvit?}
Here we will briefly reprise Casimir's original calculation~\cite{Casimir1948}.
  Let us consider a perfectly conducting box of length $L$.
  We will place another perfectly conducting plate, of area $L^2$, a distance $a$ from the $xy$ wall.   
If we consider the quantized electromagnetic field,  
then the energy of the field in its ground state is
\begin{equation}
E = \sum_{\text{modes}}\frac{\hbar\omega_\alpha}{2}.
\end{equation}
Since the energies for each mode are positive, and the sum extends over all possible modes,
 this enery is extremely divergent.
  However, if we consider energy \emph{differences} between two configurations 
we can find a finite result.  
This subtraction is a form of renormalization.  

We can write down the allowed modes for a rectangular cavity defined by, 
\begin{equation}
0 \le x \le L, \quad 0 \le y\le L, \quad 0 \le z \le a.
\end{equation}
From the boundary conditions, and transversality of the field,
 the mode functions for a perfectly conducting box 
are\footnote{Eq. (8.62) in `` Quantum and Atom Optics'', by Daniel A. Steck 
  \cite{SteckNotes}}:
\begin{align}
\mathbf{f}_{\mathbf{k},\zeta}(\mathbf{r}) =& \sqrt{\frac{8}{V}}\bigg[ 
\hat{x}(\hat{\epsilon}_{\mathbf{k},\zeta}\cdot\hat{x})\cos k_xx\sin k_yy\sin k_z z\nonumber\\
&+\hat{y}(\hat{\epsilon}_{\mathbf{k},\zeta}\cdot\hat{y})\sin k_xx\cos k_yy\sin k_z z\nonumber\\
&+\hat{z}(\hat{\epsilon}_{\mathbf{k},\zeta}\cdot\hat{z})\sin k_xx\sin k_yy\cos k_z z\bigg],
\end{align}
where $\hat{\epsilon}_{\mathbf{k},\zeta}$ is the polarization vector associated with
 the mode with wavevector $\vect{k}$ and the polarization index $\zeta$ takes on two values.
  The conducting boundary conditions mean that the fields must vanish on the surfaces, which requires that
\begin{equation}
k_x = \frac{pi n_x}{L},\quad  k_y = \frac{\pi n_y}{L}, \quad k_z=\frac{\pi n_z}{a},
\end{equation}
where $n_x,n_y,n_z$ are positive integers. 
 The requirement that $\nabla\cdot\vect{E}=0$, then requires that 
$\vect{k}\cdot\vect{E}=0$, which limits the polarizations to two values.
  We can write the frequency for a particular mode as
\begin{equation}
  \omega = c|\vect{k}| = c\sqrt{k_x^2+k_y^2+k_z^2}.
\end{equation}
Then if we take the limit $L\rightarrow\infty$, we can replace the sums over
 these modes with integrals over a continuum, 
\begin{equation}
E = \hbar c\bigg(\frac{L}{\pi}\bigg)^2{\sum_{n_z=0}^\infty}'\int_0^\infty dk_x
\int_0^\infty dk_y \left(\sqrt{k_x^2+k_y^2+\frac{n_z^2\pi^2}{a^2}}\right),
\end{equation}
where ${\sum_n}'$ includes a factor of $1/2$ for the $n=0$ term.
  (At $n_z=0$, there is only one polarization, while for larger $n$ there are 
two polarizations.  
When we come to consider non-zero temperature effects later, 
we will encounter similar summations.)

We can introduce polar coordinates to evaluate the integral over $k_x,k_y$.  
If we define $\kappa=\sqrt{k_x^2+k_y^2}$, and carry out the angular integral in 
the upper quarter plane, we find
\begin{equation}
E = \hbar c\bigg(\frac{L}{\pi}\bigg)^2\frac{\pi}{2}{\sum_{n_z=0}^\infty}'
\int_0^\infty d\kappa\,\kappa \sqrt{\kappa^2+\frac{n_z^2\pi^2}{a^2}}.
\end{equation}

On its own this energy is highly divergent.  
We will then subtract off the energy when the plates are moved farther apart.
In the limit of large $a$ we can also convert the sum over integers into 
another integral.  
The result of this subtraction is
\begin{equation}
E = \hbar c\bigg(\frac{L}{\pi}\bigg)^2\frac{\pi}{2}\int_0^\infty d\kappa\,\kappa
 \left({\sum_{n_z=0}^\infty}'\sqrt{\kappa^2+\frac{n_z^2\pi^2}{a^2}}
-\frac{\pi}{a}\int_0^\infty dk_z\sqrt{\kappa^2+k_z^2}\right).
\end{equation}
At this point Casimir introduces a function $f(k/k_m)$ to regularize the calculation.
This physically amounts to including some dispersion to model an 
ultra-violet (UV) cutoff.  
It is known \comment{Kramers-Kr\"onig} that the metal becomes transparent to 
photons at high enough energies, or short enough wavelengths,
 where $k_m$ indicates the cutoff.
  $f(k/k_m)$ is one for $k\ll 1$, and approaches zero for $k\gg k_m$.  

If we change variable again to $u=(a\kappa/\pi)^2$, then
\begin{equation}
E = \hbar c\bigg(\frac{L}{\pi}\bigg)^2\frac{\pi}{2}\frac{1}{2}
\left(\frac{\pi}{a}\right)^3\int_0^\infty du 
\left({\sum_{n_z=0}^\infty}'\sqrt{u+n_z^2}-\frac{\pi}{a}\int_0^\infty dn\sqrt{u+n^2}\right)
f\left(\frac{\pi\sqrt{n^2+u}}{a k_m}\right).
\end{equation}
%(\pi/a)^2 \frac{1}{2} from change of variable
If we now use the Euler-Maclaurin formula,
\begin{equation}
{\sum_{n=0}^\infty}' F(n)-\int_0^\infty dn\, F(n)
=-\frac{1}{12}F'(0)+\frac{1}{24\times 30}F'''(0),
\end{equation}
where primes denote differentiation w.r.t the argument, 
\begin{equation}
F(n)=\int_{n^2}^\infty dw\, \sqrt{w}f\left(\frac{\pi w}{ak_m}\right),
\end{equation}
where we have changed variables in the $u$ integral to $w = \sqrt{n^2+u}$
Then taking 
\begin{align}
F'(n) & = -\frac{2n^2}f\left(\frac{\pi n^2}{a k_m}\right)\\
F'(0) & = 0\\
F'''(0) & = -4
\end{align}
\comment{huh?  Introduced $f(k)$ to regularize Euler-Maclaurin}.

The end result of this, this that the renormalied vacuum energy per unit area is
\begin{equation}
\frac{E}{L^2} = -\frac{\hbar c\pi^2}{720 a^3}
\end{equation}

\begin{figure}
\center
\includegraphics[width=10cm]{fig/intro/twoplanes_wave}
\caption{Sketch of allowed modes between perfectly conducting plates.}
\end{figure}

This example calculation is also carried out in the initial chapters of Ref.s~\cite{Milton2001,Bordag2009,Dalvit2011}.

\subsection{Casimir-Polder energy}

The Casimir force is also important for atoms near surfaces.  
This variant is known as the Casimir-Polder force, 
after a paper by Casimir and Polder where they computed the force between an 
atom and a perfect conductor accounting retardation due to the finite speed 
of light~\cite{CasimirPolder1948}.  
In this case, the atom feels an attractive potential to a surface a distance $d$ away,
\begin{equation}
V_{CP} =-\frac{3\hbar c\alpha_0}{64\pi^2\epsilon_0 d^4}.
\end{equation}

\comment{Note Dan's notes as example calculations.  }

\subsection{Van der Waals forces}

The formalism was extended to include dielectric media by Lifshitz~\cite{Lifshitz1956}.
  \comment{He also worked alongside co-workers Dzayolshinkii and Abrisokov to
 further compute the force from using Feynman diagrammatic methods\cite{Dzyaloshinskii1959,Dzyaloshinskii1961}}.  
\comment{Point of view: Due to thermal fluctuations in medium.}

Mclachlan\cite{McLachlan1963} cites these guys?


\subsection{Scale and Physical Interpretation}

\begin{itemize}
\item Ground state energy or zero point enery.  
Emphasizes boundary conditions and restricted spectrum of fluctuations.  
\item Atoms emitting and absorbing virtual photons.  
\item Feynman Diagram.  
Atom-Wall
\begin{figure}
  \centering
\begin{fmffile}{atom-loop}
  \begin{fmfgraph*}(100,60)
    \fmfleft{i}
    \fmfright{o}
    \fmftop{t}
    \fmf{plain}{i,v1}
    \fmf{plain}{v2,o}
    \fmf{plain,label=$|e\rangle$}{v1,v2}
    \fmf{photon,left=0.5,tension=.4}{v1,vt,v2}
    \fmf{phantom,tension=1}{t,vt}
    \fmflabel{$|g\rangle$}{i}
    \fmflabel{$|g\rangle$}{o}
    \fmflabel{$\gamma$}{vt}
  \end{fmfgraph*}
\end{fmffile}
\caption{Atom interacting with wall via emitting and absorbing photons.  }
\end{figure}

Wall-Wall Effective action.

\begin{figure}
\centering
\begin{fmffile}{wall-wall}
\begin{fmfgraph}(50,30)
 \fmftop{t0,t1,t2,t3}
 \fmfbottom{b0,b1,b2,b3}
 \fmf{phantom}{t1,v1}
 \fmf{phantom}{t2,v3}
 \fmf{phantom}{b1,v2}
 \fmf{phantom}{b2,v4}
 \fmffreeze
\fmf{photon}{v1,v3}
\fmf{photon}{v2,v4}
\fmf{fermion,tension=0}{v1,v2}
\fmf{fermion,tension=0,left}{v2,v1}
\fmf{fermion,tension=0}{v3,v4}
\fmf{fermion,tension=0,right}{v4,v3}
\end{fmfgraph}
\end{fmffile}
\caption{Casimir Energy in terms of fundamental QED processes.  Electrons are considered bound within their respective media.}
\end{figure}

\item Picture of Electrons interacting with EM field.
  Effective action at some loop order in basic QED.
  Makes contact with fundamental physics.
  Can then make sense of limits in which we are operating.
  Doing perturbation theory on QED.
  $\epsilon$ is linear response of medium to EM field, and working to leading order in $\epsilon$.
  Amounts to $\order(\alpha_0^4)$ diagram for QED with electrons bound via nuclear potential.   (Again, very low energies).
\begin{itemize}
\item Assuming electrically neutral
\item Far from atomic separations, so continuum approximation acceptable.
\end{itemize}

\item Casimir forces are short ranged forces and decay away quickly.
  The Coulomb potential between charged particles is behaves as $d^{-1}$,
 where $d$ is the particles separation.
  In comparison, the London dispersive force between two atoms decays as $d^{-7}$.
  For macroscopic bodies, the decay is slower, since we can roughly think of
 adding up the contributions from all of the constituent atoms.
  For Casimir energies, the energy decays as $d^{-3}$.  

These considerations are for zero temperature.
  It turns out that considering the effects of finite temperature, 
in particular thermal photons, leads to different distance dependence again.
  The energies typically decay more slowly in the high temperature limit.
  If the Casimir force decays as $d^{-n}$ at zero temperature, 
then we typically find that the force decays as $r^{-n+1}$ at high temperatures.  

We can estimate typical distance scales by considering the dominant 
resonances of the medium, and the thermal wavelength.
  So the Casimir force is typically then important at distances below the
 transition wavelength which is usually around $1\mu m$.    

\item The typical energy scale can also be estimated.
  If we consider an atom with a static polarizability of 
$\comment{polarizability}$, at a distance of $1\mu m$, we find the Casimir-Polder potential is \ldots. 

For macroscopic bodies we can compare the Casimir energy for perfect 
metal conductors at a distance $d$ to the energy stored in the capacitance.
  Note that we compute an energy per unit area.  
\comment{Milonni makes the comparison to capacitance, \cite{Milonni1994}}

The Casimir energy per unit area is $\frac{\hbar c\pi^2}{240 d^3}$, while the voltage for a parallel plate capacitor is.
\begin{shaded}
\begin{itemize}
\item Capacitance is $V=Q/C$.  
\item $C = \epsilon_0 A/d$.  
\item Energy is $\frac{1}{2}CV^2=\frac{Q^2}{2C} $
\end{itemize}
\end{shaded}

\end{itemize}


\begin{itemize}
%\item Cite Casimir/Casimir-Polder and Lifshitz.
\item Casimir vs Van Der Waals vs London Forces

The Casimir force is intimately related to the van der Waals forces between molecules.
Van der Waals forces are usually ascribed to the dipole fluctuations of the 
neutral atoms.
In the limit where the atoms are far apart retardation becomes important and 
the force changes decays more quickly as $r^{-7}$, as first found by Casimir and Polder.  
In this limit the forces are sometimes referred to as London dispersion forces.
Normally Casimir forces consider the 

Throughout this thesis we shall use Casimir forces to refer to the forces between
macroscopic bodies, and Casimir-Polder forces to refer to the forces between 
atoms and macroscopic bodies.  Van der Waals forces are also sometimes referred to as the 
short-distance limit of the Casimir force, where retardation can be ignored.      

\item Renormalization

As QED calculation, the Casimir energy is formally divergent and must be renormalized. 
 In our case, we typically energy differences when one of the bodies is 
removed to spatial infinity.
  In some cases, like spheres we must consider renormalization more carefully.
  In those cases, the formally infinite parameters we get renormalize the 
physical parameters of the model~\cite{Milton2001}.  

\item Non-additivity of forces:  Reference for this?  What is the typical scale of the correction for non-additivity?

\item Search for repulsive forces

\begin{itemize}
\item Casimir force important for stiction.
\item Attracts atoms, sets lower bound for how close you can get particles together.
\item Search for repulsive forces as possible trapping (Motivation for this?)
\item Can be found for magnetic media (but typically small).
  Metamaterials exhibit this for small range of frequencies.
  But Casimir broadband, and dielectric contribution ends up dominating.
  (\comment{ Cite Milonni on metametarials}.
Sufficiently anistropic dielectric media (how anisotropic? \comment{Cite Milton})
 $\epsilon_1<\epsilon_3<\epsilon_2$ over a broad enough range of frequencies 
\comment{Cite Lifshitz of liquid helium.
  Cite experiments, and note odd fluids.}.
   Geometries dependence (\comment{Cite reid paper on needle above hole}).
\end{itemize}
\item Dynamical Casimir effect/Unruh Effect?
\begin{itemize}
  \item Accelerating plate creates photons.  
\end{itemize}
\end{itemize}

% \subsection{Worst prediction in physics}

% Let us briefly note that the Casimir energy is sometimes related to the cosmological constant in general relativity.
%   ``one of the worst predictions in physics.''  
% In Einstein's General Theory of Relativity, energy is coupled to the metric of spacetime.
%   The cosmological constant acts to drive accelerating expansion~\cite{Carroll2004}.
%   Note that the cosmological constant can be provided by the vacuum energy of the quantum fields of matter on spacetime.
%   Note however that it is the energy itself that shows up, not energy differences. 
% If we attempt to identify the vacuum energy with the cosmological constant, we estimate the energy from 
% \begin{equation}
% E \sim \int_0^\Lambda dk\,k^2 (\hbar c k),
% \end{equation}
% where $\Lambda$ is a high-frequency cutoff, which denotes the short wavelength
%  below which our effective description of the world in terms of quantum 
% field theory breaks down.
%   If we pick $\Lambda=[c^3/(\hbar G)]^{1/3}$ to be the Plank wavelength, 
% where $G$ is Newton's gravitation constant, at which point quantum gravity
%  effects are expected to become important, then we predict, 
% \begin{equation}
% E \sim \hbar c \Lambda^4 
% \end{equation}
% Unfortunately, this is around $10^{120}$ times too large relative to the 
% measured value.  
% This discrepency is known as the cosmological constant problem.  
% Fortunately, we will always be considering experiments on a more terrestrial scale,
%  where it makes much more physical sense to only consider energy differences.
%  In which case, these issues do not arise, and experiments and theory are in much closer accord.    

\section{Physical relevance and experimental relevance}

\section{Experiments}
%Lamoreaux
\subsection{Casimir}
While Casimir predicted an attractive force between neutral metal bodies in 1948,
it was only precisely measured in 1997 by Lamoreaux~\cite{Lamoreaux1997}
Lamoreaux's work spurred a large amount of experimental and theoretical work.  
This experiment measured the Casimir force between a sphere above a metal plate,
and measured the force by \comment{\ldots}.  
This landmark experiment was closely followed by measurement by Mohideen\etal\cite{Mohideen1998}
using an atomic force microscope to measure the force in a sphere-plate geometry.  

The sphere-plate geometry has the experimental advantage of removing the need to carefully
align the metal plates.  Given the strength of the Casimir force it is hard to 
keep the plates exactly parallel.  

Chan\etal measured the Casimir force in a nanoelectromechanical (NEMS) system~\cite{Chan2001},
where the Casimir force modifies the frequency of a torsional oscillator suspended
above a plate.  

Bressi\etal measured the Casimir force between parallel plates~\cite{Bressi2002}.  

 Casimir force stiction in MEMS.  device failure, minimum feature size.~\cite{Buks2001}



\subsection{Casimir-Polder}
%Skip over names?  
\begin{itemize}
\item First modern attempt to measure the Casimir-Polder force by Sukenik\etal\cite{Sukenik1993}.
  Hot beam of atoms passing through cavity to detect Casimir-Polder.  
  Measured small phase-shift.  Not definitive. 
\item Cronin \cite{Perreault2005,Lonij2009}  Atom interferometry experiments for atoms near gratings.
  Pass hot atomic beam through gratings.  Useful as precision tests of theory.  
  Typically find first-principles theory based on perturbation not working as well,
  as effective descriptions which rely on experimental computations.  
\item Measured Casimir-Polder shift in Bose-condensed gas of ultra-cold atoms.
  Harber\etal\cite{Harber2005} measured the shift in the oscillation frequency
  of a Bose-Einstein condensate (BEC).  BEC used for enhanced signal, well localized.
  Casimir force not relevant to coherent aspects of BEC.
  Obrecht\etal\cite{Obrecht2007}Also measured force in the hard to measure thermal regime.  
  Can easily vary distance of atom from surface from $1\mu m$ up to $10\mu m$ from surface.
  Relate to lifshitz formula and cross-over.    
\item Antezza - chapters in Dalvit.  
\item Kimble atoms near toroidal resonators.
  Strongly Couple single atoms to optical field at single photon level.  
  Apply to toroidal resonators, and dielectric waveguides.  
  \cite{Alton2011}.
  Atoms above 1D Microcavity \cite{Hung2013}
\item Atom-chips  Schmiedmeyer\cite{Folman2000,Schneider2003}(?)  
  Trap atoms near surfaces using combination of lasers and magnetic fields from
  wires in surface.  Trap atoms very close to surface.  Relevant to idea of 
  minaturizing tabletop experiments.  
  Not explicitly using Casimir-Polder forces?  
\item Atom-chips and atom-waveguids are attempts towards building scalable 
architecture for quantum computer using atoms.
Couple atoms to dielectric waveguides, and get stronger coupling (thus faster operation)
by moving atoms more closely to couple to evanescent field.  
\end{itemize}
Challenges:  Need to characterize surface over broad frequency range.
Need methods with converge well for all geometries.  

\subsection{Current experimental directions}
\begin{itemize}
\item Thermal casimir force
Sushkov\cite{Sushkov2011}.
Fight in literature over exact model used to describe metals at finite temperature.
Drude vs plasma model.  
 Lamoreaux favors Drude model, Capasso/Mohideen favours plasma model.
\item Repulsion (Cite Capasso experiment with bromobenzene).  Possibility of stable trapping
  if one could balance repulsion/attraction.  
 Metamaterials for repulsion.  (Can't work Rosa 2010/2011 since dielectric
  background dominates~\cite{Rosa2008})
  (Cite Vogel and Welsch chapter for Green function methods.  They cite 
  Feinberg and Sucher (1970) on pg 360 for repulsion)
\item Modifications to gravity on $1\mu m$ or $1mm$ scale.  Cite Lamoreaux 2000 Paper.  Gervaci?
Yukawa type forces.  (Carrol's textbook on string theory dilaton.  
 Subtract off Casimir force background.
  Tino group.
  Use Casimir shield with fairly thick gold to have same Casimir force, and thne vary the medium behind it.
  Longer range gravity should lead to 
Requires very careful measurements, on top of carefully extracting Casimir force.   
\end{itemize}
All of these require one handle material properties reliably, and with arbitrary geometries,
and possible anisotropies.  Typically working in long-wavelength approximation which neglects strong
forces binding bodies together.    

% \subsection{Chemistry/Helium/Geckos?}

% \begin{itemize}
% \item Geckos use the Casimir force \cite{Autumn2002}.
% \item Military applications to mimic at human scale. Cite 2015 paper.   
% \end{itemize}

\section{Other Computational methods}

\begin{itemize}
\item Experiments spurred development of theory.  
\item Prior methods relied on mode-function expansion of fields, which is only possible
for simple geometries.  
\item Theory must account for material properties.  Surface roughness.  
\item Need general methods, tend to result in numerical integrals.
\item Crudest method available is the proximity force approximation.  
\item Analytical theory based on green function methods.  (Russian school, McLachlan, Schwinger)
\item Scattering approach.  
\item Worldline method.  
\end{itemize}

\subsection{Proximity Force Approximation}

\begin{itemize}
\item The proximity force approximation (PFA) is an uncontrolled approximation to
the Casimir force between generally shaped objects.  
The PFA computes the Casimir force can be found by treating each infinitesimal segment
of the surfaces as if they were planes, and integrating the force between surface patches.
\item Find first use?  Lamoreaux mentions usage.  Derjaguin?\cite{Derjaguin1956}
\item Note problem with non-additivity. The PFA explicitly assumes that the force
can be found by adding up the pair-wise contributions from each surface patch.  
However, the Casimir force is a global phenomenon, and the total Casimir force
Useful if very limited curvature, or effectively approximate geometry as planar.  
\item Only attractive.  (real electromagnetic casimir forces have possibility of 
being repulsive, even if hard to realize in general.)
\item ever extended to include material properties?
\item Despite these limitations, the PFA is relatively straightforward to implement,
and functions as an order of magnitude estimate for the Casimir force.  In the limit
of vanishing curvature, the PFA converges to the correct Casimir force.  
\end{itemize}

\subsection{Green function methods}

\begin{itemize}
\item Cite Lifshitz, Dzyaloshinski, Abrisokov.
\item Cite McLachlan~\cite{McLachlan1963}
\item Cite Schwinger~\cite{Schwinger1978, Milton1978}  Scalar green functions.  
\item Green tensor methods
\item Cite Barton
\item Cite Philbin(?)
\item Cite Vogel and Welsch
\item Green function equation
\begin{equation}
  \nabla\times\nabla\times G(\vect{x},\vect{x'}) + \frac{\omega^2}{c^2}\epsilon G(\vect{x},\vect{x'})  = \delta_{ij}\delta(\vect{x-x'})
\end{equation}
\item Energy expressions.  
\begin{equation}
  E_{CP} = \int_0^\infty ds \alpha_{ij}(is,\vect{x})G^{(S)}_{ji}(is,\vect{x},\vect{x})
\end{equation}
Focus on scattering part - renormalization.  $\alpha$ is atomic polarizability tensor,
 $G$ is electromagnetic Green tensor.  
 Evaluate at imaginary frequency. Note fully quantum theory, can account for atomic transitions.
\item Definition of $\alpha$,$G$ in terms of eigenfunctions, field modes.  
\begin{equation}
  \alpha_{ij} = \sum_n\sum_m \frac{\langle E_n | d_i|E_m\rangle \langle E_m| d_j|E_n\rangle}{E_m-E_n}
\end{equation}
\item Calculations correspond to given order of perturbation theory in QED hamiltonian.  
\item Finite temperature.  
\end{itemize}

\subsection{Scattering Approach}

Thus far, the scattering approach is the only general method of computing 
electromagnetic Casimir forces in general media.  
\begin{itemize}
\item EM field depends on scattered field.  Encoded in reflection matrices
which are related to how EM field scatters from a given surface.  
\item Developed as analytical technique, and developed into general purpose numerical technique.
Numerous authors contributed over decades.
\end{itemize}

\begin{itemize}
  \item Developed as general method, for scattering between basis modes.  
    Initially used with analytical mode expansions.  Can also use finite element basis,
    much more adapted to arbitrary bodies.  
\item Physical picture is fluctuating currents on bodies.  Interact via EM field.
  Can use Green function as if dielectric suffused all space.  
SIE equations and Green's theorem relate surface integrals.  
\end{itemize}


\begin{itemize}
\item Cite Balian and Duplantier \cite{Balian1977, Balian1978}.
\item Cite Lambrecht and French collaborators
  \cite{Lambrecht2006, MaiaNeto2008,Canaguier-Durand2012}
\item Cite Milton
\item Physical Picture based on generalized Green theorem from 
  SIE~\comment{Stratton}\cite{Stratton1941}.
\item Comment on Emig/Buscher showing you can use homogenous green function.
  Relation to Green's theorem.
\item Cite Emig,Jaffe,  and others for initial analytical techniques.  Relies on Green theorem.
\cite{Emig2004, Emig2007, Rahi2009}
\cite{Kenneth2006}
  Note use of existent analytical methods and similarities to existent 
  numerical FTDT techniques on earlier papers.  
  \cite{Rodriguez2007,Rodriguez2007a, Rodriguez2009}
\item Cite Johnson/Reid for numerical progress.\cite{Reid2009,Reid2011, Reid2013}.
\item Note success, applicability.  \comment{Cite experimental tylenol pill paper}
\end{itemize}

\subsection{Worldlines}

The worldline method is an alternative method for computing Casimir energies.
  The worldline method was initially developed as an alternative method for 
carrying out QFT calculations in terms of particle 
mechanics~\cite{McKeon1993, Strassler1992,Schubert2001}.
  In particular we can compute compute quantum effective actions in terms of
 the suitable dynamics of a quantum particle.
\comment{Other references - was one contemporaneous with Strassler? Bern-Kosower}

The basic insight is that for one loop effective actions, 
we can recast a field path integral calculation in terms of the particle path
 integral for particles travelling in closed space-time loops.
  This is quite similar to the scalar electrodynamics Feynman explored
 in his Ph.D thesis~\cite{Feynman1942, Brown2005}
\comment{Seems to actually be his 1950 QED paper RE Schubert2001}.
  Higher order loop calculations can also be carried out with more particles, 
and gauge fields can also be treated~\cite{Schubert2001}.
  For example, the worldline method has been used to compute relativistic
 field effects for QED such as the Lamb shift~\cite{Schmidt1995}.
  It has also been used as a numerical algorithm\cite{Mazur2014}.

The worldline method is heavily based on Feynman's path integral method~\cite{Feynman1948,Feynman1965}.

Our primary interest in the worldline method is for computing Casimir energies, which can be cast as effective actions.
  The worldline was first used for the Casimir energy by Gies\etal~\cite{Gies2003,Gies2006, Gies2006a}.
  While the initial work focused on the zero temperature limit, 
the worldline method has also been extended to finite temperatures~\cite{Klingmueller2008}.
  This has also been used to study the torsion of inclined planes~\cite{Weber2009},
 and planes and spheres and planes and cylinders~\cite{Weber2010, Weber2010a}.  

We will briefly introduce the method here, and discuss the method at 
length in Ch.~\ref{ch:scalar_worldlines}.  
Let the action for the scalar field be given by 
\begin{equation}
  S = \int_0^T dt \int d^3x \left[ (\partial_t\phi)^2-(\nabla\phi)^2-V(\vect{x},t)\phi^2\right],
\end{equation}
where $V(\vect{x},t)$ defines the surfaces of the objects we wish to compute
 the Casimir energy between.
  The potential is typically chosen to be $V(\vect{x},t)=\lambda\delta[\sigma(\vect{x})]$,
 where $\sigma(\vect{x})=0$ defines the surfaces.
  In the limit $\lambda\rightarrow\infty$ the potential enforces Dirichlet boundary conditions 
  on the surfaces of the bodies.  In certain geometries this recovers electromagnetic 
results assuming the surfaces are idealized perfect conductors.  
This corresponds to assuming the surfaces are idealized perfect conductors.  

We can compute the partition function for the field by Wick rotating to
 imaginary time (or finite temperature), where the partition function is now given by 
\begin{equation}
  Z = \int D\phi \exp\left\{-\int_0^T dt \int d^3x 
    \left[(\partial_t\phi)^2+(\nabla\phi)^2+V(\vect{x},t)\phi^2\right]\right\},
\end{equation}
The Gaussian integration over fields can be carried out as a functional determinant.
  In order to compute the energy we need the logarithm of the partition function,
 and various derivatives of it.
  The end result of these manipulations is that the renormalized energy can be written as 
\begin{equation}
E_{\text{scalar}} = \frac{\hbar c}{(2\pi)^{D/2}}\int_0^\infty \frac{d\cT}{\cT^{1+D/2}}
 \int d\vect{x} \dlangle e^{-\cT\langle V\rangle} -1\drangle,
\end{equation}
where $\cT$ is the loop ``time'' and governs the extent of the loops,
 $\vect{x}_0$ is the loop starting point, $\dlangle \cdots\drangle$ denotes 
an ensemble average over closed Gaussian random walks, 
and $\langle\cdots\rangle$ denotes the average of a quantity around the loop.  

\begin{figure}
\center
\includegraphics[width=10cm]{fig/intro/hit_strong_coupling}
\caption{Upper loop touches both objects and will contribute to Casimir energy.  Lower loop only touches one body, and does not contribute to Casimir energy.}
\end{figure}


\begin{itemize}
%\item Cite Schubert~\cite{Schubert2001}, Strassler~\cite{Strassler1992} on general worldline
% \begin{itemize}
% %\item Summarize Strassler.  Can compute QFT effects from worldline path integrals.  
% % \item One loop effective actions can be described as single-particle worldline path integrals.  Can apply for higher order loops, and gauge fields, Cite Schubert.    
% % \item Cite QED at one loop order paper.  Get same results.  
% % \item Note similarity to Schwinger's trick for handling loop integrals in QED.  T is Schwinger's proper time.  
% \end{itemize}
%\item Cite QED Worldline paper on numerics?
\item Cite Schaden applying to pistons\cite{Schaden2009}
\item Figure showing loops.  
\item Advantages
  \begin{itemize}
  \item Algorithm is geometry independent, and no spatial grid.
  \item parallelizable.  Computation time scales as one /resources.  
  \end{itemize}

\item Shortcomings
\begin{itemize}
  \item A scalar, not vector electromagnetism.
  \item Idealized boundary conditions.  
\end{itemize}
  
\end{itemize}


Thus far the worldline method has only been developed for scalar fields, 
without direct application to electromagnetic Casimir problems, 
other than for some speculation~\cite{Aehlig2011}.
  Currently, the potentials reflect the imposition of Dirichlet boundary 
conditions on the surfaces of the bodies, rather than a physical dielectric.
   In our work we will show how to incorporate the dielectric explicitly, 
and show how in simple geometries a new version of the scalar theory applies
 to electromagnetic problems.  

Quantization of the electromagnetic field inside dielectric has been considered
 by a number of authors.  
Some care is required to handle dispersion, since the Kramers-Kr\"onig relations
 imply this also requires dissipation \comment{citation?}.  
This is typically handled by coupling the electromagnetic field to an idealized
 medium, and coupling the medium to a bath of oscillators that models
 dissipation~\cite{Huttner1992,Dung1998}.  

Bechler has carried out path integral quantization for a harmonic medium 
including dispersion, and shown agreement with previous results in terms 
of noise operators~\cite{Bechler1999}.  
The primary results were the form of the propagator, 
rather than computations exploiting the propagator.
  There has been some work attempting to develop path integral quantization of
 the field inside dielectric neglecting dispersion~\cite{Bordag1998}.
  Unfortunately, these results are hard to interpret given that the non-physical
 degrees of freedom for the field do not cleanly decouple, as opposed to the 
usual situation in free space QED.
  The primary focus here was to explore the divergence structure of the theory
 via the Heat Kernel expansion, which corresponds to the small time expansion
 of a worldline path integral.
  We choose to avoid this issue by focusing on improved scalar models, 
that also correspond to the physical degrees of freedom for the field in certain geometries.  

Others have made models in the static limit by ignoring the fluctuations in the field,
and focusing purely on the electrostatic energy.
 Similar functional determinants to worldline methods are found,
 and are evaluated explicitly using a spacial grid~\cite{Pasquali2008}.  


\section{Continuous Quantum Measurements}

Describe open quantum systems, and include information from continuous measurements. 
\begin{itemize}
  \item 
\end{itemize}


\subsection{Quantum Trajectories}

\begin{itemize}
\item Cite Carmichael Rice JOSA paper~\cite{Carmichael1989}
\item Carmichael 1991 ~\cite{Carmichael1991}
\item Open Systems approach to Quantum Optics\cite{Carmichael1993}
\begin{itemize}
\item Motivated by photodetection, and modelling experiments developed a new approach to open systems.  
\item Sample trajectories then correspond to actually results.
\item Naturally fits Bayesian framework for interpretation of quantum state.  
\end{itemize}
\item Cite Marte, Zoller, Parkins, Gardiner (MCWF)  \cite{Dalibard1992,Dum1992,Gardiner1992}

\item Cite Holland ~\cite{Holland1996}, Meystre~\cite{Greenwood1997}.
  Applied to position measurements of atoms by detecting photons.
  Detection of photons localizes atoms.  
\item Control Theory.~\cite{Wiseman1993}  Cite Wiseman book
\begin{itemize}
  \item Continuously monitoring system to implement closed-loop feedback control.  
\end{itemize}
\item Quantum Chaos
\begin{itemize}
  \item Idea of exploring quantum-classical transition.
  Strong measurement is more classical.
  Can extract Lyupanov exponents for diverging trajectories.
  \cite{Bhattacharya2000,Habib2002,Habib2006}
  \cite{Scott2001}
  \item Describe 
\end{itemize}
\item Advantages:
\begin{itemize}
  \item Computationally efficient as simulating wave functions.
  Take ensemble average at the end to get density matrix.  
  \item Natural form for feedback control and reconstructing trajectory.  
\end{itemize}
\item Comment: Relationship of measurement with state and process tomography?  Any?  

\item Warshawski and Wiseman.
  Can describe additional uncertainty by including classical Bayesian probabilities for each indistinguishable trajectory.
  Note thesis of J. Thorne describing model for EMCCD camera.
  Given number for each pixel have probabilities for each number of photons.
  Must then incoherently average over all possible detection histories consistent with measured record.
  \cite{Warszawski2002,Warszawski2003a,Warszawski2003b}
\item Also cover results for generalized measurement functions and 
somewhat surprising notion that particle reflects from a sufficiently
strong quantum measurement.
\end{itemize}

\section{Thesis outline}

\begin{itemize}
\item Background for path integrals, scalar worldlines, and EM field quantization.
\item Cover 
\item Analytical methods and computations in simple geometries.
\item General method and numerical results.
\item Shift to quantum trajectories.  
\end{itemize}



%%% Local Variables: 
%%% mode: latex
%%% TeX-master: "thesis_master"
%%% End: 
