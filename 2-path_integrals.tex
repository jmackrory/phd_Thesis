\chapter{Path Integrals and Feynman-Kac formulae}

\label{ch:feynman_kac}

A path integral is a sum over an infinity of possible paths a system might explore.


    In honor of the pioneering work by Feynman and Kac, the fact that path integrals
    act as the solutions to diffusion/Schrodinger equations is referred to as the Feynman-Kac 
    formula.  In our parlance, this connection is also used to refer to the exact analytical
    result for a given path integral.  
    This chapter is devoted to deriving the Feynman-Kac formula, and then solving
    it for some simple geometries.  These results will be used in our analytical and numerical
    calculations of the Casimir force.  

\section{Derivation of Feynman-Kac formula }

In this section we will derive the path integral as the solution to a diffusion equation,
using techniques from quantum mechanics.  Our derivation will stay close in spirit to the 
one found in Sakurai~\cite{Sakurai1994}.  We will extend the usual quantum derivation to
including a source term, as is more common in statistics.   More formal derivations are available from mathematical~\cite{Cartier2004},
and probabilistic perspectives~\cite{Karatzas1991, Durrett1996}.  
An detailed discussion of the more formal probabilistic derivation is availabble in Steck Sec 17.9~\cite{SteckNotes} .

We aim to find a solution $f(x,t)$ to the driven diffusion equation given by 
\begin{equation}
  \partial_t f = \frac{1}{2}\nabla^2 f  - [V+\lambda]f +g,\label{eq:diffusion_equation}
\end{equation}
where the potential/killing rate is given by $V=V(\vect{x},t)$, and the source term is $g=g(\vect{x},t)$.  
    In this form $f$ corresponds to the probability distribution for a diffusing particle with a source
    of particles $g$, with spatially dependent killing rate $V$.  If we take $\tau\rightarrow -it$,
    we recover the Schr\"odinger equation. 
    We can use techniques from quantum mechanics, and introduce a Hilbert space with operators.
    The differential equation can be written in operator form using 
    \begin{gather}
      \langle \vect{x}|\op{x}_i|f\rangle = x_if(\vect{x})\\
      \langle \vect{x}|\op{p}_i|f\rangle = -i\partial_if(\vect{x}),
    \end{gather}
    The position and momentum operators have commutation relations
    \begin{gather}
      [\op{x}_i,\op{p}_j]=i\delta_{ij},
    \end{gather}
    and overlap 
    \begin{equation}
      \langle \vect{x}|\vect{p}\rangle = e^{i\vect{x\cdot p}}.
    \end{equation}
    We also need the position and momentum resolutions of the identity
    \begin{gather}
      I_x = \int d\vect{x}|\vect{x}\rangle \langle \vect{x}|, \quad
      I_p = \int \frac{d\vect{p}}{(2\pi)^D}|\vect{p}\rangle \langle \vect{p}|.
    \end{gather}
The diffusion equation~(\ref{eq:diffusion_equation}) can be written in operator form
\begin{equation}
  \langle \vect{x}|\partial_t |f(t)\rangle = -\langle \vect{x}|
  \left[\frac{1}{2}\op{\vect{p}}^2 + V(\op{\vect{x}},t)+\lambda\right]|f(t)\rangle +\langle \vect{x}|g(t)\rangle.
\end{equation}
We can solve this by introducing the evolution operator,
\begin{equation}
  U(t) = \text{T}\exp\left\{-\int_0^t ds\left[\frac{1}{2}\op{\vect{p}}^2 + V(\op{\vect{x}},s)+\lambda\right]\right\},
\end{equation}
where $\text{T}$ is the time-ordering operator.  \comment{Dyson series? - re QFT Derivation of S-matrix?}
We can then use the operator analogue of an integrating factor method, by transforming 
the vectors to the Heisenberg picture, $|f\rangle \rightarrow |\tilde{f}\rangle=U^{-1}(t)|f\rangle$.  
The transformed vectors obey
\begin{equation}
  \partial_t|\tilde{f}\rangle = U^{-1}(t)|\tilde{g}\rangle.
\end{equation}
This equation can be formally integrated with respect to time,
% \begin{equation}
%   |\tilde{f}(t)\rangle-|\tilde{f}(0)\rangle = \int_0^t ds |\tilde{g}(s)\rangle.
% \end{equation}
and after we transform back to the original vectors we find
\begin{equation}
  |f(t)\rangle = U(t)|f(0)\rangle + U(t) \int_0^t ds U^{-1}(s)|g(s)\rangle,
\end{equation}
where we used $|\tilde{g}(t)\rangle = U^{-1}(t)|g(t)\rangle$.  
If we assume we can combine  the operators, then 
\begin{equation}
  f(x_f,t) = \langle \vect{x}_f|U(t)|f(0)\rangle + \int_0^t ds \langle \vect{x}_f|U(t-s)|g(s)\rangle.
\end{equation}

\subsection{Evaluating the matrix element}

We can now evaluate the matrix elements for each piece.  In particular, 
\begin{equation}
  M = \langle \vect{x}_f|U(t)|f\rangle,
\end{equation}
where $U$ is the time-ordered evolution operator (so time increases towards the left.)
\begin{equation}
  U(t) = T\exp\left[-\int_0^t du H(u) \right] = \prod_{n=1}^N e^{-\Delta T \op{H}(n\Delta T)}
\end{equation}
If we insert position/momentum identities between each of these, we have
\begin{align}
  M_n &= \langle \vect{x}_f| e^{-\Delta T \op{H}(t_N)}|f\rangle%\\
%   &= \int \frac{d\vect{p}_N}{(2\pi)^{D/2}}\langle \vect{x}_N|e^{-\Delta T \op{H}(t_N)} |\vect{p}_N\rangle \langle \vect{p}_N|
% \prod_{j=1}^{N-1} e^{-\Delta T \op{H}(t_j)}|f\rangle\\
  = \int \prod_{k=0}^{N-1}\frac{d\vect{x}_{k}d\vect{p}_k}{(2\pi)^{D/2}}
  \prod_{j=1}^{N}\left[\langle \vect{x}_{k+1}| e^{-\Delta T \op{H}(t_k)}|\vect{p}_k\rangle
    \langle \vect{p}_k| \vect{x}_{k}\rangle \right]
  \langle \vect{x}_0| f\rangle
\end{align}
\comment{Do we need to consider how the ket's evolve?}
We use the Baker-Campbell-Hausdorff theorem to split the exponential operator into a position
and momentum pieces
\begin{equation}
  e^{-\Delta T [\op{p}^2+V(\op{x})]} = e^{-\Delta T V(\op{x})}e^{-\Delta T \op{p}^2} +\order(\Delta T^2).
\end{equation}
The postion and momentum operators then acquire the eigenvalues from operating to the left/right respectively.
After carrying out the Gaussian momentum integrals, the matrix element is 
\begin{align}
  M_n %  &= \int \prod_{k=0}^{N-1}\frac{d\vect{x}_{k}d\vect{p}_k}{(2\pi)^{D}}
  % \prod_{j=1}^{N} e^{-\Delta T\left[\vect{p}_k^2/2 + V(\vect{x}_k,t_k)\right]+i\vect{p}_k\cdot(\vect{x}_{k+1}-\vect{x}_k)}
  % f(\vect{x}_0,0)\\
&= \int \prod_{k=0}^{N-1}\frac{d\vect{x}_{k}}{(2\pi\Delta T)^{D/2}}
  \prod_{j=1}^{N} e^{-(\vect{x}_{k+1}-\vect{x}_k)^2/(2\Delta T)-\Delta T V(\vect{x}_k,t_k)}f(\vect{x}_0,0)
\end{align}
This is the fairly traditional form of the euclidean path integral.  We can transform it 
to look like the mathematical version, stressing the connection to Brownian motion
by changing integration variables.  
We now define the vector Wiener increments $\Delta \vect{W}_k = \vect{x}_{N-k-1}-\vect{x}_{N-k}$ (note that this
is backwards labelling from the usual convention).  The Jacobian determinant for this
transformation is unity. Then we can write the positions in terms of the Brownian motion
$\sum_{k=0}^{j}\Delta \vect{W}_k = \vect{x}_{N-j}-\vect{x}_N$.  Since we are treating $\vect{x}_N$ as a 
constant position (where we will be evaluating the solution), we should reference our points to that point.  
We can use
\begin{equation}
  \sum_{k=0}^{j} \Delta \vect{W}_k = \vect{x}_{N-j}-\vect{x}_N\rightarrow \vect{x}_j = \vect{x}_N+\sum_{k=0}^{N-j} \Delta \vect{W}_k.
\end{equation}
In terms of the Wiener increments the path integral is
\begin{align}
  M_n  &= \int \prod_{k=0}^{N-1}\frac{d\Delta \vect{W}_{k}}{(2\pi\Delta T)^{D/2}}
  \prod_{j=1}^{N} e^{-(\Delta \vect{W}_{k})^2/(2\Delta T)-\Delta T V(\vect{x}_N+\sum_{j=0}^{N-k}\Delta \vect{W}_j,t_k)}
  f(\vect{x}_N+\sum_{j=0}^{N-j}\Delta \vect{W}_j,0)\\
&= \int \prod_{k=0}^{N-1}\frac{d\Delta \vect{W}_{k}}{(2\pi\Delta T)^{D/2}}
  \prod_{j=1}^{N} e^{-(\Delta \vect{W}_{k})^2/(2\Delta T)-\Delta T V(\vect{x}_N+\vect{W}_{N-k},t_k)}
  f(\vect{x}_N+\vect{W}_{N},0)
\end{align}
where in the second equality we defined the Wiener process as $\vect{W}_k=\sum_{j=0}^{k} \Delta \vect{W_j}$, then we can write 
If we pass over to continuum language, where the Riemann sum becomes an integral, and $\vect{W}_k=\vect{W}(t_k)$,
(we will also define $\vect{x}=\vect{x}_N$
 is a continuous process we can write
\begin{align}
  M_n  &= \biggdlangle e^{-\int\limits_0^t du V(\vect{x}+\vect{W}(t-u),u)} f(\vect{x}+\vect{W}(t),0)\biggdrangle
\end{align}
The same style of reasoning can be used for both pieces.  \comment{Note however change in time limits on $g$
term.  Also do the time arguments of $g$ work out?}


\subsection{Path Integral without Source}
\begin{enumerate}
  \item {Fokker-Planck Equation on Hilbert space.  }
  \item {Split operators into $N$ steps.  Baker-Campbell-Hausdorff}
  %   Then, do the usual path integral tricks:
  %   \begin{align}
  %     \langle x_N|f(t)\rangle =& \langle x_N|\exp\left[-t\frac{\op{p}^2}{2}-tV(\op{x})\right]|f\rangle \\
  %     % =& \langle x_N|\prod_{k=1}^N\exp[- t/N(\frac{\op{p}^2}{2}+V(\op{x})]|f\rangle \\
  %     % =& \langle x_N|\exp[- t/N(\frac{\op{p}^2}{2}+V(\op{x})] \ldots \exp[- t/N(\frac{\op{p}^2}{2}+V(\op{x})]|f\rangle \\
  %     % =& \int \prod_{k=0}^{N-1}dx_k \langle x_N|\exp[- \Delta t(\frac{\op{p}^2}{2}+V(\op{x})]|x_{N-1}\rangle\langle x_{N-1}| \ldots \langle x_1|\exp[- \Delta t(\frac{\op{p}^2}{2}+V(\op{x})]|x_0\rangle\langle x_0|f\rangle \\
  %     % =& \int \prod_{k=1}^N\frac{dx_{k-1}dp_k}{(2\pi)}\prod_{j=1}^N e^{-\frac{\Delta T}{2}p_j^2 +ip_j(x_j-x_{j-1}) - V(x_j)\Delta T}\langle x_0|f\rangle \\
  %     =& \int \prod_{k=0}^{N-1}\frac{dx_k}{\sqrt{2\pi \Delta t}}e^{-\frac{(x_{k+1}-x_k)^2}{2\Delta t} - \Delta t V(x_k)}f(x_0)
  %   \end{align}
  %   Now somehow transform this to something like $f(x_n + \sum_{j}dW_j)$ to get the same form as Dan?
  %   Transform to integrating over the increments $dW_j = x_{j+1}-x_j$.
  %   Then $\sum_{j=0}^{N-1} dW_j = x_N - x_0$.
  %   Well, just quickly solve, and you see that $x_0 = x_N - \sum_{j=0}^{N-1}dW_j$.
  %   Similarly, $x_k = x_N - \sum_{j=k}^{N-1}dW_j$.  

  % \item {Continuum limit.  Note source of walks}
  %   \begin{align}
  %     f(x_N,t)=& \int \prod_{k=0}^{N-1}\frac{dx_k}{\sqrt{2\pi \Delta t}} e^{-\frac{(x_{k+1}-x_k)^2}{2\Delta t} - \Delta t V(x_k)}f(x_0)\\
  %     =& \int \prod_{k=0}^{N-1}\frac{d(dW_k)}{\sqrt{2\pi \Delta t}} e^{-\sum_k\frac{dW_{k+1}^2}{2\Delta t} - \sum_k\Delta t V(x_N-\sum_{j=k}^{N-1}dW_j)}f(x_N-\sum_{j=0}^{N-1}dW_j)\\
  %     =& \int \prod_{k=0}^{N-1}\frac{d(dW_k)}{\sqrt{2\pi \Delta t}} e^{-\sum_k\frac{dW_{k+1}^2}{2\Delta t} - \sum_k \Delta t V(x_N+\sum_{j=0}^{k}dW_j)}f(x_N+\sum_{j=0}^{N-1}dW_j),
  %   \end{align}
    or in continuous language:
    \begin{equation}
      f(x,t)= \dlangle e^{-\int_0^t dt'  V[x+W(t')]}f_0[x+W(t)]\drangle
    \end{equation}
  \item Note time-ordered product.
    Note that in the case of $V(x,t)$ the exponential is really the time ordered product of these things.
  \item $\delta$-pinning at $x=0$.  
    Then if we take $g(x) = \delta(x)$, then the Brownian walks will be restricted to return to the origin.
    Note that I think this definition also assumes that the brownian walks are starting from the origin.  
  \item Discuss notions of Brownian motion, Brownian bridges.  Specify in terms of increments,
    final positions.  Introduce measures like correlation?


% \subsection{Path Integral without Source}

% \begin{enumerate}
%   \item {Integrating factor method for ODE.  (Interaction picture?)}
%     Consider how to solve: 
%     \begin{equation}
%       \partial_t f = -\alpha(t) f + g(t)
%     \end{equation}
%     This can be solved via an integrating factor.  Let's introduce $h = e^{\int_0^t dt' \alpha(t')} f$.  then 
%     \begin{equation}
%       \partial_t h = e^{\int_0^t dt'\alpha(t')} [\partial_t f +\alpha(t)f(t) ] = e^{\int_0^t dt'\alpha(t')} g(t),
%     \end{equation}
%     where we used the differential equation for $f$.  This differential equation in $h$ has a solution
%     \begin{equation}
%       h(t) = h_0 + \int_0^t ds\, g(s) e^{\int_0^{s} du \alpha(u)}.  
%     \end{equation}
%     Then reverting the transformation to $h$ back to $f$, and using $f(t=0)=f_0=h_0$,  we have: 
%     \begin{align}
%       f(t) &= e^{-\int_0^t dt' \alpha(t')}h(t)  \\
% %      &= f_0e^{-\int_0^t dt' \alpha(t')} + \int_0^t ds\, g(s) e^{\int_0^{s} du \alpha(u)}e^{-\int_0^t dt' \alpha(t')}  \\
%       f(t)&= f_0e^{-\int_0^t dt' \alpha(t')} + \int_0^t ds\, g(s) e^{-\int_s^t du \alpha(u)}.
%     \end{align}
%     Now take $s \rightarrow t-v$, and $u \rightarrow t-u$.    
%     \begin{align}
%       f(t)&= f_0e^{-\int_0^t dt' \alpha(t')} + \int_0^t dv\, g(t-v) e^{-\int_{0}^v du \alpha(t-u)},
%     \end{align}

%   \item Integrating factor for PDE

%     Now imagine applying the same path integral procedure, but you're just careful to keep the operator ordering clear.  

%     So let's do the quantum mechanical approach to this.  We start from 
%     \begin{equation}
%       \partial_t f = \frac{1}{2}\partial_x^2 f - V f + g, 
%     \end{equation}
%     \todo{Sort out sign conventions for diffusion!}
%     this can be turned into an operator equation: 
%     \begin{equation}
%       \partial_t\langle x |f\rangle = -\langle x|\left[\frac{\op{p}^2}{2}+V(\op{x})\right]|f\rangle + \langle x|g\rangle,
%     \end{equation}
%     Then let us introduce
%     \begin{equation}
%       U(t) =  T \exp\left[-\int_0^t dt'\frac{\op{p}^2}{2} +V(\op{x})\right]
%       = \prod_{k=1}^N \exp\left[ -\Delta t\frac{p^2}{2}-\Delta tV(x_k)\right]
%     \end{equation}
%     (literally a path integral)?

%   \item Let us shift to an interaction picture using 
%     \begin{equation}
%       H_0 = \frac{\op{\vect{p}}^2}{2}+V(\op{x})
%     \end{equation}
%     with states transforming to $|\tilde{\psi}\rangle = e^{-H t}|\psi\rangle$. 
%     \todo{Time ordered product in evolution operator - check Sakurai}
%     The transformed states obey 
%     \begin{equation}
%       \partial_t|\tilde{\psi}\rangle = |\tilde{g}(t)\rangle = e^{-Ht}|g\rangle
%     \end{equation}
%     This can be formally integrated over time.  
%     \begin{equation}
%       |\tilde{\psi}(t)\rangle-|\tilde{\psi}(0)\rangle = \int_0^t ds e^{-Hs}|g\rangle
%     \end{equation}
%     Or 
%     \begin{equation}
%       |\psi(t)\rangle = e^{Ht}|\psi(0)\rangle +\int_0^t ds e^{-H(s-t)}|g(s)\rangle
%     \end{equation}
  \item Need careful time ordering.  
    For example,
    \begin{equation}
      \partial_t f = \frac{1}{2}\partial_x^2 f  - [V(x,t)+\lambda]f +g(x,t) ,
    \end{equation}
    has the solution
    \begin{equation}
      f(x,t) = \dlangle  f[x_0+x(t)] e^{-\lambda t - \int_0^t dt'\,V[x(t')]} + 
      \int_0^t ds\,g(x,t-s) e^{-\lambda s-\int_0^s du V(x,t-u)} \drangle 
    \end{equation}
    with initial condition $f(x,t=0)= f_0(x)$, and $\dlangle \cdots\drangle$ denotes the ensemble average over Brownian walks.
  \item Introduce double angle bracket notation, and relate to typically path integral notation.
  \item Also introduce path-average notation as time integral.
  \item Comment on subscripts denoting types of loops $\dlangle\cdots\drangle_{\vect{x}_0}$

\end{enumerate}


\subsection{Solution method}

\begin{enumerate}
  \item For actual solutions work in steady-state limit.  
  \item {Work with Laplace transform.  }
    Now let us consider steady state, for potentials $V(x,t) = V(x)$.
    In this case, we can drop the initial condition, and take $f(x,t=0)=0$, as the steady state is insensitive to the initial condition.
    In addition, we take $g(x,t)=g(x)$.   then as $t\rightarrow \infty$ we have 
    \begin{equation}
      f(x) = \dlangle \int_0^\infty ds\,g(x) e^{-\lambda s-\int_0^s du V[x(s)]} \drangle,\label{eq:path_int_solution}
    \end{equation}
    which satisfies 
    \begin{equation}
      0 = \frac{1}{2}\partial_x^2f(x) - (V+\lambda)f + g(x).  
    \end{equation}
  \item Take $g(x)=\delta(x)$.
  \item Solve associated diffusion equation.  (So works for separable geometries.  But foundational
    for approximations suited step-wise basis.)
  \item Identify value of $f(x=0)$ as path integral.  Different location choices will yield different
    functions.  
  \item Particularly interested in methods that could serve as approximations beyond PFA.  
    Interested in results that can serve as \emph{local} approximations exploiting planes.  
  \item From here on, this is mostly a collection of solutions to Fokker Plank equations with various potentials,
    notably steps, and sundry gradients of steps.
    The overall procedure is the same, just the potential changes.
    Later we will have need of these results in our analytical calculations.  The planar open
    path results are essential for numerical techniques.      
\end{enumerate}

\section{Sojourn Time and One Step Potential }

\begin{enumerate}
  \item {Take $V=\chi\Theta[x-d]$.}
  \item Quote FPE, with potential.  Note source term for closed loops.  
    We start with a single step, $V = \chi\Theta[x-d]$.  We need to solve the following Fokker-Planck equation 
    \begin{equation}
      \partial_t f = \frac{1}{2}\partial_x^2 f - (\chi\Theta(x-d)+\lambda)f + \delta(x).
    \end{equation}
    where $g$ will act as the source which pins the solutions to a particular point, $x_0$.
  \item Note using explicit delta function, rather than unified Fourier representation.  
    While it is possible to solve this using the Fourier representation of the delta function, 
    I found it more transparent to use an explicit delta function and explicitly handle the cases where the pole is in different places.  
    Cite Steck, and Hooghiemstra
  \item Quote Laplace-transformed solution, and appropriate PDE. 
    Now from Eq.~(\ref{eq:path_int_solution}) we have 
    \begin{equation}
      f(x) = \dlangle  \int_0^\infty ds\,\delta(x) e^{-\lambda s-\int_0^s du\,\chi\Theta[x(u)-d]} \drangle,
    \end{equation}
    is the solution to 
    \begin{equation}
      \partial_x^2 f = 2(\chi\Theta(x-d)+\lambda)f - 2\delta(x).  
    \end{equation}
  \item Quote general form of solution, note choosing bounded solution
    In general the solutions are of the form, 
    \begin{equation}
      f(x) = A e^{\kappa x} + B e^{-\kappa x},
    \end{equation}
    where we will fix $\kappa$ appropriately, and choose the bounded solution.
  \item Quote boundary conditions.  
    We also have to take care with the boundary conditions at surfaces.
    At the jump discontinuity at $x=d$ we have 
    \begin{equation}
      \partial_xf(d+\epsilon) - \partial_x f(d-\epsilon) = 0, \qquad f(d+\epsilon)-f(d-\epsilon) = 0.  
    \end{equation}
    For the boundary conditions at the delta function we have 
    \begin{equation}
      \partial_xf(\epsilon) -\partial_x f(-\epsilon) = -2 , \qquad f(d+\epsilon)-f(d-\epsilon) = 0,
    \end{equation}
    where both of these relations follow from integrating the PDE across the discontinuity.  
    
    \item Quote general form of solution for $d>0$.  
    Then for $d>0$ have 
    \begin{equation}
      f(x) =\left\{ 
        \begin{array}{lcr}  A e^{\sqrt{2\lambda} x} & \hspace{2cm} & x<0\\
          B e^{\sqrt{2\lambda}x} + Ce^{-\sqrt{2\lambda}x} & \hspace{2cm} & 0<x<d\\
          D e^{-\sqrt{2(\lambda+\chi)}x} & \hspace{2cm} & x>d
        \end{array}
      \right.
    \end{equation}
    where the coefficients are fixed by matching the boundary conditions together.
    This was done using Mathematica to speed up the tedious algebraic work.  
    \item{Note for $d>0$ need $A$}
    We are ultimately only interested in $f(x=0)$, which in this case means we just need to know $A$.  
    It turns out that 
    \begin{equation}
      A = \frac{1}{\sqrt{2\lambda}} - u\,e^{-2\sqrt{2\lambda}d},\end{equation}
    where
    \begin{equation}
      u = \frac{\sqrt{\lambda} -\sqrt{\lambda+\chi}}{\sqrt{\lambda} + \sqrt{\lambda+\chi}},
    \end{equation}
  \item Note reflection coefficient similarity.  
    plays a similar role to the $TE$ reflection coefficient.
    Considering we are solving a nearly identical differential equation this is not much of a surprise.  
  \item For $d<0$, can get similar result from $D$, if $\lambda \leftrightarrow \lambda+\chi$.
  \item Final result for all cases.  
    Ultimately, we find that 
    \begin{equation}
      \int_0^\infty dt e^{-\lambda t} \dlangle \frac{e^{-s \theta[x(t)-d]}}{\sqrt{2\pi t}}\drangle  
      = \frac{1}{\sqrt{2[\lambda+\chi\theta(d)]}}\left[1 - \sgn(d) u e^{-2\sqrt{2[\lambda+\chi\theta(d)]}|d|}\right],
      \label{eq:Feynman-Kac TE one step}
    \end{equation}
    where the ensemble average is over brownian bridges that satisfy $x(0)=x(T)=0$.
  \item Note normalization factor
    The factor of $\sqrt{2\pi T}$ is normalization for the use of the bridges.  
  \item For $d<0$, note we can use symmetry and substitutions. 
    \subsection{$d<0$}
    We can go through the same procedure for $d<0$.
    This extra effort will be necessary for the Casimir energy where we will have to apply these formulae over all space.    
    Then for $d>0$ have 
    \begin{equation}
      f(x) =\left\{ 
        \begin{array}{lcr}  A e^{\sqrt{2\lambda} x} & \hspace{2cm} & x<d\\
          B e^{\sqrt{2(\lambda+\chi)}x} + Ce^{-\sqrt{2(\lambda+\chi)}x} & \hspace{2cm} & d<x<0\\
          D e^{-\sqrt{2(\lambda+\chi)}x} & \hspace{2cm} & x>0
        \end{array}
      \right.
    \end{equation}
    where the coefficients are fixed by matching the boundary conditions together.
    This was done using Mathematica to speed up the tedious algebraic work.  

    We are ultimately only interested in $f(x=0)$, which in this case means we just need to know $D$ for this case.
    It turns out that 
    \begin{equation}
      D = \frac{1}{\sqrt{2\lambda}} + u e^{2\sqrt{2\lambda}d}.
    \end{equation}
    \item Final results.
    We can pull these results together to write
    \begin{equation}
      f_{TE,1}(x) = \left\{\begin{array}{lcr} 
          \dfrac{1}{\sqrt{2\lambda}}\left[1+ u e^{-2\sqrt{2\lambda}d}\right]  & \hspace{2cm} & d<0\\
          \dfrac{1}{\sqrt{2(\lambda+\chi)}}\left[1 - u e^{-2\sqrt{2(\lambda+\chi)}d}\right] & \hspace{2cm} & d>0\\
        \end{array} \right. 
    \end{equation}

  \item Identify similarity to reflection coefficients
  \item {Quote result for integration over surface.}

    We now need to evaluate $\int dx_0 \dlangle e^{-\int_0^T dt V[x_0 + B(t)-h]}\drangle$ for use with Casimir energies.   Now we have to take $h\rightarrow h-x_0$, and integrate over $x_0$.
    The integrals over position is 
    \begin{align}
      I_{TE,1} &= \int_{-\infty}^h dx_0 \frac{1}{\sqrt{2(\lambda+\chi_1)}}\left(1 - u_1e^{-2\sqrt{2(\lambda+\chi_1)}(h-x_0)} \right) 
      + \int_h^\infty dx_0 \frac{1}{\sqrt{2\lambda}}\left(1 + u_1 e^{2\sqrt{2\lambda}(h-x_0)}\right) \\
      % &= \int_h^\infty dx_0[(2\lambda)^{-1/2}+(2\kappa)^{-1/2}]  -  \int_{-\infty}^0  dx_0 \frac{1}{\sqrt{2\kappa_1}}u_1e^{2\sqrt{2\kappa_1}x_0} + \int_0^\infty dx_0 \frac{1}{\sqrt{2\lambda}}u_1 e^{-2\sqrt{2\lambda}x_0}\\
      &= I^{(1)}_{div}  +   \left(\frac{1}{4\lambda}- \frac{1}{4(\lambda+\chi_1)}\right)u_1,
    \end{align}
    where $I^{(1)}_{div} = \int_h^\infty(2\lambda)^{-1/2}+\int_{-\infty}^h[2(\lambda+\chi_1)]^{-1/2}$.  This renormalization is only necessary for the energy.  

\end{enumerate}

\subsection{Planar Dirichlet Conditions}

\begin{enumerate}
  \item (Comment- important to get this correct and simple as this is the template?)
  \item Note connection to Gies work (they don't use this).    Note we will discuss this in
    accelerated convergence techniques.  
  \item Take delta function potential a distance $d$ away.  Use open loops.  
  \item Quote potential
\end{enumerate}

\section{Two Step Potentials}

\begin{enumerate}
  \item {Take $V=\chi_1\Theta[d_1-x]+\chi_2\Theta[x-d_2]$.}
    When we do the Casimir energy between two bodies we will need to find the Feynman-Kac formula assuming two step discontinuities.
    We will use exactly the same procedure as above, but with an extra step. 

    Here we are solving this with $V = \chi_1\Theta(-d_1-x) + \chi_2\Theta(d_2-x)$.
    We have solutions, 
    \begin{equation}
      f(x) = \left\{ \begin{array}{lcr}
          A e^{\sqrt{2(\lambda+\chi_1)}x}   & \hspace{1cm} & x<0\\
          B e^{\sqrt{2(\lambda+\chi_1)}x} + C e^{-\sqrt{2(\lambda+\chi_1)}x}  & \hspace{1cm} & 0<x<h\\
          D e^{\sqrt{2\lambda}x} + F e^{-\sqrt{2\lambda}x}  & \hspace{1cm} & h<x<h+d\\
          G e^{-\sqrt{2(\lambda+\chi_2)}x} & \hspace{1cm} & x>d_2
        \end{array}
      \right.
    \end{equation}
    We will take $d_1 = h, d_2 = d+h$.
    We can then check that our final solutions are independent of $h$ once we have integrated over position.
    We would expect the Casimir energy to only depend on $d$ in this case.   
  \item {Solve in each region, (quote results)}
    We then need 
    \begin{equation}
      f_{TE,12}[x-(h-x_0)] = \left\{ \begin{array}{ccr}
          \dfrac{1}{\sqrt{2(\lambda+\chi_1)}} + e^{-2\sqrt{2(\lambda+\chi_1)}(h-x_0)}\dfrac{u_2 e^{-2\sqrt{2\lambda}d} - u_1}{\sqrt{2(\lambda+\chi_1)}(1-u_1u_2 e^{-2\sqrt{2\lambda}d})} & \hspace{1cm} & h>x_0\\
          \frac{1}{\sqrt{2\lambda}} + \dfrac{2u_1u_2 e^{-2\sqrt{2\lambda}d} + u_1 e^{2\sqrt{2\lambda}(h-x_0)} +u_2 e^{-2\sqrt{2\lambda}(d+h-x_0)}}{\sqrt{2\lambda}(1-u_1u_2 e^{-2\sqrt{2\lambda}d})} & \hspace{1cm} & h<x_0<h+d\\
          \dfrac{1}{\sqrt{2(\lambda+\chi_2)}} + e^{2\sqrt{2(\lambda+\chi_2)}(d+(h-x_0))}\dfrac{(u_1 e^{-2\sqrt{2\lambda}d}-u_2)}{\sqrt{2(\lambda+\chi_2)}(1-u_1u_2 e^{-2\sqrt{2\lambda}d})} & \hspace{1cm} & h+d<x_0
        \end{array}
      \right.
    \end{equation}
    with 
    \begin{equation}
      u_i = \frac{\sqrt{\lambda} -\sqrt{\lambda+\chi_i}}{\sqrt{\lambda} + \sqrt{\lambda+\chi_i}},
    \end{equation}
  \item Identify similarity to reflection coefficients/ Fabry-Perot resonance condition.
  \item Leave in Laplace transformed version for analytical comparisons.    
  \item {Carry out integral over position. Note divergent terms, these are cancelled out by 
    suitable renormalization.}
    We now need to evaluate $\int dx_0 \dlangle e^{-\int_0^T dt V[x_0 + B(t)-h]}\drangle$ for use with Casimir energies.   Now we have to take $h\rightarrow h-x_0$, and integrate over $x_0$. 
    First up we need $I_{12}=\int dx_0 f_{12}[x-(h-x_0)]$
    \begin{align}
      I_{TE,12} %=& \int_{-\infty}^h  dx_0  f_{x_0<h} + \int_{h}^{h+d}  dx_0  f_{h<x_0<h+d} + \int_{h+d}^\infty dx_0 f_{x_0>h+d}\\
      =&\int_{-\infty}^h dx_0 \left[\dfrac{1}{\sqrt{2(\lambda+\chi_1)}} + e^{-2\sqrt{2(\lambda+\chi_1)}(h-x_0)}\dfrac{u_2 e^{-2\sqrt{2\lambda}d} - u_1}{\sqrt{2(\lambda+\chi_1)}(1-u_1u_2 e^{-2\sqrt{2\lambda}d})}\right] \nonumber\\
      & +\int_{h}^{h+d}dx_0\left[\frac{1}{\sqrt{2\lambda}} + \frac{2u_1u_2 e^{-2\sqrt{2\lambda}d} + u_1 e^{2\sqrt{2\lambda}(h-x_0)} +u_2 e^{-2\sqrt{2\lambda}(d+h-x_0)}}{\sqrt{2\lambda}(1-u_1u_2 e^{-2\sqrt{2\lambda}d})} \right]\nonumber\\
      &+ \int_{h+d}^\infty dx_0 \left[\dfrac{1}{\sqrt{2(\lambda+\chi_2)}} + e^{2\sqrt{2(\lambda+\chi_2)}(d+(h-x_0))}\dfrac{u_1 e^{-2\sqrt{2\lambda}d}-u_2}{\sqrt{2(\lambda+\chi_2)}(1-u_1u_2 e^{-2\sqrt{2\lambda}d})}\right]
    \end{align}

     We need 
    \begin{gather}
      \int_{-\infty}^h dx_0 e^{-2\sqrt{2(\lambda+\chi_1)}(h-x_0)} = \int_{-\infty}^0 dx_0 e^{2\sqrt{2(\lambda+\chi_1)}(x_0)} = \frac{1}{2\sqrt{2(\lambda+\chi_1)}}\\
      \int_{h+d}^\infty dx_0 e^{2\sqrt{2(\lambda+\chi_2)}(d+(h-x_0))} = \int_0^\infty e^{-2\sqrt{2(\lambda+\chi_2)}x_0} = \frac{1}{2\sqrt{2(\lambda+\chi_2)}}\\
      \int_{h}^{h+d} dx_0 e^{-2\sqrt{2\lambda}(x_0-h)} = \int_0^d  dx_0 e^{-2\sqrt{2\lambda}x_0} = \frac{1 - e^{-2\sqrt{2\lambda}d}}{2\sqrt{2\lambda}}\\
      \int_{h}^{h+d} dx_0 e^{2\sqrt{2\lambda}(x_0-d-h)} = \int_{-d}^0  dx_0 e^{2\sqrt{2\lambda}x_0} = \frac{1 - e^{-2\sqrt{2\lambda}d}}{2\sqrt{2\lambda}}
    \end{gather}
   \item Final integrated value
    So that we get:
    \begin{align}
      I_{TE,12} =& -I_{div} + \dfrac{u_2 e^{-2\sqrt{2\lambda}d}-u_1}{4(\lambda+\chi_1)(1-u_1u_2 e^{-2\sqrt{2\lambda}d})} +\frac{2u_1u_2 e^{-\sqrt{2\lambda}d}d}{\sqrt{2\lambda}(1-u_1u_2 e^{-2\sqrt{2\lambda}d})} + (u_1+u_2)\frac{(1-e^{-2\sqrt{2\lambda}d})}{4\lambda(1-u_1u_2e^{-2\sqrt{2\lambda}d})}\nonumber\\
      & +\frac{u_1 e^{-2\sqrt{2\lambda}d} - u_2}{4(\lambda+\chi_2)(1-u_1u_2 e^{-2\sqrt{2\lambda}d})},
    \end{align}
    where $I_{div} = [2(\lambda+\chi_1)]^{-1/2}\int_{-\infty}^h dx_0  +  [2(\lambda+\chi_2)]^{-1/2}\int_{h+d}^\infty dx_0  + (2\lambda)^{-1/2}d$.
\end{enumerate}

\section{Feynman-Kac formula for TM Potentials}

\begin{enumerate}
  \item {Note that TM potential has highly singular potential.  Must be regularized.}
  \item Smooth out contribution by analytically averaging over subpaths.
    In handling the TM case we will first have need to handle that singular potential.
  \item We shall do this first on it's own, since we will have to use those results in any numerical method.
    In addition, it vastly simplifies down.
    We will find that the potential enforces a boundary condition.  

  \item Quote PDE, and exact form of the potential
    We will find the solution to 
    \begin{equation}
      \partial_t f = \frac{1}{2}\partial_x^2f -V_{TM} f - \lambda f + \delta(x-c)
    \end{equation}
    where 
    \begin{equation}
      V_{TM}(x) = \frac{\Xi^2}{2}[\delta(x)]^2 - \frac{\Xi}{2}\delta'(x).
    \end{equation}
  \item Introduce regularized version.  
    Since that is highly singular, we will instead deal with 
    \begin{equation}
      \mathfrak{M}(x) = \lim_{a\rightarrow 0} \frac{\Xi^2}{2a^2}\Theta(a/2-|x|) - \frac{\Xi}{2a}[\delta(x)-\delta(x-a)],
    \end{equation}
    which corresponds to the gradients of a suitably regularized version of $\epsilon_r = 1+\chi\Theta(x)$.  
  \item Outline of next few sections.  Derive boundary conditions for regularized potential.  
  \item Derive Feynman-Kac formula between fixed end points.  Absolutely essential for numerical
    work.
\end{enumerate}

\subsection{Transfer Layer Boundary conditions for TM Potential}

\begin{enumerate}
  \item {Quote regularized potential.  Note that it corresponds to exponential interpolation.}
    Here we will be trying to solve
    \begin{equation}
      \partial_x^2f =\frac{\Xi^2}{a^2}\Theta(a/2-|x-d|) - \frac{\Xi}{a}[\delta(x-d)-\delta(x-d-a)]f
    \end{equation}
    which arises as the TM potential for crossing the surface associated with $\epsilon_r = 1+\chi\Theta(x-d)$.  We will try to solve this in the limit $a\rightarrow 0$, so we have dropped the other terms.  
    As usual, the solutions will be of the form $f = \alpha_+ e^{\kappa x}+\alpha_- e^{-\kappa x}$, where $\kappa$ is chosen to satisy the differential equation, and $\alpha_\pm$ are fixed by the boundary conditions.  Note that we have three delta functions to handle here.  At each delta function $\gamma \delta(x-s)$ we have the following boundary conditions:
    \begin{equation}
      f'(s+\epsilon)-f'(s-\epsilon) = \gamma f(s),\qquad f(s+\epsilon)-f(s-\epsilon) = 0.
    \end{equation}

  \item {Can find BC's due to this regularized surface.}
  \item Eliminate middle region using continuity conditions.  
  \item Write out form of solutions
    Let us see if we can solve this by eliminating all reference to the middle region.  
    \begin{align}
      f_{\text{mid}} =& Be^{\sqrt{2\lambda + \Xi^2/a^2}x} + C e^{-\sqrt{2\lambda + \Xi^2/a^2}x}\\
      =& Be^{\Xi x/a} + C e^{-\Xi x/a}
    \end{align}
  \item Write out continuity conditions
    and boundary conditions, 
    \begin{subequations}
      \begin{align}
        f(d + \epsilon)-f(d -\epsilon) =& 0\\
        f(d+a+\epsilon)- f(d+a-\epsilon)=& 0\\
        f'(d + \epsilon) -f'(d -\epsilon)=& -2\sigma f(d)\\
        f'(d+a+\epsilon) -f'(d+a-\epsilon)=& +2\sigma f(d+a)
      \end{align}
    \end{subequations}
  \item Find relations between $f(d\pm \epsilon)$ and $f'(d\pm \epsilon)$.
    We will solve for $f(d+a+\epsilon)$, and $f'(d+a+\epsilon)$, trying to eliminate $B$ and $C$, which we will fix in terms of $f(d-\epsilon),f'(d-\epsilon)$.  We will thus have derived the relation between $f$ and its derivatives on both sides of the potential.      
    Our first conditions are
    \begin{align}
      f(d+\epsilon) - f(d-\epsilon) =& 0\\
      \rightarrow f(d-\epsilon) = Be^{\Xi d/a} + C e^{-\Xi d/a}\label{eq:M c1}
    \end{align}
    Similarly, at the second edge  we find that 
    \begin{align}
      f(d+a+\epsilon) - f(d+a-\epsilon) =& 0\\
      \rightarrow f(d+a-\epsilon) = Be^{\Xi d/a+\Xi} + C e^{-\Xi d/a-\Xi}\label{eq:M c2}
    \end{align}
    Now the derivative conditions at $d$
    \begin{align}
      f'(d +\epsilon)-f'(d-\epsilon) =& -\frac{\Xi}{a}f(d)\\
      \rightarrow f'(d-\epsilon)=& \frac{\Xi}{a}\left(Be^{\Xi d/a} + C e^{-\Xi d/a}\right) + \frac{\Xi}{a}\left(Be^{\Xi d/a} - C e^{-\Xi d/a}\right)\\
      =& 2\frac{\Xi}{a}Be^{\Xi d/a} \label{eq:M d1}
    \end{align}
    And the derivative conditions at $d+a$ yield 
    \begin{align}
      f'(d+a+\epsilon) -f'(d+a-\epsilon)=& \frac{\Xi}{a}f(d+a)\\
      \rightarrow f'(d+a+\epsilon)=&\frac{\Xi}{a}\left(Be^{\Xi (d+a)/a} + C e^{-\Xi (d+a)/a} \right)+ \frac{\Xi}{a}\left(Be^{\Xi (d+a)/a} - C e^{-\Xi (d+a)/a} \right)\\
      =&2\frac{\Xi}{a}Be^{\Xi d/a}e^{\Xi}\label{eq:M d2}
    \end{align}
  \item Solve for B/C.
    Eq.~(\ref{eq:M d1}) fixes $B$, which we can use in to Eq.~(\ref{eq:M d2}), so that 
    \begin{equation}
      f'(d+a+\epsilon) = e^{\Xi}f'(d-\epsilon).
    \end{equation}
    So far so good.  Next up the normal continuity condition.  
    \begin{align}
      f(d-\epsilon) =& B e^{\Xi d/a} + C e^{-\Xi d/a}\\
      =& \frac{a}{2\Xi} f' + C e^{-\Xi d/a} = C e^{-\Xi d/a},
    \end{align}
    which says that $B\sim a$, so we will drop it from this part of the calculation.  
    Now use this in Eq.~(\ref{eq:M d2}) to find
    \begin{equation}
      f(d+a+\epsilon) = C e^{-\Xi d/a} e^{-\Xi} =  e^{-\Xi} f(d-\epsilon).  
    \end{equation}
  \item Eliminate references to internal state.  Take limit of regularization going to zero.
  \item {Quote final boundary conditions.}
    We can now take $a\rightarrow 0 $ everywhere.  So we have the following boundary conditions due to an interface $\mathfrak{M}$:
    \begin{align}
      \Aboxed{
        f(d-\epsilon) =& e^{\Xi}f(d+\epsilon), \qquad
        f'(d -\epsilon)= e^{-\Xi}f'(d+\epsilon)
      }
    \end{align}
    \label{sec:TM boundary condition}

\end{enumerate}

\subsection{Finding the Feynman-Kac formula}

\begin{enumerate}
  \item {Using above effective BC solve for Feynman-Kac eqn for open bridge.}
    \item Quote form of solution, and PDE it solves.  
    \item Note how to recover various cases by flipping signs etc.  
    We are now in position to derive an expression the ensemble average of paths going from $0\rightarrow c$ through $V_{TM}$. 
    \begin{equation}
      f_{0\rightarrow c}=\int_0^\infty dt e^{-\lambda t}\frac{e^{-c^2/(2t)}}{\sqrt{2\pi t}} \dlangle e^{-s\int_0^T dt V_{TM}}\drangle.
    \end{equation}
    This is the solution to 
    \begin{equation}
      0 = \frac{1}{2} \partial_x^2 f - V_{TM}f - \lambda f + \delta(x-c)
    \end{equation}
    We will find the solutions for the case where $0<c<d$.  From this we can see how we need to change the signs to recover the other cases.  Note that the boundary conditions at $x=c$ are: ${f(c+\epsilon)=f(c-\epsilon)}, {f(c+\epsilon)-f(c-\epsilon)= -2}$.  As we found in Sec.~(\ref{sec:TM boundary condition}), at $x=d$ we need $f(d-\epsilon) = e^{\Xi}f(d+\epsilon),f'(d -\epsilon)= e^{-\Xi}f'(d+\epsilon)$.  
  \item Form of solutions in each region, and fixing values via BCs from previous section
    For example, with $c<d$ we have 
    \begin{equation}
      f  = \left\{\begin{array}{ccr} A e^{\sqrt{2\lambda} x} & \hspace{2cm} & x<c\\
          B e^{\sqrt{2\lambda} x} + C e^{-\sqrt{2\lambda} x}  & \hspace{2cm} & c<x<d\\
          D e^{-\sqrt{2\lambda} x}& \hspace{2cm} & c<x<d\\
        \end{array}
      \right. .
    \end{equation}
    In the case where $d>0$, we will need $f(0) = A$.  In the other case where $d<0$, we will need $f(0) = D$.  
    Mathematica when asked to solve the boundary conditions finds that 
    \begin{equation}
      f = \left\{ \begin{array}{ccr} 
          \dfrac{e^{-\sqrt{2\lambda}|c|}}{\sqrt{2\lambda}} + \dfrac{e^{-\sqrt{2\lambda}(2d-c)}}{\sqrt{2\lambda}}\dfrac{e^{2\Xi}-1}{e^{2\Xi} +1}  &   \hspace{2cm}  & d>c,  d>0\\
          \dfrac{ e^{-\sqrt{2\lambda}|c|}}{\sqrt{2\lambda}} \dfrac{2e^\Xi}{1 + e^{2\Xi}} & \hspace{2cm} & d>c,d<0 \\
          \dfrac{ e^{-\sqrt{2\lambda}|c|}}{\sqrt{2\lambda}} \dfrac{2e^\Xi}{1 + e^{2\Xi}} & \hspace{2cm} & c>d,d>0 \\
          \dfrac{ e^{-\sqrt{2\lambda}|c|}}{\sqrt{2\lambda}} - \dfrac{e^{\sqrt{2\lambda}(2d-c)}}{\sqrt{2\lambda}}\dfrac{e^{2\Xi}-1}{e^{2\Xi}+1} & \hspace{2cm} & c>d, d<0
          \\
        \end{array}
      \right.
    \end{equation}
    \item Simplify down to crossing/no-crossing cases.
    That seems complicated, but we can note that if there is a crossing ($|c|<|d|$) then we get 
    \begin{equation}
      f_{nc}=\dfrac{e^{-\sqrt{2\lambda}|c|}}{\sqrt{2\lambda}} + \sgn(d)\dfrac{e^{-\sqrt{2\lambda}|2d-c|}}{\sqrt{2\lambda}}\dfrac{e^{2\Xi}-1}{e^{2\Xi} +1}
    \end{equation}
    For the case of crossings ($d(d-c)<0$) we get
    \begin{equation}
      f_c = \dfrac{ e^{-\sqrt{2\lambda}|c|}}{\sqrt{2\lambda}} \dfrac{2e^\Xi}{1 + e^{2\Xi}}.
    \end{equation}
  \item {Invert Laplace transform to get real-time version.}

    Ultimately, we will want to isolate the potential term.  
    If we use 
    \begin{equation}
      \mathcal{L}^{-1}\left[ \frac{e^{-\sqrt{2\lambda}x}}{\sqrt{2\lambda}}   \right] = \frac{e^{-x^2/(2t)}}{\sqrt{2\pi t}},
    \end{equation}
    which is exactly the factor we isolated in front of $\mathcal{M}$.  
  \item {Final results.  Can now use as basis for numerical methods since depends on loop-time.}
    So we have:
    \begin{align}
      \frac{e^{-c^2/(2T)}}{\sqrt{2\pi T}} \dlangle e^{-\int_0^T dt V_{TM}(x-d)}\drangle 
      &=  \bigg( \frac{e^{-c^2/(2T)}}{\sqrt{2\pi T}} 
      + \sgn(d)\dfrac{e^{-(2d-c)^2/(2T)}}{\sqrt{2\pi T}}\dfrac{e^{2\Xi}-1}{e^{2\Xi} +1}\bigg)\nonumber\\
      \theta(|d|-|c|) 
      &+ \frac{e^{-c^2/(2T)}}{\sqrt{2\pi T}}\dfrac{2e^\Xi}{1 + e^{2\Xi}}\theta(|c|-|d|)
    \end{align}
    
    % Dan has:
    % \begin{equation}
    %   \dlangle \exp\left[-\int_0^T dt V_{TM}(x-d)\right]\drangle = 1 + \frac{\sinh(\Xi/2)}{\cosh\Xi}[\sgn(d-c)e^{\sgn(d)\Xi/2} + \sgn(d)e^{-\sgn(d)\Xi/2}]e^{\left[c^2-(|d|+|c-d|)^2\right]/2t}.
    % \end{equation}
    % This agrees with my expressions for all cases.  This is useful for our numerical work.  
    % Let us check out the $\Xi$ prefactor against my work.  
    % \begin{align}
    %   F_{d>0,c<d} =&\frac{\sinh(\Xi/2)}{\cosh\Xi}[\sgn(d-c)e^{\sgn(d)\Xi/2} + \sgn(d)e^{-\sgn(d)\Xi/2}]\\
    %   =&\frac{\sinh(\Xi/2)}{\cosh\Xi}[e^{\Xi/2} + e^{-\Xi/2}] = \frac{e^{\Xi} - e^{-\Xi}}{e^\Xi+ e^\Xi}  
    % \end{align}
    % works.
    % \begin{align}
    %   F_{d>0,d<c} =&\frac{\sinh(\Xi/2)}{\cosh\Xi}[-e^{\Xi/2} + e^{-\Xi/2}] = -\frac{ e^{-\Xi} + e^{\Xi} -2}{e^\Xi + e^{-\Xi}},
    % \end{align}
    % works, 
    % \begin{align}
    %   F_{d<0,c<d} =\frac{\sinh(\Xi/2)}{\cosh\Xi}[e^{-\Xi/2} -e^{\Xi/2}]=\frac{[(e^{\Xi/2}- e^{-\Xi/2})(e^{-\Xi/2} -e^{\Xi/2})]}{e^\Xi + e^{-\Xi}}=-\frac{e^\Xi + e^{-\Xi} -2}{e^\Xi + e^{-\Xi}},
    % \end{align}
    % works, and 
    % \begin{align}
    %   F_{d<0,d<c} =&\frac{\sinh(\Xi/2)}{\cosh\Xi}[-e^{\Xi/2} -e^{-\Xi/2}] = -\frac{e^\Xi - e^{-\Xi}}{e^\Xi + e^{-\Xi}}
    % \end{align}
  \item Figure of TM potential.  Note connection to Neumann BC and reflection.  
\end{enumerate}

\section{Single TM potential and Step}

\begin{enumerate}
  \item {Solve FK for TM potential plus step. }
  \item Quote form of solution, and PDE it solves
    Let's now find: 
    \begin{equation}
      f = \int_0^\infty dT \frac{1}{\sqrt{2\pi T}}\langle e^{-\int_0^T dt V_{TM} + s\Theta}\rangle 
    \end{equation}
    This is the steady state solution to 
    \begin{equation}
      \partial_t f = \frac{1}{2}\partial_x^2f -(V_{TM} + s\Theta(x-d)+\lambda)f +\delta(x). 
    \end{equation}

  \item Quote boundary conditions at the boundary, and delta function.  
    For the boundary conditions at the delta function we have 
    \begin{equation}
      \partial_xf(\epsilon) -\partial_x f(-\epsilon) = -2 , \qquad f(d+\epsilon)-f(d-\epsilon) = 0,
    \end{equation}
    where both of these relations follow from integrating the PDE across the discontinuity.
    And the boundary conditions courtesy of $V_{TM}$ are
    \begin{align}
      f(d-\epsilon) = e^{\Xi}f(d+\epsilon)\\
      f'(d-\epsilon) = e^{-\Xi}f'(d+\epsilon).
    \end{align}

    This problem can be solved in exactly the same fashion as the equivalent TE problems.
    Perhaps unsurprisingly, you find that the two slab results can also be ported over, but with their TM reflection coefficients.  

  \item {Leave as Laplace-transform.  Note integral identities mean we can apply this for analytical
    result analytically.  }
    \item Quote solution.  
  \begin{equation}
      f_{TM,1}(x) = \left\{\begin{array}{lcr} 
          \dfrac{1}{\sqrt{2\lambda}}\left[1+ u' e^{-2\sqrt{2\lambda}d}\right]  & \hspace{2cm} & d<0\\
          \dfrac{1}{\sqrt{2(\lambda+\chi)}}\left[1 - u' e^{-2\sqrt{2(\lambda+\chi)}d}\right] & \hspace{2cm} & d>0\\
        \end{array} \right. 
    \end{equation}
    where
    \begin{equation}
      u' = \frac{\sqrt{\lambda}e^{2\Xi} -\sqrt{\lambda+\chi}}{\sqrt{\lambda}e^{2\Xi} + \sqrt{\lambda+\chi}},
    \end{equation}
    plays a similar role to the $TM$ reflection coefficient, since $e^{2\Xi} = (1+\chi)$.   
  \item Final result.  
    Ultimately, we find that 
    \begin{equation}
      \int_0^\infty dT e^{-\lambda T} \dlangle \frac{e^{-\int_0^T dt V_{TM}+s\theta[x(t)-d]}}{\sqrt{2\pi T}}\drangle  =
      \frac{1}{\sqrt{2[\lambda+\chi\theta(d)]}}\left[1 - \sgn(d) u' e^{-2\sqrt{2[\lambda+\chi\theta(d)]}|d|}\right],\label{eq:Feynman-Kac TM one step}
    \end{equation}
    where the ensemble average is over brownian bridges that satisfy $x(0)=x(T)=0$.
    The factor of $\sqrt{2\pi T}$ is normalization for the use of the bridges.  
\end{enumerate}


\section{Two TM Step Potentials}

\begin{enumerate}
  \item {Quote full potential, and note boundary conditions.}
  \item Quote full potential, and PDE.
  \item Comment on how to handle other surface.
    When we do the Casimir energy between two bodies we will need to find the Feynman-Kac formula assuming two step discontinuities.
    We will use exactly the same procedure as above, but with an extra step. 

    The potential is then 
    \begin{equation}
      V = \chi_1\Theta(h-x) + \chi_2\Theta(x-h-d) + \mathfrak{M}(-\Xi,h) + \mathfrak{M}(\Xi,h+d),
    \end{equation}
    where we took $\Xi \rightarrow -\Xi$ on one step since $\partial^2_x\theta(-x) = -\delta'$.  
  \item Exploit symmetry of effective boundary conditions.  
  \item {Quote results.  Note that $r\rightarrow r'$.}

    We then need to solve this for the various cases of $h,h+d $ on each side of $x=0$:
    \begin{equation}
      f_{TM,12}[x-(h-x_0)] = \left\{ \begin{array}{ccr}
          \dfrac{1}{\sqrt{2(\lambda+\chi_1)}} + e^{-2\sqrt{2(\lambda+\chi_1)}(h-x_0)}
\dfrac{u'_2 e^{-2\sqrt{2\lambda}d} - u'_1}{\sqrt{2(\lambda+\chi_1)}(1-u'_1u'_2 e^{-2\sqrt{2\lambda}d})} & \hspace{1cm} & h>x_0\\
          \frac{1}{\sqrt{2\lambda}} + 
          \dfrac{2u'_1u'_2 e^{-2\sqrt{2\lambda}d} + u'_1 e^{2\sqrt{2\lambda}(h-x_0)} +u'_2 e^{-2\sqrt{2\lambda}(d+h-x_0)}}
          {\sqrt{2\lambda}(1-u'_1u'_2 e^{-2\sqrt{2\lambda}d})} & \hspace{1cm} & h<x_0<h+d\\
          \dfrac{1}{\sqrt{2(\lambda+\chi_2)}} + e^{2\sqrt{2(\lambda+\chi_2)}(d+(h-x_0))}
          \dfrac{(u'_1 e^{-2\sqrt{2\lambda}d}-u'_2)}{\sqrt{2(\lambda+\chi_2)}(1-u'_1u'_2 e^{-2\sqrt{2\lambda}d})} & \hspace{1cm} & h+d<x_0
        \end{array}
      \right.
    \end{equation}
    \item {Quote various limits for $h,h+d$ on either side of 0}
    For have $h,h+d>0$
    \begin{align}
      f(x) =\frac{1}{\sqrt{2(\lambda+\chi_1)}} + e^{-2\sqrt{2(\lambda+\chi_1)}h}
      \frac{u'_2 e^{-2\sqrt{2\lambda}d}-u'_1 }{\sqrt{2(\lambda+\chi_1)}(1-u'_1u'_2 e^{-2\sqrt{2\lambda}d})}
    \end{align}

    For $h<0, h+d>0$ we need 
    \begin{equation}
      f(x) = \frac{1}{\sqrt{2\lambda}} + \frac{2u'_1u'_2 e^{-2\sqrt{2\lambda}d} + u'_1 e^{2\sqrt{2\lambda}h} +u'_2 e^{-2\sqrt{2\lambda}(d+h)}}{\sqrt{2\lambda}(1-u'_1u'_2 e^{-2\sqrt{2\lambda}d})}
    \end{equation}

    Finally, for $h,h+d<0$ we get 
    \begin{equation}
      f(x) =  \frac{1}{\sqrt{2(\lambda+\chi_2)}} + \frac{e^{2\sqrt{2(\lambda+\chi_2)}(d+h)}(u'_1 e^{-2\sqrt{2\lambda}d} - u'_2)}{\sqrt{2(\lambda+\chi_2)}(1-u'_1u'_2 e^{-2\sqrt{2\lambda}d})},
    \end{equation}
    with 
    \begin{equation}
      u'_i = \frac{e^{2\Xi}\sqrt{\lambda} -\sqrt{\lambda+\chi_i}}{e^{2\Xi}\sqrt{\lambda} + \sqrt{\lambda+\chi_i}},
    \end{equation}
  \item {Integrate over position}
    In exactly the same fashion, we can integrate over position.  
    \begin{align}
      I_{TM,12} =& -I_{div} + \dfrac{u'_2 e^{-2\sqrt{2\lambda}d}-u'_1}{4(\lambda+\chi_1)(1-u'_1u'_2 e^{-2\sqrt{2\lambda}d})} +\frac{2u'_1u'_2 e^{-\sqrt{2\lambda}d}d}{\sqrt{2\lambda}(1-u'_1u'_2 e^{-2\sqrt{2\lambda}d})} + (u'_1+u'_2)\frac{(1-e^{-2\sqrt{2\lambda}d})}{4\lambda(1-u'_1u'_2e^{-2\sqrt{2\lambda}d})}\nonumber\\
      & +\frac{u'_1 e^{-2\sqrt{2\lambda}d} - u'_2}{4(\lambda+\chi_2)(1-u'_1u'_2 e^{-2\sqrt{2\lambda}d})},
    \end{align}
    where $I_{div} = [2(\lambda+\chi_1)]^{-1/2}\int_{-\infty}^h dx_0  +  [2(\lambda+\chi_2)]^{-1/2}\int_{h+d}^\infty dx_0  + (2\lambda)^{-1/2}d$.
  \item Comment on divergent parts cancelling out. 
\end{enumerate}



%%% Local Variables: 
%%% mode: latex
%%% TeX-master: "thesis_master"
%%% End: 
