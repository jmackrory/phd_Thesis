\chapter{Path Integrals and Feynman--Kac formulae}

\label{ch:feynman_kac}

A path integral is a sum over all possible paths a system might explore.
The path integral is a powerful way to represent stochastic processes, and path integals have been used
extensively in statistics~\cite{Karatzas1991, Durrett1996,DelMoral2004}.  
Although path integrals are primarily used in a perturbative sense within physics, 
they can be solved exactly in some geometries.
These analytical solutions are particularly useful in accelerating numerical methods relying 
on stochastic simulations, such as occur in finance.  

In honor of the pioneering work by Feynman~\cite{Feynman1950} and Kac~\cite{Kac1949}, the fact that path integrals
act as the solutions to diffusion or Schrodinger equations is referred to as the Feynman-Kac 
formula.  This thesis will also sometimes use the same label to refer to the analytical
result for a particular path integral.  

This chapter is devoted to deriving the Feynman--Kac formula, and then solving
it for planar geometries that are useful for both TE and TM polarizations.
These analytical formulae will be used in Ch.~\ref{ch:analytical} to show agreement 
with other results for both Casimir and Casimir--Polder energies.
The Feynman--Kac formula is also essential for handling the singular potentials.  
In fact, these analytical results are necessary to even allow the TM numerics to proceed at all.

\section{Derivation of the Feynman-Kac formula }

In this section we will derive the path integral as the solution to a diffusion equation,
using techniques from quantum mechanics.
Our derivation will stay close in spirit to the  one found in Sakurai~\cite{Sakurai1994}.
But, we will extend the usual quantum derivation to including a source term, as is more common in statistics.
More formal derivations are available from mathematical~\cite{Cartier2004},
and probabilistic perspectives~\cite{Karatzas1991, Durrett1996}.  
A detailed discussion of the more formal probabilistic derivation is available in Steck \S 17.9~\cite{SteckNotes}.

We aim to find a solution $f(x,t)$ to the driven diffusion equation
\begin{equation}
  \partial_t f = \frac{1}{2}\nabla^2 f  - [V+\lambda]f +g,\label{eq:diffusion_equation}
\end{equation}
where the potential/killing rate is given by $V=V(\vect{x},t)$, and the source term is $g=g(\vect{x},t)$.  
In this form $f$ corresponds to the probability distribution for a diffusing particle with a source
of particles $g$, with spatially dependent killing rate $V$.  
The Schr\"odinger equation is recovered under the $\tau\rightarrow -it$ substitution.
    We can use techniques from quantum mechanics, and introduce a Hilbert space with operators.
    The differential equation can be written in operator form using the same Hilbert space 
    conventions~(\ref{eq:commutation})--(\ref{eq:identity}) we have used for the worldline path integrals.
    Note however that we will consider a Hilbert space with $\hbar=1$ in all operator equations.
    % \begin{gather}
    %   \langle \vect{x}|\op{x}_i|f\rangle = x_if(\vect{x})\qquad
    %   \langle \vect{x}|\op{p}_i|f\rangle = -i\partial_if(\vect{x}),
    % \end{gather}
    % The position and momentum operators have commutation relations
    % \begin{gather}
    %   [\op{x}_i,\op{p}_j]=i\delta_{ij},
    % \end{gather}
    % and overlap 
    % \begin{equation}
    %   \langle \vect{x}|\vect{p}\rangle = e^{i\vect{x\cdot p}}.
    % \end{equation}
    % We also need the position and momentum resolutions of the identity
    % \begin{gather}
    %   I_x = \int d\vect{x}|\vect{x}\rangle \langle \vect{x}|, \quad
    %   I_p = \int \frac{d\vect{p}}{(2\pi)^D}|\vect{p}\rangle \langle \vect{p}|.
    % \end{gather}
The diffusion equation~(\ref{eq:diffusion_equation}) can be written in operator form
\begin{equation}
  \partial_t\langle \vect{x}| |f(t)\rangle = -\langle \vect{x}|
  \left[\frac{1}{2}\op{\vect{p}}^2 + V(\op{\vect{x}},t)+\lambda\right]|f(t)\rangle +\langle \vect{x}|g(t)\rangle.
\end{equation}
We can solve this by introducing the evolution operator,
\begin{equation}
  U(t) = \text{T}\exp\left\{-\int_0^t ds\left[\frac{1}{2}\op{\vect{p}}^2 + V(\op{\vect{x}},s)+\lambda\right]\right\},
\end{equation}
where $\text{T}$ is the time-ordering operator.  \comment{Dyson series? - re QFT Derivation of S-matrix?}
We can then use the operator analogue of an integrating factor method, by transforming 
the vectors to the Heisenberg picture, $|f\rangle \rightarrow |\tilde{f}\rangle=U^{-1}(t)|f\rangle$.  
The transformed vectors obey
\begin{equation}
  \partial_t|\tilde{f}\rangle = U^{-1}(t)|\tilde{g}\rangle.
\end{equation}
This equation can be formally integrated with respect to time,
% \begin{equation}
%   |\tilde{f}(t)\rangle-|\tilde{f}(0)\rangle = \int_0^t ds |\tilde{g}(s)\rangle.
% \end{equation}
and after transforming back to the original vectors we find
\begin{equation}
  |f(t)\rangle = U(t)|f(0)\rangle + U(t) \int_0^t ds\, U^{-1}(s)|g(s)\rangle,
\end{equation}
where $|\tilde{g}(t)\rangle = U^{-1}(t)|g(t)\rangle$.  
If we assume we can combine  the operators, then 
\begin{equation}
  f(\vect{x}_f,t) = \langle \vect{x}_f|U(t)|f(0)\rangle + \int_0^t ds \langle \vect{x}_f|U(t-s)|g(s)\rangle.
\end{equation}

\subsection{Evaluating the matrix element}

We can now evaluate the matrix elements for each piece.  In particular, 
\begin{equation}
  M = \langle \vect{x}_f|U(t)|f\rangle,
\end{equation}
where $U$ is the time-ordered evolution operator (so time increases towards the left.)
\begin{equation}
  U(t) = T\exp\left[-\int_0^t du\, H(u) \right] = \prod_{n=1}^N e^{-\Delta t \op{H}(n\Delta t)}
\end{equation}
The path integral litany of 1) splitting evolution operator into short-time evolution operators,
2) inserting identities, 3) splitting position and momentum operators, 4) replacing operators with eigenvalues
5) integrating out momenta 
can be followed once again. 
If we insert position/momentum identities between each of these, we have
\begin{align}
  M_n %&= \langle \vect{x}_f| e^{-\Delta t \op{H}(t_N)}|f\rangle%\\
%   &= \int \frac{d\vect{p}_N}{(2\pi)^{D/2}}\langle \vect{x}_N|e^{-\Delta t \op{H}(t_N)} |\vect{p}_N\rangle \langle \vect{p}_N|
% \prod_{j=1}^{N-1} e^{-\Delta t \op{H}(t_j)}|f\rangle\\
  = e^{-\lambda t}\int \prod_{k=0}^{N-1}\frac{d\vect{x}_{k}d\vect{p}_k}{(2\pi)^{D/2}}
  \prod_{j=1}^{N}\left[\langle \vect{x}_{k+1}| e^{-\Delta t \op{H}(t_k)}|\vect{p}_k\rangle
    \langle \vect{p}_k| \vect{x}_{k}\rangle \right]
  \langle \vect{x}_0| f\rangle
\end{align}
(Note that since $\lambda$ in constant, $e^{-\lambda \cT}$ can be factored out immediately)
The exponential operator is split into position and momentum pieces using the Baker-Campbell-Hausdorff theorem,
\begin{equation}
  e^{-\Delta t \op{p}^2+V(\op{x})} = e^{-\Delta t V(\op{x})}e^{-\Delta t \op{p}^2} +\order(\Delta t^2).
\end{equation}
The position and momentum operators then acquire the eigenvalues from operating to the left and right respectively.
After carrying out the Gaussian momentum integrals, the matrix element is 
\begin{align}
  M_n %  &= \int \prod_{k=0}^{N-1}\frac{d\vect{x}_{k}d\vect{p}_k}{(2\pi)^{D}}
  % \prod_{j=1}^{N} e^{-\Delta t\left[\vect{p}_k^2/2 + V(\vect{x}_k,t_k)\right]+i\vect{p}_k\cdot(\vect{x}_{k+1}-\vect{x}_k)}
  % f(\vect{x}_0,0)\\
&= e^{-\lambda t}\int \prod_{k=0}^{N-1}\frac{d\vect{x}_{k}}{(2\pi\Delta t)^{D/2}}
  \prod_{j=1}^{N} e^{-(\vect{x}_{k+1}-\vect{x}_k)^2/(2\Delta t)-\Delta t V(\vect{x}_k,t_k)}f(\vect{x}_0,0)
\end{align}
This is the traditional discrete form of the euclidean path integral.
  We can transform it to look like the mathematical version, 
stressing the connection to Brownian motion by changing integration variables.  
We define the vector Wiener increments $\Delta \vect{W}_k = \vect{x}_{N-k-1}-\vect{x}_{N-k}$ 
(note that this is labelling moves backwards in time from the usual convention).  
Then we can write the positions in terms of the Brownian motion
$\sum_{k=0}^{j}\Delta \vect{W}_k = \vect{x}_{N-j}-\vect{x}_N$.
Note that evaluate the solutions at a constant fixed position $\vect{x}_N=\vect{x}_f$
So our coordinates should be defined relative to that point, which prompts defining
\begin{equation}
%  \sum_{k=0}^{j} \Delta \vect{W}_k = \vect{x}_{N-j}-\vect{x}_N\rightarrow 
\vect{x}_j = \vect{x}_N+\sum_{k=0}^{N-j} \Delta \vect{W}_k= \vect{x}_N+\vect{W}_j
\end{equation}
where in the second equality we defined the Wiener process as $\vect{W}_j:=\sum_{k=0}^{N-j} \Delta \vect{W_j}$.
In terms of the Wiener paths, the path integral is
\begin{align}
  M_n %  &= \int \prod_{k=0}^{N-1}\frac{d(\Delta \vect{W}_{k})}{(2\pi\Delta t)^{D/2}}
  % \prod_{j=1}^{N} e^{-(\Delta \vect{W}_{k})^2/(2\Delta t)-\Delta t V(\vect{x}_N+\sum_{j=0}^{N-k}\Delta \vect{W}_j,t_k)}
  % f\big(\vect{x}_N+\sum_{j=0}^{N-j}\Delta \vect{W}_j,0\big)\\
&= e^{-\lambda t}\int \prod_{k=0}^{N-1}\frac{d(\Delta \vect{W}_{k})}{(2\pi\Delta t)^{D/2}}
  \prod_{j=1}^{N} e^{-(\Delta \vect{W}_{k})^2/(2\Delta t)-\Delta t V(\vect{x}_N+\vect{W}_{N-k},t_k)}
  f(\vect{x}_N+\vect{W}_{N},0).
\end{align}
After taking the continuum limit $N\rightarrow\infty$, where the Riemann sum becomes an integral, and 
the stochastic paths $\vect{W}_k=\vect{W}(t_k)$ are continuously varying stochastic processes 
 is a continuous process we can write
\begin{align}
  M_n  &= \biggdlangle \exp\bigg[-\lambda t-\int_0^t du\, V(\vect{x}+\vect{W}(t-u),u)\bigg] f(\vect{x}+\vect{W}(t),0)\biggdrangle
\end{align}
The same style of reasoning can be used for both pieces.  \comment{Note however change in time limits on $g$
term.  Also do the time arguments of $g$ work out?}

After plugging this result back in to the evolution operator, the solution to the diffusion equation is 
\begin{align}
  f(\vect{x},t) = &\biggdlangle \exp\bigg[-\lambda t -\int_0^t du\, V[\vect{x}+\vect{W}(t-u),u]\bigg] f[\vect{x}+\vect{W}(t),0]\biggdrangle\nonumber\\
  &+\int_0^t ds\,\biggdlangle \exp\bigg[-\lambda s -\int_0^s du\, V[\vect{x}+\vect{W}(s-u),u]\bigg] g[\vect{x}+\vect{W}(s),s]\biggdrangle
  \label{eq:feynman-kac}
\end{align}

\subsection{Steady State Brownian Bridge Path Integral}

In worldline path integrals such as Eq.~(\ref{eq:TE_worldline}), the Brownian motions are pinned to close on themselves.
We will follow a solution outline used by Hooghiemstra~\cite{Hooghiemstra2002}, 
and expanded upon and explained by Steck in \S 17.9 -- \S 17.11~\cite{SteckNotes}.

An analytical form for the path integral can be derived in conjunction with the diffusion equation.  
In what may seem like circular logic, the diffusion equation can be solved, and by 
the Feynman-Kac formula that solution is an analytical expression for a path integral.  
These results from simple planar geometries form a possible basis for more complicated geometries,
since they can be chained together in a spatially dependent manner.  

In the worldline path integral, we require path integrals for path-averaged potentials with closed paths.
The solution to the diffusion equation can be converted into the right form with some manipulation.

First, take the $t\rightarrow\infty$ limit.  In the steady state limit, the initial
condition is irrelevant, so for simplicity it can be set to zero.
Second, the source function can be used to construct the path pinning by setting $g(\vect{x})=\delta(\vect{x}-\vect{c})$, so
that only paths at satisfying $\vect{x}(t)=c$ contribute.  After those manipulations, the general solution is
\begin{align}
  f(\vect{x}) = \int_0^{\infty} ds\,
  \biggdlangle \delta[\vect{x}+\vect{W}(s)-\vect{c}] e^{-\lambda s -\int_0^s du\, V[\vect{x}+\vect{W}(s-u),u]}\biggdrangle,
  \label{eq:f_soln}
\end{align}
which has the form of a Laplace transform in $\lambda$.  
The $\delta$-function selects out paths that satisfy $\vect{W}(s)=\vect{c-x}$. In order to select
out Brownian paths starting from $\vect{x}=0$ and propagating to $\vect{x}=\vect{c}$, we should 
consider the solution at the origin $f(\vect{x}=0)$.  

While the full path integral is concerned with closed paths where $\vect{c}=0$, 
these path integral expressions will also be used to
accelerate convergence for discrete steps along a path, from one discrete position $\vect{x}_{k}$ to 
another $\vect{x}_{k+1}$.  That corresponds requires shifting all coordinates in the problem so that 
$\vect{x}_k=0$, and $\vect{c}=\vect{x}_{k+1}-\vect{x}_k$.  

The path integral (\label{eq:f_soln}) satisfies the following steady-state diffusion equation
\begin{equation}
  0 = \frac{1}{2}\nabla^2f(x) - (V+\lambda)f + \delta(\vect{x}-\vect{c}),
\end{equation}
which for simple geometries can be solved analytically.  The closed form for the path
integral is then found by inverting the Laplace transform in Eq.~(\ref{eq:f_soln}).

This might seem circular having passed from wave equations we cannot solve, to path integrals,
and back to diffusion equations we can only solve in particular geometries.
However, our goal with the path integral is to stitch together results for 
simple geometries together to handle more complicated geometries.
For example, the notion of locally planar could be defined in terms of the nearest surface-normal.  
This is not the same approximation as the proximity force approximation, 
which approximates the bodies \emph{globally} by their planes.
This is a local approximation based on a particular point in space, 
and varies as the path moves through space.



%  \item {Fokker-Planck Equation on Hilbert space.  }
%  \item {Split operators into $N$ steps.  Baker-Campbell-Hausdorff}
  %   Then, do the usual path integral tricks:
  %   \begin{align}
  %     \langle x_N|f(t)\rangle =& \langle x_N|\exp\left[-t\frac{\op{p}^2}{2}-tV(\op{x})\right]|f\rangle \\
  %     % =& \langle x_N|\prod_{k=1}^N\exp[- t/N(\frac{\op{p}^2}{2}+V(\op{x})]|f\rangle \\
  %     % =& \langle x_N|\exp[- t/N(\frac{\op{p}^2}{2}+V(\op{x})] \ldots \exp[- t/N(\frac{\op{p}^2}{2}+V(\op{x})]|f\rangle \\
  %     % =& \int \prod_{k=0}^{N-1}dx_k \langle x_N|\exp[- \Delta t(\frac{\op{p}^2}{2}+V(\op{x})]|x_{N-1}\rangle\langle x_{N-1}| \ldots \langle x_1|\exp[- \Delta t(\frac{\op{p}^2}{2}+V(\op{x})]|x_0\rangle\langle x_0|f\rangle \\
  %     % =& \int \prod_{k=1}^N\frac{dx_{k-1}dp_k}{(2\pi)}\prod_{j=1}^N e^{-\frac{\Delta t}{2}p_j^2 +ip_j(x_j-x_{j-1}) - V(x_j)\Delta t}\langle x_0|f\rangle \\
  %     =& \int \prod_{k=0}^{N-1}\frac{dx_k}{\sqrt{2\pi \Delta t}}e^{-\frac{(x_{k+1}-x_k)^2}{2\Delta t} - \Delta t V(x_k)}f(x_0)
  %   \end{align}
  %   Now somehow transform this to something like $f(x_n + \sum_{j}dW_j)$ to get the same form as Dan?
  %   Transform to integrating over the increments $dW_j = x_{j+1}-x_j$.
  %   Then $\sum_{j=0}^{N-1} dW_j = x_N - x_0$.
  %   Well, just quickly solve, and you see that $x_0 = x_N - \sum_{j=0}^{N-1}dW_j$.
  %   Similarly, $x_k = x_N - \sum_{j=k}^{N-1}dW_j$.  

  % \item {Continuum limit.  Note source of walks}
  %   \begin{align}
  %     f(x_N,t)=& \int \prod_{k=0}^{N-1}\frac{dx_k}{\sqrt{2\pi \Delta t}} e^{-\frac{(x_{k+1}-x_k)^2}{2\Delta t} - \Delta t V(x_k)}f(x_0)\\
  %     =& \int \prod_{k=0}^{N-1}\frac{d(dW_k)}{\sqrt{2\pi \Delta t}} e^{-\sum_k\frac{dW_{k+1}^2}{2\Delta t} - \sum_k\Delta t V(x_N-\sum_{j=k}^{N-1}dW_j)}f(x_N-\sum_{j=0}^{N-1}dW_j)\\
  %     =& \int \prod_{k=0}^{N-1}\frac{d(dW_k)}{\sqrt{2\pi \Delta t}} e^{-\sum_k\frac{dW_{k+1}^2}{2\Delta t} - \sum_k \Delta t V(x_N+\sum_{j=0}^{k}dW_j)}f(x_N+\sum_{j=0}^{N-1}dW_j),
  %   \end{align}
  %   or in continuous language:
  %   \begin{equation}
  %     f(x,t)= \dlangle e^{-\int_0^t dt'  V[x+W(t')]}f_0[x+W(t)]\drangle
  %   \end{equation}
  % \item Note time-ordered product.
  %   Note that in the case of $V(x,t)$ the exponential is really the time ordered product of these things.
  % \item $\delta$-pinning at $x=0$.  
  %   Then if we take $g(x) = \delta(x)$, then the Brownian walks will be restricted to return to the origin.
  %   Note that I think this definition also assumes that the brownian walks are starting from the origin.  
   Discuss notions of Brownian motion, Brownian bridges.  Specify in terms of increments,
    final positions.  Introduce measures like correlation?

  % \item For actual solutions work in steady-state limit.  
  % \item {Work with Laplace transform.  }
  %   Now let us consider steady state, for potentials $V(x,t) = V(x)$.
  %   In this case, we can drop the initial condition, and take $f(x,t=0)=0$, as the steady state is insensitive to the initial condition.
  %   In addition, we take $g(x,t)=g(x)$.   then as $t\rightarrow \infty$ we have 
  %   \begin{equation}
  %     f(x) = \dlangle \int_0^\infty ds\,g(x) e^{-\lambda s-\int_0^s du V[x(s)]} \drangle,\label{eq:path_int_solution}
  %   \end{equation}
  %   which satisfies 
  % \item Take $g(x)=\delta(x)$.
  % \item Solve associated diffusion equation.  (So works for separable geometries.  But foundational
  %   for approximations suited step-wise basis.)
  % \item Identify value of $f(x=0)$ as path integral.  Different location choices will yield different
  %   functions.  
   % Particularly interested in methods that could serve as approximations beyond PFA.  
   %  Interested in results that can serve as \emph{local} approximations exploiting planes.  

   % From here on, this is mostly a collection of solutions to Fokker Plank equations with various potentials,
   %  notably steps, and sundry gradients of steps.
   %  The overall procedure is the same, just the potential changes.
   %  Later we will have need of these results in our analytical calculations.  The planar open
   %  path results are essential for numerical techniques.      

    The remainder of this chapter is devoted to solving the diffusion equation~(\ref{eq:diffusion_eq})
    for some simple planar geometries.
    These geometries are important for analytically calculating Casimir and Casimir--Polder
    energies from the worldline path integrals.  
    
\section{Sojourn Time and One Step Potential }

As a first example, consider a step potential $V=\chi\Theta(x-d)$, where $\Theta(x)$ is the Heaviside step
function.  The Heaviside step function $\Theta(x)$ is unity for positive arguments, zero for negative arguments 
and takes a value of $1/2$ at the origin.
The step potential will be useful in computing the Casimir--Polder energy for an atom 
above a dielectric half-space in the worldline formalism.
It is also related to the Sojourn time statistic, which measures the time a Brownian motion $x(t)$ spends 
beyond a certain point $d$,
\begin{equation}
  T_S:= \int_0^T dt\,\Theta[x(t)-d].
\end{equation}
(The following is a much briefer discussion of the sojourn time, 
which is discussed for various Brownian motions by Hooghiemstra who used the procedure 
we follow~\cite{Hooghiemstra2002}.  Hooghiemstra's derivation is expanded and discussed in more detail
in \S17.10 of Steck~\cite{SteckNotes}.)

First we must solve
\begin{equation}
  0 = \frac{1}{2}\partial_x^2 f - [\chi\Theta(x-d)+\lambda]f + \delta(x).
\end{equation}
We initially assume that the surface is to the right of the origin, $d>0$.
In this work we will use an explicit delta function, rather than a Fourier representation.

In general the solutions are of the form, 
\begin{equation}
  f(x) = A e^{\kappa x} + B e^{-\kappa x},
\end{equation}
where $\kappa$ will be determined to satisfy the equation in each region, and $A$ and $B$ will be determined by 
the boundary conditions.
At step discontinuities (such as at $x=d)$ the solution and its derivative must be continuous across the surface,
    \begin{equation}
      \partial_xf(d+0_+) - \partial_x f(d-0_+) = 0, \qquad f(d+0_+)-f(d-0_+) = 0,  
    \end{equation}
    where $0_+$ is a small positive real number.
At a $\delta$-function singularity the derivative of the solution is discontinuous, but the function
is continuous.
    \begin{equation}
      \partial_xf(0_+) -\partial_x f(-0_+) = -2 , \qquad f(d+0_+)-f(d-0_+) = 0.
    \end{equation}
Both of these relations follow from integrating the diffusion equation across the relevant discontinuity.  
Then for $d>0$, taking the bounded solution in each region, we have 
\begin{equation}
  f(x) =\left\{ 
    \begin{array}{lcr}  A e^{\sqrt{2\lambda} x} & \hspace{2cm} & x<0\\
      B e^{\sqrt{2\lambda}x} + Ce^{-\sqrt{2\lambda}x} & \hspace{2cm} & 0<x<d\\
      D e^{-\sqrt{2(\lambda+\chi)}x} & \hspace{2cm} & x>d
    \end{array}
  \right.
\end{equation}
where the coefficients are fixed by matching the boundary conditions together at the interfaces at
$x=0$ and $x=d$.  This was done using Mathematica to speed up the tedious algebraic work.  
We are ultimately only interested in $f(x=0)$, which in this case means we just need to know $A$.
Algebra, or a suitable software package such as Mathematica, shows that 
\begin{equation}
  A = \frac{1}{\sqrt{2\lambda}} - u\supTE\,e^{-2\sqrt{2\lambda}d},
\end{equation}
where
\begin{equation}
  u\supTE = \frac{\sqrt{\lambda} -\sqrt{\lambda+\chi}}{\sqrt{\lambda} + \sqrt{\lambda+\chi}}.
\end{equation}
The reflection coefficient $u\supTE$ plays the same role as the TE-reflection coefficients in the Lifshitz 
formula.  
A similar computation can be carried out for $d<0$, which corresponds to finding the solution inside
the medium.
For $d<0$, can get similar result from $D$, if $\lambda \leftrightarrow \lambda+\chi$.
Ultimately, we find that 
We can pull these results together to write
\begin{equation}
  f_{TE,1}(x) = \left\{\begin{array}{lcr} 
      \dfrac{1}{\sqrt{2\lambda}}\left[1+ u\supTE\, e^{-2\sqrt{2\lambda}d}\right]  & \hspace{2cm} & d<0\\
      \dfrac{1}{\sqrt{2(\lambda+\chi)}}\left[1 - u\supTE\, e^{-2\sqrt{2(\lambda+\chi)}d}\right] & \hspace{2cm} & d>0\\
    \end{array} \right. 
\end{equation}
The subscript deontes the relevance of this solution to a single body in the TE polarization.  
Ultimately, we need to relate the solution to the diffusion equation back to the path integral via
\begin{equation}
 f\subTE= \int_0^\infty d\cT e^{-\lambda \cT} \biggdlangle \delta(x)e^{-\chi \theta[x(\cT)-d]}\biggdrangle_{W(t)}  
 = \int_0^\infty d\cT e^{-\lambda \cT} \biggdlangle \frac{e^{-\chi \int_0^\cT dt\,\theta[x(\cT)-d]}}{\sqrt{2\pi \cT}}\biggdrangle_{x(\cT)}  
  \label{eq:Feynman-Kac TE one step}
\end{equation}
where the first ensemble average is over free Wiener paths, while the second is over Brownian bridges that satisfy $x(0)=x(\cT)=0$.
The factor of $(2\pi T)^{-1/2}$ is normalization for the use of closed Brownian bridges.  
In this case, we will not undo the Laplace transform, since later results can convert the relevant 
Casimir energies to exploit this form of the energy. 

It is possible to generalize this calculation to compute the equivalent formulae for open Brownian
bridges from $0\rightarrow c$, as discussed in App. B of Ref.~\cite{Mackrory2016}.
These formulae may be useful in accelerating numerical techniques, with relatively coarse bridges.
However the formulae are quite complicated and involve diverging integrands, which makes it difficult to implement them 
efficient.

   % Identify similarity to reflection coefficients
   % {Quote result for integration over surface.}

   %  We now need to evaluate $\int dx_0 \dlangle e^{-\int_0^T dt V[x_0 + B(t)-h]}\drangle$ for use with Casimir energies.
   % Now we have to take $h\rightarrow h-x_0$, and integrate over $x_0$.
   %  The integrals over position is 
   %  \begin{align}
   %    I_{TE,1} &= \int_{-\infty}^h dx_0 \frac{1}{\sqrt{2(\lambda+\chi_1)}}\left(1 - u_1e^{-2\sqrt{2(\lambda+\chi_1)}(h-x_0)} \right) 
   %    + \int_h^\infty dx_0 \frac{1}{\sqrt{2\lambda}}\left(1 + u_1 e^{2\sqrt{2\lambda}(h-x_0)}\right) \\
   %    % &= \int_h^\infty dx_0[(2\lambda)^{-1/2}+(2\kappa)^{-1/2}]  -  \int_{-\infty}^0  dx_0 \frac{1}{\sqrt{2\kappa_1}}u_1e^{2\sqrt{2\kappa_1}x_0} + \int_0^\infty dx_0 \frac{1}{\sqrt{2\lambda}}u_1 e^{-2\sqrt{2\lambda}x_0}\\
   %    &= I^{(1)}_{div}  +   \left(\frac{1}{4\lambda}- \frac{1}{4(\lambda+\chi_1)}\right)u_1,
   %  \end{align}
   %  where $I^{(1)}_{div} = \int_h^\infty(2\lambda)^{-1/2}+\int_{-\infty}^h[2(\lambda+\chi_1)]^{-1/2}$.  This renormalization is only necessary for the energy.  


\subsection{Planar Dirichlet Conditions}

It is possible to take the strong-coupling limit where $\chi\rightarrow \infty$. 
In that Dirichlet limit, the reflection coefficient goes to -1.
The path integral solution is
\begin{equation}
  f_{D,1}(x) = \left\{\begin{array}{lcr} 
      \dfrac{1}{\sqrt{2\lambda}}\left[1 - e^{-2\sqrt{2\lambda}d}\right]  & \hspace{2cm} & d<0\\
      0 & \hspace{2cm} & d>0\\
    \end{array} \right. 
\end{equation}
The path integral can be converted back to the time domain by inverting the Laplace transform with
\begin{equation}
  \mathcal{L}^{-1}\bigg(\frac{1}{\sqrt{2\lambda}}e^{-2\sqrt{2\lambda}d}\bigg) 
  = \frac{1}{\sqrt{2\pi t}}e^{-d^2/(2t)}.\label{eq:Laplace_Gaussian}
\end{equation}

This result can straightforwardly be generalized to open Brownian bridges for paths between $x$ and $y$, and a surface at $d$.
\comment{so do that?}
The solution is  
\begin{equation}
  f_{D,1}(x,y) = \left\{\begin{array}{lcr} 
      \dfrac{e^{-(x-y)^2/(2t)}}{\sqrt{2\pi t}}\left[1 - e^{-2(d-x)(d-y)/t}\right]  & \hspace{0.5cm} & (x-d)(y-d)>0\\
      0 & \hspace{0.5cm} & (x-d)(y-d)<0\\
    \end{array} \right. 
\end{equation}
The lower solution applies when $x$ and $y$ are on different sides of the surface and the path 
must cross through.  The upper solution applies when the points are both on the same side.
As the points $x,y$ get closer to the surface, the probability of touching the surface increases, and the amplitude
decreases.

This analytical could be applied to enhance Dirichlet worldline methods with little cost,
since path resolution is one of the dominant sources of error in numerical worldlines~\cite{Mackrory2016}.

% \begin{enumerate}
%   \item Note connection to Gies work (they don't use this).    Note we will discuss this in
%     accelerated convergence techniques.  
%   \item Take delta function potential a distance $d$ away.  Use open loops.  
%   \item Quote potential
% \end{enumerate}

\section{Two Step Potentials}

The next case of importance is for two step potentials, where $V=\chi_1\Theta[d_1-x]+\chi_2\Theta[x-d_2]$.
The bodies are separated by a distance $d=d_2-d_1$.
This potential is useful for analytically calculating the Casimir energy between two dielectric half-spaces,
or the Casimir--Polder energy for an atom between two half-spaces.
The same procedure as one half-space can be used, albeit with another set of boundary conditions to manage.   
The solution must be continuous at both surfaces $x=d_1,d_2$, and have continuous first derivative.

    % When we do the Casimir energy between two bodies we will need to find the Feynman-Kac formula assuming two step discontinuities.
    % We will use exactly the same procedure as above, but with an extra step. 

    % Here we are solving this with $V = \chi_1\Theta(-d_1-x) + \chi_2\Theta(d_2-x)$.


% For the case of 
%     We have solutions, 
%     \begin{equation}
%       f(x) = \left\{ \begin{array}{lcr}
%           A e^{\sqrt{2(\lambda+\chi_1)}x}   & \hspace{1cm} & x<0\\
%           B e^{\sqrt{2(\lambda+\chi_1)}x} + C e^{-\sqrt{2(\lambda+\chi_1)}x}  & \hspace{1cm} & 0<x<h\\
%           D e^{\sqrt{2\lambda}x} + F e^{-\sqrt{2\lambda}x}  & \hspace{1cm} & h<x<h+d\\
%           G e^{-\sqrt{2(\lambda+\chi_2)}x} & \hspace{1cm} & x>d_2
%         \end{array}
%       \right.
%     \end{equation}
%     We will take $d_1 = h, d_2 = d+h$.
%     We can then check that our final solutions are independent of $h$ once we have integrated over position.
%     We would expect the Casimir energy to only depend on $d$ in this case.   

Skipping what is some tedious algebra, the solution for closed Brownian paths can be be written 
in each of three regions.  First, for paths starting inside the left hand body, when $d_1>x_0$, 
\begin{equation}
  f_{\text{TE},12}^{(I)}(x_0)= \dfrac{1}{\sqrt{2(\lambda+\chi_1)}} 
  + e^{-2\sqrt{2(\lambda+\chi_1)}(d_1-x_0)}\dfrac{u\supTE_2 e^{-2\sqrt{2\lambda}d} - u\supTE_1}
  {\sqrt{2(\lambda+\chi_1)}(1-u\supTE_1u\supTE_2 e^{-2\sqrt{2\lambda}d})}
\end{equation}
where the reflection coefficients are given by
\begin{equation}
  u\supTE_i = \frac{\sqrt{\lambda} -\sqrt{\lambda+\chi_i}}{\sqrt{\lambda} + \sqrt{\lambda+\chi_i}}.
\end{equation}
In between the half-spaces the solution is 
\begin{equation}
  f^{(II)}_{\text{TE},12}(x_0)=\frac{1}{\sqrt{2\lambda}} 
  + \dfrac{2u\supTE_1u\supTE_2 e^{-2\sqrt{2\lambda}d} + u\supTE_1 e^{2\sqrt{2\lambda}(d_1-x_0)} 
    +u\supTE_2 e^{-2\sqrt{2\lambda}(d_2-x_0)}}{\sqrt{2\lambda}(1-u\supTE_1u\supTE_2 e^{-2\sqrt{2\lambda}d})}
\end{equation}
Finally, in the right hand planar body, for $d_2<0$, the solution is 
\begin{equation}
  f^{(III)}_{\text{TE},12}(x_0) = 
  \dfrac{1}{\sqrt{2(\lambda+\chi_2)}} 
  + e^{2\sqrt{2(\lambda+\chi_2)}(d_2-x_0)}\dfrac{(u\supTE_1 e^{-2\sqrt{2\lambda}d}-u\supTE_2)}
  {\sqrt{2(\lambda+\chi_2)}(1-u\supTE_1u\supTE_2 e^{-2\sqrt{2\lambda}d})}.
\end{equation}
The ``12'' subscript denotes that both left and right or first and second bodies are present.
The equivalent single body expressions can be derived from these.  For example, the solution for body 1 can
be found by keeping $d_1$ fixed, but setting $d=d_2$ to infinity.  

We will not attempt to invert the Laplace transform for these two-body formulae.  
Note the presence of $1-u\supTE_1u\supTE_2e^{-2\sqrt{2\lambda}d}$ in the denominators.  Given we are solving essentially the same
problem as when we derived the Lifshitz formula in Sec.~\ref{sec:lifshitz}, this is perhaps not surprising. 
However, in this case there is no need to make any conditions on which modes contribute.  
Later we will convert the worldline energy into a suitable form to use these results as written.  

In the strong coupling ($u\supTE_i\rightarrow 1$) limit, 
the Laplace transform can be inverted by expanding the denominator of these expressions using 
$(1-s)^{-1}=\sum_{n}(-1)^n s^n$ for $s<1$.  The $n^{\text{th}}$ order term corresponds to the $n^\text{th}$ reflection
from the far surface and the governing exponential term would be of the form $e^{-2\sqrt{2\lambda}(n+1)d}$.
After inverting the Laplace transforms, each Gaussian in the sum would be of the form $\exp[-2(n+1)^2d^2/t]$, corresponding
to the probability for a Brownian path to bounce $n$ times between points a distance $d$ apart.
This naturally goes over to the reflection picture discussed in \S21.1.5.3 of Steck~\cite{Steck}.

\section{Feynman-Kac formula for Singular Potentials}

The preceding has been a prelude for the truly important work of this method.
Exactly the same methods can be used to yield sensible, finite results for the path integral 
against singular potentials.  This is particularly true for $V\subTE$ and $V\subTM$, where 
\begin{equation}
  V\subTM(x) = \frac{1}{2}\big[(\partial_x\log\sqrt{\epsilon})^2-\partial_x^2\log\sqrt{\epsilon}\big].
\end{equation}
(We will focus on $V\subTM$, since $\mu=1$ for almost all materials.)
If we consider a dielectric with a step-function discontuity, $\epsr = 1+\chi\Theta(x-d)$, the resulting
potential is highly singular.  
The first derivative of a step-function is a $\delta$-function, and 
the second derivative is $\delta'$.  We define a parameter $\Xi:=\log\sqrt{1+\chi}$, so that we can write
$\log\sqrt{\epsr}(x) = \Xi\theta(x-d)$, and the potential is
\begin{equation}
  V\subTM(x) \sim \frac{1}{2}\big[\Xi^2\delta^2(x-d)-\Xi \delta'(x-d)\big].
\end{equation}
As written this is horrifically divergent, and does not make sense (what is the effect of $\delta^2$ under an integral?). 
The only way to make sense of this is to regularize the singularity in the step-function,
and take the limit of vanishing regularization at the end of the computation. 

We will consider an exponential interpolation of the dielectric between the two values $1$ and $1+\chi$,
over a distance $a$,
\begin{equation}
  \epsr\supa(x)= \left\{\begin{array}{ccr} 
      1 & \hspace{0.5cm} &x<d\\ 
      e^{2\Xi (x-d)/a} & \hspace{0.5cm}&d<x<d+a\\
      1+\chi & \hspace{0.5cm}&x>d+a
    \end{array}.
  \right.
\end{equation}
The logarithm of the dielectric can then be written
\begin{equation}
  \log\sqrt{\epsr\supa}(x)  
= \left\{\begin{array}{ccr} 
      0 & \hspace{0.5cm} &x<d\\ 
      \frac{\Xi}{a}(x-d)& \hspace{0.5cm}&d<x<d+a\\
      \Xi & \hspace{0.5cm}&x>d+a
    \end{array}.
  \right.
\end{equation}
The regularized TM potential is now given by 
\begin{equation}
  V\subTM\supa(x) = \frac{\Xi^2}{2a^2}\Theta(x-d)\Theta(d+a-x)-\frac{\Xi}{a}[\delta(x-d)-\delta(x-d-a)\big].
\end{equation}
The worst singularity present are the $\delta$-functions from the second derivative.
While the potential is still horribly singular, and cannot be handled nicely for just a single path, the ensemble
averaged expression over many such paths is well behaved.  And that ensemble averaged expression can be
found by solving the associated diffusion equation.  Fortunately, this is smooth enough to get sensible results.


\begin{figure}
  \hspace{2cm} FIGURE FROM POSTER
  \caption[Regularized TM Potential]{Schematic drawing of a regularized TM potential at a surface. 
    Vertical arrows denote $\delta$-functions.}
\end{figure}

\begin{enumerate}
  % \item Smooth out contribution by analytically averaging over subpaths.
  %   In handling the TM case we will first have need to handle that singular potential.
  % \item We shall do this first on it's own, since we will have to use those results in any numerical method.
  %   In addition, it vastly simplifies down.
  %   We will find that the potential enforces a boundary condition.  

  \item Quote PDE, and exact form of the potential
    We will find the solution to 
    \begin{equation}
      \partial_t f = \frac{1}{2}\partial_x^2f -V_{TM} f - \lambda f + \delta(x-c)
    \end{equation}
    where 
\end{enumerate}

\subsection{Transfer Layer Boundary conditions for TM Potential}

As a first step, we will show that the TM potential imposes an effective boundary condition at the interface. 
The diffusion equation can be solved inside the boundary layer $x\in(d,d+a)$, and all references 
to the interior can be eliminated.  We will then have conditions relating the solution and its derivatives
on either side of the surface.  Since we will eventually consider the surface to be infinitesimally thin,
we will not consider pinning a point inside the surface.  

The analytical solutions for the path integral obey
\begin{equation}
  \frac{1}{2}\partial_x^2f =\frac{1}{2}\bigg[\lambda+\frac{\Xi^2}{a^2}\Theta(x-d)\Theta(d+a-x)
  - \frac{\Xi}{a}[\delta(x-d)-\delta(x-d-a)]\bigg]f
\end{equation}
Since $a$ is small, $\Xi/a$ is large relative to $\lambda$, which can be ignored in the thin-surface limit.    
As with the other cases, the solutions are exponentials of the form $f = \alpha_+ e^{\kappa x}+\alpha_- e^{-\kappa x}$, 
where $\kappa$ is chosen to satisy the differential equation, and $\alpha_\pm$ are fixed by the boundary conditions.
At each delta function $(\Xi/a)\delta(x-s)$, the solution $f$ has a jump in the derivative, but is still continuous
\begin{equation}
  f'(s+\epsilon)-f'(s-\epsilon) = \frac{\Xi}{a} f(s),\qquad f(s+\epsilon)-f(s-\epsilon) = 0.
\end{equation}

Let $f_{\text{mid}}$ be the solution in the middle region for $x\in(d,d+a)$,   
    \begin{align}
      f_{\text{mid}} %=& Be^{\sqrt{2\lambda + \Xi^2/a^2}x} + C e^{-\sqrt{2\lambda + \Xi^2/a^2}x}\\
      =& Be^{\Xi x/a} + C e^{-\Xi x/a}.
    \end{align}
    Our goal is to derive the effective conditions imposed on the solutions just outside the surface.
    Let the solutions from the left be $f_1:=f(d-0_+)$, and the gradient $f_1:=f'(d-0_+)$,
    with corresponding solutions on the right $f_2:=f(d+a+0_+)$, $f_2':=f(d+a+0_+)$.  As far as this 
    calculation is concerned, these are unspecified real constants. 

    The continuity conditions at $x=d$ and $x=d+a$ require that
    \begin{subequations}
      \begin{align}
        f_{\text{mid}}(d)-f_1 &= 0\\
        f_2- f_{\text{mid}}(d+a)&= 0\\
        f_{\text{mid}}'(d) -f_1'&= -\frac{\Xi}{a} f_1\\
        f_2' -f'_{\text{mid}}(d+a)&= +\frac{\Xi}{a} f_2.
      \end{align}
    \end{subequations}
   These equations can be solved for the internal parameters $B$ and $C$, as well as two of the external 
   parameters, $f_2$ and $f_2'$.  As a result, there will be simple effective boundary conditions between
   the field values on either side of the boundary layer.  
   Given this calculation's importance to our subsequent, we will actually go through the elementary algebra.  

   % Find relations between $f(d\pm \epsilon)$ and $f'(d\pm \epsilon)$.
   %  We will solve for $f(d+a+\epsilon)$, and $f'(d+a+\epsilon)$,
   %  trying to eliminate $B$ and $C$, which we will fix in terms of $f(d-\epsilon),f'(d-\epsilon)$.
   %  We will thus have derived the relation between $f$ and its derivatives on both sides of the potential.      
    The continuity conditions on $f_1$ and $f_2$ require that
    \begin{align}
%      f(d+\epsilon) - f(d-\epsilon) =& 0\\
%      \rightarrow 
      f_1 &= Be^{\Xi d/a} + C e^{-\Xi d/a}\label{eq:M c1}\\
    % \end{align}
    % Similarly, at the second edge  we find that 
    % \begin{align}
    %   f(d+a+\epsilon) - f(d+a-\epsilon) =& 0\\
    %   \rightarrow 
      f_2 &= Be^{\Xi d/a+\Xi} + C e^{-\Xi d/a-\Xi}\label{eq:M c2}.
    \end{align}
    The derivative conditions then require that
    \begin{align}
%      f'(d +\epsilon)-f'(d-\epsilon) =& -\frac{\Xi}{a}f(d)\\
%      \rightarrow f'_1%=& \frac{\Xi}{a}\left(Be^{\Xi d/a} + C e^{-\Xi d/a}\right) + \frac{\Xi}{a}\left(Be^{\Xi d/a} - C e^{-\Xi d/a}\right)\\
      f_1'&= 2\frac{\Xi}{a}Be^{\Xi d/a} \label{eq:M d1}\\
    % \end{align}
    % And the derivative conditions at $d+a$ yield 
    % \begin{align}
       % f'(d+a+\epsilon) -f'(d+a-\epsilon)=& \frac{\Xi}{a}f(d+a)\\
      % \rightarrow f'(d+a+\epsilon)=&\frac{\Xi}{a}\left(Be^{\Xi (d+a)/a} + C e^{-\Xi (d+a)/a} \right)+ \frac{\Xi}{a}\left(Be^{\Xi (d+a)/a} - C e^{-\Xi (d+a)/a} \right)\\
      f_2'&=2\frac{\Xi}{a}Be^{\Xi d/a}e^{\Xi}\label{eq:M d2}
    \end{align}
    From the derivative conditions, $f_1'$ and $f_2'$ can be related to one another as follows
%    Eq.~(\ref{eq:M d1}) fixes $B$, which we can use in to Eq.~(\ref{eq:M d2}), so that 
    \begin{equation}
      f'_2 = e^{\Xi}f'_1.
    \end{equation}
    In addition, under the assumption that the external gradients $f_1'$, $f_2'$ are order one,
    that also implies that $B\sim\order(a)$.  That implies that in the continuity conditions 
    $B$ can be dropped relative to $C$ which must also be order one.  On setting $B=0$, it is clear 
    from Eqs.~(\ref{eq:M c1})-(\ref{eq:M c2}) that 
% Next up the normal continuity condition.  
%     \begin{align}
%       f_1 =& B e^{\Xi d/a} + C e^{-\Xi d/a}\\
%       =& \frac{a}{2\Xi} f' + C e^{-\Xi d/a} = C e^{-\Xi d/a},
%     \end{align}
%     which says that $B\sim a$, so we will drop it from this part of the calculation.  
%     Now use this in Eq.~(\ref{eq:M d2}) to find
    \begin{equation}
      f_2 =  e^{-\Xi} f_1.  
    \end{equation}
    At this point, the conditions between the solutions outside the boundary layer have been derived,
    and the  $a\rightarrow 0$ limit can be taken everywhere.  
    % Eliminate references to internal state.  Take limit of regularization going to zero.
    % {Quote final boundary conditions.}
    % We can now take $a\rightarrow 0 $ everywhere. 
    
    The regularized dielectric step produces the following effective boundary matching conditions at an interface
    \begin{align}
      \Aboxed{
        f(d+0_+) =& e^{-\Xi}f(d-0_+), \qquad
        f'(d+0_+)= e^{\Xi}f'(d-0_+)
      }\label{eq:TM_boundary_conditions}
    \end{align}
    \label{sec:TM boundary condition}
    These boundary conditions can then be used to find the ensemble average solution.  This is essential
    to take into account analytically, since the chance that a randomly drawn path will satisfy these conditions 
    is negligible.  

\subsection{Finding the Feynman-Kac formula}

We can now find the Feynman-Kac formula for open Brownian bridges near a single TM-potential, which 
imposes the boundary conditions~\ref{eq:TM_boundary_conditions}.  This solution is the ensemble average 
over all Brownian paths between the end points.

We aim to find a closed form solution for the path integral over paths between starting at $x=0$ and terminating at $c$ after time $t$ and 
interacting with potential $V\subTM$,
\begin{equation}
  f\subTM(c,d,\Xi)=\int_0^\infty dt\, e^{-\lambda t}\frac{e^{-c^2/(2t)}}{\sqrt{2\pi t}} \dlangle e^{-\int_0^T dt V_{TM}}\drangle.
\end{equation}
The Gaussian factor in $c$ is the normalization for a 1D Brownian bridge between $x=0$ and $c$ in time $t$.
This path integral is found by solving the following diffusion equation
\begin{equation}
  0 = \frac{1}{2} \partial_x^2 f\subTM - (V\subTM+\lambda)f\subTM + \delta(x-c).
\end{equation}
In this case the there are TM boundary conditions~(\ref{eq:TM_boundary_conditions}) imposed at $x=d$,
and $\delta$-function boundary conditions imposed at $x=c$.  
The solutions naturally decompose into two cases: first,  when the end points of the path are on the same
side of the surface,  and when they are on different sides.
% We will find the solutions for the case where $0<c<d$.
% From this we can see how we need to change the signs to recover the other cases.
% Note that the boundary conditions at $x=c$ are: ${f(c+\epsilon)=f(c-\epsilon)}, {f(c+\epsilon)-f(c-\epsilon)= -2}$.
  
%  Form of solutions in each region, and fixing values via BCs from previous section
%     For example, with $c<d$ we have 
%     \begin{equation}
%       f  = \left\{\begin{array}{ccr} A e^{\sqrt{2\lambda} x} & \hspace{2cm} & x<c\\
%           B e^{\sqrt{2\lambda} x} + C e^{-\sqrt{2\lambda} x}  & \hspace{2cm} & c<x<d\\
%           D e^{-\sqrt{2\lambda} x}& \hspace{2cm} & c<x<d\\
%         \end{array}
%       \right. .
%     \end{equation}
%     In the case where $d>0$, we will need $f(0) = A$.  In the other case where $d<0$, we will need $f(0) = D$.  
%     Mathematica when asked to solve the boundary conditions finds that 
    % \begin{equation}
    %   f = \left\{ \begin{array}{ccr} 
    %       \dfrac{e^{-\sqrt{2\lambda}|c|}}{\sqrt{2\lambda}} + \dfrac{e^{-\sqrt{2\lambda}(2d-c)}}{\sqrt{2\lambda}}\dfrac{e^{2\Xi}-1}{e^{2\Xi} +1}  &   \hspace{2cm}  & d>c,  d>0\\
    %       \dfrac{ e^{-\sqrt{2\lambda}|c|}}{\sqrt{2\lambda}} \dfrac{2e^\Xi}{1 + e^{2\Xi}} & \hspace{2cm} & d>c,d<0 \\
    %       \dfrac{ e^{-\sqrt{2\lambda}|c|}}{\sqrt{2\lambda}} \dfrac{2e^\Xi}{1 + e^{2\Xi}} & \hspace{2cm} & c>d,d>0 \\
    %       \dfrac{ e^{-\sqrt{2\lambda}|c|}}{\sqrt{2\lambda}} - \dfrac{e^{\sqrt{2\lambda}(2d-c)}}{\sqrt{2\lambda}}\dfrac{e^{2\Xi}-1}{e^{2\Xi}+1} & \hspace{2cm} & c>d, d<0
    %       \\
    %     \end{array}
    %   \right.
    % \end{equation}
\comment{    Note how to recover various cases by flipping signs etc.  
    Simplify down to crossing/no-crossing cases.}
%    That seems complicated, but we can note that if there is a crossing ($|c|<|d|$) then we get 
    When both points are on the same side, or $d(d-c)>0$, the solution is 
    \begin{equation}
      f\subTM=\dfrac{e^{-\sqrt{2\lambda}|c|}}{\sqrt{2\lambda}} 
      + \sgn(d)\dfrac{e^{-\sqrt{2\lambda}|2d-c|}}{\sqrt{2\lambda}}\dfrac{e^{2\Xi}-1}{e^{2\Xi} +1}.
    \end{equation}
    When the paths must cross through the surface since $x=0$ and $c$ are on different sides of the surface, $d(d-c)<0$,
    the solution is
    \begin{equation}
      f\subTM = \dfrac{ e^{-\sqrt{2\lambda}|c|}}{\sqrt{2\lambda}} \dfrac{2e^\Xi}{1 + e^{2\Xi}}.
    \end{equation}
    In both cases the Laplace transforms in $\lambda$ can be inverted yielding Gaussians [following from Eq.~\ref{eq:Laplace_gaussian}].
    After cancelling out the Gaussian factor $e^{-c^2/(2t)}/\sqrt{2\pi t}$, the analytical expression for 
    the path integral can be isolated
    % Ultimately, we will want to isolate the potential term.  
    % If we use 
    % \begin{equation}
    %   \mathcal{L}^{-1}\left[ \frac{e^{-\sqrt{2\lambda}x}}{\sqrt{2\lambda}}   \right] = \frac{e^{-x^2/(2t)}}{\sqrt{2\pi t}},
    % \end{equation}
    % which is exactly the factor we isolated in front of $\mathcal{M}$.  
    \begin{align}
      \dlangle e^{-\int_0^t dt'\, V\subTM(x-d)}\drangle 
      &=\left\{ \begin{array}{lc} 
          1   + \sgn(d)\dfrac{e^{-2d(d-c)/t}}{\sqrt{2\pi t}}\tanh\Xi & d(d-c)>0\\
          \sech\Xi & d(d-c)<0
        \end{array}
        \right.
      \end{align}
      This result is absolutely crucial for developing numerical methods for the TM polarization.  
      Even regularized the potential is too unruly to handle on a single trajectory wise basis.  The analytical
      solution smooth the result out by averaging over all possible sub-paths, which tames the divergences. 
      The result can be extended to include any starting point $x_i$ by shifting $d\rightarrow d-x_i$,
      and identifying $c=x_f-x_i$, where $x_f$ is the final point.

      The potential has been plotted in Fig.~\ref{fig:TM_plot} as a function of ending point
      for paths starting inside and outside the barrier.  
      Note that the potential leads to larger values on the vacuum side of the material, 
      and suppresses values for starting points inside the surface or paths that must cross the surface.  

      The material constants can be rewritten in terms of $\chi$,
      \begin{align}
        \tanh\Xi &=\dfrac{e^{2\Xi}-1}{e^{2\Xi} +1} =\dfrac{\chi}{2+\chi}=\frac{\epsilon_{\text{r},1} -1}{\epsilon_{\text{r},1}+1}\\
        \sech\Xi &= \dfrac{2e^\Xi}{1 + e^{2\Xi}} = \frac{2\sqrt{1+\chi}}{2+\chi}.
      \end{align}
      In the strong-coupling limit $\tanh\Xi\rightarrow 1, \sech\Xi\rightarrow 0$, and the solution for 
      paths starting at the origin is
    \begin{align}
      f_N
      &=\theta[d(d-c)]\bigg( 1 +\sgn(d)\dfrac{e^{-2d(d-c)/t}}{\sqrt{2\pi t}}\bigg).
      \end{align}
      This is in some sense dual to the Dirichlet limit.  In that case paths reflect off with the opposite phase, 
      leading to a value of zero on the boundary.  Here, the paths reflecting off the surface add in phase, suggesting 
      a correspondence to Neuman boundary conditions.  However, this does not hold for paths starting 
      inside the surface, where the solution is the same as for Dirichlet boundary conditions.  

    \begin{figure}
      \hspace{2cm}
      PLOT OF TM SOLUTION
      \caption[Plot of TM Solution]{Plot of path-averaged TM solution for various ending points}
      \label{fig:TM_plot}
    \end{figure}
    

    % Dan has:
    % \begin{equation}
    %   \dlangle \exp\left[-\int_0^T dt V_{TM}(x-d)\right]\drangle = 1 + \frac{\sinh(\Xi/2)}{\cosh\Xi}[\sgn(d-c)e^{\sgn(d)\Xi/2} + \sgn(d)e^{-\sgn(d)\Xi/2}]e^{\left[c^2-(|d|+|c-d|)^2\right]/2t}.
    % \end{equation}
    % This agrees with my expressions for all cases.  This is useful for our numerical work.  
    % Let us check out the $\Xi$ prefactor against my work.  
    % \begin{align}
    %   F_{d>0,c<d} =&\frac{\sinh(\Xi/2)}{\cosh\Xi}[\sgn(d-c)e^{\sgn(d)\Xi/2} + \sgn(d)e^{-\sgn(d)\Xi/2}]\\
    %   =&\frac{\sinh(\Xi/2)}{\cosh\Xi}[e^{\Xi/2} + e^{-\Xi/2}] = \frac{e^{\Xi} - e^{-\Xi}}{e^\Xi+ e^\Xi}  
    % \end{align}
    % works.
    % \begin{align}
    %   F_{d>0,d<c} =&\frac{\sinh(\Xi/2)}{\cosh\Xi}[-e^{\Xi/2} + e^{-\Xi/2}] = -\frac{ e^{-\Xi} + e^{\Xi} -2}{e^\Xi + e^{-\Xi}},
    % \end{align}
    % works, 
    % \begin{align}
    %   F_{d<0,c<d} =\frac{\sinh(\Xi/2)}{\cosh\Xi}[e^{-\Xi/2} -e^{\Xi/2}]=\frac{[(e^{\Xi/2}- e^{-\Xi/2})(e^{-\Xi/2} -e^{\Xi/2})]}{e^\Xi + e^{-\Xi}}=-\frac{e^\Xi + e^{-\Xi} -2}{e^\Xi + e^{-\Xi}},
    % \end{align}
    % works, and 
    % \begin{align}
    %   F_{d<0,d<c} =&\frac{\sinh(\Xi/2)}{\cosh\Xi}[-e^{\Xi/2} -e^{-\Xi/2}] = -\frac{e^\Xi - e^{-\Xi}}{e^\Xi + e^{-\Xi}}
    % \end{align}

\section{Single TM potential and Step}

Next, we consider a dielectric step combined with a TM boundary condition.  
This is directly relevant to analytically computing the TM component of the Casimir--Polder energy for an atom near a
dielectric half-space.  We will only develop the solution for closed paths.

The path integral
    \begin{equation}
      f = \int_0^\infty dt\, e^{-\lambda t}\frac{1}{\sqrt{2\pi t}}\bigdlangle e^{-\int_0^t dt'\, [V\subTM(x,d) + \chi\Theta(x-d)]}\bigdrangle ,
    \end{equation}
    is the steady-state solution to 
    \begin{equation}
      \partial_t f = \frac{1}{2}\partial_x^2f -(V_{TM} + s\Theta(x-d)+\lambda)f +\delta(x). 
    \end{equation}
    The analytical expression is found by solving the diffusion equation directly, with 
    TM boundary conditions at $x=d$ and $\delta$-function boundary conditions at $x=0$.
    The solution for a single TM body is 
  \begin{equation}
      f_{TM,1}(x) = \left\{\begin{array}{lcr} 
          \dfrac{1}{\sqrt{2\lambda}}\left[1+ u\supTM e^{-2\sqrt{2\lambda}d}\right]  & \hspace{2cm} & d<0\\
          \dfrac{1}{\sqrt{2(\lambda+\chi)}}\left[1 - u\supTM e^{-2\sqrt{2(\lambda+\chi)}d}\right] & \hspace{2cm} & d>0\\
        \end{array} \right. 
    \end{equation}
    where
    \begin{equation}
      u\supTM = \frac{\sqrt{\lambda}e^{2\Xi} -\sqrt{\lambda+\chi}}{\sqrt{\lambda}e^{2\Xi} + \sqrt{\lambda+\chi}}.
    \end{equation}
    The reflection coefficient $u\supTM$ corresponds to the TM reflection coefficient 
    used in Sec.~\ref{sec:lifshitz}, since $e^{2\Xi}=1+\chi$.  
    This calculation can also be naturally extended to include a magnetic response.
    The parameter $\Xi$ is defined in relation to the boundary condition, while $\chi$ relates to the 
    step discontinuity.  To describe the TM potential, one could take $\chi\rightarrow (\epsr\mur-1)$
    and $\Xi\rightarrow \log\sqrt{\epsr}$, where $\epsr$ and $\mur$ denote the permittivity and permeability inside the medium.

    Ultimately, we find that 
    \begin{align}
      &\int_0^\infty dT e^{-\lambda T} \dlangle \frac{1}{\sqrt{2\pi T}}
      \exp\left(-\int_0^T dt \{V\subTM[x(t)]+s\theta[x(t)-d]\}\right)\drangle\nonumber\\
      &\hspace{1cm}=
      \frac{1}{\sqrt{2[\lambda+\chi\theta(d)]}}\left[1 - \sgn(d) u\supTM e^{-2\sqrt{2[\lambda+\chi\theta(d)]}|d|}\right],\label{eq:Feynman-Kac TM one step}
    \end{align}
    where the ensemble average is over brownian bridges that satisfy $x(0)=x(T)=0$.
    The factor of $\sqrt{2\pi T}$ is normalization for the use of the bridges.  


\section{Two TM Step Potentials}

The preceding calculations for can be easily adapted to handle two planar half-spaces subject to 
TM boundary conditions. This time consider the path integral solution for a potential
\begin{equation}
  V = \chi_1\Theta(-d/2-x)+\chi_2\Theta(x-d/2)+V\subTM(-\Xi_1,-d/2)+V\subTM(\Xi_2,d/2)
\end{equation}
In this case a little care is needed in defined the boundary conditions at the left hand surface.
Since the left body has the opposite orientation (the permittivity decreases at $x=-d/2$)
the correct TM potential for the left body has $Xi\rightarrow-\Xi_1$ everywhere.
This change reverses the nature of the discontinuities from passing through the surface: if passing 
through the surface decreases the function, and increases the gradient, then traversing the surface in the   
opposite direction should have the opposite effect.  

  % \item {Quote full potential, and note boundary conditions.}
  % \item Quote full potential, and PDE.
  % \item Comment on how to handle other surface.
  %   When we do the Casimir energy between two bodies we will need to find the Feynman-Kac formula assuming two step discontinuities.
  %   We will use exactly the same procedure as above, but with an extra step. 

  %   The potential is then 
  %   \begin{equation}
  %     V = \chi_1\Theta(h-x) + \chi_2\Theta(x-h-d) + \mathfrak{M}(-\Xi,h) + \mathfrak{M}(\Xi,h+d),
  %   \end{equation}
  %   where we took $\Xi \rightarrow -\Xi$ on one step since $\partial^2_x\theta(-x) = -\delta'$.  
  % \item Exploit symmetry of effective boundary conditions.  
  % \item {Quote results.  Note that $r\rightarrow r'$.}

%     We then need to solve this for the various cases of $h,h+d $ on each side of $x=0$:
%     \begin{equation}
%       f_{TM,12}[x-(h-x_0)] = \left\{ \begin{array}{ccr}
%           \dfrac{1}{\sqrt{2(\lambda+\chi_1)}} + e^{-2\sqrt{2(\lambda+\chi_1)}(h-x_0)}
% \dfrac{u\supTM_2 e^{-2\sqrt{2\lambda}d} - u\supTM_1}{\sqrt{2(\lambda+\chi_1)}(1-u\supTM_1u\supTM_2 e^{-2\sqrt{2\lambda}d})} & \hspace{1cm} & h>x_0\\
%           \frac{1}{\sqrt{2\lambda}} + 
%           \dfrac{2u\supTM_1u\supTM_2 e^{-2\sqrt{2\lambda}d} + u\supTM_1 e^{2\sqrt{2\lambda}(h-x_0)} +u\supTM_2 e^{-2\sqrt{2\lambda}(d+h-x_0)}}
%           {\sqrt{2\lambda}(1-u\supTM_1u\supTM_2 e^{-2\sqrt{2\lambda}d})} & \hspace{1cm} & h<x_0<h+d\\
%           \dfrac{1}{\sqrt{2(\lambda+\chi_2)}} + e^{2\sqrt{2(\lambda+\chi_2)}(d+(h-x_0))}
%           \dfrac{(u\supTM_1 e^{-2\sqrt{2\lambda}d}-u\supTM_2)}{\sqrt{2(\lambda+\chi_2)}(1-u\supTM_1u\supTM_2 e^{-2\sqrt{2\lambda}d})} & \hspace{1cm} & h+d<x_0
%         \end{array}
%       \right.
%     \end{equation}
%     \item {Quote various limits for $h,h+d$ on either side of 0}
The resulting solutions have the same form and structure as the TE solutions, but with 
the TE reflection coefficients replaced by their TM counterparts.  

For paths starting in the left body, $x_0<d_1$ the solution is
    \begin{align}
      f^{(\text{I})}\subTMtwo(x) &=
           \dfrac{1}{\sqrt{2(\lambda+\chi_1)}} + e^{-2\sqrt{2(\lambda+\chi_1)}(d_1-x_0)}
 \dfrac{u\supTM_2 e^{-2\sqrt{2\lambda}d} - u\supTM_1}{\sqrt{2(\lambda+\chi_1)}(1-u\supTM_1u\supTM_2 e^{-2\sqrt{2\lambda}d})}
% =\frac{1}{\sqrt{2(\lambda+\chi_1)}} + e^{-2\sqrt{2(\lambda+\chi_1)}(d_1-x_0)}
%       \frac{u\supTM_2 e^{-2\sqrt{2\lambda}d}-u\supTM_1 }{\sqrt{2(\lambda+\chi_1)}(1-u\supTM_1u\supTM_2 e^{-2\sqrt{2\lambda}d})}.
    \end{align}
    where the TM reflection coefficients for each body are given by 
    \begin{equation}
      u\supTM_i = \frac{e^{2\Xi}\sqrt{\lambda} -\sqrt{\lambda+\chi_i}}{e^{2\Xi}\sqrt{\lambda} + \sqrt{\lambda+\chi_i}}.
    \end{equation}
    For paths starting in the middle, $d_1<x_0<d_2$, the solution is
    \begin{equation}
      f^{(\text{II})}\subTMtwo(x)= %= \frac{1}{\sqrt{2\lambda}} + \frac{2u\supTM_1u\supTM_2 e^{-2\sqrt{2\lambda}d} + u\supTM_1 e^{2\sqrt{2\lambda}h} +u\supTM_2 e^{-2\sqrt{2\lambda}(d+h)}}{\sqrt{2\lambda}(1-u\supTM_1u\supTM_2 e^{-2\sqrt{2\lambda}d})}
          \frac{1}{\sqrt{2\lambda}} + 
          \dfrac{2u\supTM_1u\supTM_2 e^{-2\sqrt{2\lambda}d} + u\supTM_1 e^{2\sqrt{2\lambda}(d_1-x_0)} +u\supTM_2 e^{-2\sqrt{2\lambda}(d_2-x_0)}}
          {\sqrt{2\lambda}(1-u\supTM_1u\supTM_2 e^{-2\sqrt{2\lambda}d})},
    \end{equation}
    and  for paths starting in the right hand body, $d_2<x_0$, the solution is
    \begin{equation}
            f^{(\text{III})}\subTMtwo(x)= %=  \frac{1}{\sqrt{2(\lambda+\chi_2)}} + \frac{e^{2\sqrt{2(\lambda+\chi_2)}(d+h)}(u\supTM_1 e^{-2\sqrt{2\lambda}d} - u\supTM_2)}{\sqrt{2(\lambda+\chi_2)}(1-u\supTM_1u\supTM_2 e^{-2\sqrt{2\lambda}d})},
          \dfrac{1}{\sqrt{2(\lambda+\chi_2)}} + e^{2\sqrt{2(\lambda+\chi_2)}(d_2-x_0)}
          \dfrac{(u\supTM_1 e^{-2\sqrt{2\lambda}d}-u\supTM_2)}{\sqrt{2(\lambda+\chi_2)}(1-u\supTM_1u\supTM_2 e^{-2\sqrt{2\lambda}d})} .
    \end{equation}
%   \item {Integrate over position}
%     In exactly the same fashion, we can integrate over position.  
%     \begin{align}
%       I_{TM,12} =& -I_{div} + \dfrac{u\supTM_2 e^{-2\sqrt{2\lambda}d}-u\supTM_1}{4(\lambda+\chi_1)(1-u\supTM_1u\supTM_2 e^{-2\sqrt{2\lambda}d})} +\frac{2u\supTM_1u\supTM_2 e^{-\sqrt{2\lambda}d}d}{\sqrt{2\lambda}(1-u\supTM_1u\supTM_2 e^{-2\sqrt{2\lambda}d})} + (u\supTM_1+u\supTM_2)\frac{(1-e^{-2\sqrt{2\lambda}d})}{4\lambda(1-u\supTM_1u\supTM_2e^{-2\sqrt{2\lambda}d})}\nonumber\\
%       & +\frac{u\supTM_1 e^{-2\sqrt{2\lambda}d} - u\supTM_2}{4(\lambda+\chi_2)(1-u\supTM_1u\supTM_2 e^{-2\sqrt{2\lambda}d})},
%     \end{align}
%     where $I_{div} = [2(\lambda+\chi_1)]^{-1/2}\int_{-\infty}^h dx_0  +  [2(\lambda+\chi_2)]^{-1/2}\int_{h+d}^\infty dx_0  + (2\lambda)^{-1/2}d$.
%   \item Comment on divergent parts cancelling out. 
% \end{enumerate}
This result contains all of the previous results for closed paths.  By taking $\Xi\rightarrow 0$, the TE results are covered.
In taking $\chi\rightarrow 0$, the steps can be ignored, and this is the two body TM 
The equivalent single body results can be found by taking the far body to $x=\pm\infty$.  

These intimidating formulae are most useful for showing the analytical agreement with existing 
Casimir energy results.  As such it is best to leave them in this Laplace-transformed form.  
The next chapter uses the results from this chapter to rederive known analytical results for both 
polarizations for Casimir and Casimir--Polder energies.  

%%% Local Variables: 
%%% mode: latex
%%% TeX-master: "thesis_master"
%%% End: 
