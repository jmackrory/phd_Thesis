\chapter{Worldlines for Scalar Fields}
\label{ch:scalar_worldlines}
\begin{itemize}
\item Cite Schubert~\cite{Schubert2001}, Strassler~\cite{Strassler1992} on general worldline
\comment{Other references - was one contemporaneous with Strassler?}
\begin{itemize}
\item Summarize Strassler.  Can compute QFT effects from worldline path integrals.  
\item One loop effective actions can be described as single-particle worldline path integrals.  Can apply for higher order loops, and gauge fields, Cite Schubert.    
\item Cite QED at one loop order paper.  Get same results.  
\item Note similarity to Schwinger's trick for handling loop integrals in QED.  T is Schwinger's proper time.  
\end{itemize}
\item Cite QED Worldline paper on numerics?
\item Cite Gies papers~\cite{Gies2003,Gies2006, Gies2006a} (all of them!) note work on thermal/geometry~\cite{Klingmueller2008,Weber2009, Weber2010}
\end{itemize}

In this chapter we will review prior work on the worldline method, translated to our terminology
and conventions.  We will derive the method, and discuss its advantages and shortcomings.  

\section{Partition Function to Worldline path integral}

\begin{itemize}
\item Intro
We consider a scalar field coupled to a background potential $V(\vect{x},t)$.  This potential
embodies the location of the bodies we are considering.  Starting from the classical action,
we will derive the Hamiltonian for the fields, and then compute the quantum partition function.  
The partition function can be written as a path-integral, which is readily evaluated as a functional
determinant.  Ultimately we want the free energy, which can be further converted into a path integral
for a fictitious single-particle.  This single-particle path integral forms the basis of the numerical
world line method.   

\item Lagrangian - Hamiltonian
The Lagrangian for the scalar field is
\begin{equation}
  L := \int d^3x \left[ \frac{1}{2}(\partial_t\phi)^2-\frac{1}{2}(\nabla\phi)^2-V(\vect{x},t)\phi^2\right].
\end{equation}
In prior work, the potential $V(\vect{x},t)$ encodes the locations of the interacting bodies, with 
definition
\begin{equation}
  V(\vect{x}) = \lambda \sum_r \delta[\sigma_r(\vect{x}-\vect{R}_r)],
\end{equation}
    where $\lambda$ is the coupling constant, $\sigma_r(\vect{x})=0$ marks the locations of the surfaces, 
    and $\vect{R}_r$ marks the center location of each body.  The coupling constant $\lambda$ 
    is taken to infinity, which corresponds to imposing Dirichlet boundary conditions on the surfaces.

The conjugate momentum to $\phi$ is given by
\begin{equation}
  \Pi := \frac{\delta \cL}{\delta(\partial_t\phi)} = \partial_t\phi,
\end{equation}
    where $\frac{\delta}{\delta f(t)}$ denotes the functional derivative w.r.t. $f(t)$.    
The Hamiltonian can then be easily found,
\begin{align}
  H := \int d^3x\,\Pi\partial_t\phi -  L\\ 
  = \int d^3x\,\bigg[\frac{\Pi^2}{2} + \frac{1}{2}(\nabla\phi)^2 +V(\vect{x},t)\phi^2\bigg].  
\end{align}
We are now in a position to quantize the field by promoting the fields to operators, 
    $\phi\rightarrow \op{\phi}, \Pi\rightarrow\op{\Pi}$.
The fields can be promoted to operators with equal-time commutation relations
\begin{equation}
  [\op{\phi}(\vect{x},t),\op{\Pi}(\vect{x'},t)] = i\hbar \delta(\vect{x}-\vect{x'}).
\end{equation}
In exactly analogous fashion to quantum mechanics, the overlap between states is given by 
\begin{equation}
  \langle \phi|\Pi\rangle = \exp\bigg[\frac{i}{\hbar}\int d^3x \phi(\vect{x})\Pi(\vect{x})\bigg].
\end{equation}
    
    We can compute physical quantities of interest such as Casimir energies and forces
    by taking suitable derivatives of the free energy.  The free energy $\mathcal{F}=-\kB T \log Z$,
    is in turn given by the partition function $Z$.  \comment{Cite Brown, Altland-Simons}

The field partition function is 
\begin{equation}
  Z = \tr[ e^{-\beta\op{H}}] = \int d\phi \langle \phi| e^{-\beta \op{H}}|\phi\rangle,
\end{equation}
where we have evaluated the trace over the complete set of field states.  In classic path-integral
fashion the exponential operator can be split into $N$ pieces, and resolutions of the identity
in both fields and conjugate-momentum fields can be inserted between each piece.  

% \begin{align}
%   Z &= \int d\phi_0\prod_{n=1}^N d\phi_n \langle \phi_n| e^{-\Delta \beta \op{H}}|\Pi_n\rangle
%   \langle\Pi_n| \phi_{n-1}\rangle
% \end{align}

After integrating out the momentum fields, the partition function can be written as 
\item Euclidean Path integral (Generating Function) 
\begin{equation}
  Z = \int D\phi \exp\left\{-\int_0^{\hbar\beta c} d\tau \int d^3x 
    \left[ \frac{1}{2}(\partial_\tau\phi)^2+\frac{1}{2}(\nabla\phi)^2+V(\vect{x})\phi^2\right]\right\}.
\end{equation}
The functional integral over $\phi$ is Gaussian and can be formally evaluated immediately as a 
functional determinant, since the differential operator is positive operator.  
Some care is required in regularizing such infinite determinants.  
This is done in analogy with finite dimensional Gaussian integrals.  
We can consider treating the fields as only being evaluated on a finite lattice of space-time points, 
with the lattice also having a finite extent which bounds all bodies.  We will then take the limit of 
arbitrarily large volume, and lattice resolution at the end.  The gradient operators 
can be treated via their finite difference approximations, which can be thought of as sparse matrices.  

For a single real Gaussian integral, one can evaluate it as
\begin{equation}
  I_1=\int dx e^{-\alpha x^2} =\sqrt{\frac{\pi}{\alpha}}.
\end{equation}
Let us now consider the multidimensional Gaussian integral
\begin{equation}
  I_2=\int \prod_{j=1}^Ndx_j e^{-\vect{x}^T A \vect{x}},
\end{equation}    
where there is an implicit sum over $i,j$, and $A$ is a positive matrix, i.e. all of $A$'s eigenvalues $\alpha_j$ 
are positive.
    The Gaussian integral can be readily evaluated by changing integration variables $\vect{x}$ to the amplitudes of the 
    eigenvectors of $A$, $\vect{y}$ via an orthogonal transformation $O$,
    \begin{equation}
      \vect{x} = O\vect{y}.
    \end{equation}
    The Gaussian integral can be evaluated as
    \begin{equation}
      I_2 = \int \prod_{j=1}^N dy_j e^{-\alpha_jy_j^2} \propto \left[ \prod_{j=1}^N\alpha_j \right]^{-1/2}
      = {\det}^{-1/2} A.
    \end{equation}
    \comment{(Cite Srednicki or Brown for their chapter on Gaussian integrals)}
In an analogous fashion, one can formally evaluate the partition function path integral as a 
functional determinant, 
\begin{equation}
  Z \propto {\det}^{-1/2}\left[-\frac{1}{2}\partial_\tau^2-\frac{1}{2}\nabla^2+V(\vect{x})\phi^2\right].
\end{equation}
Note that the original computations for the worldline method stressed computing the quantum effective
action for the scalar field.  This yields essentially the same expression.  

The free energy for the interacting field can be written as 
\begin{equation}
  \mathcal{F} = -\kB T\log Z = \frac{1}{2}\kB T \log\det[-\frac{1}{2}\partial_\tau^2-\frac{1}{2}\nabla^2+V(\vect{x})].
  \label{eq:free-energy-det}
\end{equation}
As it stands this expression is divergent, however we will renormalize by subtracting off the 
same expression with the bodies removed to spatial infinity.  Physically this corresponds to 
computing energy differences between different configurations.  The renormalization also cancels off 
the constant normalization factors.  

The free energy can now be converted into a single-particle path integral via some formal 
manipulations.  First we will use the identity $\log\det A=\tr\log A$, which can be readily
verified for positive finite matrices.  
\begin{align}
  \log\det A &= \log\prod_j \alpha_j%\nonumber\\
  =\sum_j \log\alpha_j%\nonumber\\
  = \tr\log A,\label{eq:log-det}
\end{align}
    where we used the facts that the trace and determinant of a matrix are given by the sum
and product of its eigenvalues respectively. \comment{Check Kirsten/Vassilevich for better argument/citation}
Furthermore, we will use the integral representation of the logarithm,
\begin{equation}
  \log A -\log B= -\int_0^\infty \frac{d\cT}{\cT} (e^{-A\cT} - e^{-B\cT}).\label{eq:integral_log}
\end{equation}
This expression also relies on a difference of terms to cancel out divergent terms at $T=0$.  The 
earlier renormalization by subtracting off the vacuum energy provides exactly this subtraction. 

By applying Eqs.~(\ref{eq:log-det}) and (\ref{eq:integral_log}) to free energy~(\ref{eq:free-energy-det}),
 the renormalized free energy can be rewritten as
\begin{equation}
  \mathcal{F}-\mathcal{F}_0 = -\frac{\kB T}{2}\int_0^\infty \frac{d\cT}{\cT}
  \tr\Big[e^{-(\partial_\tau^2+\nabla^2)\cT}\big(e^{-\cT V(\vect{x})}-1\big)\Big].
\end{equation}
    


\item Worldline path integral
  \begin{equation}
    F = - \int \frac{dT}{T^{1+D/2}} \dlangle e^{-\cT\langle V\rangle} - 1\drangle,
  \end{equation}
  where $\cT$ is the loop proper time, $\langle V\rangle$ is the average of the potential around a particular loop, and $\dlangle\cdots\drangle$ denotes an ensemble average over Brownian paths.  
\item Typically take $V = \lambda\delta[\vect{x}-\sigma(\vect{x})]$, where $\sigma(\vect{x})=0$ is a function describing the surfaces.  In the limit $\lambda\rightarrow\infty$ this amounts to enforcing Dirichlet boundary conditions on the fields at the surfaces.  
\end{itemize}

    \section{Numerical method}
    \section{Deficiencies of the scalar method}
\begin{itemize}
\item Cite Schaden applying to pistons\cite{Schaden2009}
\item Figure showing loops.  
\item Advantages
  \begin{itemize}
  \item Algorithm is geometry independent, and no spatial grid.
  \item Parallelizable.  Computation time scales as inversely with resources.  
  \end{itemize}

\item Shortcomings
\begin{itemize}
  \item No coupling of photons to medium.
  \item A scalar, not vector electromagnetism.
\end{itemize}
  
\end{itemize}


\begin{shaded}
  The presence of $\delta(\vect{x}_N-\vect{x}_0)$ leads to an overall normalization constant $(2\pi\cT)^{-(D-1)/2}$.
  This follows either from Hormander's argument, that 
  \begin{equation}
    \int d^nx\, \delta[h(\vect{x})]f(\vect{x}) = \int_{h^{-1}(0)} dS\,\frac{1}{|\nabla h(\vect{x})|}f(\vect{x}),
  \end{equation}
where $S$ is defined as the surface satisfying $h(\vect{x})=0$, and 
$|\nabla h(\vect{x})|=\sqrt{\sum_k \left(\frac{\partial h}{\partial x_k}\right)^2}$.
  In our case, we are restricting a sum of $N$ Gaussian integrals to have zero total.
  If we define the increments $d\vect{x}_n = \vect{x}_{n+1}-\vect{x}_n$, then the loop constraint is $\delta(\sum_{k=0}^{N-1} \vect{x}_k)$.
  If we account for the remaining normalizations of the $\vect{x}_n$ integral, then the normalization for the loop path integral is $\sqrt{2\pi\Delta \cT N} = \sqrt{2\pi\cT}$.  
\footnote{See pages 826-828 of Dan's notes.}
\end{shaded}


%%% Local Variables: 
%%% mode: latex
%%% TeX-master: "thesis_master"
%%% End: 
