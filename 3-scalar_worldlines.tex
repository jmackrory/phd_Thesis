\chapter{Worldlines for Scalar Fields}
\label{ch:scalar_worldlines}
\begin{itemize}
\item Cite Schubert~\cite{Schubert2001}, Strassler~\cite{Strassler1992} on general worldline
\comment{Other references - was one contemporaneous with Strassler?}
\begin{itemize}
\item Summarize Strassler.  Can compute QFT effects from worldline path integrals.  
\item One loop effective actions can be described as single-particle worldline path integrals.  Can apply for higher order loops, and gauge fields, Cite Schubert.    
\item Cite QED at one loop order paper.  Get same results.  
\item Note similarity to Schwinger's trick for handling loop integrals in QED.  T is Schwinger's proper time.  
\end{itemize}
\item Cite QED Worldline paper on numerics?
\item Cite Gies papers~\cite{Gies2003,Gies2006, Gies2006a} (all of them!) note work on thermal/geometry~\cite{Klingmueller2008,Weber2009, Weber2010}
\end{itemize}

In this chapter we will review prior work on the worldline method, translated to our terminology.
We will derive the method, and discuss its advantages and shortcomings.  

\section{Partition Function to Worldline path integral}

\begin{itemize}
\item Intro
We consider a scalar field coupled to a background potential $V(\vect{x},t)$.  This potential
embodies the location of the bodies we are considering.  Starting from the classical action,
we will derive the Hamiltonian for the fields, and then compute the quantum partition function.  
The partition function can be written as a path-integral, which is readily evaluated as a functional
determinant.  Ultimately we want the free energy, which can be further converted into a path integral
for a fictitious single-particle.  This single-particle path integral forms the basis of the numerical
world line method.   

\item Lagrangian - Hamiltonian
The Lagrangian for the scalar field is
\begin{equation}
  L := \int d^3x \left[ \frac{1}{2}(\partial_t\phi)^2-\frac{1}{2}(\nabla\phi)^2-V(\vect{x},t)\phi^2\right].
\end{equation}
In prior work, the potential $V(\vect{x},t)$ encodes the locations of the interacting bodies, with 
definition
\begin{equation}
  V(\vect{x}) = \lambda \sum_r \delta[\sigma_r(\vect{x}-\vect{R}_r)],
\end{equation}
    where $\lambda$ is the coupling constant, $\sigma_r(\vect{x})=0$ marks the locations of the surfaces, 
    and $\vect{R}_r$ marks the center location of each body.  The coupling constant $\lambda$ 
    is taken to infinity, which corresponds to imposing Dirichlet boundary conditions on the surfaces.

The conjugate momentum to $\phi$ is given by
\begin{equation}
  \Pi := \frac{\delta \cL}{\delta(\partial_t\phi)} = \partial_t\phi,
\end{equation}
    where $\frac{\delta}{\delta f(t)}$ denotes the functional derivative w.r.t. $f(t)$.    
The Hamiltonian can then be easily found,
\begin{align}
  H := \int d^3x\,\Pi\partial_t\phi -  L\\ 
  = \int d^3x\,\bigg[\frac{\Pi^2}{2} + \frac{1}{2}(\nabla\phi)^2 +V(\vect{x},t)\phi^2\bigg].  
\end{align}
We are now in a position to quantize the field by promoting the fields to operators, 
    $\phi\rightarrow \op{\phi}, \Pi\rightarrow\op{\Pi}$.
The fields can be promoted to operators with equal-time commutation relations
\begin{equation}
  [\op{\phi}(\vect{x},t),\op{\Pi}(\vect{x'},t)] = i\hbar \delta(\vect{x}-\vect{x'}).
\end{equation}
In exactly analogous fashion to quantum mechanics, the overlap between states is given by 
\begin{equation}
  \langle \phi|\Pi\rangle = \exp\bigg[\frac{i}{\hbar}\int d^3x \phi(\vect{x})\Pi(\vect{x})\bigg].
\end{equation}
    
    We can compute physical quantities of interest such as Casimir energies and forces
    by taking suitable derivatives of the free energy.  The free energy $\mathcal{F}=-\kb T \log Z$,
    is in turn given by the partition function $Z$.  

The field partition function is 
\begin{equation}
  Z = \tr[ e^{-\beta\op{H}}] = \int d\phi \langle \phi| e^{-\beta \op{H}}|\phi\rangle,
\end{equation}
where we have evaluated the trace over the complete set of field states.  In classic path-integral
fashion the exponential operator can be split into $N$ pieces, and resolutions of the identity
in both fields and conjugate-momentum fields can be inserted between each piece.  
\begin{align}
  Z &= \int d\phi_0\prod_{n=1}^N d\phi_n \langle \phi_n| e^{-\Delta \beta \op{H}}|\Pi_n\langle
  \rangle\Pi_n| \phi_{n-1}\rangle
\end{align}
\comment{Cite Brown, Altland-Simons}




\item Euclidean Path integral (Generating Function) 
\begin{equation}
  Z = \int D\phi \exp\left\{-\int_0^T dt \int d^3x \left[ (\partial_t\phi)^2+(\nabla\phi)^2+V(\vect{x},t)\phi^2\right]\right\},
\end{equation}
\item Gaussian integral.  Can integrate out $\phi$ to get an effective action for particular geometry of objects.  (Or use Free energy).  
\begin{equation}
  F = -k_BT\log Z = -k_BT \log\det[-\partial_t^2-\nabla^2-V(\vect{x},t)]
\end{equation}
\item Can use $\tr\log A = \log\det A$, and integral representation of log, 
\begin{equation}
  \log A -\log B= -\int_0^\infty \frac{dT}{T} (e^{-AT} - e^{-BT}),
\end{equation}
\item Worldline path integral
  \begin{equation}
    F = - \int \frac{dT}{T^{1+D/2}} \dlangle e^{-\cT\langle V\rangle} - 1\drangle,
  \end{equation}
  where $\cT$ is the loop proper time, $\langle V\rangle$ is the average of the potential around a particular loop, and $\dlangle\cdots\drangle$ denotes an ensemble average over Brownian paths.  
\item Typically take $V = \lambda\delta[\vect{x}-\sigma(\vect{x})]$, where $\sigma(\vect{x})=0$ is a function describing the surfaces.  In the limit $\lambda\rightarrow\infty$ this amounts to enforcing Dirichlet boundary conditions on the fields at the surfaces.  
\end{itemize}

    \section{Numerical method}
    \section{Deficiencies of the scalar method}
\begin{itemize}
\item Cite Schaden applying to pistons\cite{Schaden2009}
\item Figure showing loops.  
\item Advantages
  \begin{itemize}
  \item Algorithm is geometry independent, and no spatial grid.
  \item parallelizable.  Computation time scales as one /resources.  
  \end{itemize}

\item Shortcomings
\begin{itemize}
  \item No coupling of photons to medium.
  \item A scalar, not vector electromagnetism.
\end{itemize}
  
\end{itemize}


\begin{shaded}
  The presence of $\delta(\vect{x}_N-\vect{x}_0)$ leads to an overall normalization constant $(2\pi\cT)^{-(D-1)/2}$.  This follows either from Hormander's argument, that 
  \begin{equation}
    \int d^nx\, \delta[h(\vect{x})]f(\vect{x}) = \int_{h^{-1}(0)} dS\,\frac{1}{|\nabla h(\vect{x})|}f(\vect{x}),
  \end{equation}
where $S$ is defined as the surface satisfying $h(\vect{x})=0$, and $|\nabla h(\vect{x})|=\sqrt{\sum_k \left(\frac{\partial h}{\partial x_k}\right)^2}$.  In our case, we are restricting a sum of $N$ Gaussian integrals to have zero total.  If we define the increments $d\vect{x}_n = \vect{x}_{n+1}-\vect{x}_n$, then the loop constraint is $\delta(\sum_{k=0}^{N-1} \vect{x}_k)$.  If we account for the remaining normalizations of the $\vect{x}_n$ integral, then the normalization for the loop path integral is $\sqrt{2\pi\Delta \cT N} = \sqrt{2\pi\cT}$.  
\footnote{See pages 826-828 of Dan's notes.}
\end{shaded}


%%% Local Variables: 
%%% mode: latex
%%% TeX-master: "thesis_master"
%%% End: 
