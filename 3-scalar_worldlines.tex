\chapter{Worldlines for Scalar Fields}
\label{ch:scalar_worldlines}
    \section{Partition Function to Worldline path integral}
    \section{Numerical method}
    \section{Deficiencies of the scalar method}

\begin{itemize}
\item Cite Schubert~\cite{Schubert2001}, Strassler~\cite{Strassler1992} on general worldline
\comment{Other references - was one contemporaneous with Strassler?}
\begin{itemize}
\item Summarize Strassler.  Can compute QFT effects from worldline path integrals.  
\item One loop effective actions can be described as single-particle worldline path integrals.  Can apply for higher order loops, and gauge fields, Cite Schubert.    
\item Cite QED at one loop order paper.  Get same results.  
\item Note similarity to Schwinger's trick for handling loop integrals in QED.  T is Schwinger's proper time.  
\end{itemize}
\item Cite QED Worldline paper on numerics?
\item Cite Gies papers~\cite{Gies2003,Gies2006, Gies2006a} (all of them!) note work on thermal/geometry~\cite{Klingmueller2008,Weber2009, Weber2010}
\begin{itemize}
\item Action
\begin{equation}
  S = \int_0^T dt \int d^3x \left[ (\partial_t\phi)^2-(\nabla\phi)^2-V(\vect{x},t)\phi^2\right],
\end{equation}
\item Euclidean Path integral (Generating Function) 
\begin{equation}
  Z = \int D\phi \exp\left\{-\int_0^T dt \int d^3x \left[ (\partial_t\phi)^2+(\nabla\phi)^2+V(\vect{x},t)\phi^2\right]\right\},
\end{equation}
\item Gaussian integral.  Can integrate out $\phi$ to get an effective action for particular geometry of objects.  (Or use Free energy).  
\begin{equation}
  F = -k_BT\log Z = -k_BT \log\det[-\partial_t^2-\nabla^2-V(\vect{x},t)]
\end{equation}
\item Can use $\tr\log A = \log\det A$, and integral representation of log, 
\begin{equation}
  \log A -\log B= -\int_0^\infty \frac{dT}{T} (e^{-AT} - e^{-BT}),
\end{equation}
\item Worldline path integral
  \begin{equation}
    F = - \int \frac{dT}{T^{1+D/2}} \dlangle e^{-\cT\langle V\rangle} - 1\drangle,
  \end{equation}
  where $\cT$ is the loop proper time, $\langle V\rangle$ is the average of the potential around a particular loop, and $\dlangle\cdots\drangle$ denotes an ensemble average over Brownian paths.  
\item Typically take $V = \lambda\delta[\vect{x}-\sigma(\vect{x})]$, where $\sigma(\vect{x})=0$ is a function describing the surfaces.  In the limit $\lambda\rightarrow\infty$ this amounts to enforcing Dirichlet boundary conditions on the fields at the surfaces.  
\end{itemize}
\item Cite Schaden applying to pistons\cite{Schaden2009}
\item Figure showing loops.  
\item Advantages
  \begin{itemize}
  \item Algorithm is geometry independent, and no spatial grid.
  \item parallelizable.  Computation time scales as one /resources.  
  \end{itemize}

\item Shortcomings
\begin{itemize}
  \item No coupling of photons to medium.
  \item A scalar, not vector electromagnetism.
\end{itemize}
  
\end{itemize}


%%% Local Variables: 
%%% mode: latex
%%% TeX-master: "thesis_master"
%%% End: 
