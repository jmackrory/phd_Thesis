\chapter{Electromagnetic Worldlines - Analytical Results}
\label{ch:analytical}

This chapter will show that the worldline path integrals yield the same results as the more direct 
calculations outlined in Ch.~\ref{ch:introduction}.  
We will show how to extract the Casimir--Polder energy via suitable functional derivatives of the free energy.
The analytical energies can be derived by substituting in analytical solutions for the relevant path integrals.
The worldline Casimir energies can be put in the right form by using the Gamma function, and the Laplace-Mellin
transform.  The integrals are then transformed to yield known results in planar geometries.
Finally, we examine the behaviour of the worldline path integrals in the case of nonzero temperature, 
and show how known results for the near-field and high temperature results emerge from this formalism.    
The work presented on the TE polarization has been published in Ref.~\cite{Mackrory2016}.

\section{Extracting Casimir--Polder Energies}
\label{sec:casimir-polder_worldline}
The Casimir--Polder energy for an atom interacting with macroscopic bodies can be derived 
by treating the atom as a perturbation to the permittivity and permeability.  
The atom is located at $\rA$, and has polarizability $\alpha$ and magnetizability $\beta$.
The atom is perturbs the background permittivity $\epsr(\vect{r})\rightarrow \epsr(\vect{r})+\delta\epsr(\vect{r})$,
and permeability $\mur(\vect{r})\rightarrow \mur(\vect{r})+\delta\mur(\vect{r})$, where
\begin{equation}
  \delta\epsr(\vect{r})=\frac{\alpha}{\epsilon_0}\delta(\vect{r}-\rA), 
  \quad \delta\mu_r(\vect{r})=\mu_0\beta_0\delta(\vect{r}-\rA).
\end{equation}
We will initially carry out calculations for dispersion-free media, at zero temperature.  The generalization 
to nonzero temperature will be considered in Sec.~\ref{sec:nonzero_temp}.

The Casimir energy for TE and TM polarizations was derived in Ch.~\ref{ch:EM_quantization}.
In the zero temperature limit, the energy in the EM field in the TE polarization is 
\begin{equation}
  E\subTE-E\sup0 = -\frac{\hbar c}{2}\int_0^\infty\frac{d\cT}{(2\pi\cT)^{D/2}\cT}\int d\vect{x}_0
  \biggdlangle
  \frac{e^{-\langle V\subTE(\vect{x})\rangle\cT}}{\sqrt{\langle \epsr(\vect{x})\mur(\vect{x})\rangle}} -1
  \biggdrangle_{\vect{x}(t)}.\label{eq:TE_energy}
\end{equation}
The Casimir--Polder energy comes from expanding the energy to linear order in the $\alpha$ and $\beta_0$.
Note that the $\delta$-functions are being used a short-hand notation for a sharply localized function with unit integral.
All expansions involving these $\delta$-functions can be carried out with a finite regularization of the 
$\delta$-function, and the arbitrarily sharp limit can be taken at the end of the computations.  
%This expansion corresponds to extracting the lowest order linear response.
The expansions in $\alpha$ and $\beta$ must be carried out in both $\langle\epsr\mur\rangle$, and the potential $V\subTE$.
Considering the similarities between the polarizations, we will carry these expansions out for only
one polarization, since the others follow by duality.  

% It is important be clear about exactly what the energy is, and how it is computed.  
% As discussed in \S 4.7 of Jackson~\cite{Jackson1998}, the electrostatic energy for a system depends on the 
% manner in which the system was arranged.  Different energies are found if the free charge or potential
% are specified.  The same is true of magnetostatic systems: are the potentials on the boundaries, or the free 
% currents specified.  

The energy for an atom in the electromagnetic field can be found by considering the change in the energy 
from adding the perturbation. 
The energy can be written as a functional of the permittivity and permeability, $E[\epsr,\mur]$.
The change in energy for adding an atom is then
\begin{equation}
  \delta E[\epsr,\mur] = E[\epsr+\delta\epsr,\mur+\delta\mur]-E[\epsr,\mur].
\end{equation}
Since the atom is assumed to have a small localized effect on the permittity and permeability, 
this expansion corresponds to a functional derivative.  
 Otherwise, the expansions can be carried out with $\alpha_0$ and $\beta_0$ acting as the small parameters.
The Casimir--Polder energy can be found by taking the following functional derivatives
\begin{equation}
  V\subCP(\rA) = \frac{\alpha_0}{\epsilon_0}\frac{\delta}{\delta\epsr(\rA)}E+\mu_0\beta_0\frac{\delta}{\delta\mur(\rA)}E.
\end{equation}
The Casimir--Polder energy must be renormalized by considering the change in the energy as the atom is removed 
arbitrarily from the dielectric objects.

It is necessary to expand the path-averaged permittivity and permeability,
\begin{align}
  \langle(\epsr+\delta\epsr)(\mur+\delta\mur)\rangle^{-1/2} &= \langle\epsr\mur\rangle^{-1/2}
  -\frac{1}{2}\langle \mur\delta\epsr+\epsr\delta\mur\rangle\langle\epsr\mur\rangle^{-3/2}\nonumber\\
&= \langle\epsr\mur\rangle^{-1/2}
-\frac{1}{2}\frac{\alpha_0}{\epsilon_0}\langle \mur(\vect{x})\delta(x-\rA)\rangle\langle\epsr\mur\rangle^{-3/2}\nonumber\\
&\hspace{1cm} -\frac{1}{2}\mu_0\beta_0\langle\epsr(\vect{x})\delta(\vect{x}-\rA)\rangle \langle\epsr\mur\rangle^{-3/2}
\label{eq:mueps_expansion}
\end{align}
The singular potentials $V\subTE, V\subTM$ can be expanded in the same fashion.  The expansion is carried out to linear order 
in $\delta\epsr,\delta\mur$, and all higher order terms are dropped.
\begin{align}
  \langle V\subTE[\mur +\delta\mur] \rangle 
  =& \frac{1}{2}\Big<(\nabla\log\sqrt{\mur+\delta\mur})^2-\nabla^2\log\sqrt{\mur+\delta\mur}\Big>\nonumber\\
  =& \frac{1}{8} \Big< [\nabla\log(\mur+\delta\mur)]^2\Big>
  -\frac{1}{4}\Big< \nabla^2\log(\mur+\delta\mur)\Big> \nonumber\\
  =& \langle V\subTE[\mur]\rangle+\left< \frac{1}{4} \nabla\log\mur\cdot\nabla\frac{\delta\mur}{\mur}
    -\frac{1}{4}\nabla^2\frac{\delta\mur}{\mur}\right> 
  \label{eq:VTE_expansion}
\end{align}
It is straightforward to then expand the exponential $e^{-\cT\langle V\subTE\rangle}$ to linear order in $\delta\epsr$ and $\delta\mur$.  
The terms involving derivatives $\nabla \delta\mur$ only make sense after integration by parts, 
\begin{equation}
  \int d\vect{x} f(\vect{x})\nabla\delta(\vect{x}-\rA)  = -\nabla f(\vect{x})\bigg|_{\vect{x}=\rA}.
\end{equation}
The same manipulations for the path-averaged $\delta$-function also apply to the versions involving derivatives.  

In all of these expansions, the path-averaged $\delta$-functions act to restrict the path integrals to paths starting at the atom's
position $\rA$.  
The path integral can be written schematically as some path-averaged function that depends the whole path and includes 
 a path-averaged $\delta$-function,
\begin{equation}
  I = \int d\vect{x}_0\Bigdlangle \,\langle f(\vect{x})\rangle\langle g(\vect{x})\delta(\vect{x}-\rA)\rangle\,\Bigdrangle_{\vect{x}(t)}.
\end{equation}
In the discretized notation this is 
\begin{equation}
  I = \int \prod_{n=0}^{N-1}dx_n\,P(x_0,\cdots, x_{N-1}) \frac{1}{N}\sum_{k=0}^{N-1}f(x_k)\frac{1}{N}\sum_{j=0}^{N-1}\delta(\vect{x}_j-\rA)g(\vect{x}_j)
\end{equation}
where the second $\delta$-function enforces path closure.  All of the functions are invariant under cyclic permutations 
of the path labels.  This is true of the path-averaged functions such as $\langle \epsr \mur$ and $\langle \VTM\rangle$,
and the Gaussian probability for closed Brownian bridges.
Then for each term $\delta(\vect{x}_j-\rA)$, the labels can be shifted $j$ times so that in the shifted
coordinates $\vect{x}_j\rightarrow \vect{x}_0$.  Since there is now a sum of $N$ identical terms, the 
path integral can be written as
\begin{equation}
  I = \int \prod_{n=0}^{N-1}dx_n\,P(\vect{x}_0,\cdots, \vect{x}_{N-1}) \frac{1}{N}\sum_{k=0}^{N-1}f(\vect{x}_k)\delta(\vect{x}_0-\rA)
= \Bigdlangle\, \langle f\rangle\,\Bigdrangle_{\vect{x}(t), \vect{x}(0)=\rA}.
\end{equation}
% For a closed path, integrated over all space, with integrands athat are all written as averaged around the path, 
% there is some freedom in which point of the path is called the origin.  
Since only paths the satisfy the $\delta$-function will contribute to the path integral, we are free to call the point
at $\rA$ the path origin.  
The end result is the path-averaged $\delta$-function restricts the starting point of the paths to the atom's
position $\rA$.


Using the results in Eqs.~(\ref{eq:mueps_expansion}) and (\ref{eq:VTE_expansion}), the Casimir--Polder energy
for the TE polarization can be written as
\begin{align}
    V\supTE\subCP(\vect{\rA}) &= -\frac{\hbar c}{2}\int_0^\infty\frac{d\cT}{(2\pi\cT)^{D/2}\cT}\int d\vect{x}_0
    \biggdlangle
    \left( - \frac{\langle\mur\delta\epsr+\epsr\delta\mur\rangle}
    {2\langle \epsr\mur\rangle^{3/2}}\right) 
  e^{-\langle V\subTE\rangle\cT} \nonumber\\
  &\hspace{1cm}+ e^{-\langle V\subTE\rangle\cT}\left(-\frac{\cT}{\langle\epsr\mur\rangle^{1/2}}
    \left< \frac{1}{4} \nabla(\log\mur)\cdot\nabla\frac{\delta\mur}{\mur}
      -\frac{1}{4}\nabla^2\frac{\delta\mur}{\mur}\right> \right)
    \biggdrangle_{\vect{x}(t)}.
\end{align}
Then after manipulating the path-averaged $\delta$-functions, and renormalizing the energy against the 
case when the atom is far from the bodies, the Casimir--Polder energy is
\begin{align}
    V\supTE\subCP(\vect{\rA}) &= -\frac{\hbar c}{2}\int_0^\infty\frac{d\cT}{(2\pi\cT)^{D/2}\cT}
    \biggdlangle
    \left( - \frac{\alpha_0\mur(\rA)}{2\epsilon_0\langle \epsr\mur\rangle^{3/2}}
      -\frac{\beta_0\mu_0\epsr(\rA)}{2\langle \epsr\mur\rangle^{3/2}}\right) e^{-\langle V\subTE\rangle\cT} \nonumber\\
    &\hspace{1cm}+\frac{\cT}{4}\frac{\beta_0\mu_0}{\mur(\rA)}\left[
     \nabla^2      +\nabla^2(\log\mur)+ \nabla(\log\mur)\cdot\nabla\right]
    \frac{ e^{-\langle V\subTE\rangle\cT}}{\langle\epsr\mur\rangle^{1/2}}
    \biggdrangle_{\vect{x}(t),\vect{x}(0)=\rA}.
\end{align}
Note that bracketed gradients such as $\nabla(\log\mur)$ should be interpreted as functions, while the 
other derivative operators act on everything to their right.  The remaining derivatives act with respect to 
the path origin $\vect{x}_0=\rA$.
The corresponding TM Casimir--Polder energy can be found under the duality transformation, and is given by 
\begin{align}
    V\supTM\subCP(\vect{\rA}) &= -\frac{\hbar c}{2}\int_0^\infty\frac{d\cT}{(2\pi\cT)^{D/2}\cT}
    \biggdlangle
    \left( - \frac{\alpha_0\mur(\rA)}{2\epsilon_0\langle \epsr\mur\rangle^{3/2}}
      -\frac{\beta_0\mu_0\epsr(\rA)}{2\langle \epsr\mur\rangle^{3/2}}\right) e^{-\langle V\subTM\rangle\cT} \nonumber\\
    &\hspace{1cm}+\frac{\cT}{4}\frac{\alpha_0}{\epsilon_0\epsr(\rA)}\left[
     \nabla^2      +\nabla^2(\log\epsr) + \nabla(\log\epsr)\cdot\nabla\right]
    \frac{ e^{-\langle V\subTM\rangle\cT}}{\langle\epsr\mur\rangle^{1/2}}
    \biggdrangle_{\vect{x}(t),\vect{x}(0)=\rA}.
\end{align}

These expressions can be further simplified if the atom is in a region where the dielectric is not varying spatially,
and we consider non-magnetic atoms and media where $\beta_0=0$ and $\mur=1$.  
If the permittivity is spatially constant at the atom's location then $\nabla\log\sqrt\epsr(\rA)=0$.
In this case, the TE and TM Casimir--Polder energies are given by 
\begin{align}
    V\supTE\subCP(\vect{\rA}) &= \frac{\hbar c\alpha_0}{4\epsilon_0(2\pi)^{D/2}}\int_0^\infty\frac{d\cT}{\cT^{1+D/2}}
    \biggdlangle
      \frac{1}{\langle \epsr\rangle^{3/2}}
    %   \right) \nonumber\\
    % &\hspace{1cm}
      \biggdrangle_{\vect{x}(t),\vect{x}(0)=\rA}\\
    V\supTM\subCP(\vect{\rA}) &= \frac{\hbar c\alpha_0}{4\epsilon_0(2\pi)^{D/2}}\int_0^\infty\frac{d\cT}{\cT^{1+D/2}}
    \biggdlangle
      \frac{e^{-\langle V\subTM\rangle\cT}}{\langle \epsr\rangle^{3/2}}
      -\frac{\cT}{2\epsr(\rA)} \nabla^2 \frac{ e^{-\langle V\subTM\rangle\cT}}{\langle\epsr\rangle^{1/2}}
      \biggdrangle_{\vect{x}(t),\vect{x}(0)=\rA}.
\end{align}
The TE Casimir--Polder energy will thus always be the simpler case to evaluate since it only depends on $\langle\epsr\rangle$, which
is well behaved.
In contrast, the TM Casimir--Polder energy involves the singular TM potential%  which must be regularized
% using the results in Sec.~\ref{sec:TM_potential}.  The TM Casimir--Polder energy also requires
and also requires spatial derivatives.
 Both of those factors will require some care in numerical methods involving stochastic paths against singular potentials.


\section{Rearranging Worldline Casimir Energies}

The worldline energies need some rearrangement in order to use the analytical results for 
path integrals derived in Ch.~\ref{ch:feynman_kac}.  This can be done with two integral identities.
The first converts the worldline path integral into the form where the Laplace transformed path
integral appears.  The second exponentiates $\langle\epsr\rangle$ by means of the Gamma function.  

\subsection{ Laplace-Mellin Transforms}

The worldline path integral has the form of a Mellin transform.  
The Mellin transform appears in the context of $\zeta$-function renormalization for functional determinants,
and considering its connection to the $\zeta$-function plays a role in formal number theory.  However,
for our purposes it is just an integral transform.  
The Mellin transform of a function $f$ is defined as 
\begin{equation}
\mathcal{M}[f](z)= \int_0^\infty dt\, t^{z-1}f(t),
\end{equation}
which is a function of $z$.  
In the worldline energy $f$ is the ensemble-average path integral and $z=1+D/2$.

There is a useful relationship between Laplace transforms and Mellin transforms referred to as the Laplace-Mellin theorem~\cite{Lew1975}.  
The Laplace transform was defined in Eq.~(\ref{eq:Laplace}), and the $\Gamma$ function is defined as  
\begin{equation}
\Gamma(z) = \int_0^\infty ds\, s^{z-1} e^{-s} = \mathcal{M}[e^{-s}](z),
\end{equation}
where the second equality writes the Gamma function as the Mellin transform of the exponential.  
The Laplace-Mellin theorem~\cite{Lew1975} says
\begin{equation}
  \Gamma(1-z)\mathcal{M}[f](z) = \mathcal{M}\big[\mathcal{L}[f]\big](1-z)\label{eq:Laplace-Mellin}
\end{equation}
This is most easily motivated by starting with the right hand side
\begin{equation}
\mathcal{M}\big[\mathcal{L}[f]\big](1-z) = 
\int_0^\infty ds\, s^{-z} \int_0^\infty dt\,e^{-st} f(t).
\end{equation}
The order of $s$ and $t$ integration can be swapped, and the $s$ can be transformed to $s:=t/u$, with the 
result
% =& \int_0^\infty dt\,\left[\int_0^\infty ds s^{-z} e^{-st}\right] f(t)\\
% =& \int_0^\infty dt\,\left[\int_0^\infty d\frac{u}{t}\, t^zu^{-z} e^{-u}\right] f(t) \\
\begin{align}
\mathcal{M}\big[\mathcal{L}[f]\big](1-z)=&\int_0^\infty dt\,\int_0^\infty du\, u^{-z} e^{-u}\,t^{z-1} f(t) \\
=& \Gamma(1-z)\mathcal{M}[f](z)
\end{align}
In words, the Mellin transform of a function is proportional to the Mellin transform of the Laplace transform of the function,
 and subject to a change of variable $z\rightarrow 1-z$.  This is 
directly useful to rewriting worldline path integrals in terms of their Laplace transforms, especially
since the solution method in Ch.~\ref{ch:feynman_kac} naturally yields the Laplace transform of the function.

\subsection{Inverse Moment Theorem}

One further step is required to put all of the material functions in the path integral into exponential form.
This is necessary, since the solutions from the previous chapter were for path integrals with exponential potentials.
If positive powers were required, then the usual moment generating tricks could be used 
such as $\langle x\rangle^n = \frac{d^n}{ds^n}e^{-s\langle x\rangle}\big|_{s=0}$.
However, for the inverse-moments required in the worldline method, the Gamma function must be used 
\begin{equation}
\frac{1}{\Gamma[\alpha]}\int_0^\infty ds\,s^{\alpha-1}\dlangle e^{-s(x+\beta)}\drangle  
= \dlangle \frac{1}{(x+\beta)^\alpha}\drangle\label{eq:moment_theorem}.
\end{equation}
This is restricted to $x+\beta>0$, and $\alpha>0$.
In the worldline calculations $x+\beta=\langle\epsr(\vect{x})\rangle$ and $\alpha=1/2$ for Casimir energies, and $\alpha=3/2$
for Casimir--Polder energies.  On the imaginary frequency axis, the dielectric functions
are real, positive, decaying functions, so all of these conditions are satisfied for worldline path integrals.

\subsection{Rearranging the Worldline into Analytical Form}

The path integral can be converted into a form where analytical results can be substituted in by using both of the preceding results.
As an exampe, consider the TE path integral, with dielectric function $\epsr(\vect{x})=1+\chi(\vect{x})$, where $\chi$
is a positive function of position.  In both the Casimir and Casimir--Polder cases, the energy involves 
$\langle \epsr\rangle^{-\alpha}$, with $\alpha=1/2$, and $\alpha=3/2$ respectively.  
The energy density can be rewritten using the inverse-moment theorem~(\ref{eq:moment_theorem}),  
\begin{align}
\int_0^\infty \frac{d\cT}{\cT^{1+D/2}}\biggdlangle\frac{1}{\langle 1+\chi(\vect{x})\rangle^\alpha} \biggdrangle_{\vect{x}(t)}
% = &\int_0^\infty \frac{dT}{T^{1+D/2}}\frac{1}{\Gamma[\alpha]}\int_0^\infty ds s^{\alpha-1} 
% \dlangle e^{-s(\chi \int_0^T dt \Theta(x-d) +1)}\drangle \\
% =&\frac{1}{\Gamma[\alpha]}\int_0^\infty \frac{dT}{T^{1+D/2-\alpha}}\int_0^\infty ds s^{\alpha-1} e^{-s T}
% \dlangle e^{-s \chi \int_0^T dt \Theta(x-d)}\drangle \\
=&\int_0^\infty ds\, \frac{s^{\alpha-1}}{\Gamma(\alpha)}\int_0^\infty \frac{d\cT}{\cT^{1+D/2-\alpha}}
\dlangle e^{-s\cT- \int_0^\cT dt \,\chi[\vect{x}(t)]}\drangle.
\end{align}
In the second equality the integration variable was rescaled $s\rightarrow s\cT$,
 and the definition of the path average, $\langle f\rangle = \cT^{-1}\int_0^\cT dt\, f(t)$ was used.
% where we used the Inverse-moment theorem, rescaled the $\lambda\rightarrow \lambda T$,
%  and swapped the order of integration. We can see that that $T$ integral has the form of a Mellin transform.
The energy density can be further transformed with the Laplace-Mellin transform~(\ref{eq:Laplace-Mellin}), 
\begin{align}
\int_0^\infty \frac{d\cT}{\cT^{1+D/2}}\biggdlangle\frac{1}{\langle 1+\chi(\vect{x})\rangle^\alpha} \biggdrangle_{\vect{x}(t)}
% =& \int_0^\infty \frac{dT}{T^{1+z-1/2}}e^{-sT}\dlangle \frac{e^{-s \chi \int dt_0^T dt \Theta(x-d)}}{\sqrt{T}}\drangle\\
% =&\mathcal{M}\left[e^{-sT}\dlangle \frac{e^{-s \chi \int_0^T dt \Theta(x-d)}}{\sqrt{T}}\drangle\right]\left(-z+1/2\right) \\
% =& \frac{1}{\Gamma[1+z-1/2]}\mathcal{M}\left[\int_0^\infty dT e^{-(\lambda+s)T}
% \dlangle \frac{e^{-s \chi \int_0^T dt \Theta(x-d)}}{\sqrt{T}}\drangle\right]\left(-z+1/2\right) \\
&= \int_0^\infty ds\, \frac{s^{\alpha-1}}{\Gamma(\alpha)}\int_0^\infty d\lambda\, 
\frac{\lambda^{(D-n)/2-\alpha}}{\Gamma[(D-n)/2-\alpha+1]}\nonumber\\
&\hspace{1cm}\times\int_0^\infty d\cT e^{-(\lambda+s)\cT}
\dlangle \frac{e^{-s \int_0^\cT dt\,\chi(\vect{x})}}{\cT^{n/2}}\drangle_{\vect{x}(t)}.\label{eq:Casimir_Laplace_inverse}
\end{align}
In the last line we have factored out $\cT^{n/2}$, the normalization for a $n$-dimensional brownian bridge.
This assumes that the solution was computed in $n$-dimension; we will work in the case where $n=1$.
(Despite knowing the parameters $\alpha, D$ and $n$, it is easier to track them in calculations as variables
than as numbers.)
The integral over $\cT$ is the solution to the relevant diffusion equation, as discussed in Ch.~\ref{ch:feynman_kac}.
The solutions must be appropriately scaled with $\lambda\rightarrow \lambda+s$, and $\chi\rightarrow s\chi$.

\section{Analytical  TE CP energy for an atom and a dielectric plane}
\label{sec:TE_CP}
The TE contribution to Casimir--Polder energy for an atom above a half-plane at $x=d$ is then given 
combining these formal manipulations with the relevant path integral solution.  For an atom at the origin
interacting with a planar dielectric $\epsr(z)=1+\chi\Theta(x-d)$, the path integral solution is given by
 Eq.~(\ref{eq:Feynman-Kac TE one step}).
Under the rescalings $s\rightarrow s\chi, \lambda\rightarrow \lambda+s$, the renormalized TE Casimir--Polder potential is
\begin{align}
V\subCP\supTE-V\subCP\sup0=&-\frac{\hbar c\alpha_0}{4\epsilon_0(2\pi)^{D/2}}\frac{\sqrt{\pi}}{\Gamma[\alpha]\Gamma\left[(D+1)/2-\alpha\right]}
\int_0^\infty ds\, s^{\alpha-1}\int_0^\infty d\lambda\, \lambda^{(D-1)/2-\alpha}\nonumber\\
&\times\frac{e^{-2\sqrt{2(\lambda+s)}|d|}}{\sqrt{\lambda+s}} 
\frac{\sqrt{\lambda+s(1+\chi)}-\sqrt{\lambda+s}}{\sqrt{\lambda+s(1+\chi)}+\sqrt{\lambda+s}},
\label{eq:VCP_TE_inter}
\end{align}
This can be put into the same form as the known results by changing integration variables.
Similar steps will be required to convert the other path integrals, so we will go through this once in detail.
We will confine our attention to the integral over $s$ and $\lambda$, where
\begin{equation}
  J=\int_0^\infty ds\, s^{\alpha-1}\int_0^\infty d\lambda\, \lambda^{(D-1)/2-\alpha}\frac{e^{-2\sqrt{2(\lambda+s)}|d|}}{\sqrt{\lambda+s}} 
\frac{\sqrt{\lambda+s(1+\chi)}-\sqrt{\lambda+s}}{\sqrt{\lambda+s(1+\chi)}+\sqrt{\lambda+s}}.
\end{equation}
First, change variable from $\lambda$ to $v:=\sqrt{\lambda/s+1}$, 
\begin{equation}
  J% =2\int_0^\infty ds\, s^{\alpha-1}s^{(D-1)/2-\alpha+1-1/2}\int_1^\infty dv\, (v^2-1)^{(D-1)/2-\alpha}e^{-\sqrt{8 d^2s}v}
  % \frac{\sqrt{v^2+\chi}-v}{\sqrt{v^2+\chi}+v},
  =2\int_0^\infty ds\, s^{D/2-1}\int_1^\infty dv\, (v^2-1)^{(D-1)/2-\alpha}e^{-\sqrt{8 d^2s}v}
  \frac{\sqrt{v^2+\chi}-v}{\sqrt{v^2+\chi}+v}.
\end{equation}
Next, change variable from $s$ to $t=\sqrt{8d^2 s}v$, and swap the $t$ and $v$ integrals. 
\begin{align}
  J % =4\int_1^\infty dv\, (v^2-1)^{(D-1)/2-\alpha} \int_0^\infty dt\, \frac{2t}{8d^2v^2}\left(\frac{t^2}{8d^2v^2}\right)^{D/2-1}e^{-t}
  % \frac{\sqrt{v^2+\chi}-v}{\sqrt{v^2+\chi}+v}\\
=\frac{1}{8^{D/2-1}d^D}\int_1^\infty dv\,v^{-D} (v^2-1)^{(D-1)/2-\alpha} 
  \frac{\sqrt{v^2+\chi}-v}{\sqrt{v^2+\chi}+v}\int_0^\infty dt\, t^{D-1}e^{-t}.
\end{align}
The $t$-integral can then be identified as a Gamma function, $\Gamma[D-1]$.  
Substituting this integral back into the Casimir--Polder energy~(\ref{eq:VCP_TE_inter}), while taking $D=4, \alpha=3/2$ yields
\begin{align}
  V\subCP\supTE-V\subCP\sup0=&-\frac{3\hbar c\alpha_0}{32\epsilon_0\pi^{2}d^4}
  \int_1^\infty dv\,\frac{1}{2v^4} \frac{\sqrt{v^2+\chi}-v}{\sqrt{v^2+\chi}+v}.
\end{align}
The prefactor is the Casimir--Polder energy for an atom above a perfect conductor~(\ref{eq:CP_conductor}).
This result agrees with the known result for the TE contribution to the Casimir--Polder energy.  
The integral over $v$ is the TE contribution to the efficiency, $\eta\subTE$, and can be   
 evaluated in closed form
\begin{align}
\eta\subTE(\chi)=&\frac{1}{2}\int_{1}^\infty dv\,v^{-4}\frac{v-\sqrt{v^2+\chi}}{v+\sqrt{v^2+\chi  }}\nonumber \\
=&\frac{1}{3}+2\chi^{-1}- \frac{\sqrt{\chi  (\chi +1)}}{\chi^{3/2}}
-\frac{1}{4\chi^{3/2}}\log \left[2 \chi +2 \sqrt{\chi  (\chi+1)}+1\right]
-\frac{\text{arcsinh}\left(\sqrt{\chi }\right)}{2\chi^{3/2}}.
\end{align}
The efficiency $\eta\subTE$ smoothly interpolates between $0$ and $1/6$ as $\chi$ varies from $0$ to $\infty$.
In the strong-coupling limit the TE polarization provides $1/6$ of the Casimir--Polder energy.  
The remaining $5/6$ is provided by the TM polarization.  

\section{Analytical TM CP energy for atom-plane}

The TM calculation for the Casimir--Polder energy proceeds in a similar fashion to the TE case.  
The TM Casimir--Polder energy can be split into two pieces 
\begin{align}
  V\supTM\subCP(\vect{\rA})-V\subCP\sup0 &= \frac{\hbar c\alpha_0}{4\epsilon_0(2\pi)^{D/2}}
  \bigg(\cV_{D/2,3/2}-\frac{1}{2}\nabla^2\cV_{(D-2)/2,1/2}\bigg),\label{eq:TM_CP_inter}
\end{align}
where 
\begin{equation}
  \cV_{z,\alpha}:=\int_0^\infty\frac{d\cT}{\cT^{1+z/2}}
  \biggdlangle\frac{e^{-\langle V\subTM\rangle\cT}}{\langle \epsr\rangle^{\alpha}}-1
  \biggdrangle_{\vect{x}(t),\vect{x}(0)=\rA}.
\end{equation}
Each term $\cV_{z,\alpha}$ can in turn be transformed using the combination of the Laplace-Mellin transform
and the inverse-moment theorem.  
\begin{align}
  \cV_{z,\alpha}:=&\frac{1}{\Gamma[\alpha]\Gamma[z-\alpha+1/2]}\int_0^\infty d\lambda\,\lambda^{z-1/2-\alpha}
  \int_0^\infty ds\,s^{\alpha-1}\nonumber\\
  &\times\int_0^\infty d\cT \frac{e^{-(\lambda+s)\cT}}{\sqrt{\cT}}
  \Bigdlangle e^{-\int_0^\cT dt\,[V\subTM(\vect{x}) + s\chi(\vect{x)]}}-1  \Bigdrangle_{\vect{x}(t),\vect{x}(0)=\rA}\label{eq:TM_subpart}
\end{align}
Note that the TM-potential is already in the exponential, so the TM potential and $\Xi$ does not need to be rescaled. 
In this sense $\Xi$ is an independent parameter from $\chi$. 
The analytical path integral~(\ref{eq:Feynman-Kac TM one step}) can be substituted in to Eq.~(\ref{eq:TM_subpart}),
and the integral can be transformed in the same manner as Sec.~\ref{sec:TE_CP}.
\begin{align}
  \cV_{z,\alpha}% =&\frac{(-1)\sqrt{2\pi}}{\Gamma[\alpha]\Gamma[z-\alpha+1/2]}\int_0^\infty d\lambda\,\lambda^{z-1/2-\alpha}
=&-\frac{\sqrt{\pi}\Gamma[2z]}{8^zd^{2z}\Gamma[\alpha]\Gamma[z-\alpha+1/2]}\int_1^\infty dv\,\frac{1}{v^{2z}}(v^2-1)^{z-1/2-\alpha}
  \frac{ ve^{2\Xi}-\sqrt{v^2+\chi}} {v e^{2\Xi}+\sqrt{v^2+\chi}}.
\end{align}
\comment{check power on 8?  Off by 4 should be $8^{2z}2^{-2}$?}
The two cases of interest are for $z=2,\alpha=3/2$ and $z=1,\alpha=1/2$.
\begin{align}
\cV_{2,3/2} %=&-\frac{\sqrt{\pi}\Gamma(4)}{8^{2}d^{4}\frac{\sqrt{\pi}}{2}\Gamma(1)}\int_1^\infty dv\,\frac{1}{v^{4}}(v^2-1)^{2-1/2-3/2}
  % \frac{ ve^{2\Xi}-\sqrt{v^2+\chi}} {v e^{2\Xi}+\sqrt{v^2+\chi}}.
=&-\frac{6}{32d^{4}}\int_1^\infty dv\,\frac{1}{v^{4}}\frac{ ve^{2\Xi}-\sqrt{v^2+\chi}} {v e^{2\Xi}+\sqrt{v^2+\chi}},\\
\cV_{1,1/2}=&-\frac{1}{8d^{2}}\int_1^\infty dv\,\frac{1}{v^{2}}
  \frac{ ve^{2\Xi}-\sqrt{v^2+\chi}} {v e^{2\Xi}+\sqrt{v^2+\chi}}.
\end{align}
The derivatives with respect to the starting poistion $\rA$ 
are equivalent to derivatives with respect to the distance $d$, and can be straightforwardly
evaluated.  
After substituting this back into Eq.~(\ref{eq:TM_CP_inter}), the TM Casimir--Polder energy is given by 
\begin{align}
  V\supTM\subCP(\vect{\rA})-V\subCP\sup0 &= -\frac{3\hbar c\alpha_0}{32\pi^2\epsilon_0d^4}\comment{$\frac{1}{4}$}\frac{1}{2}
  \int_1^\infty dv\,v^{-4}(1-2v^2)  \frac{ v(1+\chi)-\sqrt{v^2+\chi}}{v(1+\chi)+\sqrt{v^2+\chi}},
\end{align}
where we substituted $e^{2\Xi}=1+\chi$.  This agrees with the Lifshitz results for the TM Casimir--Polder energy
for an atom near a dielectric half-space (Eq. 14.205 in Steck ~\cite{SteckNotes}).
The TM efficiency $\eta\subTM$ also has a closed form solution, 
\begin{align}
  \eta\subTM(\chi):=\,&\frac{1}{2}
  \int_1^\infty dv\,v^{-4}(1-2v^2)  \frac{ v(1+\chi)-\sqrt{v^2+\chi}}{v(1+\chi)+\sqrt{v^2+\chi}}\nonumber\\
  =\,& \frac{7}{6} + \chi + \frac{2 - (1+\chi)^{3/2}}{2\chi} 
  - \frac{\text{arcsinh}\sqrt{\chi}}{2\chi^{3/2}}[1 + \chi + 2\chi^2(1 + \chi)] \nonumber\\ 
  &+ \frac{(1+\chi)^2}{\sqrt{2+\chi}}\left[\text{arcsinh}\sqrt{1+\chi} - \text{arcsinh}\left(\frac{1}{\sqrt{1+\chi}}\right)\right].
\end{align}
The TM polarization provides the majority of the Casimir--Polder energy for an atom-plane, with 5/6 of the 
strong-coupling result coming from the TM polarization.  From the worldline point of view, most of that 
comes from the term involving the second derivative, which suggest it is essential to correctly capture  
that.  

\section{Finding the TE Casimir energy}

The Casimir energy for two dielectric planes can also be calculated within this formalism.  
The dielectric function is given by $\epsrab(x) = 1+\chi_1\Theta(d_1-x)+\chi_2\Theta(x-d_2)$.
The calculation proceeds in the same way, except for two changes.  
First, the Casimir energy requires a further integral over the starting points of the paths.
Second, the two-body interaction energy is found by subtracting the one-body energies involving 
$\epsra$ and $\epsrb$ from the two-body  expressions with $\epsrab$.  The fully renormalized Casimir energy between two planes is
\begin{align}
  E\subTE-E\sup0 &= -\frac{\hbar c}{2(2\pi)^{D/2}}\int_0^\infty\frac{d\cT}{\cT^{1+D/2}}\int d\vect{x}_0
  \biggdlangle
  \bigg(\frac{1}{\sqrt{\langle \epsrab\rangle}}-\frac{1}{\sqrt{\epsrab(\vect{x}_0)}}\bigg) \nonumber\\
&\hspace{1cm}  -\bigg(\frac{1}{\sqrt{\langle \epsra\rangle}}-\frac{1}{\sqrt{\epsra(\vect{x}_0)}}\bigg)
  -\bigg(\frac{1}{\sqrt{\langle \epsrb\rangle}}-\frac{1}{\sqrt{\epsrb(\vect{x}_0)}}\bigg)
    \biggdrangle_{\vect{x}(t)}.
  \end{align}
  Each term is renormalized by subtracting off the constant value of the dielectric evaluated at the 
  start of the paths.  (It is also possible to renormalize the energy by instead subtracting off the vacuum value 
  $\epsr=1$ from each term.)
  The Casimir energy can be recast using the inverse-moment theorem and the Laplace-Mellin transform,  
  \begin{align}
  E\subTE-E\sup0 &= -\frac{\hbar c}{2(2\pi)^{D/2}}\int_0^\infty ds\,\frac{s^{\alpha-1}}{\Gamma(\alpha)}
  \int d\lambda \frac{\lambda^{(D-1)/2-\alpha}}{\Gamma[(D+1)/2-\alpha]}\nonumber\\
  &\hspace{0.5cm}\times\int d\vect{x}_0 \left[ \big(f\supTE_{12}(\vect{x}_0)-f_{12}\sup0\big) 
- \big(f\supTE_{1}(\vect{x}_0)-f_{1}\sup0\big)
-\big(f\supTE_{2}(\vect{x}_0)-f_{1}\sup0\big)\right].
  \end{align}
  % where $f(\vect{x}_0)$ is the path integral solution for the relevant geometry, and $f\sup0$ is the solution when the 
  % dielectric is replaced by the constant value at the path starting point.
  The values for $\alpha=1/2$ and $D=4$ will be taken at the end of the computation.
  The solutions $f\supTE_i$ are the path integral solutions
  % \begin{equation}
  %   f_i = \int_0^\infty d\cT \frac{e^{-(\lambda+s)\cT}}{\sqrt{\cT}}\dlangle e^{-s\int_0^\cT dt\, \chi_i(\vect{x})}\drangle,
  % \end{equation}  
  derived in Eqs.~(\ref{eq:Feynman-Kac TE one step}) and (\ref{eq:Feynman-Kac TE two step}) for one and 
  two dielectric steps respectively.  These are renormalized by subtracting $f\sup0_i$, the solution for a constant dielectric filling space.  
  The spatial integral can be carried out for each of the three regions: Region I where $x_0<d_1$, Region II where
  $d_1<x_0<d_2$, and Region III where $d_2<x_0$.  
  Fortunately, the solutions are simple exponentials in $x_0$, making these integrals straightforward.
  This calculation is deferred to an App.~\ref{app:nasty_calc}, since it is straightforward, but messy.
  
  After a lot of algebra, the spatial integrand can be written as
  \begin{align}
    &\int_{-\infty}^\infty dx_0\bigg(\big[f\supTE_{12}(x_0)-f_{12}\sup0\big] -\big[f\supTE_{1}(x_0)-f_{1}\sup0\big]
    -\big[f\supTE_{2}(x_0)-f_{2}\sup0\big]\bigg)\nonumber\\
    & = \frac{A\sqrt{\pi}u\supTE_1u\supTE_2e^{-2\sqrt{2(\lambda+s)}d}}{\sqrt{(\lambda+s)}(1-u\supTE_1u\supTE_2 e^{-2\sqrt{2(\lambda+s)}d})}\left(2d
     + \frac{\sqrt{2}}{\sqrt{\lambda+s(1+\chi_1)}}+\frac{\sqrt{2}}{\sqrt{\lambda+s(1+\chi_2)}}\right)
  \end{align}
  where $A$ is the (infinite) area of the dielectric planes, and the reflection cofficients for each surface are
  \begin{equation}
    u\supTE_i = \frac{\sqrt{\lambda+s}-\sqrt{\lambda+s(1+\chi_i)}}{\sqrt{\lambda+s}+\sqrt{\lambda+s(1+\chi_i)}}.
  \end{equation}
  In order to recover a finite quantity it is necessary to calculate the energy per unit area.  
  The integrals can be transformed into Lifshitz form via similar transformations to those used previously.
First, change variable from $\lambda$ to $v:=\sqrt{\lambda/s+1}$, % [or $\lambda=s(v^2-1)$]
  \begin{align}
    \frac{E\subTE-E\sup0}{A} &= -\frac{\hbar c}{(2\pi)^{D/2}}\sqrt{\pi}\int_0^\infty ds\,\frac{s^{(D-2)/2}}{\Gamma(\alpha)}
  \int_1^\infty dv\, \frac{(v^2-1)^{(D-1)/2-\alpha}}{\Gamma[(D+1)/2-\alpha]}\nonumber\\
  &\times\frac{u\supTE_1u\supTE_2e^{-2\sqrt{2s}vd}}{(1-u\supTE_1u\supTE_2 e^{-2\sqrt{2s}vd})}
\left(2d + \frac{\sqrt{2}}{\sqrt{s}\sqrt{v^2+\chi_1}}+\frac{\sqrt{2}}{\sqrt{s}{v^2+\chi_2}}\right).
  \end{align}
  Next, change variable from $s$ to $t=\sqrt{2s}$, swap the $t$ and $v$ integrals, and  
  susbstitute $\alpha=1/2$, $D=4$, with the result
  \begin{align}
     \frac{E\subTE-E\sup0}{A}
    &= -\frac{\hbar c}{8\pi^{2}}\int_0^\infty dt \,t^{D-1}  \int_1^\infty dv\, (v^2-1)\nonumber\\
    &\times\frac{u\supTE_1u\supTE_2e^{-2tvd}}{(1-u\supTE_1u\supTE_2 e^{-2tvd})}
    \left(2d + \frac{2}{t\sqrt{v^2+\chi_1}}+\frac{2}{t\sqrt{v^2+\chi_2}}\right),
  \end{align}
where the reflection coefficients have been transformed to 
\begin{equation}
   u\supTE_i = \frac{v-\sqrt{v^2+\chi_i}}{v+\sqrt{v^2+\chi_i}}.
\end{equation}
Finally, integrate by parts with respect to $v$.
The following derivatives will be of use,
\begin{align}
  \frac{d u\supTE_i}{dv} &= \frac{d}{dv}\frac{v-\sqrt{v^2+\chi_i}}{v+\sqrt{v^2+\chi_i}}
    % &= \frac{1-v/\sqrt{v^2+\chi_i}}{v+\sqrt{v^2+\chi_i}}
    %   - \frac{(v-\sqrt{v^2+\chi_i})(1+v/\sqrt{v^2+\chi_i})}{(v+\sqrt{v^2+\chi_i})^2}\\
    % &= \frac{\sqrt{v^2+\chi_i}-v}{\sqrt{v^2+\chi_i}(v+\sqrt{v^2+\chi_i})}
    %   - \frac{(v-\sqrt{v^2+\chi_i})(\sqrt{v^2+\chi_i}+v)}{\sqrt{v^2+\chi_i}(v+\sqrt{v^2+\chi_i})^2}\\
    = \frac{-2u\supTE_i}{\sqrt{v^2+\chi_i}},\\
    \frac{d}{dv}\log[1-u\supTE_1u\supTE_2 e^{-2 t v d}] & = 
    \frac{u\supTE_1u\supTE_2 e^{-2 t v d}}{1-u\supTE_1u\supTE_2 e^{-2 t v d}}\left( 2 t d+\frac{2}{\sqrt{v^2+\chi_1}}
+\frac{2}{\sqrt{v^2+\chi_2}}\right).
\end{align}
The Casimir energy between two half-spaces is then 
\begin{align}
    \frac{E\subTE-E\sup0}{A}
  &= -\frac{\hbar c}{8\pi^2}\int_0^\infty dt \,t^{3}  \int_1^\infty dv\, (v^2-1) 
  \frac{d}{dv}\bigg[\frac{1}{t}\log(1-u\supTE_1u\supTE_2 e^{-2tvd})\bigg]\nonumber\\
  &= \frac{\hbar c}{4\pi^2}\int_0^\infty dt \,t^{2}  \int_1^\infty dv\, v \log(1-u\supTE_1u\supTE_2 e^{-2tvd}),
\end{align}
since the boundary term from the integration by parts vanishes.  
This is exactly the TE component of the Lifshitz energy we derived by more straightforward means in Sec.~\ref{sec:lifshitz}.
In this derivation, the gap between the spaces was filled with vacuum $\epsilon_3=1$.  
This derivation will be extended to the TM component, and the nonzero temperature case where
there is dispersion.  

\section{Finding the TM Casimir energy}

The TM Casimir energy calculation is carried out in a similar manner to the TM case.
 Despite the similarities thus far, it is still necessary to check that this calculation also works, 
since the differences in the reflection coefficients may upset the cancellations that occured.  
Fortunately, the calculation still works, and the Lifshitz results are recovered.    

The renormalized two-body TM Casimir interaction energy is 
\begin{align}
  E\subTM-E\sup0 &= -\frac{\hbar c}{2(2\pi)^{D/2}}\int_0^\infty\frac{d\cT}{\cT^{1+D/2}}\int d\vect{x}_0
  \biggdlangle
  \bigg(\frac{e^{-\cT\langle \VTM\sup1+\VTM\sup2\rangle}}{\sqrt{\langle \epsrab\rangle}}-\frac{1}{\sqrt{\epsrab(\vect{x}_0)}}\bigg) \nonumber\\
&\hspace{1cm}  -\bigg(\frac{e^{-\cT\langle \VTM\sup1\rangle}}{\sqrt{\langle \epsra\rangle}}-\frac{1}{\sqrt{\epsra(\vect{x}_0)}}\bigg)
  -\bigg(\frac{e^{-\cT\langle\VTM\sup2\rangle}}{\sqrt{\langle \epsrb\rangle}}-\frac{1}{\sqrt{\epsrb(\vect{x}_0)}}\bigg)
    \biggdrangle_{\vect{x}(t)}.
  \end{align}
In addition, to the two-body dielectric, the path integral is augmented by the TM potentials 
at both surfaces $\VTM\sup1$ and $\VTM\sup2$.
The TM potentials $\VTM$ all vanish for the renormalization
terms, which are evaluated for the case of a constant dielectric.  
After the Laplace-Mellin and inverse-moment transforms, the TM Casimir energy is 
  \begin{align}
  E\subTM-E\sup0 &= -\frac{\hbar c}{2(2\pi)^{D/2}}\int_0^\infty ds\,\frac{s^{\alpha-1}}{\Gamma(\alpha)}
  \int_0^\infty d\lambda \frac{\lambda^{(D-1)/2-\alpha}}{\Gamma[(D+1)/2-\alpha]}\nonumber\\
  &\hspace{0.5cm}\times\int d\vect{x}_0 \left[ \big(f\supTM_{12}(\vect{x}_0)-f_{12}\sup0\big) 
- \big(f\supTM_{1}(\vect{x}_0)-f_{1}\sup0\big)
-\big(f\supTM_{2}(\vect{x}_0)-f_{1}\sup0\big)\right],
  \end{align}
where the solutions are the path integrals in Eqs.~(\ref{eq:Feynman-Kac TM one step}) and 
(\ref{eq:Feynman-Kac TM two step}).  Again, the following substitutions are required: $\Xi_i\rightarrow \Xi_i$,
$\lambda\rightarrow \lambda+s$, and $\chi_i\rightarrow s\chi_i$.
The spatial integrals over $x_0$ can be evaluated, and the results simplified.  
This algebra is again deferred to an App.~\ref{app:nasty_calc}.
In this case, the integrated solution is 
\begin{align}
&\int d\vect{x}_0 \left[ \big(f\supTM_{12}(\vect{x}_0)-f_{12}\sup0\big) 
- \big(f\supTM_{1}(\vect{x}_0)-f_{1}\sup0\big)
-\big(f\supTM_{2}(\vect{x}_0)-f_{1}\sup0\big)\right]\nonumber\\
  &=\frac{\sqrt{\pi} A\,u\supTM_1u\supTM_2e^{-2\sqrt{2(\lambda+s)}d}}{\sqrt{\lambda+s}(1-u\supTM_1u\supTM_2 e^{-2\sqrt{2(\lambda+s)}d})}\nonumber\\
  &\hspace{0.5cm}\times\bigg[2d
  -\sum_{i=1}^2\frac{\sqrt{2}e^{2\Xi_i}s\chi_i}{\sqrt{\lambda+s(1+\chi_i)}[(\lambda+s)\,e^{4\Xi_i}-\lambda-s(1+\chi_i)]}
 \bigg],
\end{align}
where there is a factor of the transverse area $A$, and the reflection coeffients are given by, 
\begin{equation}
  u\supTM_i =  \frac{e^{2\Xi_i}\sqrt{\lambda+s}-\sqrt{\lambda+s(1+\chi_i)}}{e^{2\Xi_i}\sqrt{\lambda+s}+\sqrt{\lambda+s(1+\chi_i)}}.
\end{equation}
The same transformations can be employed to transform the TM Casimir energy.
First, change variable from $\lambda$ to $v:=\sqrt{\lambda/s+1}$, % [or $\lambda=s(v^2-1)$]
  \begin{align}
  E\subTM-E\sup0 % &= -\frac{\hbar cA}{2(2\pi)^{D/2}}\int_0^\infty ds\,\frac{s^{\alpha-1}}{\Gamma(\alpha)}
&= -\frac{\hbar cA}{(2\pi)^{D/2}}\int_0^\infty ds\,\frac{s^{(D-2)/2}}{\Gamma(\alpha)}
  \int_1^\infty dv\,\frac{(v^2-1)^{(D-1)/2-\alpha}}{\Gamma[(D+1)/2-\alpha]}\nonumber\\
  &\hspace{0.5cm}
\times\frac{\sqrt{\pi} \,u\supTM_1u\supTM_2e^{-2\sqrt{2s}vd}}{(1-u\supTM_1u\supTM_2 e^{-2\sqrt{2s}vd})}
%\nonumber\\  &\hspace{0.5cm}\times
\bigg[2d
  -\sum_{i=1}^2\frac{\sqrt{2}e^{2\Xi_i}\chi_i}{\sqrt{s}\sqrt{v^2+\chi_i}[v^2\,e^{4\Xi_i}-v^2-\chi_i]}
 \bigg],
  \end{align}
where the reflection coefficients are now given by 
\begin{equation}
  u\supTM_i =  \frac{e^{2\Xi_i}v-\sqrt{v^2+\chi_i}}{e^{2\Xi_i}v+\sqrt{v+\chi_i}}.
\end{equation}
  Next, change variable from $s$ to $t=\sqrt{2s}$, and substitute $\alpha=1/2$, $D=4$, with the result
  \begin{align}
  E\subTM-E\sup0 
%   &= -\frac{\hbar cA}{(2\pi)^{D/2}}\int_0^\infty dt\,\frac{t^{D-1}}{2^{(D-2)/2}\Gamma(\alpha)}
%   \int_1^\infty dv\,\frac{(v^2-1)^{(D-1)/2-\alpha}}{\Gamma[(D+1)/2-\alpha]}\nonumber\\
%   &\hspace{0.5cm}
% \times\frac{\sqrt{\pi} \,u\supTM_1u\supTM_2e^{-2tvd}}{(1-u\supTM_1u\supTM_2 e^{-2tvd})}
% %\nonumber\\  &\hspace{0.5cm}\times
% \bigg[2d
%   -\sum_{i=1}^2\frac{2e^{2\Xi_i}\chi_i}{t\sqrt{v^2+\chi_i}[v^2\,e^{4\Xi_i}-v^2-\chi_i]}
%  \bigg]
  &= -\frac{\hbar cA}{16\pi^{2}}\int_0^\infty dt\,t^{3}
  \int_1^\infty dv\,(v^2-1)\nonumber\\
  &\hspace{0.5cm}
\times\frac{\,u\supTM_1u\supTM_2e^{-2tvd}}{(1-u\supTM_1u\supTM_2 e^{-2tvd})}
%\nonumber\\  &\hspace{0.5cm}\times
\bigg[2d
  -\sum_{i=1}^2\frac{2e^{2\Xi_i}\chi_i}{t\sqrt{v^2+\chi_i}[v^2\,e^{4\Xi_i}-v^2-\chi_i]}
 \bigg].
  \end{align}
Once again, an integration by parts w.r.t. $v$ will put the energy in Lifshitz form. The following
derivatives will be required:
\begin{align}
  \frac{d}{dp}\log[1-u\supTM_1u\supTM_2 e^{-2ptd}] 
  &= \frac{u\supTM_1u\supTM_2 e^{-2ptd}}{1-u\supTM_1u\supTM_2 e^{-2ptd}}\left( 2td -\sum_{i=1}^2\frac{d\log u\supTM_i}{dp}\right),
\end{align}
and 
\begin{align}
  \frac{d}{dp}\log[u\supTM_i] %=& \frac{d}{dp}\left(\log[e^{2\Xi_i}p - \sqrt{p^2+\chi_i}] -\log[e^{2\Xi_i}p + \sqrt{p^2+\chi_i}]\right) \nonumber\\
  =& \frac{2\chi e^{2\Xi_i}}{\sqrt{p^2+\chi_i}[e^{4\Xi_i}p^2-(p^2+\chi_i)]}.
\end{align}
After integrating by parts, the TM Casimir energy is 
  \begin{align}
  E\subTM-E\sup0 
  &= \frac{\hbar cA}{8\pi^2}\int_0^\infty dt\,t^{2}
  \int_1^\infty dv\,\log[1-u\supTM_1u\supTM_2 e^{-2tvd}].
  \end{align}
\comment{need factor of $-4$}.



\section{Nonzero Temperature and Dispersion}
\label{sec:nonzero_temp}
The preceding results were all derived for dispersion free media at zero temperature.  
These calculations can be extended to nonzero temperature and to account for dispersion.  This is needed
to describe the near-field and thermal limiting cases.  
% So I read Babb's paper\footnote{Babb, J. F. and Klimchitskaya, G. L., and Mostepanenko, V. M., 
% ``Casimir-Polder interaction between an atom and a cavity wall under the influence of real conditions'',
%  Phys. Rev. A, \textbf{70},042901,(2004)} (which Dan cites for thermal Casimir-Polder calculations.~\cite{Babb2004})
%   In it they use the free energy, which is $\mathcal{F} = -k_BT\log Z$, as the basis of their calculations.
%  I've been trying to use the mean energy, $E= -\partial_\beta\log Z$. 
The free energy for the TE and TM polarizations (for non-magnetic media) is
\begin{align}
\cF\supTE-\cF_0 %& = -k_BT \log \frac{Z_{TE}}{Z_0} \\
% & = k_B T \frac{1}{2}{\sum_{n}}'\tr\left\{ \log\left[ \frac{1}{2}
%     \left(\epsilon(is_n,\vect{x})\frac{s_n^2}{c^2} -\nabla^2\right)\right]
% -\log\left[ \frac{1}{2}\left(\frac{s_n^2}{c^2} -\nabla^2\right)\right]\right\}\\
=&-\frac{k_BT}{2}{\sum_{n=0}^\infty}'\int_0^\infty \frac{d\cT}{\cT(2\pi \cT)^{(D-1)/2}}\int d^{D-1}x_0\,\nonumber\\
&\times\dlangle e^{-s_n^2\cT /(2c^2)} -  e^{-s_n^2\langle\epsr(\vect{x},is_n)\rangle\cT /(2c^2)}
\drangle\\
% \end{align}
% \begin{align}
\cF\supTM-\cF_0 
=&-\frac{k_BT}{2}{\sum_{n=0}^\infty}'\int_0^\infty \frac{d\cT}{\cT(2\pi \cT)^{(D-1)/2}}\int d^{D-1}x_0\,\nonumber\\
&\times\dlangle e^{-s_n^2\cT /(2c^2)} -  e^{-s_n^2\langle\epsr(\vect{x},is_n)\rangle\cT /(2c^2)}
e^{-\cT\langle V\subTM^{(n)}(\vect{x})\rangle}\drangle.
\end{align}
The Casimir--Polder energy can be derived by the same reasoning used in Sec.~\ref{sec:casimir-polder_worldline}.
In this case, the results are 
\begin{align}
\cF\supTE\subCP-\cF_0 
=&-\frac{k_BT}{2}{\sum_{n=0}^\infty}'\frac{s_n^2\alpha(is_n)}{2\epsilon_0c^2}\int_0^\infty \frac{d\cT}{(2\pi \cT)^{(D-1)/2}}\int d^{D-1}x_0\,\nonumber\\
&\times \dlangle e^{-s_n^2\cT /(2c^2)} -  e^{-s_n^2\langle\epsr(\vect{x},is_n)\rangle\cT /(2c^2)}
\drangle_{\vect{x}(t),\vect{x}(0)=\rA}\\
\cF\supTM\subCP-\cF_0 
=&-\frac{k_BT}{2}{\sum_{n=0}^\infty}'\frac{\alpha(is_n)}{\epsilon_0}\int_0^\infty \frac{d\cT}{(2\pi \cT)^{(D-1)/2}}\int d^{D-1}x_0\,\nonumber\\
&\times\dlangle \frac{s_n^2}{2c^2}e^{-s_n^2\cT /(2c^2)} -  \left(\frac{s_n^2}{2c^2}-\frac{1}{4}\nabla^2\right)e^{-s_n^2\langle\epsr(\vect{x},is_n)\rangle\cT /(2c^2)}
e^{-\cT\langle V\subTM^{(n)}(\vect{x})\rangle}\drangle_{\vect{x}(t),\vect{x}(0)=\rA}.
\end{align}
There are some noteworthy features of the finite temperature worldline expression.  
First, all material functions are already in the exponent, so there is no need for the inverse-moment
theorem.  In fact, at zero temperature the frequency sum is replaced by a frequency integral, which
corresponds to the Gamma function used in the inverse-moment theorem.  
Second, in the TE polarization the dielectric path-average is proportional to $s_n^2$.
In contrast, in the TM polarization the additional TM potential is not weighted by $s_n$.  While the TM 
potential might depend on the frequency via $\epsr(\vect{x},is_n)$,
but it is not explicitly weighted by it.
This has important ramifications for the near-field and high temperature limits.
In those limits, the entire Casimir effect is provided by the TM polarization.  

At high temperature $\beta\rightarrow 0$, so the spacing between the Matsubara frequencies $s_n=2\pi n/(\hbar \beta)$
diverges.  As a result, only the first few terms significantly contribute.  In fact, the 
first mode is exponentially suppressed relative to the zero frequency mode.  Since the TE energy is 
The leading order term comes from the TM potential, which for the Casimir energy would be 
\begin{equation}
\lim_{\beta\rightarrow 0} \cF\supTM-\cF\sup0=-\frac{k_BT}{4}\int_0^\infty \frac{d\cT}{\cT(2\pi \cT)^{(D-1)/2}}\int d^{D-1}x_0\,
\dlangle 1 -  e^{-\cT\langle V\subTM^{(0)}(\vect{x})\rangle}\drangle.
\end{equation}
Similar considerations apply to both the Casimir and Casimir--Polder energy.  The TM polarization is 
similarly dominant at small distances, although showing that hinges more on the form of the solutions
and the dominant regions of the integrals.  

The high temperature limit touches on one of the arguments in the literature.  
There has been a dispute over the correct model for the frequency dependence of a realistic metal,
and the correct contribution from the TE mode at zero frequency.  This is particularly relevant 
for describe metals with effective dielectric functions.  
The Drude model diverges as $\omega^{-1}$ at DC,
whereas the plasma model diverges as $\omega^{-2}$ at zero frequency.  That extra divergence would 
lead to an extra contribution from the TE polarization at zero frequency.  Early experiments were 
unable to distinguish between the two models, although recent measurements have claimed to eliminate the 
plasma model from consideration~\cite{Sushkov2011}.

At the outset, we note that the zero temperature, far field limit can be recovered in two steps.
First, since $\beta\rightarrow\infty$, the spacing between the Matsubara frequencies approaches zero.
The sum over Matsubara modes can then be converted into a frequency integral,
$\beta^{-1}\sum'_nf(s_n)\rightarrow \frac{\hbar c}{2\pi}\int_0^\infty ds\,f(s)$.
Then assuming that the distances $d$ are longer than the dominant resonant wavelengths, the 
material response functions can be replaced with their zero-frequency values.  The frequency integrals
can then carried out, and our earlier expressions will be recovered.

\subsection{Thermal TE Casimir--Polder energy}

We will limit our discussion to the Casimir--Polder case of an atom near a dielectric plane.  
The preceding calculations are straightforwardly extended to dispersion and finite temperature.
The analytical solutions for the TE and TM path integrals for a single dielectric plane
in Eqs.~(\ref{eq:Feynman-Kac TE one step})
and (\ref{eq:Feynman-Kac TM one step}) can be substituted into the appropriate path integrals.  
After using the Laplace-Mellin transform~(\ref{eq:Laplace-Mellin}), the renormalized 
TE worldline path integral is 
\begin{align}
&\int_0^\infty d\cT\,\frac{1}{(2\pi \cT)^{(D-1)/2}}\dlangle e^{-v\cT} - e^{-v\cT \langle\epsr(is_n)\rangle}\drangle \nonumber\\
& =\frac{1}{2\pi}\int_0^\infty d\lambda\, \frac{e^{-2\sqrt{2(\lambda+v)}|d|}}{\sqrt{2(\lambda+v)}}
\frac{\sqrt{\lambda+v[1+\chi(is_n)]}-\sqrt{\lambda+v}}{\sqrt{\lambda+v[1+\chi(is_n)]}+\sqrt{\lambda+v}},
\end{align}
and there are no extra $\lambda$ terms from the Laplace-Mellin transform since ${\lambda^{(D-3)/2-1/2}=1}$
when $D=4$.
This result can be used in the TE Casimir free energy if substitution $v\rightarrow s_n^2/(2c^2).$
After changing integration variable to $p = \sqrt{1+2c^2\lambda/s_n^2}$, the free energy is
\begin{align}
\cF\supTE\subCP-\cF_0&=-k_BT{\sum_n}'\frac{s_n^3\alpha(is_n)}{4\pi\epsilon_0c^3}\int_1^\infty dp\,e^{-2s_n p|d|/c}
\frac{\sqrt{p^2+\chi(is_n)}-p}{\sqrt{p^2+\chi(is_n)}+p},
\label{eq:TE_CP_finite_temperature}
\end{align}
This is the general result for finite temperature and dispersion for the TE polarization for an atom near 
a planar dielectric.

The presence of the worldline can be used to estimate the relevant frequencies.  
Since the ensemble average is over Gaussian random walks,  the paths will typically intersect all the surfaces 
when $\cT\sim d^2$, where $d$ is the distance from the source point $x_0$ to the farthest surface.
Secondly, the frequency sum is dominated by the exponential factors, which will contribute most when $\cT s_n^2/c^2\sim 1$.
This suggests that frequencies with  $s_n^2< c^2/d^2$  will contribute most to the Casimir energy.   

We will now show that the TE contribution is negligible in both the near-field regime at zero temperature,
and the high temperature, far-field limit.  
In the near field regime, the separtion between the atom and the wall is much smaller than the atom's dominant wavelength, 
 $d\ll c/\omega_{j0}$.
In the zero temperature limit, the free energy~(\ref{eq:TE_CP_finite_temperature}) becomes
\begin{equation}
E-E_0=-\frac{\hbar}{8\pi^2\epsilon_0c^3}\int_0^\infty d\omega\,\omega^3\alpha(i\omega)
\int_1^\infty dp\,e^{-2\omega p|d|/c}\frac{\sqrt{p^2+\chi(i\omega)}-p}{\sqrt{p^2+\chi(i\omega)}+p}
\end{equation}
The integral contributes most when the exponent is order unity.
  The presence of the atom's polarizability $\alpha(i\omega)$ means that frequencies around 
  $\omega_{j0}$ will dominate the frequency integral.
  Then $p \sim  c/(d\omega_{j0})\gg 1$.
  The reflection coefficient can be approximated in this limit, 
\begin{align}
  \frac{\sqrt{p^2+\chi(i\omega)}-p}{\sqrt{p^2+\chi(i\omega)}+p}
\approx \frac{\chi(i\omega)}{4p^2}.
\end{align}
The $p$ integral can be approximately evaluated as
\begin{align}
\int_1^\infty dp\,\frac{1}{4p^2}e^{-2\omega p|d|/c}%  =& -\frac{1}{4p}e^{-2\omega p d/c}\bigg|_{p=1}^{\infty}
 % + \int_1^\infty dp\, \frac{1}{4p}\times \frac{-2\omega d}{c}e^{-2\omega pd/c}\\
\approx & \frac{1}{4}.
\end{align}
Substituting this into the energy, the result is 
\begin{align}
E-E_0=&-\frac{\hbar}{32\pi^2\epsilon_0c^3}\int_0^\infty d\omega\,\omega^3\alpha(i\omega)\chi(i\omega)\\
=&-\frac{\hbar}{32\pi^2\epsilon_0 d^3}\int_0^\infty d\omega\,\frac{\omega^3d^3}{c^3}\alpha(i\omega)\chi(i\omega)\approx 0
\end{align}
So this suppressed by $\order[(\omega d/c)^3]$ relative to the TM contributions, and be ignored.  
As already noted, in the high-temperature limit, the TE free energy is suppressed by  $e^{-s_1^2\cT/(2c^2)}$,
which comes from the first Matsubara frequency.  For a dielectric, the zeroth order contribution vanishes and 
can be ignored. 

% The TM Casimir--Polder energy at zero temperature is:
% \begin{align}
% E\supTM\subCP-E_0=&-\frac{\hbar}{2\pi}\int_0^\infty d\omega\frac{\alpha(i\omega)}{2\epsilon_0}
% \int_0^\infty d\cT\,\frac{1}{(2\pi \cT)^{3/2}}\nonumber\\
% &\times\dlangle \frac{\omega^2}{c^2}e^{-\omega^2\cT/(2c^2)}-\left(\frac{\omega^2}{c^2} 
%  - \frac{1}{2}\partial_x^2\right)e^{-\omega^2\cT \langle\epsr(i\omega)\rangle/(2c^2) - \cT\langle V_{TM}(i\omega)\rangle}\drangle
% \label{eq:TM_CP_zero_temperature},
% \end{align}

% \subsubsection{Laplace-Mellin and Feynman-Kac Formulae}
% We will again need to use the Laplace-Mellin transforms, and Feynman-Kac Formulae.
%   We quote the results:
% The Laplace-Mellin transform is
% \begin{align}
% \int_0^\infty \frac{dT}{T^{1+z}}\dlangle e^{-sT\langle\epsr\rangle - T\langle V_{TM}\rangle}\drangle =&
%  \frac{1}{\Gamma[z+1/2]}\int_0^\infty d\lambda\, \lambda^{z-1/2}\int_0^\infty dT e^{-(\lambda+s)T}
% \dlangle \frac{e^{-\int_0^T dt\,(s\chi+ V_{TM})}}{\sqrt{T}}\drangle.
% \end{align}
% For Casimir-Polder we need $z=1/2$, and for Casimir we need $z=3/2$.
%   In both cases we need $s= \omega^2/(2c^2)$.
%   We also need the actual analytical expression for that path integral.

% For one body we need:
% \begin{align}
% &\int_0^\infty dT e^{-(\lambda+s) T} \dlangle \frac{e^{-s\chi\int_0^T dt \Theta(x-d)}}{\sqrt{2\pi T}}\drangle  \nonumber\\
% &\hspace{0.5cm}=\frac{1}{\sqrt{2(\lambda+s)}}\left[1 - e^{-2\sqrt{2(\lambda+s)}|d|}\frac{\sqrt{\lambda+s(1+\chi)}
% -\sqrt{\lambda+s}e^{2\Xi}}{\sqrt{\lambda+s(1+\chi)}+\sqrt{\lambda+s}e^{2\Xi}}\right],
% \end{align}
% where $e^{2\Xi} = (1+\chi)$ comes from the contribution of $e^{-V_{TM}}$.
%   \comment{Correct signs?} For two macroscopic bodies we will need:
% \begin{align}
% &\int dx\int_0^\infty dT \frac{e^{-(\lambda +s)T}}{\sqrt{2\pi T}}\left[e^{-s\int_0^T dt\,(\chi_{12} + V_{12,TM})}
%  +1 -e^{-s\int_0^T dt\,(\chi_{1} + V_{1,TM})}-e^{-s\int_0^T dt\,(\chi_{2} + V_{2,TM})}\right]\nonumber\\ 
% =&  \dfrac{u_1'u'_2e^{-2\sqrt{2\lambda}d}}{1 - u'_1u'_2 e^{-2\sqrt{2\lambda}d}}\left[ \frac{2 d}{\sqrt{2\lambda}}
% -\frac{ e^{2\Xi_1}}{\sqrt{\lambda+s}\sqrt{\lambda+s(1+\chi_1)}}
% \frac{s\chi_1}{e^{4\Xi_1}(\lambda+s)-[\lambda+s(1+\chi_1)]}  + \{1 \leftrightarrow 2\}  \right].
% \end{align}
% where 
% \begin{equation}
% u'_i = \frac{\sqrt{\lambda+s}(1+\chi)-\sqrt{\lambda+s(1+\chi)}}{\sqrt{\lambda+s}(1+\chi)+\sqrt{\lambda+s(1+\chi)}}
% \end{equation}
% As nasty as that two-body expression may be, exactly the same tricks will work on it, 
% and it will simplify down to exactly the same form as the other polarization.  

\subsection{TM Casimir-Polder: Limiting Cases}

The TM contribution to the Casimir--Polder free energy for an atom in front of a dielectric plane 
 proceeds in the same manner as the TE case.
This time, the non-zero contributions come from the presence of $\Xi$ in the TM expressions.   
The relevant analytical expression for the path integral can be substituted in, and the Laplace-Mellin
transform can be exploited.  
\begin{align}
\cF\supTM\subCP-\cF_0%=& -k_BT{\sum_n}'\frac{\alpha(i s_n)}{4\pi\epsilon_0}\left(\frac{ s_n^2}{c^2}  - \frac{1}{2}\partial_d^2\right)\int_0^\infty d\kappa\, \frac{ s_n^2}{2c^2}\frac{e^{-2\sqrt{\kappa+1} s_n d/c}c}{ s_n\sqrt{\kappa+1}}\frac{\sqrt{\kappa+1+\chi}-\sqrt{\kappa+1}e^{2\Xi}}{\sqrt{\kappa+1+\chi}+\sqrt{\kappa+1}e^{2\Xi}} \\
=& -k_BT{\sum_n}'\frac{ s_n\alpha(i s_n)}{4\pi\epsilon_0c}
\left(\frac{ s_n^2}{c^2}  - \frac{1}{2}\partial_d^2\right)
\int_1^\infty dp\,e^{-2p s_n d/c}\frac{\sqrt{p^2+\chi}-pe^{2\Xi}}{\sqrt{p^2+\chi}+p e^{2\Xi}}.
\end{align}
In this geometry the derivatives with respect to the atom's starting postion can be exchanged for differentating
with respect to the surface's distance.  
After take the $\partial_d$ derivatives, the free energy is
\begin{align}
\cF\supTM\subCP-\cF_0%=& -k_BT{\sum_n}'\frac{ s_n\alpha(i s_n)}{4\pi\epsilon_0c}\int_1^\infty dp\,\left(\frac{ s_n^2}{c^2}  - \frac{2 s_n^2p^2}{c^2}\right)e^{-2p s_n d/c}\frac{\sqrt{p^2+\chi}-pe^{2\Xi}}{\sqrt{p^2+\chi}+p e^{2\Xi}} \\
=& -k_BT{\sum_n}'\frac{s^3_n\alpha(i s_n)}{4\pi\epsilon_0c^3}\int_1^\infty dp\,
\left(1-2p^2\right)e^{-2p s_n d/c}\frac{\sqrt{p^2+\chi(is_n)}-pe^{2\Xi(is_n)}}{\sqrt{p^2+\chi(is_n)}+p e^{2\Xi(is_n)}}.
\end{align}
Evidently our earlier results~\ref{eq:TM_CP} would also be easily recovered by taking the zero temperature limit.

\subsubsection{Zero temperature, near-field}

Let us consider the zero temperature, near field limit.  The frequency sum can be replaced by an integral.
\begin{align}
E-E_0=& -\frac{\hbar}{8\pi^2\epsilon_0c^3}\int_0^\infty d\omega \omega^3\alpha(i\omega)\int_1^\infty dp\,
\left(1-2p^2\right)e^{-2p\omega d/c}\frac{\sqrt{p^2+\chi}-pe^{2\Xi}}{\sqrt{p^2+\chi}+p e^{2\Xi}} 
\end{align}
Once again, the atom's polarizability dominates the frequency integral, so the dominant frequencies 
occur for $\omega< \omega_{j0}$.  In the near-field limit, the distances are much smaller than these
wavelengths, so $\omega d/c\ll 1$.  Since the $p$ integral is dominated by the exponential,
the relevant $p$ are large.  
% Alternatively, just take $p\sim c/(d\omega)$.
%   Since $d$ is small, then important $p$ are very large?
%   I think this implicitly takes $\omega d/c\ll1$?
In this limit, the reflection coefficient becomes
\begin{equation}
\frac{\sqrt{p^2+\chi}-p e^{2\Xi}}{\sqrt{p^2+\chi}+pe^{2\Xi}} \approx  
 -\frac{\epsr(i\omega)-1}{\epsr(i\omega)+1},
\end{equation}
where we used $\Xi = \log\sqrt{\epsr$.
Afer substituting this in, and evaluating the $p$ integral, the energy becomes
\begin{align}
E-E_0&\approx \frac{\hbar}{8\pi^2\epsilon_0c^3}\int_0^\infty d\omega\, \omega^3
\alpha(i\omega)\frac{\epsr(i\omega)-1}{\epsr(i\omega)+1}\int_1^\infty dp\,(1-2p^2)e^{-2p\omega d/c}\\
&= \frac{\hbar}{8\pi^2\epsilon_0c^3}\int_0^\infty d\omega\, \omega^3
\alpha(i\omega)\frac{\epsr(i\omega)-1}{\epsr(i\omega)+1}\left(-\frac{c^3e^{-2\omega d/c}(1+\omega d/c)^2}{2 d^3\omega^3}\right)\\
&\approx -\frac{\hbar }{16\pi^2\epsilon_0 d^3}\int_0^\infty d\omega 
\alpha(i\omega)\frac{\epsr(i\omega)-1}{\epsr(i\omega)+1},
\end{align}
which is the well known result for the van der Waals energy for an atom near a dielectric wall~\cite{}.

\subsection{High temperature, far-field}

In the high temperature limit, only the zero frequency term contributes, so the TM Casimir--Polder
energy is  
\begin{equation}
\cF\supTM\subCP-F_0=-\frac{1}{2}\beta^{-1}\frac{\alpha(0)}{2\epsilon_0}\int_0^\infty d\cT\,\frac{1}{(2\pi \cT)^{3/2}}
\dlangle -\left(-\frac{1}{2}\partial_x^2\right)e^{ - \cT\langle V_{TM}\rangle}\drangle
\end{equation}
Plugging in the analytical solution~(\ref{eq:TMpot}), and taking the derivatives gives 
\begin{align}
F-F_0=&-\frac{k_BT\alpha_0}{16\pi\epsilon_0}\frac{\epsr(0)-1}{\epsr(0)+1} 
\int_0^\infty d\cT\,\partial_d^2\frac{1}{\sqrt{2\pi }\cT^{3/2}} e^{-2 d^2/\cT}\\
=&-\frac{k_BT\alpha_0}{16\pi\epsilon_0d^3}\frac{\epsr(0)-1}{\epsr(0)+1}.
\end{align}
This is the familiar expression for the van der Waals interaction of an atom and a dielectric wall.
It seems that this duality between high temperatures and near-field is not unexpected.
This also suggests that as hard as the TM potential may be deal with, it often provides the majority
of the Casimir effect.  
We have not examined the equivalent Casimir expressions, but they could be readily evaluated. As noted 
in Eq.~\ref{eq:high_temp}, there too, the dominant contribution comes from the TM polarization, again
due to the singular TM potential.  


%%% Local Variables: 
%%% mode: latex
%%% TeX-master: "thesis_master"
%%% End: 
