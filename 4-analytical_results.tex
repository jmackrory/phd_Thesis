\chapter{Electromagnetic Worldlines - Analytical Results}
\label{ch:analytical}

This chapter will show that in planar media the worldline path integrals yield the same results as the more direct 
calculations outlined in Chapter~\ref{ch:introduction}.  
The analytical energies can be derived by evaluating an analytical average over paths for the relevant path integrals.
The worldline Casimir energies can be put in the right form to use the solutions from Chapter~\ref{ch:feynman_kac}.  
Finally, we examine the behavior of the worldline path integrals in the case of nonzero temperature, 
and show how known results for the near-field and high temperature results emerge from this formalism.    
The work presented on the TE polarization has been published in Ref.~\cite{Mackrory2016}, 
and the work on the TM polarization is also to be published.

\section{Extracting Casimir--Polder Energies}
\label{sec:casimir-polder_worldline}
The Casimir--Polder energy for an atom interacting with macroscopic bodies can be derived 
by treating the atom as a perturbation to the permittivity and permeability.  
The atom is located at $\rA$, and has polarizability $\alpha$ and magnetizability $\beta$.
The atom is perturbs the background permittivity $\epsr(\vect{r})\rightarrow \epsr(\vect{r})+\delta\epsr(\vect{r})$,
and permeability $\mur(\vect{r})\rightarrow \mur(\vect{r})+\delta\mur(\vect{r})$, where
\begin{equation}
  \delta\epsr(\vect{r})=\frac{\alpha}{\epsilon_0}\delta(\vect{r}-\rA), 
  \quad \delta\mu_r(\vect{r})=\mu_0\beta\delta(\vect{r}-\rA).
\end{equation}
Note that the delta functions are being used a short-hand notation for a sharply localized function which has unit integral.
All expansions involving these delta functions can be carried out with a finite regularization of the 
delta function, and the limit of an arbitrarily small particle can be taken at the end of the computations.  
We will initially carry out calculations for dispersion-free media, at zero temperature.  The generalization 
to nonzero temperature will be considered in Section~\ref{sec:nonzero_temp}.

The Casimir energy for TE and TM polarizations was derived in Chapter~\ref{ch:EM_quantization}.
In the zero-temperature limit, the energy in the EM field in the TE polarization is 
\begin{equation}
  E\subTE-E\sup0 = -\frac{\hbar c}{2}\int_0^\infty\frac{d\cT}{(2\pi\cT)^{D/2}\cT}\int d\vect{x}_0
  \biggdlangle
  \frac{e^{-\langle V\subTE(\vect{x})\rangle\cT}}{\sqrt{\langle \epsr(\vect{x})\mur(\vect{x})\rangle}} -1
  \biggdrangle_{\vect{x}(t)}.\label{eq:TE_energy}
\end{equation}
The Casimir--Polder energy comes from expanding the energy to linear order in the $\alpha$ and $\beta$.
The expansions in $\alpha$ and $\beta$ must be carried out in both $\langle\epsr\mur\rangle$, and the potential $V\subTE$.
%This expansion corresponds to extracting the lowest order linear response.
Considering the similarities between the polarizations, we will carry these expansions out for only
one polarization, since the others follow by duality.  

% It is important be clear about exactly what the energy is, and how it is computed.  
% As discussed in \S 4.7 of Jackson~\cite{Jackson1998}, the electrostatic energy for a system depends on the 
% manner in which the system was arranged.  Different energies are found if the free charge or potential
% are specified.  The same is true of magnetostatic systems: are the potentials on the boundaries, or the free 
% currents specified.  

% The energy for an atom in the electromagnetic field can be found by considering the change in the energy 
% from adding the perturbation. 
The energy can be written as a functional of the permittivity and permeability, $E[\epsr,\mur]$.
The change in energy for adding an atom is then
\begin{equation}
  \delta E[\epsr,\mur] = E[\epsr+\delta\epsr,\mur+\delta\mur]-E[\epsr,\mur].
\end{equation}
The expansions can be carried out with $\alpha_0/\epsilon_0$ and $\mu_0\beta_0$ acting as the small parameters.
The Casimir--Polder energy can be found by expanding the change in energy to linear order 
in $\alpha_0/\epsilon_0$ and $\mu_0\beta_0$, which corresponds to taking the following functional derivatives:
\begin{equation}
  V\subCP(\rA) = \frac{\alpha_0}{\epsilon_0}\frac{\delta}{\delta\epsr(\rA)}E+\mu_0\beta_0\frac{\delta}{\delta\mur(\rA)}E,
\end{equation}
where $\rA$ is the atom's location.  
The Casimir--Polder energy must be renormalized by considering the change in the energy as the atom is removed 
arbitrarily from the dielectric objects.

The expansion for the path-averaged permittivity and permeability is
\begin{align}
  \langle(\epsr+\delta\epsr)(\mur+\delta\mur)\rangle^{-1/2} &= \langle\epsr\mur\rangle^{-1/2}
  -\frac{1}{2}\langle \mur\delta\epsr+\epsr\delta\mur\rangle\langle\epsr\mur\rangle^{-3/2}\nonumber\\
&= \langle\epsr\mur\rangle^{-1/2}
-\frac{1}{2}\frac{\alpha_0}{\epsilon_0}\langle \mur(\vect{x})\delta(x-\rA)\rangle\langle\epsr\mur\rangle^{-3/2}\nonumber\\
&\hspace{1cm} -\frac{1}{2}\mu_0\beta_0\langle\epsr(\vect{x})\delta(\vect{x}-\rA)\rangle \langle\epsr\mur\rangle^{-3/2}.
\label{eq:mueps_expansion}
\end{align}
The singular potentials $V\subTE, V\subTM$ can be expanded in the same fashion,
\begin{align}
  \langle V\subTE[\mur +\delta\mur] \rangle 
  =& \frac{1}{2}\Big<(\nabla\log\sqrt{\mur+\delta\mur})^2-\nabla^2\log\sqrt{\mur+\delta\mur}\Big>\nonumber\\
  % =& \frac{1}{8} \Big< [\nabla\log(\mur+\delta\mur)]^2\Big>
  % -\frac{1}{4}\Big< \nabla^2\log(\mur+\delta\mur)\Big> \nonumber\\
  =& \langle V\subTE[\mur]\rangle+\left< \frac{1}{4} \nabla\log\mur\cdot\nabla\frac{\delta\mur}{\mur}
    -\frac{1}{4}\nabla^2\frac{\delta\mur}{\mur}\right> .
  \label{eq:VTE_expansion}
\end{align}
It is straightforward to then expand the exponential using
\begin{equation}
  e^{-\cT\langle V\subTE[\mur+\delta\mur]\rangle}=e^{-\cT\langle V\subTE[\mur]\rangle}(1-\cT\langle\delta V\subTE[\mur]\rangle),
\end{equation}
where $\delta V\subTE$ is the second term in Eq.~(\ref{eq:VTE_expansion}).
The terms involving $\nabla \delta\mur$ will yield derivatives (such as $\nabla^2$) acting on the path integral after a integration by parts.
% \begin{equation}
%   \int d\vect{x} f(\vect{x})\nabla\delta(\vect{x}-\rA)  = -\nabla f(\vect{x})\bigg|_{\vect{x}=\rA}.
% \end{equation}
In all of these expansions, the path-averaged delta functions act to restrict the path integrals to paths starting at the atom's
position $\rA$.  
The path integral can be written schematically as some path-averaged function that depends the whole path and includes 
 a path-averaged delta function,
\begin{equation}
  I = \int d\vect{x}_0\Bigdlangle \,\langle f(\vect{x})\rangle\langle g(\vect{x})\delta(\vect{x}-\rA)\rangle\,\Bigdrangle_{\vect{x}(t)}.
\end{equation}
In discrete notation this is 
\begin{equation}
  I = \int \prod_{n=0}^{N-1}dx_n\,P(x_0,\cdots, x_{N-1}) \frac{1}{N}\sum_{k=0}^{N-1}f(x_k)\frac{1}{N}\sum_{j=0}^{N-1}\delta(\vect{x}_j-\rA)g(\vect{x}_j)
\end{equation}
where the delta function enforces path closure.  All of the functions are invariant under cyclic permutations 
of the path labels.  This is true of the path-averaged functions such as $\langle \epsr \mur\rangle$ and $\langle \VTM\rangle$,
and the Gaussian probability for closed Brownian bridges.
Then for each term $\delta(\vect{x}_j-\rA)$, the labels can be permuted $j$ times so that in the shifted
coordinates $\vect{x}_j\rightarrow \vect{x}_0$.  Since there is now a sum of $N$ identical terms, the 
path integral can be written as
\begin{equation}
  I = \int \prod_{n=0}^{N-1}dx_n\,P(\vect{x}_0,\cdots, \vect{x}_{N-1}) \sum_{k=0}^{N-1}f(\vect{x}_k)\delta(\vect{x}_0-\rA)g(\vect{x}_0)
= \Bigdlangle\, \langle f\rangle g(\rA)\,\Bigdrangle_{\vect{x}(t), \vect{x}(0)=\rA}.\label{eq:delta_pin}
\end{equation}
% For a closed path, integrated over all space, with integrands athat are all written as averaged around the path, 
% there is some freedom in which point of the path is called the origin.  
Since only paths the satisfy the delta function constraint will contribute to the path integral, 
we are free to call the point at $\rA$ the path origin.  
% The end result is that the path-averaged delta function restricts the starting point of the paths to the atom's
% position $\rA$.

Using the results in Eqs.~(\ref{eq:mueps_expansion}) and (\ref{eq:VTE_expansion}), the Casimir--Polder energy
for the TE polarization can be written 
\begin{align}
    V\supTE\subCP(\vect{\rA}) &= \frac{\hbar c}{4}\int_0^\infty\frac{d\cT}{(2\pi\cT)^{D/2}\cT}\int d\vect{x}_0
    \biggdlangle
    \left(  \frac{\langle\mur\delta\epsr+\epsr\delta\mur\rangle}
    {\langle \epsr\mur\rangle^{3/2}}\right) 
  e^{-\langle V\subTE\rangle\cT} \nonumber\\
  &\hspace{1cm}+ e^{-\langle V\subTE\rangle\cT}\left(+\frac{\cT}{2\langle\epsr\mur\rangle^{1/2}}
    \left<  (\nabla\log\mur)\cdot\nabla\frac{\delta\mur}{\mur}
      -\nabla^2\frac{\delta\mur}{\mur}\right> \right)
    \biggdrangle_{\vect{x}(t)}.
\end{align}
Then after manipulating the path-averaged delta functions using Eq.~(\ref{eq:delta_pin}), 
and integrating by parts, the Casimir--Polder energy is
\begin{align}
    V\supTE\subCP(\vect{\rA}) &= \frac{\hbar c}{4}\int_0^\infty\frac{d\cT}{(2\pi\cT)^{D/2}\cT}
    \biggdlangle
    \left(  \frac{\alpha_0\mur(\rA)}{2\epsilon_0\langle \epsr\mur\rangle^{3/2}}
      +\frac{\beta_0\mu_0\epsr(\rA)}{2\langle \epsr\mur\rangle^{3/2}}\right) e^{-\langle V\subTE\rangle\cT} \nonumber\\
    &\hspace{1cm}-\frac{\cT}{2}\frac{\beta_0\mu_0}{\mur(\rA)}\left[
     \nabla^2      +(\nabla^2\log\mur)+ (\nabla\log\mur)\cdot\nabla\right]
    \frac{ e^{-\langle V\subTE\rangle\cT}}{\langle\epsr\mur\rangle^{1/2}}
    \biggdrangle_{\vect{x}(t),\vect{x}(0)=\rA}.
\end{align}
Note that the gradients in parentheses such as $(\nabla\log\mur)$ should be interpreted as functions, while the 
other derivative operators act on everything to their right.  The remaining derivatives act with respect to 
the path origin $\vect{x}_0=\rA$.
The corresponding TM Casimir--Polder energy can be found via the duality transformation, and is given by 
\begin{align}
    V\supTM\subCP(\vect{\rA}) &= -\frac{\hbar c}{2}\int_0^\infty\frac{d\cT}{(2\pi\cT)^{D/2}\cT}
    \biggdlangle
    \left( - \frac{\alpha_0\mur(\rA)}{2\epsilon_0\langle \epsr\mur\rangle^{3/2}}
      -\frac{\beta_0\mu_0\epsr(\rA)}{2\langle \epsr\mur\rangle^{3/2}}\right) e^{-\langle V\subTM\rangle\cT} \nonumber\\
    &\hspace{1cm}+\frac{\cT}{4}\frac{\alpha_0}{\epsilon_0\epsr(\rA)}\left[
     \nabla^2      +\nabla^2(\log\epsr) + \nabla(\log\epsr)\cdot\nabla\right]
    \frac{ e^{-\langle V\subTM\rangle\cT}}{\langle\epsr\mur\rangle^{1/2}}
    \biggdrangle_{\vect{x}(t),\vect{x}(0)=\rA}.
\end{align}
These expressions can be further simplified if the atom is in a region where the dielectric is not varying spatially
(which implies that $\nabla\log\sqrt\epsr(\rA)=0$), 
and we consider non-magnetic atoms and media where $\beta_0=0$ and $\mur=1$.  
In this case, the TE and TM Casimir--Polder energies are given by 
\begin{align}
    V\supTE\subCP(\vect{\rA}) &= \frac{\hbar c\alpha_0}{4\epsilon_0(2\pi)^{D/2}}\int_0^\infty\frac{d\cT}{\cT^{1+D/2}}
    \biggdlangle
      \frac{1}{\langle \epsr\rangle^{3/2}}
    %   \right) \nonumber\\
    % &\hspace{1cm}
      \biggdrangle_{\vect{x}(t),\vect{x}(0)=\rA}
      \label{eq:TE_Casimir_Polder}\\
    V\supTM\subCP(\vect{\rA}) &= \frac{\hbar c\alpha_0}{4\epsilon_0(2\pi)^{D/2}}\int_0^\infty\frac{d\cT}{\cT^{1+D/2}}
    \biggdlangle
      \frac{e^{-\langle V\subTM\rangle\cT}}{\langle \epsr\rangle^{3/2}}
      -\frac{\cT}{2\epsr(\rA)} \nabla^2 \frac{ e^{-\langle V\subTM\rangle\cT}}{\langle\epsr\rangle^{1/2}}
      \biggdrangle_{\vect{x}(t),\vect{x}(0)=\rA}.\label{eq:TM_Casimir_Polder}
\end{align}
These expressions should be renormalized by subtracting off the equivalent expressions with a constant dielectric of permittivity $\epsr(\rA)$.
This corresponds to finding the change in energy for the atom when it is brought to 
a finite distance from dielectric, after starting arbitrarily far away.
In non magnetic media, the TE Casimir--Polder energy is the simpler case to evaluate 
since it only depends on $\langle\epsr\rangle$, which is well behaved.
By contrast, the TM Casimir--Polder energy involves the singular TM potential, which implies the need for spatial derivatives.
Both of those factors will require some care in numerical methods involving stochastic paths interacting with 
discontinuous or singular potentials.

\section{Rearranging Worldline Casimir Energies}

The TE and TM worldline energies can be some rearranged in order to use the analytical results for 
path integrals derived in Chapter~\ref{ch:feynman_kac}.
 This can be done with two integral identities.
The first identity converts the worldline path integral into a form involving the Laplace transform of the path
integral.
The second identity puts the prefactor $\langle\epsr\rangle$ in exponential form by means of the Gamma function.  

\subsection{ Laplace-Mellin Transforms}

The worldline path integral has the form of a Mellin transform.  
The Mellin transform of a function $f$ is defined as 
\begin{equation}
\mathcal{M}[f](z)= \int_0^\infty dt\, t^{z-1}f(t).
\end{equation}
The Mellin transform also appears in the context of $\zeta$-function renormalization for functional determinants~\cite{Elizalde2008},
which is closely related to the worldline path integral.  
In the application to the worldline path integral, $f$ will be the ensemble-averaged path integral and $z$ will be $1+D/2$.

There is a useful relationship between Laplace transforms and Mellin transforms~\cite{Lew1975}.  
The Laplace transform was defined in Eq.~(\ref{eq:Laplace}), and the $\Gamma$ function is defined as  
\begin{equation}
\Gamma(z) = \int_0^\infty ds\, s^{z-1} e^{-s} = \mathcal{M}[e^{-s}](z).
\end{equation}
The Laplace-Mellin theorem~\cite{Lew1975} states that
\begin{equation}
  \Gamma(1-z)\mathcal{M}[f](z) = \mathcal{M}\big[\mathcal{L}[f]\big](1-z)\label{eq:Laplace-Mellin}.
\end{equation}
This relation is most easily motivated by starting with the right hand side:
\begin{equation}
\mathcal{M}\big[\mathcal{L}[f]\big](1-z) = 
\int_0^\infty ds\, s^{-z} \int_0^\infty dt\,e^{-st} f(t).
\end{equation}
The order of $s$ and $t$ integration can be swapped, and  $s\rightarrow t/u$, with the 
result
% =& \int_0^\infty dt\,\left[\int_0^\infty ds s^{-z} e^{-st}\right] f(t)\\
% =& \int_0^\infty dt\,\left[\int_0^\infty d\frac{u}{t}\, t^zu^{-z} e^{-u}\right] f(t) \\
\begin{align}
\mathcal{M}\big[\mathcal{L}[f]\big](1-z)=&\int_0^\infty dt\,\int_0^\infty du\, u^{-z} e^{-u}\,t^{z-1} f(t) \\
=& \Gamma(1-z)\mathcal{M}[f](z).
\end{align}
% In words, the Mellin transform of a function is proportional to the Mellin transform of the Laplace transform of the function,
%  and subject to a change of variable $z\rightarrow 1-z$. 
This result can be used to rewrite worldline path integrals in terms of their Laplace transforms. 
This is useful since the solution method in Chapter~\ref{ch:feynman_kac} naturally yields the Laplace transform of the path integral.

\subsection{Inverse Moment Theorem}

One further step is required to put all of the material functions in the path integral into exponential form.
This is necessary since the solutions from the previous chapter were for path integrals with exponential potentials.
If positive powers were required, then the usual moment generating tricks could be used 
such as $\langle x\rangle^n = \frac{d^n}{ds^n}e^{-s\langle x\rangle}\big|_{s=0}$.
However, for the inverse moments required in the worldline method, the following integral transformation
involving the Gamma function can be used 
\begin{equation}
\frac{1}{\Gamma[\alpha]}\int_0^\infty ds\,s^{\alpha-1}\dlangle e^{-s(x+\beta)}\drangle  
= \dlangle \frac{1}{(x+\beta)^\alpha}\drangle.\label{eq:moment_theorem}
\end{equation}
This is restricted to $x+\beta>0$ and $\alpha>0$.
In the worldline calculations $x+\beta$ will be $\langle\epsr(\vect{x})\rangle$, where the dielectric 
is a real, positive function.  In addition, $\alpha$ will be $1/2$ for Casimir energies, and $3/2$
for Casimir--Polder energies, respectively.  
%On the imaginary frequency axis, the dielectric functions are real, positive, decaying functions, so all of these conditions are satisfied for worldline path integrals.

\subsection{Rewriting the Worldline in Analytical Form}

As an example, consider the TE path integral, with dielectric function $\epsr(\vect{x})=1+\chi(\vect{x})$,
where $\chi(\vect{x})$ is the space-dependent dielectric susceptibility.
In both the Casimir and Casimir--Polder cases, the energy involves the factor
$\langle \epsr\rangle^{-\alpha}$, with $\alpha=1/2$ and $\alpha=3/2$ respectively.  
The energy density can be rewritten using the inverse moment theorem~(\ref{eq:moment_theorem}):  
\begin{align}
\int_0^\infty\!\! \frac{d\cT}{\cT^{1+D/2}}\biggdlangle\frac{1}{\langle 1+\chi(\vect{x})\rangle^\alpha} \biggdrangle_{\vect{x}(t)}
% = &\int_0^\infty \frac{dT}{T^{1+D/2}}\frac{1}{\Gamma[\alpha]}\int_0^\infty ds s^{\alpha-1} 
% \dlangle e^{-s(\chi \int_0^T dt \Theta(x-d) +1)}\drangle \\
% =&\frac{1}{\Gamma[\alpha]}\int_0^\infty \frac{dT}{T^{1+D/2-\alpha}}\int_0^\infty ds s^{\alpha-1} e^{-s T}
% \dlangle e^{-s \chi \int_0^T dt \Theta(x-d)}\drangle \\
\hspace{-0.15cm}=&\int_0^\infty ds\, \frac{s^{\alpha-1}}{\Gamma(\alpha)}\int_0^\infty\!\! \frac{d\cT}{\cT^{1+D/2-\alpha}}
\dlangle e^{-s\cT- \int_0^\cT dt \,\chi[\vect{x}(t)]}\drangle_{\vect{x}(t)}.
\end{align}
In the second equality the integration variable was rescaled to $s\rightarrow s\cT$,
 and the definition of the path average, $\langle f\rangle = \cT^{-1}\int_0^\cT dt\, f(t)$ was used.
% where we used the inverse moment theorem, rescaled the $\lambda\rightarrow \lambda T$,
%  and swapped the order of integration. We can see that that $T$ integral has the form of a Mellin transform.
The energy density can be further transformed with the Laplace-Mellin theorem~(\ref{eq:Laplace-Mellin}), 
\begin{align}
\int_0^\infty \frac{d\cT}{\cT^{1+D/2}}\biggdlangle\frac{1}{\langle 1+\chi(\vect{x})\rangle^\alpha} \biggdrangle_{\vect{x}(t)}
% =& \int_0^\infty \frac{dT}{T^{1+z-1/2}}e^{-sT}\dlangle \frac{e^{-s \chi \int dt_0^T dt \Theta(x-d)}}{\sqrt{T}}\drangle\\
% =&\mathcal{M}\left[e^{-sT}\dlangle \frac{e^{-s \chi \int_0^T dt \Theta(x-d)}}{\sqrt{T}}\drangle\right]\left(-z+1/2\right) \\
% =& \frac{1}{\Gamma[1+z-1/2]}\mathcal{M}\left[\int_0^\infty dT e^{-(\lambda+s)T}
% \dlangle \frac{e^{-s \chi \int_0^T dt \Theta(x-d)}}{\sqrt{T}}\drangle\right]\left(-z+1/2\right) \\
&= \int_0^\infty ds\, \frac{s^{\alpha-1}}{\Gamma(\alpha)}\int_0^\infty d\lambda\, 
\frac{\lambda^{(D-n)/2-\alpha}}{\Gamma[(D-n)/2-\alpha+1]}\nonumber\\
&\hspace{1cm}\times\int_0^\infty d\cT e^{-(\lambda+s)\cT}
\dlangle \frac{e^{-s \int_0^\cT dt\,\chi(\vect{x})}}{\cT^{n/2}}\drangle_{\vect{x}(t)}.\label{eq:Casimir_Laplace_inverse}
\end{align}
In the last line we have factored out $\cT^{n/2}$ as the normalization for a $n$-dimensional Brownian bridge, 
which assumes that the path integral solution was computed in $n$-dimensions.
We will typically work in with planar media where $n=1$.  
(Despite knowing the specific values for $\alpha, D$ and $n$, it is useful to track them in calculations.)
The Laplace transformed path integral can be computed as the solution~(\ref{eq:f_soln}) 
to the relevant diffusion equation~(\ref{eq:diffusion_eq}), as discussed in Chapter~\ref{ch:feynman_kac}.

\section{Analytical  TE Casimir--Polder Energy for an Atom and a Dielectric Plane}
\label{sec:TE_CP}

The TE contribution to the Casimir--Polder energy for an atom interacting with a dielectric body is given 
by combining these formal manipulations with the relevant path integral solution.  
For an atom at the origin, interacting with a planar dielectric interface $\epsr(z)=1+\chi\Theta(x-d)$, 
the path-integral solution is given in terms of Eq.~(\ref{eq:Feynman-Kac TE one step}).
Under the rescalings $s\rightarrow s\chi, \lambda\rightarrow \lambda+s$,
 the renormalized TE Casimir--Polder potential is
\begin{align}
V\subCP\supTE-V\subCP\sup0=&-\frac{\hbar c\alpha_0}{4\epsilon_0(2\pi)^{D/2}}
\frac{\sqrt{\pi}}{\Gamma[\alpha]\Gamma\left[(D+1)/2-\alpha\right]}
\int_0^\infty ds\, s^{\alpha-1}\int_0^\infty d\lambda\, \lambda^{(D-1)/2-\alpha}\nonumber\\
&\times\frac{e^{-2\sqrt{2(\lambda+s)}|d|}}{\sqrt{\lambda+s}} 
\frac{\sqrt{\lambda+s(1+\chi)}-\sqrt{\lambda+s}}{\sqrt{\lambda+s(1+\chi)}+\sqrt{\lambda+s}},
\label{eq:VCP_TE_inter}
\end{align}
This can be put into the same form as the known results by changing integration variables.
The integral over $s$ and $\lambda$ have the form:
\begin{equation}
  J=\int_0^\infty ds\, s^{\alpha-1}\int_0^\infty d\lambda\, \lambda^{(D-1)/2-\alpha}\frac{e^{-2\sqrt{2(\lambda+s)}|d|}}{\sqrt{\lambda+s}} 
\frac{\sqrt{\lambda+s(1+\chi)}-\sqrt{\lambda+s}}{\sqrt{\lambda+s(1+\chi)}+\sqrt{\lambda+s}}.
\end{equation}
This can be transformed by changing variable from $\lambda$ to $p:=\sqrt{\lambda/s+1}$, 
\begin{align}
  J %&=2\int_0^\infty ds\, s^{\alpha-1}s^{(D-1)/2-\alpha+1-1/2}\int_1^\infty dv\, (p^2-1)^{(D-1)/2-\alpha}e^{-\sqrt{8 d^2s}p}
 %  \frac{\sqrt{p^2+\chi}-p}{\sqrt{p^2+\chi}+p},\\
  &=2\int_0^\infty ds\, s^{D/2-1}\int_1^\infty dp\, (p^2-1)^{(D-1)/2-\alpha}e^{-\sqrt{8 d^2s}p}
  \frac{\sqrt{p^2+\chi}-p}{\sqrt{p^2+\chi}+p}.
\end{align}
After changing variables from $s$ to $t:=\sqrt{8d^2 s}\,p$, and swapping the $t$ and $p$ integrals, the result is
\begin{align}
  J % =4\int_1^\infty dp\, (p^2-1)^{(D-1)/2-\alpha} \int_0^\infty dt\, \frac{2t}{8d^2p^2}\left(\frac{t^2}{8d^2p^2}\right)^{D/2-1}e^{-t}
  % \frac{\sqrt{p^2+\chi}-p}{\sqrt{p^2+\chi}+p}\\
=\frac{1}{2^{3D/2-2}d^D}\int_1^\infty dp\,p^{-D} (p^2-1)^{(D-1)/2-\alpha} 
  \frac{\sqrt{p^2+\chi}-p}{\sqrt{p^2+\chi}+p}\int_0^\infty dt\, t^{D-1}e^{-t}.\label{eq:J}
\end{align}
The $t$ integral has the value, $\Gamma[D]$.  
Substituting the transformed integral~(\ref{eq:J}) back into the Casimir--Polder energy~(\ref{eq:VCP_TE_inter}), while setting $D=4$
and $\alpha=3/2$ yields
\begin{align}
  V\subCP\supTE-V\subCP\sup0
% =\,&-\frac{\hbar c\alpha_0}{4\epsilon_0(2\pi)^{D/2}}
% \frac{\sqrt{\pi}}{\Gamma[\alpha]\Gamma\left[(D+1)/2-\alpha\right]}\frac{\Gamma[D]}{2^{3D/2-2}d^D}\int_1^\infty dp\,p^{-D} (p^2-1)^{(D-1)/2-\alpha} 
%   \frac{\sqrt{p^2+\chi}-p}{\sqrt{p^2+\chi}+p}\\
% =\,&-\frac{\hbar c\alpha_0}{2^{2D-2} \pi^{D/2}\epsilon_0d^D}
% \frac{\sqrt{\pi}6}{(\sqrt{\pi}/2}\int_1^\infty dp\,p^{-D} (p^2-1)^{(D-1)/2-\alpha} 
%   \frac{\sqrt{p^2+\chi}-p}{\sqrt{p^2+\chi}+p}\\
=\,&-\frac{3\hbar c\alpha_0}{32\epsilon_0\pi^{2}d^4}
  \int_1^\infty dp\,\frac{1}{2p^4} \frac{\sqrt{p^2+\chi}-p}{\sqrt{p^2+\chi}+p}.
\end{align}
The prefactor is the Casimir--Polder energy for an atom above a perfect conductor~(\ref{eq:CP_conductor}).
This result agrees with the known result for the TE contribution to the Casimir--Polder energy~(
[see Eq.~(14.210) in \S 14.3 of Steck~\cite{SteckNotes}].
An ``efficiency'' $\eta\subTE$ can be defined by dividing the Casimir--Polder energy for an atom 
and a dielectric by the Casimir--Polder energy between an atom and a perfect conductor.  
The integral over $p$ is the TE contribution to the efficiency, $\eta\subTE$, and can be   
 evaluated in closed form
\begin{align}
\eta\subTE(\chi)=\,&\frac{1}{2}\int_{1}^\infty dp\,p^{-4}\frac{p-\sqrt{p^2+\chi}}{p+\sqrt{p^2+\chi  }}\nonumber \\
=\,&\frac{1}{3}+\frac{2}{\chi}- \frac{\sqrt{\chi  (\chi +1)}}{\chi^{3/2}}
-\frac{1}{4\chi^{3/2}}\log \left[2 \chi +2 \sqrt{\chi  (\chi+1)}+1\right]
-\frac{\text{arcsinh}\left(\sqrt{\chi }\right)}{2\chi^{3/2}}.
\label{eq:etaTE}
\end{align}
The efficiency $\eta\subTE$ smoothly interpolates between $0$ and $1/6$ as $\chi$ varies from $0$ to $\infty$.
In the strong-coupling limit, the TE polarization provides $1/6$ of the Casimir--Polder energy.  
The remaining $5/6$ is provided by the TM polarization.  

\section{Analytical TM Casimir--Polder Energy for an Atom and a Dielectric Plane}

The calculation for the TM Casimir--Polder energy proceeds in a similar fashion to the TE case.  
The renormalized TM Casimir--Polder energy can be split into two pieces 
\begin{align}
  V\supTM\subCP(\vect{\rA})-V\subCP\sup0 &= \frac{\hbar c\alpha_0}{4\epsilon_0(2\pi)^{D/2}}
  \bigg(\cV_{D,3/2}-\frac{1}{2}\nabla^2\cV_{(D-2),1/2}\bigg),\label{eq:TM_CP_inter}
\end{align}
where 
\begin{equation}
  \cV_{\nu,\alpha}:=\int_0^\infty\frac{d\cT}{\cT^{1+\nu/2}}
  \biggdlangle\frac{e^{-\langle V\subTM\rangle\cT}}{\langle \epsr\rangle^{\alpha}}-1
  \biggdrangle_{\vect{x}(t),\vect{x}(0)=\rA}.
\end{equation}
Each term $\cV_{\nu,\alpha}$ can in turn be transformed using the combination of the Laplace-Mellin theorem
and the inverse moment theorem.  
\begin{align}
  \cV_{\nu,\alpha}:=\,&\frac{\sqrt{\pi}}{\Gamma[\alpha]\Gamma[(\nu+1)/2-\alpha]}\int_0^\infty d\lambda\,\lambda^{(\nu-1)/2-\alpha}
  \int_0^\infty ds\,s^{\alpha-1}\nonumber\\
  &\times\int_0^\infty d\cT \frac{e^{-(\lambda+s)\cT}}{\sqrt{\cT}}
  \Bigdlangle e^{-\int_0^\cT dt\,[V\subTM(\vect{x}) + s\chi(\vect{x)]}}-1  \Bigdrangle_{\vect{x}(t),\vect{x}(0)=\rA}\label{eq:TM_subpart}.
\end{align}
Note that the TM-potential is already in the exponential, so the TM potential and $\Xi$ does not need to be rescaled by $s$. 
In this sense $\Xi$ is an independent parameter from $\chi$. 
The analytical result~(\ref{eq:Feynman-Kac TM one step}) can be substituted into Eq.~(\ref{eq:TM_subpart}),
and the integral can be transformed in the same manner as in Section~\ref{sec:TE_CP}, 
\begin{align}
  \cV_{\nu,\alpha}=\,&\frac{\sqrt{\pi}}{\Gamma[\alpha]\Gamma[(\nu+1)/2-\alpha]}\int_0^\infty d\lambda\,\lambda^{(\nu-1)/2-\alpha}
  \int_0^\infty ds\,s^{\alpha-1}  \frac{e^{-2\sqrt{2(\lambda+s)}|d|}}{\sqrt{\lambda+s}} r\subTM\\
%
% =\,&\frac{1}{\Gamma[\alpha]\Gamma[(\nu+1)/2-\alpha]}\int_0^\infty ds\,s^{\alpha-1} \int_1^\infty dp\,2sp s^{(\nu-1)/2-\alpha}(p^2-1)^{(\nu-1)/2-\alpha}
%  \frac{e^{-\sqrt{8sd^2}p}}{\sqrt{s}p} % r\subTM\\
% =\,&\frac{2}{\Gamma[\alpha]\Gamma[(\nu+1)/2-\alpha]}\int_0^\infty ds\,s^{(\nu-2)/2} \int_1^\infty dp\,(p^2-1)^{(\nu-1)/2-\alpha}
%  e^{-\sqrt{8sd^2}p}r\subTM\\
=\,&\frac{2}{\Gamma[\alpha]\Gamma[(\nu+1)/2-\alpha]}\int_0^\infty dt\,
\left(\frac{2t}{2^3 d^2p^2}\right)\left(\frac{t^2}{8 d^2p^2}\right)^{(\nu-2)/2} \int_1^\infty dp\,(p^2-1)^{(\nu-1)/2-\alpha}
 e^{-t}r\subTM\\
=\,&\frac{\sqrt{\pi}\Gamma[\nu]}{\Gamma[\alpha]\Gamma[(\nu+1)/2-\alpha]2^{3(\nu-2)/2+1} d^{\nu}}
\int_1^\infty dp\,(p^2-1)^{(\nu-1)/2-\alpha} p^{-\nu}
r\subTM\\
%
% =&\frac{(-1)\sqrt{2\pi}}{\Gamma[\alpha]\Gamma[z-\alpha+1/2]}\int_0^\infty d\lambda\,\lambda^{z-1/2-\alpha}
\end{align}


\begin{align}
  \cV_{\nu,\alpha}  =\,&-\frac{\sqrt{\pi}\Gamma[\nu]}{2^{3\nu/2+2} d^{\nu }\Gamma[\alpha]\Gamma[(\nu+1)/2 -\alpha]}
  \int_1^\infty dp\,\frac{1}{p^{\nu }}(p^2-1)^{(\nu -1)/2-\alpha}
  \frac{ pe^{2\Xi}-\sqrt{p^2+\chi}} {p e^{2\Xi}+\sqrt{p^2+\chi}}.
\end{align}
The two cases of interest are for $\nu =4, \alpha=3/2$ and $\nu =2,\alpha=1/2$.
\begin{align}
\cV_{2,3/2} %=&-\frac{\sqrt{\pi}\Gamma(4)}{8^{2}d^{4}\frac{\sqrt{\pi}}{2}\Gamma(1)}\int_1^\infty dp\,\frac{1}{p^{4}}(p^2-1)^{2-1/2-3/2}
  % \frac{ pe^{2\Xi}-\sqrt{p^2+\chi}} {p e^{2\Xi}+\sqrt{p^2+\chi}}.
=\,&-\frac{6}{32d^{4}}\int_1^\infty dp\,\frac{1}{p^{4}}\frac{ pe^{2\Xi}-\sqrt{p^2+\chi}} {p e^{2\Xi}+\sqrt{p^2+\chi}},\\
\cV_{1,1/2}=\,&-\frac{1}{8d^{2}}\int_1^\infty dp\,\frac{1}{p^{2}}
  \frac{ pe^{2\Xi}-\sqrt{p^2+\chi}} {p e^{2\Xi}+\sqrt{p^2+\chi}}.
\end{align}
In Eq.~(\ref{eq:TM_CP_inter}), the derivatives with respect to the starting position $\rA$ 
are equivalent to derivatives with respect to the distance $d$, and can be straightforwardly
evaluated.  
After substituting this back into the atom-surface energy~(\ref{eq:TM_CP_inter}), the TM Casimir--Polder energy is given by 
\begin{align}
  V\supTM\subCP(\vect{\rA})-V\subCP\sup0 &= -\frac{3\hbar c\alpha_0}{32\pi^2\epsilon_0d^4}\frac{1}{2}
  \int_1^\infty dp\,p^{-4}(1-2p^2)  \frac{ p(1+\chi)-\sqrt{p^2+\chi}}{p(1+\chi)+\sqrt{p^2+\chi}},\label{eq:TM_CP}
\end{align}
where we used $e^{2\Xi}=1+\chi$.  This agrees with the Lifshitz results for the TM Casimir--Polder energy
for an atom near a dielectric half-space [Eq.~(14.205) in Steck~\cite{SteckNotes}].
The TM efficiency $\eta\subTM$ also has a closed-form solution, 
\begin{align}
  \eta\subTM(\chi):=\,&\frac{1}{2}
  \int_1^\infty dp\,p^{-4}(1-2p^2)  \frac{ p(1+\chi)-\sqrt{p^2+\chi}}{p(1+\chi)+\sqrt{p^2+\chi}}\nonumber\\
  =\,& \frac{7}{6} + \chi + \frac{2 - (1+\chi)^{3/2}}{2\chi} 
  - \frac{\text{arcsinh}\sqrt{\chi}}{2\chi^{3/2}}[1 + \chi + 2\chi^2(1 + \chi)] \nonumber\\ 
  &+ \frac{(1+\chi)^2}{\sqrt{2+\chi}}\left[\text{arcsinh}\sqrt{1+\chi} - \text{arcsinh}\left(\frac{1}{\sqrt{1+\chi}}\right)\right],
  \label{eq:etaTM}
\end{align}
which smoothly interpolates between $0$ and $5/6$ as $\chi$ increases from $0$ to $\infty$.
The TM polarization provides the majority of the Casimir--Polder energy for between an atom and a dielectric plane.  
From the worldline point of view, most of the TM energy comes from the term involving the second derivative, 
which suggest it is essential to correctly estimate the derivatives in a numerical procedure.  

\section{Analytical TE Casimir Energy between Two Dielectric Planes}
\label{sec:TE_energy}
The Casimir energy for two dielectric planes can also be calculated within this formalism.  
The dielectric function is given by 
\begin{equation}
  \epsrab(x) = 1+\chi_1\Theta(d_1-x)+\chi_2\Theta(x-d_2).\label{eq:eps12}
\end{equation}
The calculation proceeds in the same way, except for two changes.  
First, the Casimir energy requires a further integral over the starting points of the paths.
Second, the two-body interaction energy is found by subtracting the one-body energies involving 
$\epsra$ and $\epsrb$ from the two-body  expressions with $\epsrab$.
 This subtraction renormalizes the energy by considering the change in energy as the two dielectrics are moved from arbitrarily
far apart to a finite distance from one another.  
The fully renormalized Casimir energy between two planes is
\begin{align}
  E\subTE-E\sup0 &= -\frac{\hbar c}{2(2\pi)^{D/2}}\int_0^\infty\frac{d\cT}{\cT^{1+D/2}}\int d\vect{x}_0
  \biggdlangle
  \bigg(\frac{1}{\sqrt{\langle \epsrab\rangle}}-\frac{1}{\sqrt{\epsrab(\vect{x}_0)}}\bigg) \nonumber\\
&\hspace{1cm}  -\bigg(\frac{1}{\sqrt{\langle \epsra\rangle}}-\frac{1}{\sqrt{\epsra(\vect{x}_0)}}\bigg)
  -\bigg(\frac{1}{\sqrt{\langle \epsrb\rangle}}-\frac{1}{\sqrt{\epsrb(\vect{x}_0)}}\bigg)
    \biggdrangle_{\vect{x}(t)}.\label{eq:TE_Casimir}
  \end{align}
  Each term is renormalized by subtracting off the constant value of the dielectric evaluated at the 
  start of the paths.  This is chosen to eliminate the $\cT=0$ divergence from each term individually.  
  (In this case it is also possible to renormalize the energy by instead subtracting off the value 
  of the surrounding medium, which is vacuum in this case.)  Subtracting off the one-body energies then removes 
  the divergences that occur at the interfaces at $x_0= d_1$ and $x_0= d_2$, 
  where at small $\cT$ paths can enter a region of different dielectric constant, leading to a non-zero integrand.

  The Casimir energy can be recast using the inverse moment Laplace-Mellin theorems as  
  \begin{align}
  E\subTE-E\sup0 &= -\frac{\hbar c}{2(2\pi)^{D/2}}\int_0^\infty ds\,\frac{s^{\alpha-1}}{\Gamma(\alpha)}
  \int d\lambda \frac{\lambda^{(D-1)/2-\alpha}}{\Gamma[(D+1)/2-\alpha]}\nonumber\\
  &\hspace{0.5cm}\times\int d\vect{x}_0 \left[ \big(f\supTE_{12}(\vect{x}_0)-f_{12}\sup0\big) 
- \big(f\supTE_{1}(\vect{x}_0)-f_{1}\sup0\big)
-\big(f\supTE_{2}(\vect{x}_0)-f_{1}\sup0\big)\right].
  \end{align}
  % where $f(\vect{x}_0)$ is the path integral solution for the relevant geometry, and $f\sup0$ is the solution when the 
  % dielectric is replaced by the constant value at the path starting point.
  The values for $\alpha=1/2$ and $D=4$ will be used at the end of the computation.
  The solutions $f\supTE_i$ are the path integral solutions
  % \begin{equation}
  %   f_i = \int_0^\infty d\cT \frac{e^{-(\lambda+s)\cT}}{\sqrt{\cT}}\dlangle e^{-s\int_0^\cT dt\, \chi_i(\vect{x})}\drangle,
  % \end{equation}  
  derived in Eqs.~(\ref{eq:Feynman-Kac TE one step}) and (\ref{eq:Feynman-Kac TE two step}) for one and 
  two dielectric steps respectively.
  These are renormalized by subtracting $f\sup0_i$, the solution for a constant dielectric filling space.  
  The spatial integral can be carried out for each of the three regions: Region I where $x_0<d_1$, Region II where
  $d_1<x_0<d_2$, and Region III where $d_2<x_0$.  
  Fortunately, the solutions are simple exponentials in $x_0$, making these integrals straightforward.
  This calculation is deferred to Appendix~\ref{app:nasty_calc}, since it is straightforward, but messy.
  The integrated, renormalized solution can be written as
  \begin{align}
    &\int_{-\infty}^\infty dx_0\bigg(\big[f\supTE_{12}(x_0)-f_{12}\sup0\big] -\big[f\supTE_{1}(x_0)-f_{1}\sup0\big]
    -\big[f\supTE_{2}(x_0)-f_{2}\sup0\big]\bigg)\nonumber\\
    & = \frac{A\sqrt{\pi}r\supTE_1r\supTE_2e^{-2\sqrt{2(\lambda+s)}d}}{\sqrt{(\lambda+s)}(1-r\supTE_1r\supTE_2 e^{-2\sqrt{2(\lambda+s)}d})}\left(2d
     + \frac{\sqrt{2}}{\sqrt{\lambda+s(1+\chi_1)}}+\frac{\sqrt{2}}{\sqrt{\lambda+s(1+\chi_2)}}\right),
  \end{align}
  where $A$ is the (infinite) transverse area of the dielectric planes, and the reflection coefficients for each surface are
  \begin{equation}
    r\supTE_i = \frac{\sqrt{\lambda+s}-\sqrt{\lambda+s(1+\chi_i)}}{\sqrt{\lambda+s}+\sqrt{\lambda+s(1+\chi_i)}}.
  \end{equation}
  In order to recover a finite quantity it is necessary to calculate the energy per unit area.  
  The integrals can be transformed into Lifshitz form via similar transformations to those used previously.
  The integration variable $\lambda$ is transformed to $p:=\sqrt{\lambda/s+1}$, with the result % [or $\lambda=s(p^2-1)$]
  \begin{align}
    \frac{E\subTE-E\sup0}{A} &= -\frac{\hbar c}{(2\pi)^{D/2}}\sqrt{\pi}\int_0^\infty ds\,\frac{s^{(D-2)/2}}{\Gamma(\alpha)}
  \int_1^\infty dp\, \frac{(p^2-1)^{(D-1)/2-\alpha}}{\Gamma[(D+1)/2-\alpha]}\nonumber\\
  &\times\frac{r\supTE_1r\supTE_2e^{-2\sqrt{2s}pd}}{(1-r\supTE_1r\supTE_2 e^{-2\sqrt{2s}pd})}
\left(2d + \frac{\sqrt{2}}{\sqrt{s(p^2+\chi_1)}}+\frac{\sqrt{2}}{\sqrt{s(p^2+\chi_2)}}\right),
  \end{align}
  where the reflection coefficients are now given by 
  \begin{equation}
    r\supTE_i = \frac{p-\sqrt{p^2+\chi_i}}{p+\sqrt{p^2+\chi_i}}.
  \end{equation}
  Next, the $s$ integral is transformed $s$ to $t=\sqrt{2s}$, then the $t$ and $p$ integrals are swapped, and  
  finally the values of $\alpha=1/2$ and $D=4$ are used. The result of those manipulations is
  \begin{align}
     \frac{E\subTE-E\sup0}{A}
    &= -\frac{\hbar c}{8\pi^{2}}\int_0^\infty dt \,t^{D-1}  \int_1^\infty dp\, (p^2-1)\nonumber\\
    &\times\frac{r\supTE_1r\supTE_2e^{-2tpd}}{(1-r\supTE_1r\supTE_2 e^{-2tpd})}
    \left(2d + \frac{2}{t\sqrt{p^2+\chi_1}}+\frac{2}{t\sqrt{p^2+\chi_2}}\right).
  \end{align}
Finally, the integral can be simplified by integrating by parts with respect to $p$.
The following derivatives will be of use:
\begin{align}
  \frac{d r\supTE_i}{dp} &= \frac{d}{dp}\frac{p-\sqrt{p^2+\chi_i}}{p+\sqrt{p^2+\chi_i}}
    % &= \frac{1-p/\sqrt{p^2+\chi_i}}{p+\sqrt{p^2+\chi_i}}
    %   - \frac{(p-\sqrt{p^2+\chi_i})(1+p/\sqrt{p^2+\chi_i})}{(p+\sqrt{p^2+\chi_i})^2}\\
    % &= \frac{\sqrt{p^2+\chi_i}-p}{\sqrt{p^2+\chi_i}(p+\sqrt{p^2+\chi_i})}
    %   - \frac{(p-\sqrt{p^2+\chi_i})(\sqrt{p^2+\chi_i}+p)}{\sqrt{p^2+\chi_i}(p+\sqrt{p^2+\chi_i})^2}\\
    = \frac{-2r\supTE_i}{\sqrt{p^2+\chi_i}}\\
    \frac{d}{dp}\log[1-r\supTE_1r\supTE_2 e^{-2 t p d}] & = 
    \frac{r\supTE_1r\supTE_2 e^{-2 t p d}}{1-r\supTE_1r\supTE_2 e^{-2 t p d}}\left( 2 t d+\frac{2}{\sqrt{p^2+\chi_1}}
+\frac{2}{\sqrt{p^2+\chi_2}}\right).
\end{align}
The TE Casimir energy between two half-spaces is then 
\begin{align}
    \frac{E\subTE-E\sup0}{A}
  &= -\frac{\hbar c}{8\pi^2}\int_0^\infty dt \,t^{3}  \int_1^\infty dp\, (p^2-1) 
  \frac{d}{dp}\bigg[\frac{1}{t}\log(1-r\supTE_1r\supTE_2 e^{-2tpd})\bigg]\nonumber\\
  &= \frac{\hbar c}{4\pi^2}\int_0^\infty dt \,t^{2}  \int_1^\infty dp\, p \log(1-r\supTE_1r\supTE_2 e^{-2tpd}),
\end{align}
since the boundary term from the integration by parts vanishes.  
This is exactly the TE component of the Lifshitz energy that was derived by more straightforward means in Section~\ref{sec:lifshitz}.
In this derivation, the gap between the spaces was filled with vacuum~$(\epsilon_3=1)$.  
The TE Casimir energy per unit area can be written as 
\begin{equation}
  \frac{E\subTE-E\sup0}{A} = -\frac{\hbar c\pi^2}{720d^3}\gamma\subTE(\chi_1,\chi_2),
\end{equation}
where the prefactor is the perfect conductor Casimir energy, and the ``efficiency'' $\gamma\subTE$ is
\begin{equation}
  \gamma\subTE(\chi_1,\chi_2):=-\frac{180}{\pi^4}\int_0^\infty dt\,t^{2} \int_1^\infty dp\,\log[1-r\supTE_1r\supTE_2 e^{-2tpd}].
  \label{eq:gammaTE}
\end{equation}
The efficiency is the ratio of the TE Casimir energy between the dielectrics to the total Casimir energy
between perfectly-conducting plates at the same distance.
In this case, the $\gamma\subTE$ increases monotonically between $0$ and $1/2$ as both $\chi_1$ and $\chi_2$ 
increase from $0$ to $\infty$.  In the Casimir energy
both polarizations contribute equally in strong-coupling.  
This derivation will be extended to the TM component.  

\section{Analytical TM Casimir Energy between Two Dielectric Planes}
\label{eq:TM_energy}
The TM Casimir energy calculation is carried out in a similar manner to the TE case.
Despite the similarities between the two solutions, it is still necessary to check that this calculation also works, 
since the differences in the reflection coefficients may upset the cancellations that occurred.  
The renormalized two-body TM Casimir interaction energy is 
\begin{align}
  E\subTM-E\sup0 &= -\frac{\hbar c}{2(2\pi)^{D/2}}\int_0^\infty\frac{d\cT}{\cT^{1+D/2}}\int d\vect{x}_0
  \biggdlangle
  \bigg(\frac{e^{-\cT\langle \VTM\sup1+\VTM\sup2\rangle}}{\sqrt{\langle \epsrab\rangle}}-\frac{1}{\sqrt{\epsrab(\vect{x}_0)}}\bigg) \nonumber\\
&\hspace{1cm}  -\bigg(\frac{e^{-\cT\langle \VTM\sup1\rangle}}{\sqrt{\langle \epsra\rangle}}-\frac{1}{\sqrt{\epsra(\vect{x}_0)}}\bigg)
  -\bigg(\frac{e^{-\cT\langle\VTM\sup2\rangle}}{\sqrt{\langle \epsrb\rangle}}-\frac{1}{\sqrt{\epsrb(\vect{x}_0)}}\bigg)
    \biggdrangle_{\vect{x}(t)}.\label{eq:TM_Casimir}
  \end{align}
In addition to the two-body dielectric function $\epsrab$, the path integral is augmented by the TM potentials 
at both surfaces $\VTM\sup1$ and $\VTM\sup2$.
The TM potentials all vanish in the renormalization terms, which are evaluated for the case of a constant dielectric.  
After the Laplace-Mellin and inverse moment transforms, the TM Casimir energy is 
  \begin{align}
  E\subTM-E\sup0 &= -\frac{\hbar c}{2(2\pi)^{D/2}}\int_0^\infty ds\,\frac{s^{\alpha-1}}{\Gamma(\alpha)}
  \int_0^\infty d\lambda \frac{\lambda^{(D-1)/2-\alpha}}{\Gamma[(D+1)/2-\alpha]}\nonumber\\
  &\hspace{0.5cm}\times\int d\vect{x}_0 \left[ \big(f\supTM_{12}(\vect{x}_0)-f_{12}\sup0\big) 
- \big(f\supTM_{1}(\vect{x}_0)-f_{1}\sup0\big)
-\big(f\supTM_{2}(\vect{x}_0)-f_{1}\sup0\big)\right],
  \end{align}
where the solutions are the path integrals in Eqs.~(\ref{eq:Feynman-Kac TM one step}) and 
(\ref{eq:Feynman-Kac TM two step}).  Again, the following substitutions are required: $\Xi_i\rightarrow \Xi_i$,
$\lambda\rightarrow \lambda+s$, and $\chi_i\rightarrow s\chi_i$.
This requires evaluating integrals over $x_0$ and combining the results. 
This algebra is again deferred to Appendix~\ref{app:nasty_calc}.
In this case, the integrated solution is 
\begin{align}
&\int d\vect{x}_0 \left[ \big(f\supTM_{12}(\vect{x}_0)-f_{12}\sup0\big) 
- \big(f\supTM_{1}(\vect{x}_0)-f_{1}\sup0\big)
-\big(f\supTM_{2}(\vect{x}_0)-f_{1}\sup0\big)\right]\nonumber\\
  &=\frac{\sqrt{\pi} A\,r\supTM_1r\supTM_2e^{-2\sqrt{2(\lambda+s)}d}}{\sqrt{\lambda+s}(1-r\supTM_1r\supTM_2 e^{-2\sqrt{2(\lambda+s)}d})}\nonumber\\
  &\hspace{0.5cm}\times\bigg[2d
  -\sum_{i=1}^2\frac{\sqrt{2}e^{2\Xi_i}s\chi_i}{\sqrt{\lambda+s(1+\chi_i)}[(\lambda+s)\,e^{4\Xi_i}-\lambda-s(1+\chi_i)]}
 \bigg],
\end{align}
where the integral over the transverse coordinates introduced the area $A$, and the reflection coefficients are given by, 
\begin{equation}
  r\supTM_i =  \frac{e^{2\Xi_i}\sqrt{\lambda+s}-\sqrt{\lambda+s(1+\chi_i)}}{e^{2\Xi_i}\sqrt{\lambda+s}+\sqrt{\lambda+s(1+\chi_i)}}.
\end{equation}
The same transformations as in Section~\ref{eq:TE_energy} can be employed to find the TM Casimir energy.
After changing integration variables from $\lambda$ to $p:=\sqrt{\lambda/s+1}$, the result is 
  \begin{align}
  E\subTM-E\sup0 % &= -\frac{\hbar cA}{2(2\pi)^{D/2}}\int_0^\infty ds\,\frac{s^{\alpha-1}}{\Gamma(\alpha)}
&= -\frac{\hbar cA}{(2\pi)^{D/2}}\int_0^\infty ds\,\frac{s^{(D-2)/2}}{\Gamma(\alpha)}
  \int_1^\infty dp\,\frac{(p^2-1)^{(D-1)/2-\alpha}}{\Gamma[(D+1)/2-\alpha]}\nonumber\\
  &\hspace{0.5cm}
\times\frac{\sqrt{\pi} \,r\supTM_1r\supTM_2e^{-2\sqrt{2s}pd}}{(1-r\supTM_1r\supTM_2 e^{-2\sqrt{2s}pd})}
%\nonumber\\  &\hspace{0.5cm}\times
\bigg[2d
  -\sum_{i=1}^2\frac{\sqrt{2}e^{2\Xi_i}\chi_i}{\sqrt{s}\sqrt{p^2+\chi_i}[p^2\,e^{4\Xi_i}-p^2-\chi_i]}
 \bigg],
  \end{align}
where the reflection coefficients are now given by 
\begin{equation}
  r\supTM_i =  \frac{e^{2\Xi_i}p-\sqrt{p^2+\chi_i}}{e^{2\Xi_i}p+\sqrt{p^2+\chi_i}}.
\end{equation}
 The $s$ integral can be transformed using $t=\sqrt{2s}$, and the values for $\alpha=1/2$ and $D=4$ can
be used, with the result
  \begin{align}
  E\subTM-E\sup0 
%   &= -\frac{\hbar cA}{(2\pi)^{D/2}}\int_0^\infty dt\,\frac{t^{D-1}}{2^{(D-2)/2}\Gamma(\alpha)}
%   \int_1^\infty dp\,\frac{(p^2-1)^{(D-1)/2-\alpha}}{\Gamma[(D+1)/2-\alpha]}\nonumber\\
%   &\hspace{0.5cm}
% \times\frac{\sqrt{\pi} \,r\supTM_1r\supTM_2e^{-2tpd}}{(1-r\supTM_1r\supTM_2 e^{-2tpd})}
% %\nonumber\\  &\hspace{0.5cm}\times
% \bigg[2d
%   -\sum_{i=1}^2\frac{2e^{2\Xi_i}\chi_i}{t\sqrt{p^2+\chi_i}[p^2\,e^{4\Xi_i}-p^2-\chi_i]}
%  \bigg]
  &= -\frac{\hbar cA}{16\pi^{2}}\int_0^\infty dt\,t^{3}
  \int_1^\infty dp\,(p^2-1)\nonumber\\
  &\hspace{0.5cm}
\times\frac{\,r\supTM_1r\supTM_2e^{-2tpd}}{(1-r\supTM_1r\supTM_2 e^{-2tpd})}
%\nonumber\\  &\hspace{0.5cm}\times
\bigg[2d
  -\sum_{i=1}^2\frac{2e^{2\Xi_i}\chi_i}{t\sqrt{p^2+\chi_i}[p^2\,e^{4\Xi_i}-p^2-\chi_i]}
 \bigg].
  \end{align}
Once again, an integration by parts with respect to $p$ will put the energy in standard form. The following
derivatives will be required:
\begin{align}
  \frac{d}{dp}\log[1-r\supTM_1r\supTM_2 e^{-2ptd}] 
  &= \frac{r\supTM_1r\supTM_2 e^{-2ptd}}{1-r\supTM_1r\supTM_2 e^{-2ptd}}\left( 2td -\sum_{i=1}^2\frac{d\log r\supTM_i}{dp}\right)\\
  \frac{d}{dp}\log[r\supTM_i] %=& \frac{d}{dp}\left(\log[e^{2\Xi_i}p - \sqrt{p^2+\chi_i}] -\log[e^{2\Xi_i}p + \sqrt{p^2+\chi_i}]\right) \nonumber\\
  =\,& \frac{2\chi e^{2\Xi_i}}{\sqrt{p^2+\chi_i}[e^{4\Xi_i}p^2-(p^2+\chi_i)]}.
\end{align}
After integrating by parts, the TM Casimir energy is 
  \begin{align}
  E\subTM-E\sup0 
  &= \frac{\hbar cA}{8\pi^2}\int_0^\infty dt\,t^{2}
  \int_1^\infty dp\,\log[1-r\supTM_1r\supTM_2 e^{-2tpd}].
  \end{align}
The TM Casimir energy per unit area can be written as 
\begin{equation}
  \frac{E\subTM-E\sup0}{A} = -\frac{\hbar c\pi^2}{720d^3}\gamma\subTM(\chi_1,\chi_2),
\end{equation}
where the prefactor is the perfect-conductor Casimir energy, and the efficiency $\gamma\subTM$ is
\begin{equation}
  \gamma\subTM(\chi_1,\chi_2):=-\frac{180}{\pi^4}\int_0^\infty dt\,t^{2} \int_1^\infty dp\,\log[1-r\supTM_1r\supTM_2 e^{-2tpd}].
  \label{eq:gammaTM}
\end{equation}
This result has the same form as the corresponding TE result~(\ref{eq:gammaTE}), but with the TE reflection coefficients
replaced by their TM counterparts.
At the end of the computation the relation $e^{2\Xi_i}=(1+\chi_i)$ can be used in the reflection coefficients.
Similarly to the TE case, $\gamma\subTM$ increases monotonically between $0$ and $1/2$ as both $\chi_1$ and $\chi_2$ 
increase from $0$ to $\infty$.  Despite the TE and TM polarizations having equal contributions to the 
Casimir energy in the strong-coupling limit, the TM is typically the larger of the two.

\section{Nonzero Temperature and Dispersion}
\label{sec:nonzero_temp}
The preceding results were all derived for dispersion free media at zero temperature.  
These calculations can be extended to nonzero temperature and to account for dispersion, which is needed
to describe the near-field and high temperature limiting cases.  
% So I read Babb's paper\footnote{Babb, J. F. and Klimchitskaya, G. L., and Mostepanenko, V. M., 
% ``Casimir-Polder interaction between an atom and a cavity wall under the influence of real conditions'',
%  Phys. Rev. A, \textbf{70},042901,(2004)} (which Dan cites for thermal Casimir-Polder calculations.~\cite{Babb2004})
%   In it they use the free energy, which is $\mathcal{F} = -k_BT\log Z$, as the basis of their calculations.
%  I've been trying to use the mean energy, $E= -\partial_\beta\log Z$. 
For systems in thermal equilibrium at nonzero temperature, the free energy $\cF$ is used instead of the mean energy $E$.
The free energy for the TE and TM polarizations (for non-magnetic media) is
\begin{align}
\cF\supTE
=\,&k_BT{\sum_{n=0}^\infty}'\int_0^\infty \frac{d\cT}{\cT(2\pi \cT)^{(D-1)/2}}\int d\vect{x}_0\,
%\nonumber\\&\times
\dlangle  e^{-s_n^2\langle\epsr(\vect{x},is_n)\rangle\cT /(2c^2)}\drangle_{\vect{x}(t)}\\
% \end{align}
% \begin{align}
\cF\supTM
=\,&k_BT{\sum_{n=0}^\infty}'\int_0^\infty \frac{d\cT}{\cT(2\pi \cT)^{(D-1)/2}}\int d\vect{x}_0\,
%\nonumber\\&\times
\dlangle  e^{-s_n^2\langle\epsr(\vect{x},is_n)\rangle\cT /(2c^2)}
e^{-\cT\langle V\subTM^{(n)}(\vect{x})\rangle}\drangle_{\vect{x}(t)},
\end{align}
where we have suppressed renormalization terms.  
The Casimir--Polder energies can be derived by the same reasoning used in Section~\ref{sec:casimir-polder_worldline},
extending the energies~(\ref{eq:TE_Casimir_Polder}) and (\ref{eq:TM_Casimir_Polder}) to nonzero temperature.
In this case, the results are 
\begin{align}
V\supTE\subCP
=\,&k_BT{\sum_{n=0}^\infty}'\frac{\alpha(is_n)}{\epsilon_0}\int_0^\infty \frac{d\cT}{(2\pi \cT)^{(D-1)/2}}\int d\vect{x}_0\,
\nonumber\\ &\times 
\biggdlangle  \frac{s_n^2}{2c^2}e^{-s_n^2\langle\epsr(\vect{x},is_n)\rangle\cT /(2c^2)}
\biggdrangle_{\vect{x}(t),\vect{x}(0)=\rA}\label{eq:TE_CP_thermal}\\
V\supTE\subCP
=\,&k_BT{\sum_{n=0}^\infty}'\frac{\alpha(is_n)}{\epsilon_0}\int_0^\infty \frac{d\cT}{(2\pi \cT)^{(D-1)/2}}\int d\vect{x}_0\,\nonumber\\
&\times\biggdlangle \bigg(\frac{s_n^2}{2c^2}-\frac{1}{4}\nabla^2\bigg)e^{-s_n^2\langle\epsr(\vect{x},is_n)\rangle\cT /(2c^2)}
e^{-\cT\langle V\subTM^{(n)}(\vect{x})\rangle}\biggdrangle_{\vect{x}(t),\vect{x}(0)=\rA}.\label{eq:TM_CP_thermal}
\end{align}
There are some noteworthy features of the finite-temperature worldline expression.  
First, all material functions are already in the exponent, so there is no need for the inverse moment
theorem when using results from Chapter~\ref{ch:feynman_kac} to analytically evaluate the finite-temperature Casimir energy.
In fact, at zero temperature the Matsubara sum is replaced by an integral, which
corresponds to the Gamma function used in the inverse moment theorem.  

Second, for the TE polarization the dielectric path average is proportional to $s_n^2$.
In contrast, for the TM polarization the additional TM potential is not weighted by $s_n$.  While the TM 
potential might depend on the frequency via $\epsr(\vect{x},is_n)$, it is not explicitly weighted by it.
This has important ramifications for the near-field and high temperature limits.
In those limits, the Casimir effect is dominated by the TM polarization.  

The high temperature limit touches on one of the arguments in the literature.  
There has been a dispute over the correct model for the frequency dependence of a realistic metal,
and the correct contribution from the TE mode at zero frequency~(Bordag\etal give a summary in Chapter~14 of Ref.~\cite{Bordag2009}).
This is particularly relevant for describing metals with effective dielectric functions.  
The Drude and plasma models of the dielectric function of a metal are given respectively by
\begin{equation}
  \epsr^{\text{(Drude)}}(is) = 1+\frac{\omega_p^2}{s(s+i\gamma)} \qquad 
\epsr^{\text{(plasma)}}(is) = 1+\frac{\omega_p^2}{s^2},
\end{equation}
where $\omega_p$ is the plasma frequency and $\gamma$ is the dissipation rate of the metal.
The Drude model diverges as $\omega^{-1}$ at zero frequency,
whereas the plasma model diverges as $\omega^{-2}$.  The faster divergence of the plasma
model would 
lead to an extra contribution from the TE polarization at zero frequency.  Early experiments were 
unable to distinguish between the two models, although recent measurements have claimed to eliminate the 
plasma model from consideration~\cite{Sushkov2011}.  We will assume that 
$\lim_{s\rightarrow 0}s^2\epsr(is)=0$ in the remainder of this section.

At high temperature, $\beta\rightarrow 0$, so the spacing between the Matsubara frequencies $s_n=2\pi n/(\hbar \beta)$
diverges.  As a result, only the first term significantly contributes.
  In fact, the first mode is exponentially suppressed relative to the zero frequency mode.
  Since the TE energy contribution vanishes at zero frequency, the leading order term comes from the TM energy,
  with the result that
\begin{equation}
\lim_{\beta\rightarrow 0} (\cF\supTM-\cF\sup0)=\frac{\kB T}{4}\int_0^\infty \frac{d\cT}{\cT(2\pi \cT)^{(D-1)/2}}\int d^{D-1}x_0\,
\dlangle 1 -  e^{-\cT\langle V\subTM^{(0)}(\vect{x})\rangle}\drangle.\label{eq:high_temp}
\end{equation}
Similar considerations apply to both the Casimir and Casimir--Polder energy.  The TM polarization is 
similarly dominant at small distances at zero temperature.
As noted at the end of Section~\ref{sec:nonzero_temp_path_integral}, the zero-temperature, far-field limit can
be recovered by replacing the Matsubara sum with a frequency integral.  In a similar fashion,
 Eqs.~(\ref{eq:TE_Casimir_Polder}) and (\ref{eq:TM_Casimir_Polder}) can be recovered from Eqs.~(\ref{eq:TE_CP_thermal})
 and (\ref{eq:TE_CP_thermal}).
% At the outset, we note that the zero-temperature, far-field limit can be recovered in two steps.
% First, since $\beta\rightarrow\infty$, the spacing between the Matsubara frequencies $\Delta s = 2\pi/(\hbar\beta)$, approaches zero.
% The Matsubara sum can then be converted into a frequency integral,
% $\beta^{-1}\sum'_nf(s_n)\rightarrow \frac{\hbar c}{2\pi}\int_0^\infty ds\,f(s)$.
% Then assuming that the distances $d$ are longer than the dominant resonant wavelengths, the 
% material response functions can be replaced with their zero-frequency values.  The frequency integrals
% can then carried out, and Eqs.~(\ref{eq:TE_Casimir_Polder}), (\ref{eq:TM_Casimir_Polder}), our earlier expressions will be recovered.

In general, the dominant frequencies can be estimated from the worldline expression for the Casimir energy.
Since the worldline path integral is an ensemble average is over Gaussian random walks,  the relevant 
range of $\cT$ can estimated from the distances of the problem.
The paths will typically intersect all the surfaces 
when $\cT\sim d^2$, where $d$ is the distance from the source point $x_0$ to the farthest surface.
Secondly, the frequency sum is dominated by the exponential factors with the form $e^{-s_n^2\cT/2c^2}$,
which contribute most when $\cT s_n^2/c^2\sim 1$.
This suggests that frequencies  $s_n\sim c/d$ will contribute most to the Casimir energy in general.   
However, this may be superseded by the frequency responses of the atom or medium, as indicated by 
the polarizability $\alpha(is_n)$ and the susceptibility $\chi(is_N)$, which will dominate in certain 
limits.   

\subsection{Thermal TE Casimir--Polder Energy}

We will limit our discussion for checking the high temperature and near-field limits 
to the Casimir--Polder case of an atom near a dielectric plane.  
The preceding calculations are straightforwardly extended to dispersion and finite temperature.
The analytical solutions for the TE and TM path integrals for a single dielectric plane
in Eqs.~(\ref{eq:Feynman-Kac TE one step})
and (\ref{eq:Feynman-Kac TM one step}) can be substituted into the appropriate path integrals.  
After using the Laplace-Mellin theorem~(\ref{eq:Laplace-Mellin}), the renormalized 
TE worldline path integral is 
\begin{align}
&\int_0^\infty d\cT\,\frac{1}{(2\pi \cT)^{(D-1)/2}}\dlangle e^{-v\cT} - e^{-v\cT \langle\epsr(is_n)\rangle}\drangle \nonumber\\
& =\frac{1}{2\pi}\int_0^\infty d\lambda\, \frac{e^{-2\sqrt{2(\lambda+v)}|d|}}{\sqrt{2(\lambda+v)}}
\frac{\sqrt{\lambda+v[1+\chi(is_n)]}-\sqrt{\lambda+v}}{\sqrt{\lambda+v[1+\chi(is_n)]}+\sqrt{\lambda+v}}.
\end{align}
There are no extra $\lambda$ terms from the Laplace-Mellin theorem since ${\lambda^{(D-3)/2-1/2}=1}$
when $D=4$.
This result can be used in the TE Casimir free energy if the $v$ is transformed according to $v\rightarrow s_n^2/(2c^2).$
After changing integration variable to $p = \sqrt{1+2c^2\lambda/s_n^2}$, the free energy is
\begin{align}
V\supTE\subCP-V_0&=-\kB T{\sum_n}'\frac{s_n^3\alpha(is_n)}{4\pi\epsilon_0c^3}\int_1^\infty dp\,e^{-2s_n p|d|/c}
\frac{\sqrt{p^2+\chi(is_n)}-p}{\sqrt{p^2+\chi(is_n)}+p},
\label{eq:TE_CP_finite_temperature}
\end{align}
This is the general result accounting for finite temperature and dispersion for the TE polarization for an atom near 
a planar dielectric.

We will now show that the TE contribution is negligible in the near-field regime at zero temperature.
In the near field regime, the separation between the atom and the wall is much smaller than the atom's dominant wavelength, 
 $d\ll 2\pi c/\omega_{j0}$.
In the zero temperature limit, the free energy~(\ref{eq:TE_CP_finite_temperature}) becomes
\begin{equation}
V\supTE\subCP-V_0=-\frac{\hbar}{8\pi^2\epsilon_0c^3}\int_0^\infty d\omega\,\omega^3\alpha(i\omega)
\int_1^\infty dp\,e^{-2\omega p|d|/c}\frac{\sqrt{p^2+\chi(i\omega)}-p}{\sqrt{p^2+\chi(i\omega)}+p}.\label{eq:VCP_near1}
\end{equation}
  The presence of the atom's polarizability $\alpha(i\omega)$ means that frequencies around 
  the atom's dominant transition frequency $\omega_{j0}$ will dominate the frequency integral.
  In that case, since $p \sim  c/(d\omega_{j0})$, the dominant values of $p$ are much greater than one.
  The reflection coefficient can be approximated in this limit, 
\begin{align}
  \frac{\sqrt{p^2+\chi(i\omega)}-p}{\sqrt{p^2+\chi(i\omega)}+p}
\approx \frac{\chi(i\omega)}{4p^2}.
\end{align}
The $p$ integral can be approximately evaluated as
\begin{align}
\int_1^\infty dp\,\frac{1}{4p^2}e^{-2\omega p|d|/c}%  =\,& -\frac{1}{4p}e^{-2\omega p d/c}\bigg|_{p=1}^{\infty}
 % + \int_1^\infty dp\, \frac{1}{4p}\times \frac{-2\omega d}{c}e^{-2\omega pd/c}\\
\approx & \frac{1}{4}.
\end{align}
Substituting this into the energy~(\ref{eq:VCP_near1}), the result is 
\begin{align}
V\supTE\subCP-V_0=\,&-\frac{\hbar}{32\pi^2\epsilon_0c^3}\int_0^\infty d\omega\,\omega^3\alpha(i\omega)\chi(i\omega)\\
=\,&-\frac{\hbar}{32\pi^2\epsilon_0 d^3}\int_0^\infty d\omega\,\frac{\omega^3d^3}{c^3}\alpha(i\omega)\chi(i\omega)\approx 0.
\end{align}
This is suppressed by $\order[(\omega d/c)^3]$ relative to the TM contributions and can be ignored.  

% The TM Casimir--Polder energy at zero temperature is:
% \begin{align}
% E\supTM\subCP-E_0=&-\frac{\hbar}{2\pi}\int_0^\infty d\omega\frac{\alpha(i\omega)}{2\epsilon_0}
% \int_0^\infty d\cT\,\frac{1}{(2\pi \cT)^{3/2}}\nonumber\\
% &\times\dlangle \frac{\omega^2}{c^2}e^{-\omega^2\cT/(2c^2)}-\left(\frac{\omega^2}{c^2} 
%  - \frac{1}{2}\partial_x^2\right)e^{-\omega^2\cT \langle\epsr(i\omega)\rangle/(2c^2) - \cT\langle V\subTM(i\omega)\rangle}\drangle
% \label{eq:TM_CP_zero_temperature},
% \end{align}

% \subsubsection{Laplace-Mellin and Feynman-Kac Formulae}
% We will again need to use the Laplace-Mellin theorems, and Feynman-Kac Formulae.
%   We quote the results:
% The Laplace-Mellin theorem is
% \begin{align}
% \int_0^\infty \frac{dT}{T^{1+z}}\dlangle e^{-sT\langle\epsr\rangle - T\langle V\subTM\rangle}\drangle =&
%  \frac{1}{\Gamma[z+1/2]}\int_0^\infty d\lambda\, \lambda^{z-1/2}\int_0^\infty dT e^{-(\lambda+s)T}
% \dlangle \frac{e^{-\int_0^T dt\,(s\chi+ V\subTM)}}{\sqrt{T}}\drangle.
% \end{align}
% For Casimir-Polder we need $z=1/2$, and for Casimir we need $z=3/2$.
%   In both cases we need $s= \omega^2/(2c^2)$.
%   We also need the actual analytical expression for that path integral.

% For one body we need:
% \begin{align}
% &\int_0^\infty dT e^{-(\lambda+s) T} \dlangle \frac{e^{-s\chi\int_0^T dt \Theta(x-d)}}{\sqrt{2\pi T}}\drangle  \nonumber\\
% &\hspace{0.5cm}=\frac{1}{\sqrt{2(\lambda+s)}}\left[1 - e^{-2\sqrt{2(\lambda+s)}|d|}\frac{\sqrt{\lambda+s(1+\chi)}
% -\sqrt{\lambda+s}e^{2\Xi}}{\sqrt{\lambda+s(1+\chi)}+\sqrt{\lambda+s}e^{2\Xi}}\right],
% \end{align}
% where $e^{2\Xi} = (1+\chi)$ comes from the contribution of $e^{-V\subTM}$.
%   \comment{Correct signs?} For two macroscopic bodies we will need:
% \begin{align}
% &\int dx\int_0^\infty dT \frac{e^{-(\lambda +s)T}}{\sqrt{2\pi T}}\left[e^{-s\int_0^T dt\,(\chi_{12} + V_{12,TM})}
%  +1 -e^{-s\int_0^T dt\,(\chi_{1} + V_{1,TM})}-e^{-s\int_0^T dt\,(\chi_{2} + V_{2,TM})}\right]\nonumber\\ 
% =&  \dfrac{u_1'u'_2e^{-2\sqrt{2\lambda}d}}{1 - u'_1u'_2 e^{-2\sqrt{2\lambda}d}}\left[ \frac{2 d}{\sqrt{2\lambda}}
% -\frac{ e^{2\Xi_1}}{\sqrt{\lambda+s}\sqrt{\lambda+s(1+\chi_1)}}
% \frac{s\chi_1}{e^{4\Xi_1}(\lambda+s)-[\lambda+s(1+\chi_1)]}  + \{1 \leftrightarrow 2\}  \right].
% \end{align}
% where 
% \begin{equation}
% u'_i = \frac{\sqrt{\lambda+s}(1+\chi)-\sqrt{\lambda+s(1+\chi)}}{\sqrt{\lambda+s}(1+\chi)+\sqrt{\lambda+s(1+\chi)}}
% \end{equation}
% As nasty as that two-body expression may be, exactly the same tricks will work on it, 
% and it will simplify down to exactly the same form as the other polarization.  

\subsection{Thermal TM Casimir-Polder Energy}

The TM contribution to the Casimir--Polder free energy for an atom near a dielectric plane 
 proceeds in the same manner as in the TE case.
This time, the nonzero contribution comes from the presence of the TM potential, which is reflected in
the presence of $\Xi$ in the TM expressions.   
The relevant analytical expression~(\ref{eq:Feynman-Kac TM one step}) for the path integral can be substituted in, and the Laplace-Mellin
transform can be used to write
\begin{align}
V\supTM\subCP-V_0%=\,& -k_BT{\sum_n}'\frac{\alpha(i s_n)}{4\pi\epsilon_0}\left(\frac{ s_n^2}{c^2}  - \frac{1}{2}\partial_d^2\right)\int_0^\infty d\kappa\, \frac{ s_n^2}{2c^2}\frac{e^{-2\sqrt{\kappa+1} s_n d/c}c}{ s_n\sqrt{\kappa+1}}\frac{\sqrt{\kappa+1+\chi}-\sqrt{\kappa+1}e^{2\Xi}}{\sqrt{\kappa+1+\chi}+\sqrt{\kappa+1}e^{2\Xi}} \\
=\,& -k_BT{\sum_n}'\frac{ s_n\alpha(i s_n)}{4\pi\epsilon_0c}
\left(\frac{ s_n^2}{c^2}  - \frac{1}{2}\partial_d^2\right)
\int_1^\infty dp\,e^{-2p s_n d/c}\frac{\sqrt{p^2+\chi}-pe^{2\Xi}}{\sqrt{p^2+\chi}+p e^{2\Xi}}.
\end{align}
In this geometry the derivatives with respect to the atom's starting position can be evaluated as
derivatives with respect to the surface's distance.  
After taking the derivatives with respect to distance, the free energy is
\begin{align}
V\supTM\subCP-V_0%=\,& -k_BT{\sum_n}'\frac{ s_n\alpha(i s_n)}{4\pi\epsilon_0c}\int_1^\infty dp\,\left(\frac{ s_n^2}{c^2}  - \frac{2 s_n^2p^2}{c^2}\right)e^{-2p s_n d/c}\frac{\sqrt{p^2+\chi}-pe^{2\Xi}}{\sqrt{p^2+\chi}+p e^{2\Xi}} \\
=\,& -k_BT{\sum_n}'\frac{s^3_n\alpha(i s_n)}{4\pi\epsilon_0c^3}\int_1^\infty dp\,
\left(1-2p^2\right)e^{-2p s_n d/c}\frac{\sqrt{p^2+\chi(is_n)}-pe^{2\Xi(is_n)}}{\sqrt{p^2+\chi(is_n)}+p e^{2\Xi(is_n)}}.
\end{align}
This expression can then be evaluated approximately in the near-field and high-temperature limits.  

\subsubsection{The Zero Temperature, Near-Field Limit}

Let us consider the zero-temperature, near-field limit, in which case the TM Casimir--Polder energy is
\begin{align}
V\supTM\subCP-V_0=\,& -\frac{\hbar}{8\pi^2\epsilon_0c^3}\int_0^\infty d\omega \omega^3\alpha(i\omega)\int_1^\infty dp\,
\left(1-2p^2\right)e^{-2p\omega d/c}\frac{\sqrt{p^2+\chi}-pe^{2\Xi}}{\sqrt{p^2+\chi}+p e^{2\Xi}}.
\end{align}
Once again, the atom's polarizability dominates the frequency integral, so the dominant frequencies 
occur for $\omega< \omega_{j0}$.  In the near-field limit, the distances are much smaller than these
wavelengths, so $\omega d/c\ll 1$.  Since the $p$ integral is dominated by the exponential,
the relevant $p$ are large.  
% Alternatively, just take $p\sim c/(d\omega)$.
%   Since $d$ is small, then important $p$ are very large?
%   I think this implicitly takes $\omega d/c\ll1$?
In this limit, the reflection coefficient becomes
\begin{equation}
\frac{\sqrt{p^2+\chi}-p e^{2\Xi}}{\sqrt{p^2+\chi}+pe^{2\Xi}} \approx  
 -\frac{\epsr(i\omega)-1}{\epsr(i\omega)+1},
\end{equation}
where we used $\Xi = \log\sqrt{\epsr}$.
After substituting this in, and evaluating the $p$ integral, the energy becomes
\begin{align}
V\supTM\subCP-V_0&\approx \frac{\hbar}{8\pi^2\epsilon_0c^3}\int_0^\infty d\omega\, \omega^3
\alpha(i\omega)\frac{\epsr(i\omega)-1}{\epsr(i\omega)+1}\int_1^\infty dp\,(1-2p^2)e^{-2p\omega d/c}\\
&= \frac{\hbar}{8\pi^2\epsilon_0c^3}\int_0^\infty d\omega\, \omega^3
\alpha(i\omega)\frac{\epsr(i\omega)-1}{\epsr(i\omega)+1}\left(-\frac{c^3e^{-2\omega d/c}(1+\omega d/c)^2}{2 d^3\omega^3}\right)\\
&\approx -\frac{\hbar }{16\pi^2\epsilon_0 d^3}\int_0^\infty d\omega \,
\alpha(i\omega)\frac{\epsr(i\omega)-1}{\epsr(i\omega)+1},
\end{align}
which is the well known result for the van der Waals energy for an atom near a dielectric wall~(see Eq. 14.199 of Steck~\cite{SteckNotes}).

\subsubsection{The High Temperature, Far Field Limit}

In the high-temperature limit, only the zero-frequency term contributes, so the TM Casimir--Polder
energy is  
\begin{equation}
V\supTM\subCP-V_0=-\frac{1}{2}\kB T\frac{\alpha(0)}{\epsilon_0}\int_0^\infty d\cT\,\frac{1}{(2\pi \cT)^{3/2}}
\dlangle \frac{1}{2}\partial_x^2e^{ - \cT\langle V\subTM\rangle}\drangle_{\vect{x}(t)}.
\end{equation}
In this case the solution can be directly integrated, without the need for Laplace transforms.    
After substituting in the analytical solution ${\dlangle e^{-\cT\langle V\subTM\rangle}\drangle_{\vect{x}(t)} = 1+\tanh\Xi\,e^{-2d^2/\cT}}$, and differentiating, 
the answer is
\begin{align}
V\supTM\subCP-V_0=\,&-\frac{k_BT\alpha_0}{16\pi\epsilon_0}\frac{\epsr(0)-1}{\epsr(0)+1} 
\int_0^\infty d\cT\,\partial_d^2\frac{1}{\sqrt{2\pi }\cT^{3/2}} e^{-2 d^2/\cT}\\
=\,&-\frac{k_BT\alpha_0}{16\pi\epsilon_0d^3}\frac{\epsr(0)-1}{\epsr(0)+1}.
\end{align}
This is the expected high temperature result for an atom and a dielectric wall~[see Eq.~(14.324) of Steck~\cite{SteckNotes}].
Although we have not examined the equivalent Casimir expressions, they could be readily evaluated. 
As noted in Eq.~(\ref{eq:high_temp}), it is expected that the dominant contribution comes from the TM polarization,
due to the TM potential.  
Since the TM potential also provides the majority of the Casimir energy in these limiting cases and the 
majority of the Casimir--Polder energy, it is essential to correctly account for it in a numerical method.  

%%% Local Variables: 
%%% mode: latex
%%% TeX-master: "thesis_master"
%%% End: 
