\chapter{Electromagnetic Worldlines - Analytical Results}
\label{ch:analytical}

This chapter will show that the worldline path integrals yield the same results as the more direct 
calculations outlined in Ch.~\ref{ch:introduction}.  
We will show how to extract the Casimir--Polder energy via suitable functional derivatives of the free energy.
The analytical energies can be derived by substituting in analytical solutions for the relevant path integrals.
The worldline Casimir energies can be put in the right form by using the Gamma function, and the Laplace-Mellin
transform.  The integrals are then transformed to yield known results in planar geometries.
Finally, we examine the behaviour of the worldline path integrals in the case of nonzero temperature, 
and show how known results for the near-field and high temperature results emerge from this formalism.    
The work presented on the TE polarization has been published in Ref.~\cite{Mackrory2016}.

\section{Extracting Casimir--Polder Energies}

The Casimir--Polder energy for an atom interacting with macroscopic bodies can be derived 
by treating the atom as a perturbation to the permittivity and permeability.  
The atom is located at $\rA$, and has polarizability $\alpha$ and magnetizability $\beta$.
The atom is perturbs the background permittivity $\epsr(\vect{r})\rightarrow \epsr(\vect{r})+\delta\epsr(\vect{r})$,
and permeability $\mur(\vect{r})\rightarrow \mur(\vect{r})+\delta\mur(\vect{r})$, where
\begin{equation}
  \delta\epsr(\vect{r})=\frac{\alpha}{\epsilon_0}\delta(\vect{r}-\rA), 
  \quad \delta\mu_r(\vect{r})=\mu_0\beta_0\delta(\vect{r}-\rA).
\end{equation}
We will initially carry out calculations for dispersion-free media, at zero temperature.  The generalization 
to nonzero temperature will be considered in Sec.~\ref{sec:nonzero_temp}.

The Casimir energy for TE and TM polarizations was derived in Ch.~\ref{ch:EM_quantization}.
In the zero temperature limit, the energy in the EM field in the TE polarization is 
\begin{equation}
  E\subTE-E\sup0 = -\frac{\hbar c}{2}\int_0^\infty\frac{d\cT}{(2\pi\cT)^{D/2}\cT}\int d\vect{x}_0
  \biggdlangle
  \frac{e^{-\langle V\subTE(\vect{x})\rangle\cT}}{\sqrt{\langle \epsr(\vect{x})\mur(\vect{x})\rangle}} -1
  \biggdrangle_{\vect{x}(t)}.\label{eq:TE_energy}
\end{equation}
The Casimir--Polder energy comes from expanding the energy to linear order in the $\alpha$ and $\beta_0$.
Note that the $\delta$-functions are being used a short-hand notation for a sharply localized function with unit integral.
All expansions involving these $\delta$-functions can be carried out with a finite regularization of the 
$\delta$-function, and the arbitrarily sharp limit can be taken at the end of the computations.  
%This expansion corresponds to extracting the lowest order linear response.
The expansions in $\alpha$ and $\beta$ must be carried out in both $\langle\epsr\mur\rangle$, and the potential $V\subTE$.
Considering the similarities between the polarizations, we will carry these expansions out for only
one polarization, since the others follow by duality.  

% It is important be clear about exactly what the energy is, and how it is computed.  
% As discussed in \S 4.7 of Jackson~\cite{Jackson1998}, the electrostatic energy for a system depends on the 
% manner in which the system was arranged.  Different energies are found if the free charge or potential
% are specified.  The same is true of magnetostatic systems: are the potentials on the boundaries, or the free 
% currents specified.  

The energy for an atom in the electromagnetic field can be found by considering the change in the energy 
from adding the perturbation. 
The energy can be written as a functional of the permittivity and permeability, $E[\epsr,\mur]$.
The change in energy for adding an atom is then
\begin{equation}
  \delta E[\epsr,\mur] = E[\epsr+\delta\epsr,\mur+\delta\mur]-E[\epsr,\mur].
\end{equation}
Since the atom is assumed to have a small localized effect on the permittity and permeability, 
this expansion corresponds to a functional derivative.  
 Otherwise, the expansions can be carried out with $\alpha_0$ and $\beta_0$ acting as the small parameters.
The Casimir--Polder energy can be found by taking the following functional derivatives
\begin{equation}
  V\subCP(\rA) = \frac{\alpha_0}{\epsilon_0}\frac{\delta}{\delta\epsr(\rA)}E+\mu_0\beta_0\frac{\delta}{\delta\mur(\rA)}E.
\end{equation}
The Casimir--Polder energy must be renormalized by considering the change in the energy as the atom is removed 
arbitrarily from the dielectric objects.

It is necessary to expand the path-averaged permittivity and permeability,
\begin{align}
  \langle(\epsr+\delta\epsr)(\mur+\delta\mur)\rangle^{-1/2} &= \langle\epsr\mur\rangle^{-1/2}
  -\frac{1}{2}\langle \mur\delta\epsr+\epsr\delta\mur\rangle\langle\epsr\mur\rangle^{-3/2}\nonumber\\
&= \langle\epsr\mur\rangle^{-1/2}
-\frac{1}{2}\frac{\alpha_0}{\epsilon_0}\langle \mur(\vect{x})\delta(x-\rA)\rangle\langle\epsr\mur\rangle^{-3/2}\nonumber\\
&\hspace{1cm} -\frac{1}{2}\mu_0\beta_0\langle\epsr(\vect{x})\delta(\vect{x}-\rA)\rangle \langle\epsr\mur\rangle^{-3/2}
\label{eq:mueps_expansion}
\end{align}
The singular potentials $V\subTE, V\subTM$ can be expanded in the same fashion.  The expansion is carried out to linear order 
in $\delta\epsr,\delta\mur$, and all higher order terms are dropped.
\begin{align}
  \langle V\subTE[\mur +\delta\mur] \rangle 
  =& \frac{1}{2}\Big<(\nabla\log\sqrt{\mur+\delta\mur})^2-\nabla^2\log\sqrt{\mur+\delta\mur}\Big>\nonumber\\
  =& \frac{1}{8} \Big< [\nabla\log(\mur+\delta\mur)]^2\Big>
  -\frac{1}{4}\Big< \nabla^2\log(\mur+\delta\mur)\Big> \nonumber\\
  =& \langle V\subTE[\mur]\rangle+\left< \frac{1}{4} \nabla\log\mur\cdot\nabla\frac{\delta\mur}{\mur}
    -\frac{1}{4}\nabla^2\frac{\delta\mur}{\mur}\right> 
  \label{eq:VTE_expansion}
\end{align}
It is straightforward to then expand the exponential $e^{-\cT\langle V\subTE\rangle}$ to linear order in $\delta\epsr$ and $\delta\mur$.  
The terms involving derivatives $\nabla \delta\mur$ only make sense after integration by parts, 
\begin{equation}
  \int d\vect{x} f(\vect{x})\nabla\delta(\vect{x}-\rA)  = -\nabla f(\vect{x})\bigg|_{\vect{x}=\rA}.
\end{equation}
The same manipulations for the path-averaged $\delta$-function also apply to the versions involving derivatives.  

In all of these expansions, the path-averaged $\delta$-functions act to restrict the path integrals to paths starting at the atom's
position $\rA$.  
The path integral can be written schematically as some path-averaged function that depends the whole path and includes 
 a path-averaged $\delta$-function,
\begin{equation}
  I = \int d\vect{x}_0\Bigdlangle \,\langle f(\vect{x})\rangle\langle g(\vect{x})\delta(\vect{x}-\rA)\rangle\,\Bigdrangle_{\vect{x}(t)}.
\end{equation}
In the discretized notation this is 
\begin{equation}
  I = \int \prod_{n=0}^{N-1}dx_n\,P(x_0,\cdots, x_{N-1}) \frac{1}{N}\sum_{k=0}^{N-1}f(x_k)\frac{1}{N}\sum_{j=0}^{N-1}\delta(\vect{x}_j-\rA)g(\vect{x}_j)
\end{equation}
where the second $\delta$-function enforces path closure.  All of the functions are invariant under cyclic permutations 
of the path labels.  This is true of the path-averaged functions such as $\langle \epsr \mur$ and $\langle \VTM\rangle$,
and the Gaussian probability for closed Brownian bridges.
Then for each term $\delta(\vect{x}_j-\rA)$, the labels can be shifted $j$ times so that in the shifted
coordinates $\vect{x}_j\rightarrow \vect{x}_0$.  Since there is now a sum of $N$ identical terms, the 
path integral can be written as
\begin{equation}
  I = \int \prod_{n=0}^{N-1}dx_n\,P(\vect{x}_0,\cdots, \vect{x}_{N-1}) \frac{1}{N}\sum_{k=0}^{N-1}f(\vect{x}_k)\delta(\vect{x}_0-\rA)
= \Bigdlangle\, \langle f\rangle\,\Bigdrangle_{\vect{x}(t), \vect{x}(0)=\rA}.
\end{equation}
% For a closed path, integrated over all space, with integrands athat are all written as averaged around the path, 
% there is some freedom in which point of the path is called the origin.  
Since only paths the satisfy the $\delta$-function will contribute to the path integral, we are free to call the point
at $\rA$ the path origin.  
The end result is the path-averaged $\delta$-function restricts the starting point of the paths to the atom's
position $\rA$.


Using the results in Eqs.~(\ref{eq:mueps_expansion}) and (\ref{eq:VTE_expansion}), the Casimir--Polder energy
for the TE polarization can be written as
\begin{align}
    V\supTE\subCP(\vect{\rA}) &= -\frac{\hbar c}{2}\int_0^\infty\frac{d\cT}{(2\pi\cT)^{D/2}\cT}\int d\vect{x}_0
    \biggdlangle
    \left( - \frac{\langle\mur\delta\epsr+\epsr\delta\mur\rangle}
    {2\langle \epsr\mur\rangle^{3/2}}\right) 
  e^{-\langle V\subTE\rangle\cT} \nonumber\\
  &\hspace{1cm}+ e^{-\langle V\subTE\rangle\cT}\left(-\frac{\cT}{\langle\epsr\mur\rangle^{1/2}}
    \left< \frac{1}{4} \nabla(\log\mur)\cdot\nabla\frac{\delta\mur}{\mur}
      -\frac{1}{4}\nabla^2\frac{\delta\mur}{\mur}\right> \right)
    \biggdrangle_{\vect{x}(t)}.
\end{align}
Then after manipulating the path-averaged $\delta$-functions, and renormalizing the energy against the 
case when the atom is far from the bodies, the Casimir--Polder energy is
\begin{align}
    V\supTE\subCP(\vect{\rA}) &= -\frac{\hbar c}{2}\int_0^\infty\frac{d\cT}{(2\pi\cT)^{D/2}\cT}
    \biggdlangle
    \left( - \frac{\alpha_0\mur(\rA)}{2\epsilon_0\langle \epsr\mur\rangle^{3/2}}
      -\frac{\beta_0\mu_0\epsr(\rA)}{2\langle \epsr\mur\rangle^{3/2}}\right) e^{-\langle V\subTE\rangle\cT} \nonumber\\
    &\hspace{1cm}+\frac{\cT}{4}\frac{\beta_0\mu_0}{\mur(\rA)}\left[
     \nabla^2      +\nabla^2(\log\mur)+ \nabla(\log\mur)\cdot\nabla\right]
    \frac{ e^{-\langle V\subTE\rangle\cT}}{\langle\epsr\mur\rangle^{1/2}}
    \biggdrangle_{\vect{x}(t),\vect{x}(0)=\rA}.
\end{align}
Note that bracketed gradients such as $\nabla(\log\mur)$ should be interpreted as functions, while the 
other derivative operators act on everything to their right.  The remaining derivatives act with respect to 
the path origin $\vect{x}_0=\rA$.
The corresponding TM Casimir--Polder energy can be found under the duality transformation, and is given by 
\begin{align}
    V\supTM\subCP(\vect{\rA}) &= -\frac{\hbar c}{2}\int_0^\infty\frac{d\cT}{(2\pi\cT)^{D/2}\cT}
    \biggdlangle
    \left( - \frac{\alpha_0\mur(\rA)}{2\epsilon_0\langle \epsr\mur\rangle^{3/2}}
      -\frac{\beta_0\mu_0\epsr(\rA)}{2\langle \epsr\mur\rangle^{3/2}}\right) e^{-\langle V\subTM\rangle\cT} \nonumber\\
    &\hspace{1cm}+\frac{\cT}{4}\frac{\alpha_0}{\epsilon_0\epsr(\rA)}\left[
     \nabla^2      +\nabla^2(\log\epsr) + \nabla(\log\epsr)\cdot\nabla\right]
    \frac{ e^{-\langle V\subTM\rangle\cT}}{\langle\epsr\mur\rangle^{1/2}}
    \biggdrangle_{\vect{x}(t),\vect{x}(0)=\rA}.
\end{align}

These expressions can be further simplified if the atom is in a region where the dielectric is not varying spatially,
and we consider non-magnetic atoms and media where $\beta_0=0$ and $\mur=1$.  
If the permittivity is spatially constant at the atom's location then $\nabla\log\sqrt\epsr(\rA)=0$.
In this case, the TE and TM Casimir--Polder energies are given by 
\begin{align}
    V\supTE\subCP(\vect{\rA}) &= \frac{\hbar c\alpha_0}{4\epsilon_0(2\pi)^{D/2}}\int_0^\infty\frac{d\cT}{\cT^{1+D/2}}
    \biggdlangle
      \frac{1}{\langle \epsr\rangle^{3/2}}
    %   \right) \nonumber\\
    % &\hspace{1cm}
      \biggdrangle_{\vect{x}(t),\vect{x}(0)=\rA}\\
    V\supTM\subCP(\vect{\rA}) &= \frac{\hbar c\alpha_0}{4\epsilon_0(2\pi)^{D/2}}\int_0^\infty\frac{d\cT}{\cT^{1+D/2}}
    \biggdlangle
      \frac{e^{-\langle V\subTM\rangle\cT}}{\langle \epsr\rangle^{3/2}}
      -\frac{\cT}{2\epsr(\rA)} \nabla^2 \frac{ e^{-\langle V\subTM\rangle\cT}}{\langle\epsr\rangle^{1/2}}
      \biggdrangle_{\vect{x}(t),\vect{x}(0)=\rA}.
\end{align}
The TE Casimir--Polder energy will thus always be the simpler case to evaluate since it only depends on $\langle\epsr\rangle$, which
is well behaved.
In contrast, the TM Casimir--Polder energy involves the singular TM potential%  which must be regularized
% using the results in Sec.~\ref{sec:TM_potential}.  The TM Casimir--Polder energy also requires
and also requires spatial derivatives.
 Both of those factors will require some care in numerical methods involving stochastic paths against singular potentials.


\section{Rearranging Worldline Casimir Energies}

The worldline energies need some rearrangement in order to use the analytical results for 
path integrals derived in Ch.~\ref{ch:feynman_kac}.  This can be done with two integral identities.
The first converts the worldline path integral into the form where the Laplace transformed path
integral appears.  The second exponentiates $\langle\epsr\rangle$ by means of the Gamma function.  

\subsection{ Laplace-Mellin Transforms}

The worldline path integral has the form of a Mellin transform.  
The Mellin transform appears in the context of $\zeta$-function renormalization for functional determinants,
and considering its connection to the $\zeta$-function plays a role in formal number theory.  However,
for our purposes it is just an integral transform.  
The Mellin transform of a function $f$ is defined as 
\begin{equation}
\mathcal{M}[f](z)= \int_0^\infty dt\, t^{z-1}f(t),
\end{equation}
which is a function of $z$.  
In the worldline energy $f$ is the ensemble-average path integral and $z=1+D/2$.

There is a useful relationship between Laplace transforms and Mellin transforms referred to as the Laplace-Mellin theorem~\cite{Lew1975}.  
The Laplace transform was defined in Eq.~(\ref{eq:Laplace}), and the $\Gamma$ function is defined as  
\begin{equation}
\Gamma(z) = \int_0^\infty ds\, s^{z-1} e^{-s} = \mathcal{M}[e^{-s}](z),
\end{equation}
where the second equality writes the Gamma function as the Mellin transform of the exponential.  
The Laplace-Mellin theorem~\cite{Lew1975} says
\begin{equation}
  \Gamma(1-z)\mathcal{M}[f](z) = \mathcal{M}\big[\mathcal{L}[f]\big](1-z)\label{eq:Laplace-Mellin}
\end{equation}
This is most easily motivated by starting with the right hand side
\begin{equation}
\mathcal{M}\big[\mathcal{L}[f]\big](1-z) = 
\int_0^\infty ds\, s^{-z} \int_0^\infty dt\,e^{-st} f(t).
\end{equation}
The order of $s$ and $t$ integration can be swapped, and the $s$ can be transformed to $s:=t/u$, with the 
result
% =& \int_0^\infty dt\,\left[\int_0^\infty ds s^{-z} e^{-st}\right] f(t)\\
% =& \int_0^\infty dt\,\left[\int_0^\infty d\frac{u}{t}\, t^zu^{-z} e^{-u}\right] f(t) \\
\begin{align}
\mathcal{M}\big[\mathcal{L}[f]\big](1-z)=&\int_0^\infty dt\,\int_0^\infty du\, u^{-z} e^{-u}\,t^{z-1} f(t) \\
=& \Gamma(1-z)\mathcal{M}[f](z)
\end{align}
In words, the Mellin transform of a function is proportional to the Mellin transform of the Laplace transform of the function,
 and subject to a change of variable $z\rightarrow 1-z$.  This is 
directly useful to rewriting worldline path integrals in terms of their Laplace transforms, especially
since the solution method in Ch.~\ref{ch:feynman_kac} naturally yields the Laplace transform of the function.

\subsection{Inverse Moment Theorem}

One further step is required to put all of the material functions in the path integral into exponential form.
This is necessary, since the solutions from the previous chapter were for path integrals with exponential potentials.
If positive powers were required, then the usual moment generating tricks could be used 
such as $\langle x\rangle^n = \frac{d^n}{ds^n}e^{-s\langle x\rangle}\big|_{s=0}$.
However, for the inverse-moments required in the worldline method, the Gamma function must be used 
\begin{equation}
\frac{1}{\Gamma[\alpha]}\int_0^\infty ds\,s^{\alpha-1}\dlangle e^{-s(x+\beta)}\drangle  
= \dlangle \frac{1}{(x+\beta)^\alpha}\drangle\label{eq:moment_theorem}.
\end{equation}
This is restricted to $x+\beta>0$, and $\alpha>0$.
In the worldline calculations $x+\beta=\langle\epsr(\vect{x})\rangle$ and $\alpha=1/2$ for Casimir energies, and $\alpha=3/2$
for Casimir--Polder energies.  On the imaginary frequency axis, the dielectric functions
are real, positive, decaying functions, so all of these conditions are satisfied for worldline path integrals.

\subsection{Rearranging the Worldline into Analytical Form}

The path integral can be converted into a form where analytical results can be substituted in by using both of the preceding results.
As an exampe, consider the TE path integral, with dielectric function $\epsr(\vect{x})=1+\chi(\vect{x})$, where $\chi$
is a positive function of position.  In both the Casimir and Casimir--Polder cases, the energy involves 
$\langle \epsr\rangle^{-\alpha}$, with $\alpha=1/2$, and $\alpha=3/2$ respectively.  
The energy density can be rewritten using the inverse-moment theorem~(\ref{eq:moment_theorem}),  
\begin{align}
\int_0^\infty \frac{d\cT}{\cT^{1+D/2}}\biggdlangle\frac{1}{\langle 1+\chi(\vect{x})\rangle^\alpha} \biggdrangle_{\vect{x}(t)}
% = &\int_0^\infty \frac{dT}{T^{1+D/2}}\frac{1}{\Gamma[\alpha]}\int_0^\infty ds s^{\alpha-1} 
% \dlangle e^{-s(\chi \int_0^T dt \Theta(x-d) +1)}\drangle \\
% =&\frac{1}{\Gamma[\alpha]}\int_0^\infty \frac{dT}{T^{1+D/2-\alpha}}\int_0^\infty ds s^{\alpha-1} e^{-s T}
% \dlangle e^{-s \chi \int_0^T dt \Theta(x-d)}\drangle \\
=&\int_0^\infty ds\, \frac{s^{\alpha-1}}{\Gamma(\alpha)}\int_0^\infty \frac{d\cT}{\cT^{1+D/2-\alpha}}
\dlangle e^{-s\cT- \int_0^\cT dt \,\chi[\vect{x}(t)]}\drangle.
\end{align}
In the second equality the integration variable was rescaled $s\rightarrow s\cT$,
 and the definition of the path average, $\langle f\rangle = \cT^{-1}\int_0^\cT dt\, f(t)$ was used.
% where we used the Inverse-moment theorem, rescaled the $\lambda\rightarrow \lambda T$,
%  and swapped the order of integration. We can see that that $T$ integral has the form of a Mellin transform.
The energy density can be further transformed with the Laplace-Mellin transform~(\ref{eq:Laplace-Mellin}), 
\begin{align}
\int_0^\infty \frac{d\cT}{\cT^{1+D/2}}\biggdlangle\frac{1}{\langle 1+\chi(\vect{x})\rangle^\alpha} \biggdrangle_{\vect{x}(t)}
% =& \int_0^\infty \frac{dT}{T^{1+z-1/2}}e^{-sT}\dlangle \frac{e^{-s \chi \int dt_0^T dt \Theta(x-d)}}{\sqrt{T}}\drangle\\
% =&\mathcal{M}\left[e^{-sT}\dlangle \frac{e^{-s \chi \int_0^T dt \Theta(x-d)}}{\sqrt{T}}\drangle\right]\left(-z+1/2\right) \\
% =& \frac{1}{\Gamma[1+z-1/2]}\mathcal{M}\left[\int_0^\infty dT e^{-(\lambda+s)T}
% \dlangle \frac{e^{-s \chi \int_0^T dt \Theta(x-d)}}{\sqrt{T}}\drangle\right]\left(-z+1/2\right) \\
&= \int_0^\infty ds\, \frac{s^{\alpha-1}}{\Gamma(\alpha)}\int_0^\infty d\lambda\, 
\frac{\lambda^{(D-n)/2-\alpha}}{\Gamma[(D-n)/2-\alpha+1]}\nonumber\\
&\hspace{1cm}\times\int_0^\infty d\cT e^{-(\lambda+s)\cT}
\dlangle \frac{e^{-s \int_0^\cT dt\,\chi(\vect{x})}}{\cT^{n/2}}\drangle_{\vect{x}(t)}.\label{eq:Casimir_Laplace_inverse}
\end{align}
In the last line we have factored out $\cT^{n/2}$, the normalization for a $n$-dimensional brownian bridge.
This assumes that the solution was computed in $n$-dimension; we will work in the case where $n=1$.
(Despite knowing the parameters $\alpha, D$ and $n$, it is easier to track them in calculations as variables
than as numbers.)
The integral over $\cT$ is the solution to the relevant diffusion equation, as discussed in Ch.~\ref{ch:feynman_kac}.
The solutions must be appropriately scaled with $\lambda\rightarrow \lambda+s$, and $\chi\rightarrow s\chi$.

\section{Analytical  TE CP energy for an atom and a dielectric plane}
\label{sec:TE_CP}
The TE contribution to Casimir--Polder energy for an atom above a half-plane at $x=d$ is then given 
combining these formal manipulations with the relevant path integral solution.  For an atom at the origin
interacting with a planar dielectric $\epsr(z)=1+\chi\Theta(x-d)$, the path integral solution is given by
 Eq.~(\ref{eq:Feynman-Kac TE one step}).
Under the rescalings $s\rightarrow s\chi, \lambda\rightarrow \lambda+s$, the renormalized TE Casimir--Polder potential is
\begin{align}
V\subCP\supTE-V\subCP\sup0=&-\frac{\hbar c\alpha_0}{4\epsilon_0(2\pi)^{D/2}}\frac{\sqrt{\pi}}{\Gamma[\alpha]\Gamma\left[(D+1)/2-\alpha\right]}
\int_0^\infty ds\, s^{\alpha-1}\int_0^\infty d\lambda\, \lambda^{(D-1)/2-\alpha}\nonumber\\
&\times\frac{e^{-2\sqrt{2(\lambda+s)}|d|}}{\sqrt{\lambda+s}} 
\frac{\sqrt{\lambda+s(1+\chi)}-\sqrt{\lambda+s}}{\sqrt{\lambda+s(1+\chi)}+\sqrt{\lambda+s}},
\label{eq:VCP_TE_inter}
\end{align}
This can be put into the same form as the known results by changing integration variables.
Similar steps will be required to convert the other path integrals, so we will go through this once in detail.
We will confine our attention to the integral over $s$ and $\lambda$, where
\begin{equation}
  J=\int_0^\infty ds\, s^{\alpha-1}\int_0^\infty d\lambda\, \lambda^{(D-1)/2-\alpha}\frac{e^{-2\sqrt{2(\lambda+s)}|d|}}{\sqrt{\lambda+s}} 
\frac{\sqrt{\lambda+s(1+\chi)}-\sqrt{\lambda+s}}{\sqrt{\lambda+s(1+\chi)}+\sqrt{\lambda+s}}.
\end{equation}
First, change variable from $\lambda$ to $v:=\sqrt{\lambda/s+1}$, 
\begin{equation}
  J% =2\int_0^\infty ds\, s^{\alpha-1}s^{(D-1)/2-\alpha+1-1/2}\int_1^\infty dv\, (v^2-1)^{(D-1)/2-\alpha}e^{-\sqrt{8 d^2s}v}
  % \frac{\sqrt{v^2+\chi}-v}{\sqrt{v^2+\chi}+v},
  =2\int_0^\infty ds\, s^{D/2-1}\int_1^\infty dv\, (v^2-1)^{(D-1)/2-\alpha}e^{-\sqrt{8 d^2s}v}
  \frac{\sqrt{v^2+\chi}-v}{\sqrt{v^2+\chi}+v}.
\end{equation}
Next, change variable from $s$ to $t=\sqrt{8d^2 s}v$, and swap the $t$ and $v$ integrals. 
\begin{align}
  J % =4\int_1^\infty dv\, (v^2-1)^{(D-1)/2-\alpha} \int_0^\infty dt\, \frac{2t}{8d^2v^2}\left(\frac{t^2}{8d^2v^2}\right)^{D/2-1}e^{-t}
  % \frac{\sqrt{v^2+\chi}-v}{\sqrt{v^2+\chi}+v}\\
=\frac{1}{8^{D/2-1}d^D}\int_1^\infty dv\,v^{-D} (v^2-1)^{(D-1)/2-\alpha} 
  \frac{\sqrt{v^2+\chi}-v}{\sqrt{v^2+\chi}+v}\int_0^\infty dt\, t^{D-1}e^{-t}.
\end{align}
The $t$-integral can then be identified as a Gamma function, $\Gamma[D-1]$.  
Substituting this integral back into the Casimir--Polder energy~(\ref{eq:VCP_TE_inter}), while taking $D=4, \alpha=3/2$ yields
\begin{align}
  V\subCP\supTE-V\subCP\sup0=&-\frac{3\hbar c\alpha_0}{32\epsilon_0\pi^{2}d^4}
  \int_1^\infty dv\,\frac{1}{2v^4} \frac{\sqrt{v^2+\chi}-v}{\sqrt{v^2+\chi}+v}.
\end{align}
The prefactor is the Casimir--Polder energy for an atom above a perfect conductor~(\ref{eq:CP_conductor}).
This result agrees with the known result for the TE contribution to the Casimir--Polder energy.  
The integral over $v$ is the TE contribution to the efficiency, $\eta\subTE$, and can be   
 evaluated in closed form
\begin{align}
\eta\subTE(\chi)=&\frac{1}{2}\int_{1}^\infty dv\,v^{-4}\frac{v-\sqrt{v^2+\chi}}{v+\sqrt{v^2+\chi  }}\nonumber \\
=&\frac{1}{3}+2\chi^{-1}- \frac{\sqrt{\chi  (\chi +1)}}{\chi^{3/2}}
-\frac{1}{4\chi^{3/2}}\log \left[2 \chi +2 \sqrt{\chi  (\chi+1)}+1\right]
-\frac{\text{arcsinh}\left(\sqrt{\chi }\right)}{2\chi^{3/2}}.
\end{align}
The efficiency $\eta\subTE$ smoothly interpolates between $0$ and $1/6$ as $\chi$ varies from $0$ to $\infty$.
In the strong-coupling limit the TE polarization provides $1/6$ of the Casimir--Polder energy.  
The remaining $5/6$ is provided by the TM polarization.  

\section{Analytical TM CP energy for atom-plane}

The TM calculation for the Casimir--Polder energy proceeds in a similar fashion to the TE case.  
The TM Casimir--Polder energy can be split into two pieces 
\begin{align}
  V\supTM\subCP(\vect{\rA})-V\subCP\sup0 &= \frac{\hbar c\alpha_0}{4\epsilon_0(2\pi)^{D/2}}
  \bigg(\cV_{D/2,3/2}-\frac{1}{2}\nabla^2\cV_{(D-2)/2,1/2}\bigg),\label{eq:TM_CP_inter}
\end{align}
where 
\begin{equation}
  \cV_{z,\alpha}:=\int_0^\infty\frac{d\cT}{\cT^{1+z/2}}
  \biggdlangle\frac{e^{-\langle V\subTM\rangle\cT}}{\langle \epsr\rangle^{\alpha}}-1
  \biggdrangle_{\vect{x}(t),\vect{x}(0)=\rA}.
\end{equation}
Each term $\cV_{z,\alpha}$ can in turn be transformed using the combination of the Laplace-Mellin transform
and the inverse-moment theorem.  
\begin{align}
  \cV_{z,\alpha}:=&\frac{1}{\Gamma[\alpha]\Gamma[z-\alpha+1/2]}\int_0^\infty d\lambda\,\lambda^{z-1/2-\alpha}
  \int_0^\infty ds\,s^{\alpha-1}\nonumber\\
  &\times\int_0^\infty d\cT \frac{e^{-(\lambda+s)\cT}}{\sqrt{\cT}}
  \Bigdlangle e^{-\int_0^\cT dt\,[V\subTM(\vect{x}) + s\chi(\vect{x)]}}-1  \Bigdrangle_{\vect{x}(t),\vect{x}(0)=\rA}\label{eq:TM_subpart}
\end{align}
Note that the TM-potential is already in the exponential, so the TM potential and $\Xi$ does not need to be rescaled. 
In this sense $\Xi$ is an independent parameter from $\chi$. 
The analytical path integral~(\ref{eq:Feynman-Kac TM one step}) can be substituted in to Eq.~(\ref{eq:TM_subpart}),
and the integral can be transformed in the same manner as Sec.~\ref{sec:TE_CP}.
\begin{align}
  \cV_{z,\alpha}% =&\frac{(-1)\sqrt{2\pi}}{\Gamma[\alpha]\Gamma[z-\alpha+1/2]}\int_0^\infty d\lambda\,\lambda^{z-1/2-\alpha}
=&-\frac{\sqrt{\pi}\Gamma[2z]}{8^zd^{2z}\Gamma[\alpha]\Gamma[z-\alpha+1/2]}\int_1^\infty dv\,\frac{1}{v^{2z}}(v^2-1)^{z-1/2-\alpha}
  \frac{ ve^{2\Xi}-\sqrt{v^2+\chi}} {v e^{2\Xi}+\sqrt{v^2+\chi}}.
\end{align}
\comment{check power on 8?  Off by 4 should be $8^{2z}2^{-2}$?}
The two cases of interest are for $z=2,\alpha=3/2$ and $z=1,\alpha=1/2$.
\begin{align}
\cV_{2,3/2} %=&-\frac{\sqrt{\pi}\Gamma(4)}{8^{2}d^{4}\frac{\sqrt{\pi}}{2}\Gamma(1)}\int_1^\infty dv\,\frac{1}{v^{4}}(v^2-1)^{2-1/2-3/2}
  % \frac{ ve^{2\Xi}-\sqrt{v^2+\chi}} {v e^{2\Xi}+\sqrt{v^2+\chi}}.
=&-\frac{6}{32d^{4}}\int_1^\infty dv\,\frac{1}{v^{4}}\frac{ ve^{2\Xi}-\sqrt{v^2+\chi}} {v e^{2\Xi}+\sqrt{v^2+\chi}},\\
\cV_{1,1/2}=&-\frac{1}{8d^{2}}\int_1^\infty dv\,\frac{1}{v^{2}}
  \frac{ ve^{2\Xi}-\sqrt{v^2+\chi}} {v e^{2\Xi}+\sqrt{v^2+\chi}}.
\end{align}
The derivatives with respect to the starting poistion $\rA$ 
are equivalent to derivatives with respect to the distance $d$, and can be straightforwardly
evaluated.  
After substituting this back into Eq.~(\ref{eq:TM_CP_inter}), the TM Casimir--Polder energy is given by 
\begin{align}
  V\supTM\subCP(\vect{\rA})-V\subCP\sup0 &= -\frac{3\hbar c\alpha_0}{32\pi^2\epsilon_0d^4}\comment{$\frac{1}{4}$}\frac{1}{2}
  \int_1^\infty dv\,v^{-4}(1-2v^2)  \frac{ v(1+\chi)-\sqrt{v^2+\chi}}{v(1+\chi)+\sqrt{v^2+\chi}},
\end{align}
where we substituted $e^{2\Xi}=1+\chi$.  This agrees with the Lifshitz results for the TM Casimir--Polder energy
for an atom near a dielectric half-space (Eq. 14.205 in Steck ~\cite{SteckNotes}).
The TM efficiency $\eta\subTM$ also has a closed form solution, 
\begin{align}
  \eta\subTM(\chi):=\,&\frac{1}{2}
  \int_1^\infty dv\,v^{-4}(1-2v^2)  \frac{ v(1+\chi)-\sqrt{v^2+\chi}}{v(1+\chi)+\sqrt{v^2+\chi}}\\
  =\,& \frac{7}{6} + \chi + \frac{2 - (1+\chi)^{3/2}}{2\chi} 
  - \frac{\text{arcsinh}\sqrt{\chi}}{2\chi^{3/2}}[1 + \chi + 2\chi^2(1 + \chi)] \nonumber\\ 
  &+ \frac{(1+\chi)^2}{\sqrt{2+\chi}}\left[\text{arcsinh}\sqrt{1+\chi} - \text{arcsinh}\left(\frac{1}{\sqrt{1+\chi}}\right)\right]
\end{align}

\section{Finding the TE Casimir energy}

% For non-magnetic media, the worldline Casimir energies for both polarizations are
% \begin{align}
%     E\subTE-E\sup0 &= -\frac{\hbar c}{2(2\pi)^{D/2}}\int_0^\infty\frac{d\cT}{\cT^{1+D/2}}\int d\vect{x}_0
%     \biggdlangle
%     \frac{1}{\sqrt{\langle \epsr\rangle}}-1    \biggdrangle_{\vect{x}(t)}\label{eq:log_Z_TE}\\
%     E\subTM-E\sup0 &= -\frac{\hbar c}{2(2\pi)^{D/2}}\int_0^\infty\frac{d\cT}{\cT^{1+D/2}}\int d\vect{x}_0
%     \biggdlangle
%     \frac{1}{\sqrt{\langle \epsr(z)\rangle}} e^{-\langle V\subTM(z)\rangle\cT}-1
%     \biggdrangle_{\vect{x}(t)}.\label{eq:log_Z_TM}
% \end{align}
The Casimir energy for two dielectric planes can also be calculated within this formalism.  
The dielectric function is given by $\epsrab(x) = 1+\chi_1\Theta(d_1-x)+\chi_2\Theta(x-d_2)$.
The calculation proceeds in the same way, except for two changes.  

First, the Casimir energy requires a further integral over the starting points of the paths.
Second, the two-body interaction energy is found by subtracting the one-body energies involving $\epsra,\epsrb$ from the two-body
 expressions with $\epsrab$.  The fully renormalized Casimir energy between two planes is
\begin{align}
  E\subTE-E\sup0 &= -\frac{\hbar c}{2(2\pi)^{D/2}}\int_0^\infty\frac{d\cT}{\cT^{1+D/2}}\int d\vect{x}_0
  \biggdlangle
  \bigg(\frac{1}{\sqrt{\langle \epsrab\rangle}}-\frac{1}{\sqrt{\epsrab(\vect{x}_0)}}\bigg) \nonumber\\
&\hspace{1cm}  -\bigg(\frac{1}{\sqrt{\langle \epsra\rangle}}-\frac{1}{\sqrt{\epsra(\vect{x}_0)}}\bigg)
  -\bigg(\frac{1}{\sqrt{\langle \epsrb\rangle}}-\frac{1}{\sqrt{\epsrb(\vect{x}_0)}}\bigg)
    \biggdrangle_{\vect{x}(t)}.
  \end{align}
  Each term is renormalized by subtracting off the constant value of the dielectric evaluated at the 
  start of the paths.  (It is also possible to renormalize the energy by instead subtracting off the vacuum value 
  $\epsr=1$ from each term.)

  The Casimir energy can be recast using the inverse-moment theorem and the Laplace-Mellin transform.  
  The energy is 
  \begin{align}
  E\subTE-E\sup0 &= -\frac{\hbar c}{2(2\pi)^{D/2}}\int_0^\infty ds\,\frac{s^{\alpha-1}}{\Gamma(\alpha)}
  \int d\lambda \frac{\lambda^{(D-1)/2-\alpha}}{\Gamma[(D+1)/2-\alpha]}\nonumber\\
  &\hspace{0.5cm}\times\int d\vect{x}_0 \left[ \big(f_{12}(\vect{x}_0)-f_{12}\sup0\big) - \big(f_{1}(\vect{x}_0)-f_{1}\sup0\big)
-\big(f_{2}(\vect{x}_0)-f_{1}\sup0\big)\right]
  \end{align}
  where $\alpha=1/2$, and $D/2$ will be taken at the end of the computation.
  The solutions $f_i$ are the path integral solutions
  \begin{equation}
    f_i = \int_0^\infty d\cT \frac{e^{-(\lambda+s)\cT}}{\sqrt{\cT}}\dlangle e^{-s\int_0^\cT dt\, \chi_i(\vect{x})}\drangle,
  \end{equation}  
  derived in Eqs.~(\ref{eq:Feynman-Kac TE one step}) and (\ref{eq:Feynman-Kac TE two step}) for one and 
  two planar bodies respectively, and are renormalized by subtracting the solution for a constant dielectric filling space.  
  The spatial integral can be carried out for each of the three regions: Region I where $x_0<d_1$, Region II where
  $d_1<x_0<d_2$, and Region III where $d_2<x_0$.  
  Fortunately, the solutions are simple exponentials in $x_0$, making these integrals straightforward.
  This calculation is deferred to an App.~\ref{app:nasty_calc}, since it is straightforward, but messy.
  
  \comment{Guessing answer need to fix intermediate algebra}
  After a lot of algebra, the spatial integrand can be written as
  \begin{align}
    &\int_{-\infty}^\infty dx_0\bigg[\big(f_{12}(\vect{x}_0)-f_{12}\sup0\big) -\big(f_{1}(\vect{x}_0)-f_{1}\sup0\big)
    -\big(f_{2}(\vect{x}_0)-f_{2}\sup0\big)\bigg]\nonumber\\
    & = \sqrt{2\pi}\frac{u\supTE_1u\supTE_2e^{-2\sqrt{2(\lambda+s)}d}}{\sqrt{2(\lambda+s)}(1-u\supTE_1u\supTE_2 e^{-2\sqrt{2(\lambda+s)}d})}\left(2d
     + \frac{\sqrt{2}}{\sqrt{\lambda+s(1+\chi_1)}}+\frac{\sqrt{2}}{\sqrt{\lambda+s(1+\chi_2)}}\right)
  \end{align}
  where 
  \begin{equation}
    u\supTE_i = \frac{\sqrt{\lambda+s}-\sqrt{\lambda+s(1+\chi_i)}}{\sqrt{\lambda+s}+\sqrt{\lambda+s(1+\chi_i)}}.
  \end{equation}
  The integrals can be transformated into Lifshitz form via similar transformations to those used previously.
First change variable from $\lambda$ to $v:=\sqrt{\lambda/s+1}$,  [or $\lambda=s(v^2-1)$]
  \begin{align}
    E\subTE-E\sup0 %  &= -\frac{\hbar c}{2(2\pi)^{D/2}}\int_0^\infty ds\,\frac{s^{\alpha-1}}{\Gamma(\alpha)}s^{(D-1)/2-\alpha}
%   \int_1^\infty dv\, 2vs\frac{(v^2-1)^{(D-1)/2-\alpha}}{\Gamma[(D+1)/2-\alpha]}\nonumber\\
%   &\times\sqrt{2\pi}\frac{u\supTE_1u\supTE_2e^{-2\sqrt{2s}vd}}{\sqrt{2s}v(1-u\supTE_1u\supTE_2 e^{-2\sqrt{2s}vd})}
% \left(2d + \frac{\sqrt{2}}{\sqrt{s}\sqrt{v^2+\chi_1}}+\frac{\sqrt{2}}{\sqrt{s}{v^2+\chi_2}}\right)\\
&= -\frac{\hbar c}{(2\pi)^{D/2}}\sqrt{\pi}\int_0^\infty ds\,\frac{s^{(D-2)/2}}{\Gamma(\alpha)}
  \int_1^\infty dv\, \frac{(v^2-1)^{(D-1)/2-\alpha}}{\Gamma[(D+1)/2-\alpha]}\nonumber\\
  &\times\frac{u\supTE_1u\supTE_2e^{-2\sqrt{2s}vd}}{(1-u\supTE_1u\supTE_2 e^{-2\sqrt{2s}vd})}
\left(2d + \frac{\sqrt{2}}{\sqrt{s}\sqrt{v^2+\chi_1}}+\frac{\sqrt{2}}{\sqrt{s}{v^2+\chi_2}}\right)
  \end{align}
  Next, change variable from $s$ to $t=\sqrt{2s}$, swap the $t$ and $v$ integrals, and  
  susbstitute $\alpha=1/2$, $D=4$, with the result
  \begin{align}
    E\subTE-E\sup0
    &= -\frac{\hbar c}{8\pi^{2}}\int_0^\infty dt \,t^{D-1}  \int_1^\infty dv\, (v^2-1)\nonumber\\
    &\times\frac{u\supTE_1u\supTE_2e^{-2tvd}}{(1-u\supTE_1u\supTE_2 e^{-2tvd})}
    \left(2d + \frac{2}{t\sqrt{v^2+\chi_1}}+\frac{2}{t\sqrt{v^2+\chi_2}}\right),
  \end{align}
where the reflection coefficients have been transformed to 
\begin{equation}
   u\supTE_i = \frac{v-\sqrt{v^2+\chi_i}}{v+\sqrt{v^2+\chi_i}}.
\end{equation}
This expression can be put into the Lifshitz form by integrating by parts\footnote{
I realized this was necessary while reading Kimball Milton's scientific biography of Schwinger's
work on the Casimir effect where he mentions an inexplicable integration by parts in Schwingers 1992 and 1993 PNAS papers.}
with respect to $v$.
The following derivatives will be of use,
\begin{align}
  \frac{d u\supTE_i}{dv} &= \frac{d}{dv}\frac{v-\sqrt{v^2+\chi_i}}{v+\sqrt{v^2+\chi_i}}
    % &= \frac{1-v/\sqrt{v^2+\chi_i}}{v+\sqrt{v^2+\chi_i}}
    %   - \frac{(v-\sqrt{v^2+\chi_i})(1+v/\sqrt{v^2+\chi_i})}{(v+\sqrt{v^2+\chi_i})^2}\\
    % &= \frac{\sqrt{v^2+\chi_i}-v}{\sqrt{v^2+\chi_i}(v+\sqrt{v^2+\chi_i})}
    %   - \frac{(v-\sqrt{v^2+\chi_i})(\sqrt{v^2+\chi_i}+v)}{\sqrt{v^2+\chi_i}(v+\sqrt{v^2+\chi_i})^2}\\
    = \frac{-2u\supTE_i}{\sqrt{v^2+\chi_i}},\\
    \frac{d}{dv}\log[1-u\supTE_1u\supTE_2 e^{-2 t v d}] & = 
    \frac{u\supTE_1u\supTE_2 e^{-2 t v d}}{1-u\supTE_1u\supTE_2 e^{-2 t v d}}\left( 2 t d+\frac{2}{\sqrt{v^2+\chi_1}}
+\frac{2}{\sqrt{v^2+\chi_2}}\right).
\end{align}
The Casimir energy between two half-spaces is then 
\begin{align}
  E\subTE-E\sup0
  &= -\frac{\hbar c}{8\pi^2}\int_0^\infty dt \,t^{2}  \int_1^\infty dv\, (v^2-1) \frac{d}{dv} \log(1-u\supTE_1u\supTE_2 e^{-2tvd})\\
  &= \frac{\hbar c}{4\pi^2}\int_0^\infty dt \,t^{2}  \int_1^\infty dv\, v \log(1-u\supTE_1u\supTE_2 e^{-2tvd}),
\end{align}
since the boundary term from the integration by parts vanishes.  
This is exactly the TE component of the Lifshitz energy we derived by more straightforward means in Sec.~\ref{sec:lifshitz}.
In this derivation, the gap between the spaces was filled with vacuum $\epsilon_3=1$.  
This derivation will be extended to the TM component, and the nonzero temperature case where
there is dispersion.  



\section{Finding the TM Casimir energy}

From November 2013 we have
\begin{align}
I_{tot}% -I_1-I_2 + I_0 
=& 
  \frac{2r'_1r'_2 e^{-\sqrt{2\lambda}d}d}{\sqrt{2\lambda}(1-r'_1r'_2 e^{-2\sqrt{2\lambda}d})} 
+\dfrac{r'_2 e^{-2\sqrt{2\lambda}d}}{4\kappa_1(1-r'_1r'_2 e^{-2\sqrt{2\lambda}d})} 
+\frac{r'_1 e^{-2\sqrt{2\lambda}d}}{4\kappa_2(1-r'_1r'_2 e^{-2\sqrt{2\lambda}d})}  \nonumber\\
& - (r'_1+r'_2)\frac{e^{-2\sqrt{2\lambda}d}}{4\lambda(1-r'_1r'_2e^{-2\sqrt{2\lambda}d})}
 -   \left(\frac{r'_1}{4\kappa_1}-\frac{(r'_1+r'_2)}{4\lambda}  
+ \frac{r'_2}{4\kappa_2}\right)\frac{r'_1r'_2 e^{-2\sqrt{2\lambda}d}}{(1-r'_1r'_2 e^{-2\sqrt{2\lambda}d})},
\end{align}
where $\kappa = \lambda+\chi$, and 
\begin{equation}
r' =  \frac{\epsilon\sqrt{\lambda}-\sqrt{\kappa}}{\epsilon\sqrt{\lambda}+\sqrt{\kappa}}
\end{equation}

\comment{Let's now set $\chi_1=\chi_2=\chi$}
Now factor out the common terms: 
\begin{align}
I_{tot}=&  \frac{e^{-2\sqrt{2\lambda}d}}{(1-r'^2e^{-2\sqrt{2\lambda}d}}\left[
 \frac{2r'^2d}{\sqrt{2\lambda}}+ \dfrac{r'}{2\kappa}  -\frac{ r'}{2\lambda} 
- \left(\frac{r'}{2\kappa}-\frac{r'}{2\lambda}\right)r'^2\right]\\
=&  \frac{e^{-2\sqrt{2\lambda}d}}{(1-r'^2e^{-2\sqrt{2\lambda}d}}\left[
\frac{2r'^2d}{\sqrt{2\lambda}}+ \frac{r'}{2}(1-r'^2)\left(\dfrac{1}{\kappa}  -\frac{1}{\lambda}\right)\right],
\end{align}


% \begin{align}
% E =& - \partial_\beta \log Z_{TE}\\
%  =& -\frac{\hbar c}{8\pi^2}\int_0^\infty \frac{dT}{T^{1+D/2}}\int dx_0 
% \dlangle \frac{1}{\langle \epsilon\rangle^{\alpha}}e^{-\int_0^T\mathfrak{M}dt}\drangle\\
% =& - \frac{\hbar c}{8\pi^2\Gamma[(D+1)/2+\alpha]\Gamma(\alpha)} \int_0^\infty d\lambda 
% \lambda^{(D-1)/2-\alpha}\int dx_0 \int_0^\infty ds\, s^{\alpha-1}\int_0^\infty dT e^{-\lambda T}
% \dlangle \frac{e^{-sT\int_0^T dt  \epsilon-\int_0^T\mathfrak{M}dt}}{\sqrt{T}}\drangle,
% \end{align}

$I_{tot}$ is the (renormalized) expression for 
\begin{equation}
I_{tot}=\int dx_0\int_0^\infty dT \frac{e^{-\lambda T}}{\sqrt{T}}\dlangle e^{-\mathfrak{M}-\int_0^T dt V(x_0+B(t))  } \drangle,
\end{equation}

\begin{enumerate}
\item We get the energy by making the following replacements \comment{$\lambda\rightarrow \lambda+s$, $\chi\rightarrow s\chi$.  } 
% (However, this does not apply to terms in $e^{\Xi}$.  )
%  \begin{align}
% E=&  C\int_0^\infty d\lambda \lambda^{(D-1)/2-\alpha}\int_0^\infty ds\, s^{\alpha-1}\nonumber\\
% &\times  \frac{e^{-2\sqrt{2(\lambda+s)}d}}{1-r'^2e^{-2\sqrt{2(\lambda+s)}d}}
% \left[\frac{2r'^2d}{\sqrt{2(\lambda+s)}}+ \frac{r'}{2}(1-r'^2)\left(\dfrac{1}{\lambda+s+s\chi}  -\frac{1}{\lambda+s}\right)\right],
% \end{align}
% and 
% \begin{equation}
% r' =  \frac{(1+\chi)\sqrt{\lambda+s}-\sqrt{\lambda+ s + s\chi}}{(1+\chi)\sqrt{\lambda+s}+\sqrt{\lambda+s+s\chi}},
% \end{equation}
% \begin{equation}
% C = -\frac{\hbar c\sqrt{2\pi}}{8\pi^2\Gamma[(D+1)/2+\alpha]\Gamma(\alpha)} 
% \end{equation}

\item Now take $\lambda = qs$.  
%  \begin{align}
% E=&  C\int_0^\infty dq q^{(D-1)/2-\alpha}\int_0^\infty ds\, s^{\alpha-1}s^{1+(D-1)/2-\alpha}\nonumber\\
% &\times  \frac{e^{-2\sqrt{2(q+1)s}d}}{1-r'^2e^{-2\sqrt{2(q+1)s}d}}\left[
% \frac{2r'^2d}{\sqrt{2s(q+1)}}+ \frac{r'}{2}(1-r'^2)\left(\dfrac{1}{s(q+1+\chi)}  -\frac{1}{s(q+1)}\right)\right],
% \end{align}
% and 
% \begin{equation}
% r' =  \frac{(1+\chi)\sqrt{q+1}-\sqrt{q+ 1 + \chi}}{(1+\chi)\sqrt{q+1}+\sqrt{q+1+\chi}},
% \end{equation}
\item Now take $p = \sqrt{q+1}$.  $\rightarrow  q = p^2-1, dq = 2p dp$.  
%  \begin{align}
% E%=&   C\int_1^\infty dp(2p) (p^2-1)^{(D-1)/2-\alpha}\int_0^\infty ds\, s^{\alpha-1}s^{1+(D-1)/2-\alpha}\nonumber\\
% % &\times  \frac{e^{-2\sqrt{2s}pd}}{1-r'^2e^{-2\sqrt{2s}pd}}\left[\frac{2r'^2d}{\sqrt{2s}p}+ 
% % \frac{r'}{2}(1-r'^2)\left(\dfrac{1}{s(p^2+\chi)}  -\frac{1}{s p^2}\right)\right]\\
% =&  2C\int_1^\infty dp (p^2-1)^{(D-1)/2-\alpha}\int_0^\infty ds\, s^{(D-1)/2}\nonumber\\
% &\times\frac{e^{-2\sqrt{2s}pd}}{1-r'^2e^{-2\sqrt{2s}pd}}\left[\frac{2r'^2d}{\sqrt{2s}}- 
% \frac{r'}{2}(1-r'^2)\left(\dfrac{\chi}{s(p^2+\chi)p}\right)\right],
% \end{align}
% and 
% \begin{equation}
% r' =  \frac{(1+\chi)p-\sqrt{p^2 + \chi}}{(1+\chi)p+\sqrt{p^2+\chi}},
% \end{equation}
\item Finally, take 
$s = \xi^2$
\item Now take $p = \sqrt{q+1}$.  $\rightarrow  q = p^2-1, dq = 2p dp$.  
 \begin{align}
E% =&  4C\int_1^\infty dp (p^2-1)^{(D-1)/2-\alpha}\int_0^\infty d\xi\,\xi\xi^{D-1}\nonumber\\
% &\times  \frac{e^{-2\sqrt{2}\xi pd}}{1-r'^2e^{-2\sqrt{2}\xi pd}}\left[\frac{2r'^2d}{\sqrt{2}\xi}-
%  \frac{r'}{2}(1-r'^2)\left(\frac{ \chi}{\xi^2(p^2+\chi)p}\right)\right]\\
=&  4C\int_1^\infty dp (p^2-1)^{(D-1)/2-\alpha}\int_0^\infty d\xi\,\xi^{D-1}\nonumber\\
&\times  \frac{r'^2e^{-2\sqrt{2}\xi pd}}{1-r'^2e^{-2\sqrt{2}\xi pd}}\left[\frac{2d}{\sqrt{2}}-
 \frac{1}{2}\left(\frac{1}{r'}-r'\right)\left(\dfrac{\chi}{\xi(p^2+\chi)p}\right)\right],
\end{align}
and 
\begin{equation}
r' =  \frac{(1+\chi)p-\sqrt{p^2 + \chi}}{(1+ \chi)p+\sqrt{p^2+\chi}},
\end{equation}

\item Now take $u = \sqrt{2}\xi d$ 
 \begin{align}
E% =&  4C\frac{1}{2^{5/2}d^4}\int_1^\infty dp (p^2-1)\int_0^\infty du\,u^{3} 
% \frac{r'^2e^{-2u p}}{1-r'^2e^{-2u p}}\left[2d- \left(\frac{1}{r'}-r'\right)\dfrac{\chi d}{u(p^2+\chi)p}\right]\\
=&  C\frac{\sqrt{2}}{d^3}\int_1^\infty dp (p^2-1)\int_0^\infty du\,u^{3} 
 \frac{r'^2e^{-2u p}}{1-r'^2e^{-2u p}}\left[1- \left(\frac{1}{r'}-r'\right)\dfrac{\chi }{2u(p^2+\chi)p}\right],
\end{align}
and 
\begin{equation}
r' =  \frac{(1+\chi)p-\sqrt{p^2 + \chi}}{(1+ \chi)p+\sqrt{p^2+\chi}},
\end{equation}
\comment{proportional to Schwinger}
\end{enumerate}

\subsection{Integrate by parts w.r.t. $p$}
Let's check if this also goes for the TM case.  
\begin{align}
\frac{d}{dp}\ln[1-r'^2 e^{-2pu}] %=& \frac{-2r' \frac{dr'}{dp} e^{-2pu} + 2u r'^2 e^{-2pu}}{1-r'^2 e^{-2pu}} \\
% &= \frac{r'^2 e^{-2pu}}{1-r'^2 e^{-2pu}}\left( 2u -\frac{2}{r'} \frac{dr'}{dp}\right)\\
% &= 2u\frac{r'^2 e^{-2pu}}{1-r'^2 e^{-2pu}}\left( 1 -\frac{1}{ur'} \frac{dr'}{dp}\right)\\
&= 2u\frac{r'^2 e^{-2pu}}{1-r'^2 e^{-2pu}}\left( 1 -\frac{1}{u} \frac{d\ln[r']}{dp}\right)
\end{align}
Now use 
\begin{align}
\frac{d}{dp}\ln[r'] =& \frac{d}{dp}\left(\log[e^{2\Xi}p - \sqrt{p^2+\chi}] -\ln[e^{2\Xi}p + \sqrt{p^2+\chi}]\right) \\
%=& \frac{e^{2\Xi} - \frac{p}{\sqrt{p^2+\chi}}}{e^{2\Xi}p-\sqrt{p^2+\chi}} -\frac{e^{2\Xi} + \frac{p}{\sqrt{p^2+\chi}}}{e^{2\Xi}p + \sqrt{p^2+\chi}}\\ 
%=& \frac{\left[e^{2\Xi} - \frac{p}{\sqrt{p^2+\chi}}\right]\left[e^{2\Xi}p + \sqrt{p^2+\chi}\right]-\left[e^{2\Xi} + \frac{p}{\sqrt{p^2+\chi}}\right]\left[e^{2\Xi}p-\sqrt{p^2+\chi}\right]}{e^{4\Xi}p^2-(p^2+\chi)}\\
% =& \frac{\left[e^{4\Xi}p  - e^{2\Xi}\frac{p^2}{\sqrt{p^2+\chi}} +e^{2\Xi}\sqrt{p^2+\chi} - p\right]}{e^{4\Xi}p^2-(p^2+\chi)}\nonumber\\
% & - \frac{\left[e^{4\Xi}p + e^{2\Xi}\frac{p^2}{\sqrt{p^2+\chi}} -e^{2\Xi}\sqrt{p^2+\chi}-p\right]}{e^{4\Xi}p^2-(p^2+\chi)}\\
% =& \frac{2e^{2\Xi}\sqrt{p^2+\chi} - 2e^{2\Xi}\frac{p^2}{\sqrt{p^2+\chi}} }{e^{4\Xi}p^2-(p^2+\chi)}\nonumber\\
=& \frac{2\chi e^{2\Xi}}{\sqrt{p^2+\chi}[e^{4\Xi}p^2-(p^2+\chi)]}
\end{align}
Then 
\begin{align}
\frac{d}{dp}\ln[1-r'^2 e^{-2pu}] &= 
2u\frac{r'^2 e^{-2pu}}{1-r'^2 e^{-2pu}}\left( 1 -\frac{2\chi e^{2\Xi}}{u\sqrt{p^2+\chi}[e^{4\Xi}p^2-(p^2+\chi)]}\right)\label{eq:TM_integration_by_parts}
\end{align}

% % Ultimately, we want this to be proportional to 
%  \begin{align}
%  C=&\frac{r'^2 e^{-2pu}}{1-r'^2 e^{-2pu}}\left[1 - \left(\frac{1}{r'}-r'\right)\dfrac{\chi }{2u(p^2+\chi)p}\right]\\
% % =&\frac{r'^2 e^{-2pu}}{1-r'^2 e^{-2pu}}\left[1 - \left(\frac{e^{2\Xi}p+\sqrt{p^2+\chi}}{e^{2\Xi}p-\sqrt{p^2+\chi}}
% % -\frac{e^{2\Xi}p-\sqrt{p^2+\chi}}{e^{2\Xi}p+\sqrt{p^2+\chi}}\right)\dfrac{\chi }{2u(p^2+\chi)p}\right]\\
% % =&\frac{r'^2 e^{-2pu}}{1-r'^2 e^{-2pu}}\left[1 - \frac{4pe^{2\Xi}\sqrt{p^2+\chi}}{e^{4\Xi}p^2-(p^2+\chi)}\dfrac{\chi }{2u(p^2+\chi)p}\right]\\
%  =&\frac{r'^2 e^{-2pu}}{1-r'^2 e^{-2pu}}\left[1 - \frac{2\chi e^{2\Xi}}{u\sqrt{p^2+\chi}[e^{4\Xi}p^2-(p^2+\chi)]}\right]\label{eq:CTM}\\
%  \end{align}
% % So this will work.  
The TM Casimir energy is given by
\begin{align}
E_{TM} = & -\frac{\hbar c}{8\pi^2 d^3}\int_0^\infty du\,u^{3} \int_1^\infty dp\, (p^2-1) 
\frac{r'^2e^{-2u p}}{1-r'^2e^{-2u p}}\left[1- \left(\frac{1}{r'}-r'\right)\dfrac{\chi }{2u(p^2+\chi)p}\right].
\end{align}
We can use Eq.~(\ref{eq:TM_integration_by_parts}) to integrate with respect to $p$.  
\begin{align}
I_2 =& \int_1^\infty dp\, (p^2-1) \frac{r'^2e^{-2u p}}{1-r'^2e^{-2u p}}
\left[1- \left(\frac{1}{r'}-r'\right)\dfrac{\chi }{2u(p^2+\chi)p}\right]\\
=& \left[(p^2-1)\frac{1}{2u}\log[1-r'^2 e^{-2pu}]\right]_{p=1}^\infty - \int_1^\infty dp\,\frac{p}{u}\log[1-r'^2 e^{-2pu}]
\end{align}
Which gives us 
\begin{align}
\Aboxed{E_{TM}= & \frac{\hbar c}{8\pi^2 }\int_0^\infty d\xi\,\xi^{2} \int_1^\infty dp\, p \log[1-r'^2 e^{-2p\xi d}]}
\end{align}
\comment{Am I off by $1/4$?  That should be a matter of more careful accounting.}


\section{Finite Temperature and Dispersion}
\label{sec:nonzero_temp}
We will handle the finite temperature (and dispersion) in the atom-plane geometry.
  We derive the partition function for finite temperature for both the TE and TM polarizations.
  So far, I have done both Casimir and Casimir-Polder energies for TE, and only Casimir-Polder for TM.
  % The zero temperature limits all work out nicely.
  % I am stumbling a little over the right way to do the high temperature limit here.  

So I read Babb's paper\footnote{Babb, J. F. and Klimchitskaya, G. L., and Mostepanenko, V. M., 
``Casimir-Polder interaction between an atom and a cavity wall under the influence of real conditions'',
 Phys. Rev. A, \textbf{70},042901,(2004)} (which Dan cites for thermal Casimir-Polder calculations.~\cite{Babb2004})
  In it they use the free energy, which is $\mathcal{F} = -k_BT\log Z$, as the basis of their calculations.
%  I've been trying to use the mean energy, $E= -\partial_\beta\log Z$. 



The TE Casimir--Polder free energy is is 
\begin{align}
E-E_0&=-k_BT{\sum_n}\,\frac{ \omega_n^2}{2\epsilon_0c^2}\alpha(i\omega_n)\int_0^\infty d\cT\,
\frac{1}{(2\pi \cT)^{3/2}}\dlangle e^{-\cT\frac{\omega_n^2}{2c^2}} 
-e^{-\frac{\cT \omega_n^2\langle\epsilon(i\omega_n)\rangle}{2c^2}}\drangle\label{eq:TE_thermal_energy}
\end{align}

We will focus on the Casimir-Polder result, since I can then compare to Dan's expressions in the notes.
  For the Casimir results, I will collate some results from the literature to compare in the nonzero
 temperature/dispersive cases.  


% \begin{align}
% E-E_0&=-k_BT \frac{\hbar\beta}{2\pi}\int_{-\infty}^\infty d\omega\int_0^\infty \frac{dT}{T}
%\int d^3x\,\frac{1}{(2\pi T)^{3/2}}\dlangle e^{-T\frac{\omega^2}{2c^2}} 
%-  e^{-\frac{ T \omega^2\langle\epsilon(i\omega)\rangle}{2c^2}}\drangle,
% \end{align}


% \subsection{Far-field approximation, zero temperature}

% Let us consider how to take the far-field approximation from these expressions.
  % Since we are taking an ensemble average over Gaussian random walks, so the loops will intersect all the surfaces when $T\sim d^2$, where $d$ is the distance from the source point $x_0$ to the farthest surface.
  % Secondly, the frequency integral is dominated by the exponential factors, which will contribute most when $T\omega^2/c^2\sim 1$.
  % This suggests that frequencies with $d^2\omega^2/c^2<1$ will contribute most.
  % In the limit where $d/c$ is large, only frequencies near $0$ will contribute, and we can approximate $\epsilon(i\omega) \approx \epsilon(0)$ everywhere.


\subsubsection{Feynman-Kac formula}

From Dan's work (or my re-working of it) we have the Laplace-Mellin transform for the single body Feynman-Kac formula,
% \begin{align}
% \int_0^\infty \frac{dT}{T^{1+z}}\dlangle e^{-s[T+ \chi\int_0^Tdt\Theta(x-d)]}\drangle 
% =& \int_0^\infty \frac{dT}{T^{1+z-1/2}}e^{-sT}\dlangle \frac{e^{-s \chi \int _0^T dt \Theta(x-d)}}{\sqrt{T}}\drangle\\
% =& \frac{1}{\Gamma[z+1/2]}\int_0^\infty d\lambda\, \lambda^{z-1/2}\int_0^\infty dT e^{-(\lambda+s)T}
% \dlangle \frac{e^{-s \chi \int_0^T dt \Theta(x-d)}}{\sqrt{T}}\drangle.
% \end{align}
% In our case $z=1/2$, and $s= \omega^2/(2c^2)$.
%   We also need the actual analytical expression for that path integral,
% \begin{equation}
% \int_0^\infty dT e^{-(\lambda+s) T} \dlangle \frac{e^{-s\chi\int_0^T dt \Theta(x-d)}}{\sqrt{2\pi T}}\drangle  
% =\frac{1}{\sqrt{2(\lambda+s)}}\left[1 - e^{-2\sqrt{2(\lambda+s)}|d|}\frac{\sqrt{\lambda+s(1+\chi)}
% -\sqrt{\lambda+s}}{\sqrt{\lambda+s(1+\chi)}+\sqrt{\lambda+s}}\right].
% \end{equation}
We will need to apply both of these results as 
\begin{align}
\int_0^\infty dT\,\frac{1}{(2\pi T)^{3/2}}\dlangle e^{-sT} - e^{-sT \langle\epsilon(i\omega)\rangle}\drangle 
& =\frac{1}{2\pi}\int_0^\infty d\lambda\, \frac{e^{-2\sqrt{2(\lambda+s)}|d|}}{\sqrt{2(\lambda+s)}}
\frac{\sqrt{\lambda+s[1+\chi(i\omega)]}-\sqrt{\lambda+s}}{\sqrt{\lambda+s[1+\chi(i\omega)]}+\sqrt{\lambda+s}}
\end{align}

\begin{align}
E-E_0&=-k_BT{\sum_n}'\frac{\omega_n^3\alpha(i\omega_n)}{4\pi\epsilon_0c^3}\int_1^\infty dp\,e^{-2\omega_n p|d|/c}
\frac{\sqrt{p^2+\chi(i\omega_n)}-p}{\sqrt{p^2+\chi(i\omega_n)}+p},
\label{eq:TE_CP_finite_temperature}
\end{align}
This is the general result for finite temperature and dispersion for the TE polarization for an atom near 
a planar dielectric.
  We can also take the zero temperature limit.
  In the limit $\beta\rightarrow \infty$ the difference between Matsubara frequencies approaches zero,
 $\Delta\omega_n =\frac{2\pi}{\beta\hbar}$.  Then we can take $\sum_n\Delta\omega_n \rightarrow \int_0^\infty d\omega$.
\begin{align}
E-E_0&=-\frac{\hbar}{8\pi^2\epsilon_0c^3}\int_0^\infty d\omega\,\omega^3\alpha(i\omega)
\int_1^\infty dp\,e^{-2\omega p|d|/c}\frac{\sqrt{p^2+\chi(i\omega)}-p}{\sqrt{p^2+\chi(i\omega)}+p},\label{eq:TE_CP_zero_temperature}
\end{align}
where now $\omega$ is a continuous variable.  

\subsection{Various Limiting Cases}

Let us now consider the various limits for the Casimir-Polder case.
  Since we are taking an ensemble average over Gaussian random walks,
 the loops will intersect all the surfaces when $T\sim d^2$, 
where $d$ is the distance from the source point $x_0$ to the farthest surface.
  Secondly, the frequency sum/integral is dominated by the exponential factors,
 which will contribute most when $T\omega_n^2/c^2\sim 1$.
  This suggests that frequencies with  $\omega_n^2< c^2/d^2$  will contribute most.   

\subsubsection{Zero temperature, far-field limit}

We start from Eq.~(\ref{eq:TE_CP_zero_temperature}).
  If we also take the limit where $d/c$ is large, only frequencies near $0$ will contribute,
 and we can approximate $\epsilon(i\omega) \approx \epsilon(0)$, $\alpha(i\omega)\approx\alpha_0$ everywhere.  
\begin{align}
E-E_0&=-\frac{\hbar}{8\pi^2\epsilon_0c^3}\int_0^\infty d\omega\,\omega^3\alpha(i\omega)
\int_1^\infty dp\,e^{-2\omega p|d|/c}\frac{\sqrt{p^2+\chi(0)}-p}{\sqrt{p^2+\chi(0)}+p}
\end{align}

 Now evaluate the $\omega$ integral, 
\begin{equation}
\int_{0}^\infty d\omega\,\omega^3e^{-2\omega r d/c} = \frac{3 c^4}{8 p^4 d^4}
\end{equation}
which leaves a by now familiar integral:
\begin{align}
E-E_0&= -\frac{3\hbar c\alpha_0}{64\pi^2 \epsilon_0 d^4}\int_1^\infty dp\,p^{-4}\frac{\sqrt{p^2+\chi}-p}{\sqrt{p^2+\chi}+p}
\end{align}

\subsubsection{Near field, low temperature}
Let's now work in the limit where $d<<c/\omega_{j0}$.
  In this case all frequencies contribute, but we can convert the sum into an integral.
  The difference here is that all of the frequency dependence of $\epsilon(\omega)$  will matter.  
\begin{equation}
E-E_0=-\frac{\hbar}{8\pi^2\epsilon_0c^3}\int_0^\infty d\omega\,\omega^3\alpha(i\omega)
\int_1^\infty dp\,e^{-2\omega p|d|/c}\frac{\sqrt{p^2+\chi(i\omega)}-p}{\sqrt{p^2+\chi(i\omega)}+p}
\end{equation}
The integral contributes most when the exponent is order unity.
  The presence of $\alpha$ means that frequencies around $\omega_{j0}$ will dominate the frequency integral.
  Then $p \sim  c/(d\omega_{j0})\gg 1$.
  Let's use that fact to approximate the reflection coefficient, 
and see if we can reproduce the known van der Waals result.  
\begin{align}
  \frac{\sqrt{p^2+\chi(i\omega)}-p}{\sqrt{p^2+\chi(i\omega)}+p}\approx 
& \frac{ p + \frac{\chi}{2p}-p}{2p+\frac{\chi(i\omega)}{2p}} \approx \frac{\chi}{4p^2} 
\end{align}
Plugging this in, we can then evaluate the $p$ integral
\begin{align}
E-E_0=&-\frac{\hbar}{8\pi^2\epsilon_0c^3}\int_0^\infty d\omega\,\omega^3\alpha(i\omega)\chi(i\omega)
\int_1^\infty dp\,\frac{1}{4p^2}e^{-2\omega p|d|/c}
\end{align}

Let's try to work on that integral a bit.  
\begin{align}
\int_1^\infty dp\,\frac{1}{4p^2}e^{-2\omega p|d|/c}%  =& -\frac{1}{4p}e^{-2\omega p d/c}\bigg|_{p=1}^{\infty}
 % + \int_1^\infty dp\, \frac{1}{4p}\times \frac{-2\omega d}{c}e^{-2\omega pd/c}\\
=& \frac{1}{4}e^{-2\omega d/c} - \frac{2\omega d}{c} \int_1^\infty dp\, \frac{e^{-2\omega pd/c}}{p}.
\end{align}

Recall, we are working in the so-called near-field limit where $d\omega_{j0}/c<<1.$  
I think we can get away with approximating this as just the exponential term.  

\begin{align}
E-E_0=&-\frac{\hbar}{32\pi^2\epsilon_0c^3}\int_0^\infty d\omega\,\omega^3\alpha(i\omega)\chi(i\omega)e^{-2\omega d/c}\\
=&-\frac{\hbar}{32\pi^2\epsilon_0 d^3}\int_0^\infty d\omega\,\frac{\omega^3d^3}{c^3}\alpha(i\omega)\chi(i\omega)e^{-2\omega d/c}\approx 0
\end{align}
Since I think we are working wiht $d\omega/c$ is very small, so this term is tiny.  

(From Dan's analysis in the notes, apparently we can drop this term, or rather it is negligible in comparison to the $TM$ contribution.)


\subsubsection{High Temperature, (far field ?)}  

As we noted earlier, the renormalized partition function vanishes for $\omega_0$.
The leading term is $\omega_1$, which will be exponentially suppressed relative to the TM contribution.  


% This is the general case.  

% The presence of $i\langle \partial_\omega\epsilon(i\omega_n)$ in our expression is acceptable,
% since $\epsilon(i\omega_n)$ is in itself a real function, so $k_BT\epsilon(i\omega_n)$ is also real.
%  As it stands, this factor of $i$ will then combine with further factors of $i$ from the form of the derivative.
%  For example, the response of a harmonic oscillator with frequency  $\omega_0$ is $\epsilon(\omega) = A/(\omega^2-\omega_0^2)$.
%  The derivative is $\partial_\omega\epsilon(\omega) = -2A\omega/(\omega^2-\omega_0^2)^2$.  
% Now if we consider imaginary frequencies then $\epsilon(i\omega_n) = -A/(\omega_n^2+\omega_0^2)$, 
%and $i\partial_\omega \epsilon(\omega)\big|_{\omega=i\omega_n} = -i (iA\omega_n)/(\omega_n^2+\omega_0^2)^2$, which is real.
%    We would get the same result in evaluating $k_BT\epsilon(i\omega_n)$ directly.  


% \subsection{TE Polarization: Casimir}

% Let's try to do this for the Casimir energy due to TE as well.  
% We will start from 
% \begin{equation}
% E-E_0=k_BT{\sum_n}'\int_0^\infty \frac{dT}{T}\int dx\,\frac{1}{(2\pi T)^{3/2}}\dlangle e^{-T\frac{\omega_n^2}{2c^2}}
%  -  e^{-\frac{ T \omega_n^2\langle\epsilon(i\omega_n)\rangle}{2c^2}}\drangle\label{eq:casimir_partition_function}
% \end{equation}
% We will need to also subtract off the renormalized one body energies.  
% As before, we need the Feynman-Kac formula,
% \begin{align}
% &\int dx\int_0^\infty dT \frac{e^{-\lambda T}}{\sqrt{2\pi T}}\left[e^{-T\langle\epsilon_{12}\rangle}+
%  e^{-T\langle\epsilon_{0}\rangle} - e^{-T\langle\epsilon_{1}\rangle}- e^{-T\langle\epsilon_{2}\rangle}\right]\nonumber\\ 
% =&  \frac{u_1u_2 e^{-2\sqrt{2\lambda}d}}{\sqrt{2\lambda}(1-u_1u_2 e^{-2\sqrt{2\lambda}d})}
% \left( 2d + \frac{\sqrt{2}}{\sqrt{(\lambda+\chi_1)}} + \frac{\sqrt{2}}{\sqrt{(\lambda+\chi_2)}} \right)
% \end{align}
% where $u_i = (\sqrt{\lambda}-\sqrt{\lambda+\chi})/(\sqrt{\lambda}+\sqrt{\lambda+\chi})$.
%   We will also need 
% \begin{align}
% \int_0^\infty \frac{dT}{T^{1+z}}\dlangle e^{-s[T+ \chi\int_0^Tdt\Theta(x-d)]}\drangle=& 
% \frac{1}{\Gamma[z+1/2]}\int_0^\infty d\lambda\, \lambda^{z-1/2}\int_0^\infty dT e^{-(\lambda+s)T}
% \dlangle \frac{e^{-s \chi \int_0^T dt \Theta(x-d)}}{\sqrt{T}}\drangle.
% \end{align}
% In this case $z=3/2$, and $s= \omega^2/(2c^2)$, so we need to take 
% $\lambda\rightarrow \lambda+ \omega_n^2/(2c^2)$, and $\chi\rightarrow s\omega_n^2/(2c^2)$.
%   Putting these identities together in Eq.~(\ref{eq:casimir_partition_function})yields 
% \begin{align}
% E-E_0=&k_BT{\sum_n}'\frac{1}{\Gamma[2]2\pi}\int_0^\infty d\lambda\, \lambda 
%  \frac{u_1u_2 e^{-2\sqrt{2\lambda+\omega_n^2/c^2}d}}{\sqrt{2\lambda+\omega_n^2/c^2}(1-u_1u_2 e^{-2\sqrt{2\lambda+\omega_n^2/c^2}d})}\nonumber\\
% &\times \left( 2d + \frac{\sqrt{2}}{\sqrt{\lambda+\omega_n^2/(2c^2)(1+\chi_1)}}
%  + \frac{\sqrt{2}}{\sqrt{\lambda+\omega_n^2/(2c^2)/(1+\chi_2)}} \right)
% \end{align}
% with 
% \begin{equation}
% u_i = \frac{\sqrt{\lambda+\omega_n^2/(2c^2)}-\sqrt{\lambda+\omega_n^2/(2c^2)(1+\chi_i)}}
% {\sqrt{\lambda+\omega_n^2/(2c^2)}+\sqrt{\lambda+\omega_n^2/(2c^2)(1+\chi_i)}}
% \end{equation}

% We'll now make some variable changes to put this in a more tractable form.
%   First up, let's define $\lambda = \kappa \omega_n^2/(2c^2)$.  
% \begin{align}
% E-E_0%=&k_BT{\sum_n}'\frac{1}{2\pi}\int_0^\infty d\kappa\,\frac{\omega_n^4}{4c^4} \kappa  \frac{cu_1u_2 e^{-2\omega_n\sqrt{\kappa+1}d/c}}{\omega_n\sqrt{\kappa+1}(1-u_1u_2 e^{-2\omega_n\sqrt{\kappa+1}d/c})}\nonumber\\
% %&\times \left( 2d + \frac{2c}{\omega_n\sqrt{\kappa+1+\chi_1}} + \frac{2 c}{\omega_n\sqrt{\kappa+1+\chi_2}} \right)\\
% =&k_BT{\sum_n}'\frac{1}{2\pi}\int_0^\infty d\kappa\,\frac{\omega_n^2}{2c^2} \kappa 
%  \frac{u_1u_2 e^{-2\omega_n\sqrt{\kappa+1}d/c}}{\sqrt{\kappa+1}(1-u_1u_2 e^{-2\omega_n\sqrt{\kappa+1}d/c})}
% \left( \frac{\omega_nd}{c} + \frac{1}{\sqrt{\kappa+1+\chi_1}} + \frac{1}{\sqrt{\kappa+1+\chi_2}} \right)
% \end{align}
% with
% \begin{equation}
% u_i = \frac{\sqrt{\kappa+1}-\sqrt{\kappa+1+\chi_i}}{\sqrt{\kappa +1}+\sqrt{\kappa+1+\chi_i}}.
% \end{equation}
% Next up define $\kappa+1 = p^2$.  
% \begin{align}
% E-E_0%=&k_BT{\sum_n}'\frac{\omega_n^2}{4\pi c^2}\int_1^\infty dp\,2p (p^2-1)  \frac{u_1u_2 e^{-2\omega_n pd/c}}{p(1-u_1u_2 e^{-2\omega_npd/c})}\left( \frac{\omega_nd}{c} + \frac{1}{\sqrt{p^2+\chi_1}} + \frac{1}{\sqrt{p^2+\chi_2}} \right)\\
% =&k_BT{\sum_n}'\frac{\omega_n^2}{2\pi c^2}\int_1^\infty dp\,(p^2-1)  
% \frac{u_1u_2 e^{-2\omega_n pd/c}}{(1-u_1u_2 e^{-2\omega_npd/c})}
% \left( \frac{\omega_nd}{c} + \frac{1}{\sqrt{p^2+\chi_1}} + \frac{1}{\sqrt{p^2+\chi_2}} \right)
% \end{align}
% with
% \begin{equation}
% u_i = \frac{p-\sqrt{p^2+\chi_i}}{p+\sqrt{p^2+\chi_i}}.
% \end{equation}
% Finally, let's integrate by parts with respect to $p$.  
% \begin{shaded}
%  First up take the derivative of the reflection coefficient, 
% \begin{align}
% \frac{dr}{dp} =& \frac{d}{dp} \frac{p-\sqrt{p^2+\chi}}{p+\sqrt{p^2+\chi}}
% = \frac{1-\frac{2p}{2\sqrt{p^2+\chi}}}{p+\sqrt{p^2+\chi}} - (p-\sqrt{p^2+\chi})
% \frac{(1+\frac{2p}{2\sqrt{p^2+\chi}})}{(p+\sqrt{p^2+\chi})^2} 
% %=& \frac{1}{\sqrt{p^2+\chi}}\frac{\sqrt{p^2+\chi}-p}{p+\sqrt{p^2+\chi}} - (p-\sqrt{p^2+\chi})\frac{p+\sqrt{p^2+\chi}}{\sqrt{p^2+\chi}(p+\sqrt{p^2+\chi})^2} \\
% = -\frac{2r}{\sqrt{p^2+\chi}}
% \end{align}
% We can also write:
% \begin{align}
% \frac{d}{dp}[1-r_1r_2 e^{-2p\omega_n d/c }]% &= -\left( r_1\frac{dr_2}{dp} e^{-2p\omega_n d/c} + r_2\frac{dr_1}{dp} e^{-2p\omega_n d/c} - 2\xi r_1r_2d e^{-2p\omega_n d/c}\right)\\
% % &= -\left( -2 \frac{r_1r_2}{\sqrt{p^2+\chi_2}} -2 \frac{r_1r_2}{\sqrt{p^2+\chi_1}}- 2r_1r_2\frac{\omega_nd}{c} \right)e^{-2p\omega_n d/c}\\
% &= 2\left(\frac{\omega_nd}{c} +\frac{1}{\sqrt{p^2+\chi_1}} +\frac{1}{\sqrt{p^2+\chi_2}}\right) r_1r_2e^{-2p\omega_n d/c},
% \end{align}
% which suggests 
% \begin{align}
% \Aboxed{\frac{d}{dp}\ln[1-r_1r_2 e^{-2p\omega_n d/c}]
% &= \frac{2r_1r_2 e^{-2p\omega_n d/c}}{1 - r_1r_2  e^{-2p\omega_n d/c}}
% \left(\frac{\omega_n d}{c}+\frac{1}{\sqrt{p^2+\chi_1}}+\frac{1}{\sqrt{p^2+\chi_2}}\right)}
% \end{align}
% \end{shaded}

% So after integration by parts our the renormalized Casimir energy becomes
% \begin{align}
% E-E_0 & % = k_BT{\sum_n}'\frac{\omega_n^2}{2\pi c^2}\int_1^\infty dp\,(p^2-1)\frac{ r_1r_2e^{-2\omega_n p d/c}}{(1 -r_1r_2 e^{-2\omega_n pd/c})}\left[ \frac{\omega_n d}{c} +\frac{1}{\sqrt{p^2+\chi_1}}+\frac{1}{\sqrt{p^2+\chi_2}}\right]\\
% % & = k_BT{\sum_n}'\frac{\omega_n^2}{2\pi c^2}\left\{\frac{1}{2}\log\left[1-r_1r_2 e^{-2\omega_n p d/c}\right](p^2-1)\bigg|_{p=1}^\infty - \int_1^\infty dp \,p\log\left[1-r_1r_2 e^{-2\omega_n p d/c}\right]\right\}\\
% & = -k_BT{\sum_n}'\frac{\omega_n^2}{2\pi c^2}\int_1^\infty dp \,p
% \log\left[1-r_1r_2 e^{-2\omega_n p d/c}\right]\label{eq:Casimir_energy_finite_temperature}
% \end{align}
% We can take the zero temperature limit here as well:
% \begin{align}
% E-E_0& = -\frac{\hbar c}{4\pi^2}\int_0^\infty dk\,k^2\int_1^\infty dp \,p
% \log\left[1-r_1r_2 e^{-2k p d}\right]\label{eq:Casimir_energy_zero_temperature}
% \end{align}

\section{TM Polarization:Partition Function}

In this case we are starting from the TM polarization
% \begin{equation}
% Z_{TM} = \int D\psi \exp\left[ -\frac{\epsilon_0}{2}\int_0^\beta d\beta'\int d^3x\,
%  \left( \frac{\epsilon(x)}{\hbar^2}(\partial_{\beta'}\psi)^2 + c^2\frac{1}{\epsilon}|\nabla\sqrt{\epsilon}\psi|^2\right)\right] .
% \end{equation}

Let us change to using $\tau = \beta \hbar c$ as our temperature coordinate.  Then   
\begin{equation}
Z_{TM} = \int D\psi \exp\left[ -\frac{\epsilon_0 c^2 }{2 \hbar c}\int_0^{\hbar\beta c} 
d\tau'\int d^3x\, \psi\left( \epsilon(x)(\partial_{\tau}
  -\sqrt{\epsilon}\nabla \epsilon^{-1}\nabla\sqrt{\epsilon} -\nabla^2\right)\psi\right].
\end{equation}

As before, we introduce the Matsubara frequencies $\omega_n$
% We will now introduce the Matsubara frequencies $\omega_n = (2\pi n)/(\beta \hbar)$, with 
% \begin{equation}
% \psi(\beta,x) = \sum_{n=-\infty}^{\infty}e^{i\frac{\omega_n}{c}\tau} \psi_n(x),
% \end{equation}
% where $\tau = \beta\hbar c$, and the $\psi_n$ are complex variables.  We will also need to use  
% \begin{equation}
% \int_0^{\beta \hbar c}d\tau e^{i\frac{(\omega_n+\omega_m)}{c}\tau} = \beta\hbar c \delta_{n,-m},
% \end{equation}
% and $\psi_n^* = \psi_{-n}$ since $\psi^*(\beta, x) = \psi(\beta, x)$.  
Then we have 
\begin{equation}
Z_{TM} = \prod_{n=-\infty}^{\infty} \int D\psi_n\exp\left[ -\frac{\beta \epsilon_0 c^2 }{2}\int d^3x\, 
\psi_n^*\left(\epsilon(i\omega_n,x)\frac{\omega_n^2}{c^2}+   V_{TM} -\nabla^2\right)\psi_n\right], 
\end{equation}
where $V_{TM} = (\nabla\ln\sqrt{\epsilon})^2-\nabla^2\log\sqrt{\epsilon}$.
  Note that the presence of $V_{TM}$ implies there will be a contribution to the $n=0$ Matsubara term,
 which is good, since we know that the TM energy is the dominant contribution in that case.
This is what should give us a the dominant near-field, and high temperature results.  
   We will renormalize this against vacuum, 
% \begin{equation}
% \log Z_{TE} -\log Z_0= -{\sum_{n=0}^\infty}'\left\{\log\det\left[ \frac{1}{2}
% \left(\epsilon(i\omega_n,\vect{x})\frac{\omega_n^2}{c^2} +V_{TM}-\nabla^2\right)\right]
%  - \log\det\left[ \frac{1}{2}\left(\frac{\omega_n^2}{c^2} -\nabla^2\right)\right]\right\}
% \end{equation}
% Note that for $n=0$ the Matsubara frequency $\omega_n=0$, so we have 
% \begin{align}
% &\log\det\left[ \frac{1}{2}\left(\epsilon(0,\vect{x})\frac{\omega_0^2}{c^2}+V_{TM} -\nabla^2\right)\right]
%  - \log\det\left[ \frac{1}{2}\left(\frac{\omega_0^2}{c^2} -\nabla^2\right)\right] \nonumber\\
% &= \log\det\left[\frac{1}{2}\left(V_{TM} -\nabla^2\right)\right]
%  - \log\det\left[ \frac{1}{2}\left( -\nabla^2\right)\right]\ne  0
% \end{align}
% This assumes that $\epsilon(\omega)$ has at most a simple pole at zero frequency,
%  such that $\lim_{\omega\rightarrow 0}\omega^2\epsilon(\omega)=0.$    
Then renormalized Casimir energy is 
\begin{align}
E-E_0 & = -k_BT \log \frac{Z_{TM}}{Z_0} \\
&=k_BT{\sum_n}'\int_0^\infty \frac{dT}{T}\int d^3x\,\frac{1}{(2\pi T)^{3/2}}\dlangle e^{-T\frac{\omega_n^2}{2c^2}}
 -  e^{-\frac{ T \omega_n^2\langle\epsilon(i\omega_n)\rangle}{2c^2} - \frac{T}{2}\langle V_{TM}\rangle}\drangle\label{eq:TMworldline_partition_function}
\end{align}

\subsection{Casimir-Polder energy}

The renormalized TM Casimir--Polder free energy is 
\begin{align}
E-E_0=\,&-k_BT{\sum_n}'\frac{\alpha(i\omega_n)}{2\epsilon_0}\int_0^\infty d\cT\,
\frac{1}{(2\pi \cT)^{3/2}}\nonumber\\
&\times\dlangle \frac{\omega_n^2}{c^2}e^{-\cT\frac{\omega_n^2}{2c^2}}
-\left(\frac{\omega_n^2}{c^2}  - \frac{1}{2}\partial_x^2\right)e
^{-\frac{ \cT \omega_n^2\langle\epsilon(i\omega_n)\rangle}{2c^2} - \cT\langle V_{TM}\rangle}\drangle\label{eq:TM_CP_finite_temperature}.
\end{align}
As before, we can straightforwardly take the zero temperature limit: 
\begin{align}
E-E_0=&-\frac{\hbar}{2\pi}\int_0^\infty d\omega\frac{\alpha(i\omega)}{2\epsilon_0}
\int_0^\infty d\cT\,\frac{1}{(2\pi \cT)^{3/2}}\nonumber\\
&\times\dlangle \frac{\omega^2}{c^2}e^{-\cT\frac{\omega^2}{2c^2}}-\left(\frac{\omega^2}{c^2}  - \frac{1}{2}\partial_x^2\right)e^{-\frac{ \cT \omega^2\langle\epsilon(i\omega)\rangle}{2c^2} - \cT\langle V_{TM}(i\omega)\rangle}\drangle\label{eq:TM_CP_zero_temperature},
\end{align}

\subsubsection{Laplace-Mellin and Feynman-Kac Formulae}
We will again need to use the Laplace-Mellin transforms, and Feynman-Kac Formulae.
  We quote the results:
The Laplace-Mellin transform is
\begin{align}
\int_0^\infty \frac{dT}{T^{1+z}}\dlangle e^{-sT\langle\epsilon\rangle - T\langle V_{TM}\rangle}\drangle =&
 \frac{1}{\Gamma[z+1/2]}\int_0^\infty d\lambda\, \lambda^{z-1/2}\int_0^\infty dT e^{-(\lambda+s)T}
\dlangle \frac{e^{-\int_0^T dt\,(s\chi+ V_{TM})}}{\sqrt{T}}\drangle.
\end{align}
For Casimir-Polder we need$z=1/2$, and for Casimir we need $z=3/2$.
  In both cases we need $s= \omega^2/(2c^2)$.
  We also need the actual analytical expression for that path integral.

For one body we need:
\begin{align}
&\int_0^\infty dT e^{-(\lambda+s) T} \dlangle \frac{e^{-s\chi\int_0^T dt \Theta(x-d)}}{\sqrt{2\pi T}}\drangle  \nonumber\\
&\hspace{0.5cm}=\frac{1}{\sqrt{2(\lambda+s)}}\left[1 - e^{-2\sqrt{2(\lambda+s)}|d|}\frac{\sqrt{\lambda+s(1+\chi)}
-\sqrt{\lambda+s}e^{2\Xi}}{\sqrt{\lambda+s(1+\chi)}+\sqrt{\lambda+s}e^{2\Xi}}\right],
\end{align}
where $e^{2\Xi} = (1+\chi)$ comes from the contribution of $e^{-V_{TM}}$.
  \comment{Correct signs?} For two macroscopic bodies we will need:
\begin{align}
&\int dx\int_0^\infty dT \frac{e^{-(\lambda +s)T}}{\sqrt{2\pi T}}\left[e^{-s\int_0^T dt\,(\chi_{12} + V_{12,TM})}
 +1 -e^{-s\int_0^T dt\,(\chi_{1} + V_{1,TM})}-e^{-s\int_0^T dt\,(\chi_{2} + V_{2,TM})}\right]\nonumber\\ 
=&  \dfrac{u_1'u'_2e^{-2\sqrt{2\lambda}d}}{1 - u'_1u'_2 e^{-2\sqrt{2\lambda}d}}\left[ \frac{2 d}{\sqrt{2\lambda}}
-\frac{ e^{2\Xi_1}}{\sqrt{\lambda+s}\sqrt{\lambda+s(1+\chi_1)}}
\frac{s\chi_1}{e^{4\Xi_1}(\lambda+s)-[\lambda+s(1+\chi_1)]}  + \{1 \leftrightarrow 2\}  \right].
\end{align}
where 
\begin{equation}
u'_i = \frac{\sqrt{\lambda+s}(1+\chi)-\sqrt{\lambda+s(1+\chi)}}{\sqrt{\lambda+s}(1+\chi)+\sqrt{\lambda+s(1+\chi)}}
\end{equation}
As nasty as that two-body expression may be, exactly the same tricks will work on it, 
and it will simplify down to exactly the same form as the other polarization.  

\subsection{TM Casimir-Polder: Limiting Cases}

We will work with the case of an atom in front of a dielectric surface.
  Let's first plug in the relevant one-body Feynman-Kac formula into the partition function.
 
\begin{align}
E-E_0=& -k_BT{\sum_n}'\frac{\alpha(i\omega_n)}{2\epsilon_0}\int_0^\infty dT\,
\frac{1}{(2\pi T)^{3/2}}\left(\frac{\omega_n^2}{c^2}  - \frac{1}{2}\partial_x^2\right)
\dlangle e^{-T\frac{\omega_n^2}{2c^2}}-e^{-\frac{ T \omega_n^2\langle\epsilon(i\omega_n)\rangle}{2c^2} - T\langle V_{TM}\rangle}\drangle \\
=& -k_BT{\sum_n}'\frac{\alpha(i\omega_n)}{4\pi\epsilon_0}\left(\frac{\omega_n^2}{c^2}  
- \frac{1}{2}\partial_d^2\right)\int_0^\infty d\lambda\, 
\frac{e^{-2\sqrt{2(\lambda+\frac{\omega_n^2}{2c^2})}d}}{\sqrt{2\lambda+\omega_n^2/c^2}}
\frac{\sqrt{\lambda+\frac{\omega_n^2}{2c^2}(1+\chi)}-\sqrt{\lambda+\frac{\omega_n^2}{2c^2}}e^{2\Xi}}
{\sqrt{\lambda+\frac{\omega_n^2}{2c^2}(1+\chi)}+\sqrt{\lambda+\frac{\omega_n^2}{2c^2}}e^{2\Xi}} 
\end{align}
Now make our usual substitutions: $\lambda = \kappa\omega_n^2/(2c^2)$, and then $p^2 = \kappa+1$.  
\begin{align}
E-E_0%=& -k_BT{\sum_n}'\frac{\alpha(i\omega_n)}{4\pi\epsilon_0}\left(\frac{\omega_n^2}{c^2}  - \frac{1}{2}\partial_d^2\right)\int_0^\infty d\kappa\, \frac{\omega_n^2}{2c^2}\frac{e^{-2\sqrt{\kappa+1}\omega_n d/c}c}{\omega_n\sqrt{\kappa+1}}\frac{\sqrt{\kappa+1+\chi}-\sqrt{\kappa+1}e^{2\Xi}}{\sqrt{\kappa+1+\chi}+\sqrt{\kappa+1}e^{2\Xi}} \\
=& -k_BT{\sum_n}'\frac{\omega_n\alpha(i\omega_n)}{4\pi\epsilon_0c}
\left(\frac{\omega_n^2}{c^2}  - \frac{1}{2}\partial_d^2\right)
\int_1^\infty dp\,e^{-2p\omega_n d/c}\frac{\sqrt{p^2+\chi}-pe^{2\Xi}}{\sqrt{p^2+\chi}+p e^{2\Xi}} 
\end{align}

Take the $\partial_d$ derivatives, get 
\begin{align}
E-E_0%=& -k_BT{\sum_n}'\frac{\omega_n\alpha(i\omega_n)}{4\pi\epsilon_0c}\int_1^\infty dp\,\left(\frac{\omega_n^2}{c^2}  - \frac{2\omega_n^2p^2}{c^2}\right)e^{-2p\omega_n d/c}\frac{\sqrt{p^2+\chi}-pe^{2\Xi}}{\sqrt{p^2+\chi}+p e^{2\Xi}} \\
=& -k_BT{\sum_n}'\frac{\omega^3_n\alpha(i\omega_n)}{4\pi\epsilon_0c^3}\int_1^\infty dp\,
\left(1-2p^2\right)e^{-2p\omega_n d/c}\frac{\sqrt{p^2+\chi}-pe^{2\Xi}}{\sqrt{p^2+\chi}+p e^{2\Xi}} 
\end{align}

\subsubsection{Zero temperature, far-field}
In the far-field of the atom, we have $d\omega_{j0}/c>>1$, so replace $\epsilon,\alpha$ by their d.c. values.  
\begin{align}
E-E_0=& -\frac{\hbar}{2\pi}\int_0^\infty d\omega \frac{\omega^3\alpha_0}{4\pi\epsilon_0c^3}
\int_1^\infty dp\,\left(1-2p^2\right)e^{-2p\omega d/c}\frac{\sqrt{p^2+\chi}-p(1+\chi)}{\sqrt{p^2+\chi}+p(1+\chi)}\\
=& -\frac{3\hbar c\alpha_0}{64\pi^2\epsilon_0d^4}\int_1^\infty dp\,p^{-4}
\left(1-2p^2\right)\frac{\sqrt{p^2+\chi}-p(1+\chi)}{\sqrt{p^2+\chi}+p(1+\chi)},
\end{align}
yet another familiar integral (up to a lingering sign on a reflection coefficient?
 I seem to have currently stumbled onto the correct choice. )  

\subsubsection{Zero temperature, near-field}

In this limit you take $d\rightarrow 0$, but all frequencies contribute, as governed by $\alpha(i\omega)$.
    We have frequency integral determined by $\alpha$ so frequencies $w < w_{j0}$ will dominate.
  Alternatively, just take $p\sim c/(d\omega)$.
  Since $d$ is small, then important $p$ are very large?
  I think this implicitly takes $\omega d/c<<1$?
\begin{align}
E-E_0=& -\frac{\hbar}{8\pi^2\epsilon_0c^3}\int_0^\infty d\omega \omega^3\alpha(i\omega)\int_1^\infty dp\,
\left(1-2p^2\right)e^{-2p\omega d/c}\frac{\sqrt{p^2+\chi}-pe^{2\Xi}}{\sqrt{p^2+\chi}+p e^{2\Xi}} 
\end{align}
The reflection coefficient becomes
\begin{equation}
\frac{\sqrt{p^2+\chi}-p(1+\chi)}{\sqrt{p^2+\chi}+p(1+\chi)} \approx \frac{ p-p(1+\chi)}{p+p(1+\chi)} =
 -\frac{\epsilon(i\omega)-1}{\epsilon(i\omega)+1}.
\end{equation}
Plug this in, and evaluate $p$ integral
\begin{align}
E-E_0\approx& \frac{\hbar}{8\pi^2\epsilon_0c^3}\int_0^\infty d\omega \omega^3
\alpha(i\omega)\frac{\epsilon(i\omega)-1}{\epsilon(i\omega)+1}\int_1^\infty dp\,2p^2e^{-2p\omega d/c}\\
=& \frac{\hbar}{8\pi^2\epsilon_0c^3}\int_0^\infty d\omega \omega^3
\alpha(i\omega)\frac{\epsilon(i\omega)-1}{\epsilon(i\omega)+1}\left(-\frac{c^3e^{-2\omega d/c}(1+\omega d/c)^2}{2 d^3\omega^3}\right)\\
&\approx -\frac{\hbar }{16\pi^2\epsilon_0 d^3}\int_0^\infty d\omega 
\alpha(i\omega)\frac{\epsilon(i\omega)-1}{\epsilon(i\omega)+1}.
\end{align}
which is the correct answer ( an expected result since we were angling for this result by making these limits.
  BUt reassuring to see it emerge nonetheless).  

\subsection{High temperature, far-field(?)}

% We are again working in a far-field limit.
%   At high temperature $\beta\rightarrow 0$, so $\omega_i\rightarrow \infty$.
%   $\omega_n = 2\pi/(\beta\hbar)$.  
% The renormalized energy is 
% \begin{align}
% E-E_0=& -\partial_\beta{\sum_n}'\frac{\omega^3_n\alpha(i\omega_n)}{4\pi\epsilon_0c^3}\int_1^\infty dp\,
% \left(1-2p^2\right)e^{-2p\omega_n d/c}\frac{\sqrt{p^2+\chi}-p(1+\chi)}{\sqrt{p^2+\chi}+p (1+\chi)} 
% \end{align}
% Other work tells us that only $\omega_0$ contributes here.
%   Naively taking $\omega_0=0$, we have \emph{no} $\beta$ dependence anywhere.
%   So that just gives us zero?  Perhaps we have to be careful with the order of operations here 
% - or it is ok to do the $\beta$ derivative right now?
%   We will use $\partial_\beta\omega_n = \partial_\beta[2\pi/(\beta\hbar)] = -2\pi/(\beta^2\hbar) = -k_BT \omega_n$.

% \begin{align}
% E-E_0=& -\partial_\beta{\sum_n}'\frac{\omega^3_n\alpha(i\omega_n)}{4\pi\epsilon_0c^3}\int_1^\infty dp\,\left(1-2p^2\right)e^{-2p\omega_n d/c}\frac{\sqrt{p^2+\chi}-p(1+\chi)}{\sqrt{p^2+\chi}+p (1+\chi)} 
% \end{align}
% \subsubsection{Keeping $\omega_1$}
% So if $\omega_0$ does not contribute, let's try the next term.  Since $\omega_1d/c>>1$, we have only small $p$ contributing.  Since our integral's lower bound is $p=1$, only $p$ close to 1 will contribute.  Let's use $p = 1+s$.    
% \begin{align}
% E-E_0=& -\partial_\beta{\sum_n}'\frac{\omega^3_1\alpha(i\omega_1)}{4\pi\epsilon_0c^3}e^{-2\omega_1 d/c}\int_0^\infty ds\,\left(1-2(1-s)^2\right)e^{-2s\omega_1 d/c}\frac{\sqrt{(1+s)^2+\chi}-(1+s)(1+\chi)}{\sqrt{(1+s)p^2+\chi}+(1+s) (1+\chi)}
% \end{align}
% If we approximate the reflection coefficient at $s=0$ weget
% \begin{align}
% E-E_0=& -\partial_\beta\frac{\omega^3_1\alpha(i\omega_1)}{4\pi\epsilon_0c^3}e^{-2\omega_1 d/c}\int_0^\infty ds\,\left(1-2\right)e^{-2s\omega_1 d/c}\frac{\sqrt{1+\chi}-(1+\chi)}{\sqrt{1+\chi}+ (1+\chi)}\\
% =& -\partial_\beta\frac{\omega^3_1\alpha(i\omega_1)}{4\pi\epsilon_0c^3}\frac{\sqrt{1+\chi}-1 }{\sqrt{1+\chi}+1}e^{-2\omega_1 d/c}\frac{c}{2\omega_1 d}.  
% \end{align}
% Which just decays exponentially with distance (rapidly no less).  Hmm.

% \subsubsection{Retrying with free energy}

Let's try this using the free energy, $F = -\beta^{-1}\log Z$.
  The Casimir energy is starts from 
\begin{equation}
F-F_0=-\beta^{-1}{\sum_n}'\frac{\alpha(i\omega_n)}{2\epsilon_0}\int_0^\infty dT\,
\frac{1}{(2\pi T)^{3/2}}\dlangle \frac{\omega_n^2}{c^2}e^{-T\frac{\omega_n^2}{2c^2}}-\left(\frac{\omega_n^2}{c^2}  
- \frac{1}{2}\partial_x^2\right)e^{-\frac{ T \omega_n^2\langle\epsilon(i\omega_n)\rangle}{2c^2} - T\langle V_{TM}\rangle}\drangle
\end{equation}
If we only keep the term with $\omega_0=0$, we have 
\begin{equation}
F-F_0=-\frac{1}{2}\beta^{-1}\frac{\alpha(0)}{2\epsilon_0}\int_0^\infty dT\,\frac{1}{(2\pi T)^{3/2}}
\dlangle -\left(-\frac{1}{2}\partial_x^2\right)e^{ - T\langle V_{TM}\rangle}\drangle
\end{equation}
Previously, we've done the calculations for the Feynman-Kac formula for just the $TM$ potential.
  Since the Laplace-tranform is trivial, we can do it immediately.  We get 
\begin{align}
\dlangle e^{-\int_0^T dt V_{TM}}\drangle &= 1 + \frac{\sinh(\Xi/2)}{\cosh\Xi}[e^{\Xi/2} + e^{-\Xi/2}]e^{-2 d^2/T}\\
%&= 1 + \frac{(e^{\Xi/2} - e^{-\Xi/2})}{(e^{\Xi/2} + e^{-\Xi/2})}{e^\Xi + e^{-\Xi}}e^{-2 d^2/T}\\
%&= 1 + \frac{e^{\Xi} - e^{-\Xi}}{e^\Xi + e^{-\Xi}}e^{-2 d^2/T}\\
%&= 1 + \frac{e^{2\Xi} - 1}{e^{2\Xi} + 1} e^{-2 d^2/T}\\
&= 1 + \frac{\epsilon(0) - 1}{\epsilon(0)+1}e^{-2 d^2/T}.
\end{align}
Plugging this in, and taking the derivative 
\begin{align}
F-F_0=&-\frac{k_BT\alpha_0}{16\pi\epsilon_0}\frac{\epsilon(0)-1}{\epsilon(0)+1} 
\int_0^\infty dT\,\partial_d^2\frac{1}{\sqrt{2\pi }T^{3/2}} e^{-2 d^2/T}\\
=&-\frac{k_BT\alpha_0}{16\pi\epsilon_0d^3}\frac{\epsilon(0)-1}{\epsilon(0)+1}.
\end{align}
This is the familiar expression for the van der Waals interaction of an atom and a dielectric wall.

 


%%% Local Variables: 
%%% mode: latex
%%% TeX-master: "thesis_master"
%%% End: 
