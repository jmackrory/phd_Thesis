\chapter{Electromagnetic Worldlines - Analytical Results}
\label{ch:analytical}

This chapter will show that the worldline path integrals yield the same results as the simpler 
calculations outlined in Ch.~\ref{ch:introduction}.  
We will show how to extract the Casimir--Polder energy via suitable functional derivatives of the free energy.
The analytical energies can be derived by substituting in the Feynman-Kac formulae for the relevant path integrals.
In order to put the Casimir energies in the right form the Gamma function is used, and the Laplace-Mellin
transform is introduced.  The integrals are then transformed to yield known results in planar geometries.
Finally, we examine the results in the case of non-zero temperature, and show how known results for 
the near-field and high temperature results emerge from this formalism.    
The work presented on the TE polarization has been published in Ref.~\cite{Mackrory2016}.

\section{Extracting Casimir--Polder Energies}

The Casimir--Polder energy for an atom interacting with macroscopic bodies can be derived 
by treating the atom as a perturbation to the permittivity and permeability.  
The atom is located at $\rA$, and has polarizability $\alpha$ and magnetizability $\beta$.
The atom is perturbs the background permittivity $\epsr(\vect{r})\rightarrow \epsr(\vect{r})+\delta\epsr(\vect{r})$,
and permeability $\mur(\vect{r})\rightarrow \epsr(\vect{r})+\delta\mur(\vect{r})$, where
\begin{equation}
  \delta\epsr(\vect{r})=\frac{\alpha}{\epsilon_0}\delta(\vect{r}-\rA), 
  \quad \delta\mu_r(\vect{r})=\mu_0\beta_0\delta(\vect{r}-\rA).
\end{equation}
Note that we will initially consider frequency independent media, at zero temperature.  The generalization 
to nonzero temperature will be considered in Sec.~\ref{sec:nonzero_temp}.

The full Casimir energy for TE and TM polarizations was derived as the free energy in Ch.~\ref{ch:EM_quantization}.
In the zero temperature limit, the energy in the EM field is 
\begin{equation}
    E\subTE-E\sup0 = -\frac{\hbar c}{2}\int_0^\infty\frac{d\cT}{(2\pi\cT)^{D/2}\cT}\int d\vect{x}_0
    \biggdlangle
    \frac{1}{\sqrt{\langle \epsr(z)\mur(z)\rangle}} e^{-\langle V\subTE(z)\rangle\cT}-1
    \biggdrangle_{\vect{x}(t)}.\label{eq:TE_energy}
  \end{equation}
The Casimir--Polder energy comes from expanding the energy to linear order in the $\alpha$ and $\beta_0$.
Note that the $\delta$-functions are being used a short-handed for a well localized function with unit integral.
All expansions involving these $\delta$-functions can be carried out with a finite regularization of the 
$\delta$-function, and the arbitrarily sharp limit can be taken after the expansions.  
%This expansion corresponds to extracting the lowest order linear response.
The expansion must be carried out in both $\langle\epsr\mur\rangle$, and the potential $V\subTE$.
Considering the similarities between the polarizations, we will carry these expansions out for only
one polarization, since the others follow by duality.  

% It is important be clear about exactly what the energy is, and how it is computed.  
% As discussed in \S 4.7 of Jackson~\cite{Jackson1998}, the electrostatic energy for a system depends on the 
% manner in which the system was arranged.  Different energies are found if the free charge or potential
% are specified.  The same is true of magnetostatic systems: are the potentials on the boundaries, or the free 
% currents specified.  

The energy for an atom in the electromagnetic field can be found by considering the change in the energy 
from adding the perturbation. 
The energy can be written as a functional\footnote{A functional is a function that maps another function to a number.
For example, the classical action $S[x(t)]=\int dt L(x,\dot{x})$ maps the trajectory of a classical path $x(t)$ to a single number,
 via the time integral of the Lagrangian.
The Euler-Lagrange equations of motion come from taking the functional derivative of the action with respect to small path variations.} 
of the permittivity $\epsr$, and permeability $\mur$,
 then this energy difference is 
\begin{equation}
  \delta E[\epsr,\mur] = E[\epsr+\delta\epsr,\mur+\delta\mur]-E[\epsr,\mur] 
\end{equation}
Since the atom is assumed to have a small localized effect on the permittity and permeability, this expansion corresponds to a functional derivative.  
The functional derivative for a functional
 $F[f(x)]$ is formally defined as 
\begin{equation}
  \frac{\delta F}{\delta f(x')} := \lim_{a\rightarrow 0} \frac{F[f(x)+a\delta(x-x')]-F[f(x)]}{a}.
\end{equation}
As a practical matter, the functional derivative $\delta f(t)/\delta f(t') = \delta(t-t')$ generalizes
 the notation that $\delta x_i/\delta x_j = \delta_{ij}$ from ordinary multivariate calculus.  Otherwise,
the expansions can be carried out with $\alpha_0$ and $\beta_0$ acting as the small parameters.
The Casimir--Polder energy can be found by taking the following functional derivatives
\begin{equation}
  V\subCP(\rA) = \frac{\alpha_0}{\epsilon_0}\frac{\delta}{\delta\epsr(\rA)}E+\mu_0\beta_0\frac{\delta}{\delta\mur(\rA)}E.
\end{equation}

The Casimir--Polder energy must be renormalized by considering the change in the energy as the atom is removed 
to $\infty$.

It is necessary to expand the path-averaged permittivity and permeability,
\begin{align}
  \langle(\epsr+\delta\epsr)(\mur+\delta\mur)\rangle^{-1/2} &= \langle\epsr\mur\rangle^{-1/2}
  -\frac{1}{2}\langle \mur\delta\epsr+\epsr\delta\mur\rangle\langle\epsr\mur\rangle^{-3/2}\\
&= \langle\epsr\mur\rangle^{-1/2}
-\frac{1}{2}\frac{\alpha_0}{\epsilon_0}\langle \mur(\vect{x})\delta(x-\rA)\rangle\langle\epsr\mur\rangle^{-3/2}\nonumber\\
&\hspace{1cm} -\frac{1}{2}\mu_0\beta_0\langle\epsr(\vect{x})\delta(\vect{x}-\rA)\rangle \langle\epsr\mur\rangle^{-3/2}
\label{eq:mueps_expansion}
\end{align}
The path-averaged $\delta$-functions act to restrict the path integrals to paths starting at the atom's
position $\rA$.

  Consider a path integral over closed paths where the integrand depends on the whole path and includes 
 a path-averaged $\delta$-function,
\begin{equation}
  I = \int d\vect{x}_0\Bigdlangle \,\langle f\rangle\langle\delta\rangle\,\Bigdrangle_{\vect{x}(t)}.
\end{equation}
In the discretized notation this is 
\begin{equation}
  I = \int \prod_{n=0}^{N-1}dx_n\,P(x_0,\cdots, x_{N-1}) \frac{1}{N}\sum_{k=0}^{N-1}f(x_k)\frac{1}{N}\sum_{j=0}^{N-1}\delta(\vect{x}_j-\rA)
\end{equation}
where the second $\delta$-function enforces path closure.  All of the functions are invariant under cyclic permutations 
of the path labels.  This is true of the path-averaged functions, 
and the Gaussian probability $P(x_0,\cdots x_{N-1})$, the exponent of which 
can be written as  $\sum_{j=0}^{N-1}(x_{j+1}-x_j)^2$, subject to the condition that $x_N=x_0$.
Then for each term $\delta(\vect{x}_j-\rA)$, the labels can be shifted $j$ times so that in the shifted
coordinates $\vect{x}_j\rightarrow \vect{x}_0$.  Since there is now a sum of $N$ identical terms, the 
path integral can be written as
\begin{equation}
  I = \int \prod_{n=0}^{N-1}dx_n\,P(\vect{x}_0,\cdots, \vect{x}_{N-1}) \frac{1}{N}\sum_{k=0}^{N-1}f(\vect{x}_k)\delta(\vect{x}_0-\rA)
= \Bigdlangle\, \langle f\rangle\,\Bigdrangle_{\vect{x}(t), \vect{x}(0)=\rA}.
\end{equation}
For a closed path, integrated over all space, with integrands athat are all written as averaged around the path, 
there is some freedom in which point of the path is called the origin.  
Since only paths the satisfy the $\delta$-function will contribute to the path integral, we are free to call the point
at $\rA$ the path origin.  
The end result is the path-averaged $\delta$-function restricts the starting point of the paths to the atom's
position $\rA$.
\label{sec:path_average_delta}

The singular potentials $V\subTE, V\subTM$ can be expanded in the same fashion.  The expansion is carried out to linear order 
in $\delta\epsr,\delta\mur$, and all higher order terms are dropped.
\begin{align}
  \langle V\subTE[\mur +\delta\mur] \rangle 
  =& \frac{1}{2}\langle (\nabla\log\sqrt{\mur+\delta\mur})^2-\nabla^2\log\sqrt{\mur+\delta\mur}\rangle\nonumber\\
  =& \frac{1}{8} \langle [\nabla\log(\mur+\delta\mur)]^2\rangle
  -\frac{1}{4}\langle \nabla^2\log(\mur+\delta\mur)\rangle \nonumber\\
  =& \langle V\subTE[\mur]\rangle+\left< \frac{1}{4} \nabla\log\mur\cdot\nabla\frac{\delta\mur}{\mur}
    -\frac{1}{4}\nabla^2\frac{\delta\mur}{\mur}\right> 
  \label{eq:VTE_expansion}
\end{align}
The expansion to linear order in $\delta\mur $involving the exponential $e^{-\cT\langle V\subTE\rangle}$ is straightforward.  
The terms involving derivatives $\nabla \delta\mur$ only make sense after integration by parts, 
\begin{equation}
  \int d\vect{x} f(\vect{x})\nabla\delta(\vect{x}-\rA)  = -\nabla f(\vect{x})\bigg|_{\vect{x}=\rA}.
\end{equation}
The same manipulations for the path-averaged $\delta$-function also apply to these versions involving derivatives.  

Using the results in Eqs.~(\ref{eq:mueps_expansion}) and (\ref{eq:VTE_expansion}), the Casimir--Polder energy
for the TE polarization can be written as
\begin{align}
    V\supTE\subCP(\vect{\rA}) &= -\frac{\hbar c}{2}\int_0^\infty\frac{d\cT}{(2\pi\cT)^{D/2}\cT}\int d\vect{x}_0
    \biggdlangle
    \left( - \frac{\langle\mur\delta\epsr+\epsr\delta\mur\rangle}
    {2\langle \epsr\mur\rangle^{3/2}}\right) 
  e^{-\langle V\subTE\rangle\cT} \nonumber\\
  &\hspace{1cm}+ e^{-\langle V\subTE\rangle\cT}\left(-\frac{\cT}{\langle\epsr\mur\rangle^{1/2}}
    \left< \frac{1}{4} \nabla(\log\mur)\cdot\nabla\frac{\delta\mur}{\mur}
      -\frac{1}{4}\nabla^2\frac{\delta\mur}{\mur}\right> \right)
    \biggdrangle_{\vect{x}(t)}.
\end{align}
Then after manipulating the path-averaged $\delta$-functions, and renormalizing the energy against the 
case when the atom is far from the bodies, the Casimir--Polder energy is
\begin{align}
    V\supTE\subCP(\vect{\rA}) &= -\frac{\hbar c}{2}\int_0^\infty\frac{d\cT}{(2\pi\cT)^{D/2}\cT}
    \biggdlangle
    \left( - \frac{\alpha_0\mur(\rA)}{2\epsilon_0\langle \epsr\mur\rangle^{3/2}}
      -\frac{\beta_0\mu_0\epsr(\rA)}{2\langle \epsr\mur\rangle^{3/2}}\right) e^{-\langle V\subTE\rangle\cT} \nonumber\\
    &\hspace{1cm}+\frac{\cT}{4}\frac{\beta_0\mu_0}{\mur(\rA)}\left[
     \nabla^2      +\nabla^2(\log\mur)+ \nabla(\log\mur)\cdot\nabla\right]
    \frac{ e^{-\langle V\subTE\rangle\cT}}{\langle\epsr\mur\rangle^{1/2}}
    \biggdrangle_{\vect{x}(t),\vect{x}(0)=\rA}.
\end{align}
Note that bracketed gradients such as $\nabla(\log\mur)$ should be interpreted as functions, while the 
other derivative operators act on everything to their right.  The remaining derivatives act with respect to 
the path origin $\vect{x}_0=\rA$.
The corresponding TM Casimir--Polder energy can be found under the duality transformation, and is given by 
\begin{align}
    V\supTM\subCP(\vect{\rA}) &= -\frac{\hbar c}{2}\int_0^\infty\frac{d\cT}{(2\pi\cT)^{D/2}\cT}
    \biggdlangle
    \left( - \frac{\alpha_0\mur(\rA)}{2\epsilon_0\langle \epsr\mur\rangle^{3/2}}
      -\frac{\beta_0\mu_0\epsr(\rA)}{2\langle \epsr\mur\rangle^{3/2}}\right) e^{-\langle V\subTM\rangle\cT} \nonumber\\
    &\hspace{1cm}+\frac{\cT}{4}\frac{\alpha_0}{\epsilon_0\epsr(\rA)}\left[
     \nabla^2      +\nabla^2(\log\epsr) + \nabla(\log\epsr)\cdot\nabla\right]
    \frac{ e^{-\langle V\subTM\rangle\cT}}{\langle\epsr\mur\rangle^{1/2}}
    \biggdrangle_{\vect{x}(t),\vect{x}(0)=\rA}.
\end{align}

These expressions can be further simplified if the atom is in a region where the dielectric is not varying spatially,
and we consider non-magnetic atoms and media where $\beta_0=0$ and $\mur=1$.  
If the permittivity is spatially constant at the atom's location then $\nabla\log\sqrt\epsr(\rA)=0$.
In this case, the TE and TM Casimir--Polder energies are given by 
\begin{align}
    V\supTE\subCP(\vect{\rA}) &= \frac{\hbar c\alpha_0}{4\epsilon_0(2\pi)^{D/2}}\int_0^\infty\frac{d\cT}{\cT^{1+D/2}}
    \biggdlangle
      \frac{1}{\langle \epsr\rangle^{3/2}}
    %   \right) \nonumber\\
    % &\hspace{1cm}
      \biggdrangle_{\vect{x}(t),\vect{x}(0)=\rA}\\
    V\supTM\subCP(\vect{\rA}) &= \frac{\hbar c\alpha_0}{4\epsilon_0(2\pi)^{D/2}}\int_0^\infty\frac{d\cT}{\cT^{1+D/2}}
    \biggdlangle
      \frac{e^{-\langle V\subTE\rangle\cT}}{\langle \epsr\rangle^{3/2}}
    %   \right) \nonumber\\
    % &\hspace{1cm}
      -\frac{\cT}{2\epsr(\rA)} \nabla^2 \frac{ e^{-\langle V\subTE\rangle\cT}}{\langle\epsr\rangle^{1/2}}
      \biggdrangle_{\vect{x}(t),\vect{x}(0)=\rA}.
\end{align}
The TE Casimir--Polder energy will thus always be the simpler case to evaluate since it only depends on $\langle\epsr\mur\rangle$.
In contrast, the TM Casimir--Polder energy also requires the singular TM potential which must be regularized
using the results in Sec.~\ref{sec:TM_potential}.  The TM Casimir--Polder energy also requires spatial derivatives,
which will require some care in numerical methods involving stochastic paths against singular potentials.

In this limit the worldline Casimir energy for both polarizations are
\begin{align}
    E\subTE-E\sup0 &= -\frac{\hbar c}{2(2\pi)^{D/2}}\int_0^\infty\frac{d\cT}{\cT^{1+D/2}}\int d\vect{x}_0
    \biggdlangle
    \frac{1}{\sqrt{\langle \epsr\rangle}}-1    \biggdrangle_{\vect{x}(t)}\label{eq:log_Z_TE}\\
    E\subTM-E\sup0 &= -\frac{\hbar c}{2(2\pi)^{D/2}}\int_0^\infty\frac{d\cT}{\cT^{1+D/2}}\int d\vect{x}_0
    \biggdlangle
    \frac{1}{\sqrt{\langle \epsr(z)\rangle}} e^{-\langle V\subTM(z)\rangle\cT}-1
    \biggdrangle_{\vect{x}(t)}.\label{eq:log_Z_TM}
\end{align}

From now on we will work within the context of non-magnetic media with $\mur=1, \beta_0=0$.
The essential ideas will have to be developed anyway for both polarizations.

% \subsection{Functional derivative of $\log Z_{TM}$}
% We then get
% \begin{align}
% \log Z =&\int_0^\infty \frac{dT}{8\pi^2T^3}\int dx_0\bigdlangle\langle\epsilon\rangle^{-1/2}e^{- T\langle V\rangle}\bigdrangle\\
% % =& \int_0^\infty \frac{dT}{8\pi^2T^3}\int dx_0\biggdlangle \left(\langle\epsilon\rangle^{-1/2}
% % -\frac{\alpha\langle\delta\rangle}{2\langle\epsilon\rangle^{3/2}}\right)
% % \left( 1 + \frac{\alpha T}{4}\langle\nabla^2(\epsilon^{-1}\delta)
% % -\nabla\log(\epsilon)\cdot\nabla\epsilon^{-1}\delta\rangle\right)e^{-T\langle V\rangle}\biggdrangle\\
% =& \int_0^\infty \frac{dT}{8\pi^2T^3}\int dx_0\biggdlangle
% \left(\langle\epsilon\rangle^{-1/2}-\frac{\alpha\langle\delta\rangle}{2\langle\epsilon\rangle^{3/2}}  
% + \frac{\alpha T}{4}\langle\epsilon\rangle^{-1/2}\langle \nabla^2(\epsilon^{-1}\delta)-
% \nabla\log(\epsilon)\cdot\nabla\epsilon^{-1}\delta\rangle\right)e^{-T\langle V\rangle}\biggdrangle
% \end{align}

% % Now consider $\langle \delta\,f\rangle $
% % \begin{align}
% % \int dx_0\int Dx\langle \delta(x)f(x)\rangle P(x) 
% % =&  \int dx_0\prod_{j=1}^{N}dx_n\delta(x_N-x_0)\prod_k P(x_k) \frac{1}{N}\sum_j \delta(x_j-x')f(x_j) \nonumber\\
% % = &f(x')\int \prod_{j=1}^N\delta(x_N-x') dx_jP(x_j) = f(x')\int Dx P(x)_{x_0=x(T)=x'}
% % \end{align}
% % This is a trace, which is invariant under cylic permutations.
% %   We can cyclically permute all the loop labels around so that we call the point that passes through $x'$ the loop origin.
% %   Doing this for every term, means we get $N$ copies of the same integrals.  We can then call $x_0=x'$, and drop that sum over $j$.   
% The same logic works for derivatives, and $\delta'$.
%   So we can find the linear term in $\alpha$.
%   This is effectively the functional derivative with respect to $\epsilon$.  
% \begin{align}
% \frac{\delta}{\delta\epsilon(x')}\log Z=& \int_0^\infty \frac{dT}{16\pi^2T^3}\int dx_0
% \biggdlangle\bigg[ -\frac{\delta(x_0-x')}{\langle\epsilon\rangle^{3/2}}\nonumber\\
% &  +  \frac{T}{2}\langle\epsilon\rangle^{-1/2}\left(\nabla^2[\epsilon^{-1}(x_0)\delta(x_0-x')]
% -\nabla\log(\epsilon)\cdot\nabla\epsilon^{-1}\delta(x_0-x')\rangle\right)\bigg]e^{-T\langle V\rangle}\biggdrangle
% \end{align}
% Now integrate by parts on the delta functions, and we get 
% \begin{align}
% \frac{\delta}{\delta\epsilon(x')}\log \ZTM=& \int_0^\infty \frac{dT}{16\pi^2T^3}\int dx_0\delta(x_0-x')
% \biggdlangle-\frac{e^{T\langle V\rangle}}{\langle\epsilon\rangle^{3/2}}
%   +  \frac{T}{2}\epsilon^{-1}(x_0)\nabla'^2\left(\langle\epsilon\rangle^{-1/2} e^{T\langle V\rangle}\right)\nonumber\\
% & \hspace{3cm}+\epsilon^{-1}\frac{T}{2}\nabla'\cdot 
% \frac{\nabla'\epsilon(x')}{\epsilon(x')}\langle\epsilon\rangle^{-1/2}e^{-T\langle V\rangle}\biggdrangle
% \end{align}
% In cases with piece wise constant media, we can drop the final term, saying $\nabla'\epsilon(x')\approx 0$.
%   I will then use $\int dx \delta'(x-y)f(x) = \partial_y f(y) = \partial_y\int dx f(x)$ to pull the derivatives out.  
% \begin{align}
% \Aboxed{\frac{\delta}{\delta\epsilon(x')}\log \ZTM=& \frac{1}{16\pi^2}\int_0^\infty dT
% \left[ \frac{1}{2}\epsilon^{-1}(x_0)\nabla'^2\biggdlangle
% \frac{ e^{-\int dt V}}{\langle\epsilon\rangle^{1/2}T^2}\biggdrangle
% -\biggdlangle\frac{e^{-\int dt V}}{\langle\epsilon\rangle^{3/2}T^3} \biggdrangle \right]}
% \end{align}

\section{Rearranging Worldline Casimir Energies}

The worldline energies need some rearrangement in order to use the analytical results for 
path integrals derived in Ch.~\ref{ch:feynman_kac}.  This can be done with couple integral identities.
The first converts the worldline path integral into the form where the Laplace transformed path
integral appears.  The second exponentiates $\langle\epsr\mur\rangle$ by means of the Gamma function.  

\subsection{ Laplace-Mellin Transforms}

The worldline path integral has the form of a Mellin transform.  The Mellin transform of a function $f$ is 
defined as 
\begin{equation}
\mathcal{M}[f](z)= \int_0^\infty dt\, t^{z-1}f(t).
\end{equation}
The result of the Mellin transform is a function $\tilde{f}(z)$.
In the worldline energy $f$ is the ensemble-average path integral, while $z=1+D/2$.
The Mellin transform appears in the context of $\zeta$-function renormalization for functional determinants,
and considering its connection to the $\zeta$-function plays a role in formal number theory.  However,
for our purposes it is just an integral transform.  

There is a useful relationship between Laplace transforms and Mellin transforms referred to as the Laplace-Mellin theorem~\cite{Lew1975}.  
The Laplace transform (that we've been implicitly using) of a function $f(t)$ is defined as
\begin{equation}
\mathcal{L}[f](s) = \int_0^\infty dt e^{-st} f(t).
\end{equation}
We will also need the $\Gamma$ function which is defined as  
\begin{equation}
\Gamma(z) = \int_0^\infty ds\, s^{z-1} e^{-s} = \mathcal{M}[e^{-s}](z),
\end{equation}
where the second equality writes the Gamma function as the Mellin transform of the exponential.  
The Laplace-Mellin theorem~\cite{Lew1975} says
\begin{equation}
  \Gamma(1-z)\mathcal{M}[f](z) = \mathcal{M}\big[\mathcal{L}[f]\big](1-z)\label{eq:Laplace-Mellin}
\end{equation}
This is most easily motivated by starting with the right hand side
\begin{equation}
\mathcal{M}\big[\mathcal{L}[f]\big](1-z) = 
\int_0^\infty ds\, s^{-z} \int_0^\infty dt\,e^{-st} f(t).
\end{equation}
We assume that the order of $s$ and $t$ integration can be swapped, and change integration variable to $s=t/u$,
% =& \int_0^\infty dt\,\left[\int_0^\infty ds s^{-z} e^{-st}\right] f(t)\\
% =& \int_0^\infty dt\,\left[\int_0^\infty d\frac{u}{t}\, t^zu^{-z} e^{-u}\right] f(t) \\
\begin{align}
\mathcal{M}\big[\mathcal{L}[f]\big](1-z)=&\int_0^\infty dt\,\int_0^\infty du\, u^{-z} e^{-u}\,t^{z-1} f(t) \\
=& \Gamma(1-z)\mathcal{M}[f](z)
\end{align}
In words, the Mellin transform of a function is proportional to the Mellin transform of the functions
Laplace transform, subject to $z\rightarrow 1-z$ and dividing by a constant $\Gamma(1-z)$.  This is 
directly useful to rewriting worldline path integrals in terms of their Laplace transforms, especially
since the solution method in Ch.~\ref{ch:feynman_kac} naturally yields the Laplace transform of the function.

\subsection{Inverse Moment Theorem}

One further step is required to put all of the pieces involving material functions in the path integrals into exponential form.
Again, the solutions derived in the previous chapter assumed path integrals with exponential potentials.
So the terms in $\langle\epsr\mur\rangle$ should be exponentiated.  If we required positive powers,
then the usual moment generating tricks could be used such as $\langle x\rangle^n = \frac{d^n}{ds^n}e^{-s\langle x\rangle}\big|_{s=0}$.
However, for the inverse-moments required in the worldline method, the Gamma function must be used 
\begin{equation}
\frac{1}{\Gamma[\alpha]}\int_0^\infty ds\,s^{\alpha-1}\dlangle e^{-s(x+\beta)}\drangle  
= \dlangle \frac{1}{(x+\beta)^\alpha}\drangle\label{eq:moment_theorem}.
\end{equation}
This is restricted to $x+\beta>0$, and $\alpha>0$.  In the worldline calculations $\alpha=1/2$ or $\alpha=3/2$
for Casimir and Casimir--Polder energies respectively.  On the imaginary frequency axis, the response functions
are real, positive, decaying functions, so all of these conditions are satisfied for worldline path integrals.

% We can do this via a straightforward approach: 
% \begin{align}
% \frac{1}{\Gamma[\alpha]}\int_0^\infty ds\,s^{\alpha-1}\dlangle e^{-s(x+\beta)}\drangle 
% % =&\frac{1}{\Gamma[\alpha]}\int_0^\infty ds\,s^{\alpha-1}\int dx f(x) e^{-s(x+\beta)}\\
% % =& \frac{1}{\Gamma[\alpha]}\int dx f(x) \int_0^\infty ds\,s^{\alpha-1} e^{-s(x+\beta)} \\
% % =&\frac{1}{\Gamma[\alpha]}\int dx f(x) \frac{1}{(x+\beta)^\alpha}\int_0^\infty dt\,t^{\alpha-1} e^{-t} \\
% =&\int dx f(x) \frac{1}{(x+\beta)^\alpha} \\
% =& \dlangle \frac{1}{(x+\beta)^\alpha}\drangle
% \end{align}

\subsection{Rearranging the Worldline}

Consider the TE path integral, with dielectric function $\epsr(\vect{x})=1+\chi(\vect{x})$, where $\chi$
is a positive function of position.  In both the Casimir and Casimir--Polder cases, the energy involves 
$\langle \epsr\rangle^{-\alpha}$, with $\alpha=1/2$, and $\alpha=3/2$ respectively.  
Using the moment theorem.~(\ref{eq:moment_theorem}),  we can write the energy density 
\begin{align}
\int_0^\infty \frac{d\cT}{\cT^{1+D/2}}\biggdlangle\frac{1}{\langle 1+\chi(\vect{x})\rangle^\alpha} \biggdrangle_{\vect{x}(t)}
% = &\int_0^\infty \frac{dT}{T^{1+D/2}}\frac{1}{\Gamma[\alpha]}\int_0^\infty ds s^{\alpha-1} 
% \dlangle e^{-s(\chi \int_0^T dt \Theta(x-d) +1)}\drangle \\
% =&\frac{1}{\Gamma[\alpha]}\int_0^\infty \frac{dT}{T^{1+D/2-\alpha}}\int_0^\infty ds s^{\alpha-1} e^{-s T}
% \dlangle e^{-s \chi \int_0^T dt \Theta(x-d)}\drangle \\
=&\int_0^\infty ds\, \frac{s^{\alpha-1}}{\Gamma(\alpha)}\int_0^\infty \frac{d\cT}{\cT^{1+D/2-\alpha}}
\dlangle e^{-s\cT- \int_0^\cT dt \,\chi[\vect{x}(t)]}\drangle.
\end{align}
In the second equality the integration variable was rescaled $s\rightarrow s\cT$,
 and the definition of the path average, $\langle f\rangle = \cT^{-1}\int_0^\cT dt\, f(t)$ was used.
% where we used the Inverse-moment theorem, rescaled the $\lambda\rightarrow \lambda T$,
%  and swapped the order of integration. We can see that that $T$ integral has the form of a Mellin transform.
Using the Laplace-Mellin transform~(\ref{eq:Laplace-Mellin}), the energy density can be written as
\begin{align}
&\int_0^\infty \frac{d\cT}{\cT^{1+D/2-\alpha}}e^{-s\cT}\dlangle e^{-s\int_0^\cT dt\,\chi[\vect{x}(t)]}\drangle \nonumber\\
% =& \int_0^\infty \frac{dT}{T^{1+z-1/2}}e^{-sT}\dlangle \frac{e^{-s \chi \int dt_0^T dt \Theta(x-d)}}{\sqrt{T}}\drangle\\
% =&\mathcal{M}\left[e^{-sT}\dlangle \frac{e^{-s \chi \int_0^T dt \Theta(x-d)}}{\sqrt{T}}\drangle\right]\left(-z+1/2\right) \\
% =& \frac{1}{\Gamma[1+z-1/2]}\mathcal{M}\left[\int_0^\infty dT e^{-(\lambda+s)T}
% \dlangle \frac{e^{-s \chi \int_0^T dt \Theta(x-d)}}{\sqrt{T}}\drangle\right]\left(-z+1/2\right) \\
&\hspace{1cm}= \int_0^\infty d\lambda\, \frac{\lambda^{(D-1)/2-\alpha}}{\Gamma[D/2-\alpha+1/2]}\int_0^\infty d\cT e^{-(\lambda+s)\cT}
\dlangle \frac{e^{-s \int_0^\cT dt\,\chi(\vect{x})}}{\sqrt{\cT}}\drangle.\label{eq:Casimir_Laplace_inverse}
\end{align}
The integral over $\cT$ is the solution to the relevant diffusion equation, as discussed in Ch.~\ref{ch:feynman_kac}.
The solutions must be appropriately scaled with $\lambda\rightarrow \lambda+s$, and $\chi\rightarrow s\chi$.

\section{Analytical  TE CP energy for an atom and a dielectric plane}

The TE contribution to Casimir--Polder energy for an atom above a half-plane at $x=d$ is then given 
combining these formal manipulations with the relevant path integral solution.  For an atom at the origin
interacting with a planar dielectric $\epsr(z)=1+\chi\Theta(x-d)$, the path integral solution is given by
 Eq.~(\ref{eq:Feynman-Kac TE one step}).
Under the rescalings $s\rightarrow s\chi, \lambda\rightarrow \lambda+s$, the renormalized TE Casimir--Polder potential is
\begin{align}
V\subCP\supTE=&-\frac{\hbar c\alpha_0}{4\epsilon_0(2\pi)^{D/2}}\frac{\sqrt{\pi}}{\Gamma[\alpha]\Gamma\left[(D+1)/2-\alpha\right]}
\int_0^\infty ds\, s^{\alpha-1}\int_0^\infty d\lambda\, \lambda^{(D-1)/2-\alpha}\nonumber\\
&\times\frac{e^{-2\sqrt{2(\lambda+s)}|d|}}{\sqrt{\lambda+s}} 
\frac{\sqrt{\lambda+s(1+\chi)}-\sqrt{\lambda+s}}{\sqrt{\lambda+s(1+\chi)}+\sqrt{\lambda+s}},
\end{align}
This can be put into the same form as the known results by changing integration variables.
Similar steps will be required to convert the other planar energies, so we will go through this once in detail.
We will confine our attention to the integral
\begin{equation}
  J=\int_0^\infty ds\, s^{\alpha-1}\int_0^\infty d\lambda\, \lambda^{(D-1)/2-\alpha}\frac{e^{-2\sqrt{2(\lambda+s)}|d|}}{\sqrt{\lambda+s}} 
\frac{\sqrt{\lambda+s(1+\chi)}-\sqrt{\lambda+s}}{\sqrt{\lambda+s(1+\chi)}+\sqrt{\lambda+s}},
\end{equation}
First change variable from $\lambda$ to $v:=\sqrt{\lambda/s+1}$, 
\begin{equation}
  J% =2\int_0^\infty ds\, s^{\alpha-1}s^{(D-1)/2-\alpha+1-1/2}\int_1^\infty dv\, (v^2-1)^{(D-1)/2-\alpha}e^{-\sqrt{8 d^2s}v}
  % \frac{\sqrt{v^2+\chi}-v}{\sqrt{v^2+\chi}+v},
  =2\int_0^\infty ds\, s^{D/2-1}\int_1^\infty dv\, (v^2-1)^{(D-1)/2-\alpha}e^{-\sqrt{8 d^2s}v}
  \frac{\sqrt{v^2+\chi}-v}{\sqrt{v^2+\chi}+v},
\end{equation}
Next, change variable from $s$ to $t=\sqrt{8d^2 s}v$, and swap the $t$ and $v$ integrals. 
\begin{align}
  J % =4\int_1^\infty dv\, (v^2-1)^{(D-1)/2-\alpha} \int_0^\infty dt\, \frac{2t}{8d^2v^2}\left(\frac{t^2}{8d^2v^2}\right)^{D/2-1}e^{-t}
  % \frac{\sqrt{v^2+\chi}-v}{\sqrt{v^2+\chi}+v}\\
=\frac{1}{8^{D/2-1}d^D}\int_1^\infty dv\,v^{-D} (v^2-1)^{(D-1)/2-\alpha} 
  \frac{\sqrt{v^2+\chi}-v}{\sqrt{v^2+\chi}+v}\int_0^\infty dt\, t^{D-1}e^{-t}
\end{align}
The $t$-integral can then be identified as a $\Gamma[D-1]$.  
% \begin{align}
% V_D(\chi,d)%=&-\frac{\sqrt{\pi}}{\Gamma[\alpha]\Gamma\left[(D+1)/2-\alpha\right]}(8 d^2)^{\alpha-\alpha-(D+1)/2-1/2}\nonumber \\
% %&\times\int_0^\infty ds s^{\alpha-1}\int_0^\infty d\lambda \lambda^{(D-1)/2-\alpha}\frac{e^{-\sqrt{(\lambda+s)}}}{\sqrt{\lambda+s}} \frac{\sqrt{\lambda+s(1+\chi)}-\sqrt{\lambda+s}}{\sqrt{\lambda+s(1+\chi)}+\sqrt{\lambda+s}}\\
% =&\frac{\sqrt{\pi}}{2^{3D/2}\Gamma[\alpha]\Gamma\left[(D+1)/2-\alpha\right]d^D}\nonumber \\
% &\times(-1)\int_0^\infty ds s^{\alpha-1}\int_0^\infty d\lambda \lambda^{(D-1)/2-\alpha}
% \frac{e^{-\sqrt{(\lambda+s)}}}{\sqrt{\lambda+s}} 
% \frac{\sqrt{\lambda+s(1+\chi)}-\sqrt{\lambda+s}}{\sqrt{\lambda+s(1+\chi)}+\sqrt{\lambda+s}},
% \end{align}
% Secondly we will substitute $\lambda +s = u^2$,
% \begin{align}
% J_{\alpha,D} % =& (-1)\int_0^\infty ds s^{\alpha-1}\int_{\sqrt{s}}^\infty du\,(2u) (u^2-s)^{(D-1)/2-\alpha}\frac{e^{-u}}{u} \frac{\sqrt{u^2+s\chi}-u}{\sqrt{u^2+s\chi  }+u}\\
% =& 2\int_0^\infty ds s^{\alpha-1}\int_{\sqrt{s}}^\infty du\, (u^2-s)^{(D-1)/2-\alpha}e^{-u} 
% \frac{u-\sqrt{u^2+s\chi}}{u+\sqrt{u^2+s\chi  }}.
% \end{align}
% Now take $u = v\sqrt{s}$.  
% \begin{align}
% J_{\alpha,D} % =& 2\int_0^\infty ds s^{\alpha-1}\int_{1}^\infty du\, \sqrt{s}(v^2s-s)^{(D-1)/2-\alpha}e^{-\sqrt{s}v}\frac{v-\sqrt{v^2+\chi}}{v+\sqrt{v^2+\chi  }}\\
% % =& 2\int_{1}^\infty du\,(v^2-1)^{(D-1)/2-\alpha}\frac{v-\sqrt{v^2+\chi}}{v+\sqrt{v^2+\chi  }}\int_0^\infty ds s^{\alpha-1+1/2+(D-1)/2-\alpha}e^{-\sqrt{s}v}\\
% =& 2\int_{1}^\infty du\,(v^2-1)^{(D-1)/2-\alpha}\frac{v-\sqrt{v^2+\chi}}{v+\sqrt{v^2+\chi  }}
% \int_0^\infty ds s^{D/2-1}e^{-\sqrt{s}v}
% \end{align}
% Lastly take $t= v\sqrt{s}$.  or $s = v^{-2}t^2$, with $ds = 2v^{-2}t dt$.  
\begin{align}
J % =& 2\int_{1}^\infty dv\,(v^2-1)^{(D-1)/2-\alpha}\frac{v-\sqrt{v^2+\chi}}{v+\sqrt{v^2+\chi  }}\int_0^\infty  dt (2v^{-2}t)\left(v^{-2}t^{2}\right)^{D/2-1}e^{-t}\\
% =& 4\int_{1}^\infty dv\,(v^2-1)^{(D-1)/2-\alpha}\frac{v-\sqrt{v^2+\chi}}{v+\sqrt{v^2+\chi  }}\int_0^\infty  dt v^{-D} t^{D-1}e^{-t}\\
=& 4\Gamma[D]\int_{1}^\infty dv\,v^{-D}(v^2-1)^{(D-1)/2-\alpha}\frac{v-\sqrt{v^2+\chi}}{v+\sqrt{v^2+\chi  }}
\end{align}

So we have:
\begin{align}
V_D(\chi,d)=\frac{\sqrt{\pi}\Gamma[D]}{2^{3D/2-2}\Gamma[\alpha]\Gamma\left[(D+1)/2-\alpha\right]d^D}
\int_{1}^\infty dv\,v^{-D}(v^2-1)^{(D-1)/2-\alpha}\frac{v-\sqrt{v^2+\chi}}{v+\sqrt{v^2+\chi  }}\nonumber \\
\end{align}

Let's evaluate this for the case we care about: $D=4,\alpha=3/2$, 
for the energy of an atom above a dielectric half-space.
\begin{align}
V_D(\chi,d)=&\frac{\sqrt{\pi}\Gamma[4]}{2^{4}\Gamma[3/2]d^4}\int_{1}^\infty dv\,v^{-4}\frac{v-\sqrt{v^2+\chi}}{v+\sqrt{v^2+\chi  }}\nonumber \\
% =&\frac{3}{48d^4\chi^{3/2}}\left\{4 \chi^{3/2}+24\chi^{1/2}-12 \sqrt{\chi  (\chi +1)}-3 \log \left[2 \chi +2 \sqrt{\chi  (\chi+1)}+1\right]-6 \text{arcsinh}\left(\sqrt{\chi }\right)\right\}\\
=&\frac{3}{4d^4}\left\{ \frac{1}{3}+2\chi^{-1}- \chi^{-3/2}\sqrt{\chi  (\chi +1)} 
-\frac{\log \left[2 \chi +2 \sqrt{\chi  (\chi+1)}+1\right]}{4 \chi^{3/2}}
-\frac{\text{arcsinh}\left(\sqrt{\chi }\right)}{2\chi^{3/2}}\right\},
\end{align}

% \subsection{Sinh-log}
% Comparison with Dan's expression suggests that 
% \begin{equation}
%  \sinh ^{-1}\left(\sqrt{x}\right)=\frac{1}{2}\log \left(2 x+2 \sqrt{x (x+1)}+1\right)
% \end{equation}

% We can try to verify this by writing 
% \begin{equation}
% \sinh(x) = \frac{1}{2}(e^{x}-e^{-x}),
% \end{equation}
% or 
% \begin{equation}
% x = \frac{1}{2}\left(e^{\text{arcsinh}[x]} - e^{-\text{arcsinh}[x]}\right)
% \end{equation}
% Take $u = e^{\text{arcsinh}[x]}$.  
% \begin{align}
% 2x = u - 1/u,
% \end{align}
% or
% \begin{equation}
% u^2 - 2xu - 1 = 0,
% \end{equation}
% which has solutions, 
% \begin{equation}
% u = \frac{2x \pm \sqrt{4x^2 + 4}}{2}  = x \pm \sqrt{x^2+1}.
% \end{equation}
% Now $u>0,$ so only the postive solution is valid.  
% \begin{align}
% e^{as} =& x \pm \sqrt{x^2+1}\\
% \Aboxed{\text{arcsinh}[x] =& \log(x + \sqrt{x^2+1})}
% \end{align}


\section{Finding the TM CP energy}

Now we can use the Feynman-Kac formula for the combined $TM$ potential,
 and step to calculate the energy of an atom with a wall due to the TM polarization.
  We will again have to use the Laplace-Mellin transform, and the scaled gamma function to develop the formalism.  

We can plug this back in to the potential.  
\begin{align}
\mathcal{V}_D(\chi,d)=-\frac{1}{\Gamma[\alpha]\Gamma\left[(D+1)/2-\alpha\right]}
\int_0^\infty ds s^{\alpha-1}\int_0^\infty d\lambda \lambda^{(D-1)/2-\alpha}\dlangle 
\int_0^\infty dT \frac{e^{-(\lambda+s)T}}{\sqrt{T}}e^{-\int_0^T dt\, V_{TM} + s\chi\Theta(x-d)}\drangle
\end{align}

\subsection{Changing Variables}
So this is what we really want.  Time to start simplifying this integral.  
We will renormalize against $d\rightarrow \infty$, and drop the normalization factor.  
Dan focuses on the case where we normalize against $e^{-c^2/(2t)}/\sqrt{t}$, 
as opposed to $e^{-c^2/(2t)}/\sqrt{2\pi t}$.  \comment{I will multiply my results by $\sqrt{2\pi}$.  }

\begin{align}
\mathcal{V}_D(\chi,d)=-\frac{\sqrt{\pi}}{\Gamma[\alpha]\Gamma\left[(D+1)/2-\alpha\right]}
\int_0^\infty ds s^{\alpha-1}\int_0^\infty d\lambda \lambda^{(D-1)/2-\alpha}
\frac{\left(\sqrt{\lambda+s}e^{2\Xi }- \sqrt{\lambda +s(1 +\chi)}\right) }
{\sqrt{\lambda +s} e^{2 \Xi }+\sqrt{\lambda +s(1+\chi)}}\frac{e^{-2\sqrt{2(\lambda+s) } |d|}}{\sqrt{(\lambda+s)}}
\end{align}
What follows is a lot of changes of variables to turn this into something like an integral against a decaying exponential.  
This follows in exact analogy with the TE case.  
% $u=\lambda+s$.  
% \begin{align}
% \mathcal{V}_D(\chi,d)=-\frac{\sqrt{\pi}}{\Gamma[\alpha]\Gamma\left[(D+1)/2-\alpha\right]}
% \int_0^\infty ds s^{\alpha-1}\int_s^\infty du\, u^{-1/2}(u-s)^{(D-1)/2-\alpha}
% \frac{\left(\sqrt{u}e^{2\Xi }- \sqrt{u + s\chi}\right) }{\sqrt{u} e^{2 \Xi }+\sqrt{u + s\chi}}e^{-2\sqrt{2u } |d|}
% \end{align}
% Now take $u\rightarrow u/8d^2, s\rightarrow s/(8d^2)$.  
% \begin{align}
% \mathcal{V}_D(\chi,d)% =&-\frac{\sqrt{\pi}}{\Gamma[\alpha]\Gamma\left[(D+1)/2-\alpha\right]}\frac{1}{(8d^2)^\alpha}\frac{1}{(8d^2)^{(D-1)/2-\alpha+1-1/2}}\nonumber\\
% % & \times\int_0^\infty ds s^{\alpha-1}\int_s^\infty du\, u^{-1/2}(u-s)^{(D-1)/2-\alpha}\frac{\left(\sqrt{u}e^{2\Xi }- \sqrt{u + s\chi}\right) }{\sqrt{u} e^{2 \Xi }+\sqrt{u + s\chi}}e^{-\sqrt{u }}\\
% =&-\frac{\sqrt{\pi}}{\Gamma[\alpha]\Gamma\left[(D+1)/2-\alpha\right]}\frac{1}{(8d^2)^{D/2}}\nonumber\\
% & \times\int_0^\infty ds s^{\alpha-1}\int_s^\infty du\, u^{-1/2} (u-s)^{(D-1)/2-\alpha}
% \frac{\left(\sqrt{u}e^{2\Xi }- \sqrt{u + s\chi}\right) }{\sqrt{u} e^{2 \Xi }+\sqrt{u + s\chi}}e^{-\sqrt{u }}
% \end{align}
% Now define $u = sv$.  
% \begin{align}
% \mathcal{V}_D=&-\frac{\sqrt{\pi}}{\Gamma[\alpha]\Gamma\left[(D+1)/2-\alpha\right]}\frac{1}{(8d^2)^{D/2}}\nonumber\\
% & \times\int_0^\infty ds s^{(D-2)/2}\int_1^\infty dv \, v^{-1/2}(v-1)^{(D-1)/2-\alpha}
% \frac{\left(\sqrt{v}e^{2\Xi }- \sqrt{v + \chi}\right) }{\sqrt{v} e^{2 \Xi }+\sqrt{v + \chi}}e^{-\sqrt{s v}}
% \end{align}
% Now define $ s = t^2, v = w^2$.  
% \begin{align}
% \mathcal{V}_D%=&-\frac{4\sqrt{\pi}}{\Gamma[\alpha]\Gamma\left[(D+1)/2-\alpha\right]}\frac{1}{(8d^2)^{D/2}}\int_0^\infty dt\, t^{(D-2)+1}\int_1^\infty dw\,(w^2-1)^{(D-1)/2-\alpha}\frac{\left(w e^{2\Xi }- \sqrt{w^2 + \chi}\right) }{w e^{2 \Xi }+\sqrt{w^2 + \chi}}e^{-tw}\\
% =&-\frac{4\sqrt{\pi}}{\Gamma[\alpha]\Gamma\left[(D+1)/2-\alpha\right]}\frac{1}{(8d^2)^{D/2}}
% \int_0^\infty dt\, t^{D-1}\int_1^\infty dw\,(w^2-1)^{(D-1)/2-\alpha}
% \frac{\left(w e^{2\Xi }- \sqrt{w^2 + \chi}\right) }{w e^{2 \Xi }+\sqrt{w^2 + \chi}}e^{-tw}
% \end{align}
% Now take swap the $t,w$ integrals, and turn the $t$ integral into a Gamma function.  
% \begin{align}
% \mathcal{V}_D%=&-\frac{4\sqrt{\pi}}{\Gamma[\alpha]\Gamma\left[(D+1)/2-\alpha\right]}\frac{1}{(8d^2)^{D/2}}\int_1^\infty dw\,(w^2-1)^{(D-1)/2-\alpha}\frac{\left(w e^{2\Xi }- \sqrt{w^2 + \chi}\right) }{w e^{2 \Xi }+\sqrt{w^2 + \chi}}\int_0^\infty dt\, t^{D-1}e^{-tw}\\
% =&-\frac{4\sqrt{\pi}\Gamma[D]}{\Gamma[\alpha]\Gamma\left[(D+1)/2-\alpha\right]}
% \frac{1}{(8d^2)^{D/2}}\int_1^\infty dw\,w^{-D}(w^2-1)^{(D-1)/2-\alpha}
% \frac{\left(w e^{2\Xi }- \sqrt{w^2 + \chi}\right) }{w e^{2 \Xi }+\sqrt{w^2 + \chi}}
% \end{align}

So we finally have
\begin{align}
\Aboxed{\mathcal{V}_D=&-\frac{\Gamma[D]\sqrt{\pi}}{2^{3D/2-2}\Gamma[\alpha]\Gamma\left[(D+1)/2-\alpha\right]}
\frac{1}{d^{D}}\int_1^\infty dw\,w^{-D}(w^2-1)^{(D-1)/2-\alpha}\frac{w(1+\chi)- \sqrt{w^2 + \chi} }{w (1+\chi)+\sqrt{w^2 + \chi}}},
\end{align}
where we used $\Xi = \log\sqrt{1+\chi}$.  

Now for the full atom-wall potential we need 
\begin{align}
V = \mathcal{V}_D(\chi,d;3/2) - \frac{1}{2}\partial_d^2\mathcal{V}_{D-2}(x,d;1/2)
\end{align}
%\comment{The scaled ``Gies'' formulation uses  $V = V_D - V_{D-2}''$}


\subsection{Evaluating integral for specific $D$, and $\alpha$}

We need $D=4$, $\alpha=3/2$.  
\begin{align}
\mathcal{V}_D(\chi;d;3/2)%=&-\frac{\Gamma[D]\sqrt{\pi}}{2^{4}\Gamma[3/2]\Gamma\left[(D+1)/2-3/2\right]}\frac{1}{d^D}\int_1^\infty dw\,w^{-D}(w^2-1)^{(D-1)/2-3/2}\frac{w(1+\chi)- \sqrt{w^2 + \chi} }{w (1+\chi)+\sqrt{w^2 + \chi}}\\
=&-\frac{3}{4d^4}\int_1^\infty dw\,w^{-4}\frac{w(1+\chi)- \sqrt{w^2 + \chi} }{w (1+\chi)+\sqrt{w^2 + \chi}}
\end{align}

We also need $D=2$, $\alpha=1/2$
\begin{align}
\mathcal{V}_{D-2}(\chi;d;1/2)%=&-\frac{\Gamma[D-2]\sqrt{\pi}}{2^{(3(D-2)/2-2)}\Gamma[1/2]\Gamma\left[(D-1)/2-1/2\right]}\frac{1}{d^{D-2}}\int_1^\infty dw\,w^{-(D-2)}(w^2-1)^{(D-3)/2-1/2}\frac{w(1+\chi)- \sqrt{w^2 + \chi} }{w (1+\chi)+\sqrt{w^2 + \chi}}\\
=&-\frac{1}{2d^2}\int_1^\infty dw\,w^{-2}\frac{w(1+\chi)- \sqrt{w^2 + \chi} }{w (1+\chi)+\sqrt{w^2 + \chi}}
\end{align}
\comment{Doesn't $V_2$ go to zero for $D=2$ due to $\Gamma[0]$ in the denominator?}

We finally need 
\begin{align}
\frac{1}{2}\partial_d^2(\mathcal{V}_{D-2}(\chi;d;1/2)=
&-\frac{3}{2d^4}\int_1^\infty dw\,w^{-2}\frac{w(1+\chi)- \sqrt{w^2 + \chi} }{w (1+\chi)+\sqrt{w^2 + \chi}}
\end{align}

Now if we go way back up and compare to averages we needed for the Casimir energy, we wanted: 

\begin{equation}
V(d) = -\frac{1}{2}\alpha_0\langle E^2\rangle = -\frac{1}{2}\hbar c\frac{\delta}{\delta\epsilon(d)}\log Z
\end{equation}
In addition, there is a factor of $-\hbar c\alpha_0/(16\epsilon_0\pi^2)$ 
from the original path integral expression, which includes a $1/2$ from the functional derivative 
($-\delta/(\delta\epsilon)$, and $-1/2$ from the $\det[A]^{-1/2}$ from the initial Gaussian integral.  
So the final Casimir energy is 
\begin{equation}
V = -\frac{3\hbar c\alpha_0}{64\pi^2\epsilon_0d^4} \int_1^\infty dw\,w^{-4}(1-2w^2)
\frac{w(1+\chi)- \sqrt{w^2 + \chi} }{w (1+\chi)+\sqrt{w^2 + \chi}},
\end{equation}
which is the same as the Lifshitz integral.  

\section{Finding the TE Casimir energy}

This section is quite similar to the corresponding Casimir-Polder energy for a test particle placed near a plane.
  In this cae however, we will deal with two planar dielectrics.
  We will thus need the solutions for that Fokker-Planck equation in all regions, and then have to evaluate an integral over space.
  That is the only conceptual difference, otherwise this follows the same pattern.  

I have found a result by Zhou and Spruch, \footnote{Zhou, F, and Spruch, L., ``van der Waals and retardation (Casimir) interactions of an electron or atom with multilayered walls'', Phys. Rev. A, \textbf{52}, 297, (1995)}, where they have in Eq. (3.19),
\begin{align}
\frac{F}{L^2} =& -\frac{\hbar}{2\pi^2c^3}\int_0^\infty d\xi \xi^3 \epsilon_3^{3/2}\int_1^\infty dp\,p^2\left[ \frac{r_{13}r_{23}e^{-2\sqrt{\epsilon_3}p\xi l/c}}{1 - r_{13}r_{23}e^{-2\sqrt{\epsilon_3}p\xi l/c}} + \frac{r'_{13}r'_{23}e^{-2\sqrt{\epsilon_3}p\xi l/c}}{1 - r'_{13}r'_{23}e^{-2\sqrt{\epsilon_3}p\xi l/c}}\right]\\
\Aboxed{\frac{F}{L^2}=& -\frac{\hbar c}{2\pi^2d^4}\int_0^\infty du u^3\int_1^\infty dp\,p^2\left[ \frac{r^2e^{-2pu}}{1 - r^2e^{-2pu}} + \frac{r'^2e^{-2pu}}{1 - r'^2e^{-2pu}}\right]}
\end{align}


   {Carry out integral over position. Note divergent terms, these are cancelled out by 
    suitable renormalization.}
    We now need to evaluate $\int dx_0 \dlangle e^{-\int_0^T dt V[x_0 + B(t)-h]}\drangle$ for use with Casimir energies.   Now we have to take $h\rightarrow h-x_0$, and integrate over $x_0$. 
    First up we need $I_{12}=\int dx_0 f_{12}[x-(h-x_0)]$
    \begin{align}
      I_{TE,12} %=& \int_{-\infty}^h  dx_0  f_{x_0<h} + \int_{h}^{h+d}  dx_0  f_{h<x_0<h+d} + \int_{h+d}^\infty dx_0 f_{x_0>h+d}\\
      =&\int_{-\infty}^h dx_0 \left[\dfrac{1}{\sqrt{2(\lambda+\chi_1)}} + e^{-2\sqrt{2(\lambda+\chi_1)}(h-x_0)}\dfrac{u_2 e^{-2\sqrt{2\lambda}d} - u_1}{\sqrt{2(\lambda+\chi_1)}(1-u_1u_2 e^{-2\sqrt{2\lambda}d})}\right] \nonumber\\
      & +\int_{h}^{h+d}dx_0\left[\frac{1}{\sqrt{2\lambda}} + \frac{2u_1u_2 e^{-2\sqrt{2\lambda}d} + u_1 e^{2\sqrt{2\lambda}(h-x_0)} +u_2 e^{-2\sqrt{2\lambda}(d+h-x_0)}}{\sqrt{2\lambda}(1-u_1u_2 e^{-2\sqrt{2\lambda}d})} \right]\nonumber\\
      &+ \int_{h+d}^\infty dx_0 \left[\dfrac{1}{\sqrt{2(\lambda+\chi_2)}} + e^{2\sqrt{2(\lambda+\chi_2)}(d+(h-x_0))}\dfrac{u_1 e^{-2\sqrt{2\lambda}d}-u_2}{\sqrt{2(\lambda+\chi_2)}(1-u_1u_2 e^{-2\sqrt{2\lambda}d})}\right]
    \end{align}

    %  We need 
    % \begin{gather}
    %   \int_{-\infty}^h dx_0 e^{-2\sqrt{2(\lambda+\chi_1)}(h-x_0)} = \int_{-\infty}^0 dx_0 e^{2\sqrt{2(\lambda+\chi_1)}(x_0)} = \frac{1}{2\sqrt{2(\lambda+\chi_1)}}\\
    %   \int_{h+d}^\infty dx_0 e^{2\sqrt{2(\lambda+\chi_2)}(d+(h-x_0))} = \int_0^\infty e^{-2\sqrt{2(\lambda+\chi_2)}x_0} = \frac{1}{2\sqrt{2(\lambda+\chi_2)}}\\
    %   \int_{h}^{h+d} dx_0 e^{-2\sqrt{2\lambda}(x_0-h)} = \int_0^d  dx_0 e^{-2\sqrt{2\lambda}x_0} = \frac{1 - e^{-2\sqrt{2\lambda}d}}{2\sqrt{2\lambda}}\\
    %   \int_{h}^{h+d} dx_0 e^{2\sqrt{2\lambda}(x_0-d-h)} = \int_{-d}^0  dx_0 e^{2\sqrt{2\lambda}x_0} = \frac{1 - e^{-2\sqrt{2\lambda}d}}{2\sqrt{2\lambda}}
    % \end{gather}
    % Final integrated value
    % So that we get:
    % \begin{align}
    %   I_{TE,12} =& -I_{div} + \dfrac{u_2 e^{-2\sqrt{2\lambda}d}-u_1}{4(\lambda+\chi_1)(1-u_1u_2 e^{-2\sqrt{2\lambda}d})} +\frac{2u_1u_2 e^{-\sqrt{2\lambda}d}d}{\sqrt{2\lambda}(1-u_1u_2 e^{-2\sqrt{2\lambda}d})} + (u_1+u_2)\frac{(1-e^{-2\sqrt{2\lambda}d})}{4\lambda(1-u_1u_2e^{-2\sqrt{2\lambda}d})}\nonumber\\
    %   & +\frac{u_1 e^{-2\sqrt{2\lambda}d} - u_2}{4(\lambda+\chi_2)(1-u_1u_2 e^{-2\sqrt{2\lambda}d})},
    % \end{align}
    % where $I_{div} = [2(\lambda+\chi_1)]^{-1/2}\int_{-\infty}^h dx_0  +  [2(\lambda+\chi_2)]^{-1/2}\int_{h+d}^\infty dx_0  + (2\lambda)^{-1/2}d$.





\subsection{Rescaling TE again}

Let's try to check our TE results, by rescaling that again.  
\begin{align}
E =& - \partial_\beta \log Z_{TE}\\
 =& -\frac{\hbar c}{8\pi^2}\int_0^\infty \frac{dT}{T^{1+D/2}}\int dx_0 \dlangle \frac{1}{\langle \epsilon\rangle^{\alpha}}\drangle\\
=& - \frac{\hbar c}{8\pi^2\Gamma[(D+1)/2+\alpha]\Gamma(\alpha)} 
\int_0^\infty d\lambda \lambda^{(D-1)/2-\alpha}\int dx_0 \int_0^\infty ds\, s^{\alpha-1}
\int_0^\infty dT e^{-\lambda T}\dlangle \frac{e^{-s\int_0^T dt  \epsilon}}{\sqrt{T}}\drangle\\
=& - \frac{\hbar c}{8\pi^2\Gamma[(D+1)/2+\alpha]\Gamma(\alpha)} 
\int_0^\infty d\lambda \lambda^{(D-1)/2-\alpha}\int_0^\infty ds\, s^{\alpha-1} I(\lambda+s,s\chi)
\end{align}

$I_{tot}$ is the (renormalized) expression for 
\begin{equation}
I_{tot}(\lambda,\chi)=\int dx_0\int_0^\infty dT \frac{e^{-\lambda T}}{\sqrt{T}}\dlangle e^{-\int_0^T dt \chi\Theta(x_0+B(t))  } \drangle,
\end{equation}
we found that 
\begin{align}
I_{tot}(\lambda,\chi) =& I_{12}-I_1-I_2 + I_0 \\
=&  \sqrt{2\pi}\frac{r_1r_2 e^{-2\sqrt{2\lambda}d}}{\sqrt{2\lambda}(1-r_1r_2 e^{-2\sqrt{2\lambda}d})}
\left( 2d + 2\frac{\sqrt{2}}{\sqrt{\lambda+\chi}}\right)
\end{align}

Plugging all this in, we have 
\begin{align}
E  =& - \frac{\hbar c\sqrt{\pi}}{8\pi^2\Gamma[(D+1)/2+\alpha]\Gamma(\alpha)} 
\int_0^\infty d\lambda \lambda^{(D-1)/2-\alpha}\int_0^\infty ds\, s^{\alpha-1}
\frac{2r^2 e^{-2\sqrt{2\lambda}d}}{\sqrt{\lambda+s}(1-r^2 e^{-2\sqrt{2(\lambda+s)}d})}
\left( d + \frac{\sqrt{2}}{\sqrt{\lambda+s+s\chi}} \right),
\end{align}
with 
\begin{equation}
r = \frac{ \sqrt{\lambda+s} - \sqrt{\lambda+s+s\chi}}{ \sqrt{\lambda+s} + \sqrt{\lambda+s+s\chi}}
\end{equation}
Substitution time!
\begin{enumerate}
\item $\lambda = s\kappa$.
\begin{align}
E  =& - \frac{\hbar c\sqrt{\pi}}{8\pi^2\Gamma[(D+1)/2+\alpha]\Gamma(\alpha)} \int_0^\infty d\kappa 
\kappa^{(D-1)/2-\alpha}\int_0^\infty ds\, s^{(D-1)/2-\alpha+1} s^{\alpha-1}\nonumber\\
&\times \frac{2r^2 e^{-2\sqrt{2(\kappa+1)s}d}}{\sqrt{s(\kappa+1)}(1-r^2 e^{-2\sqrt{2s(\kappa+1)}d})}
\left( d + \frac{\sqrt{2}}{\sqrt{s(\kappa+1+\chi)}} \right),
\end{align}
with 
\begin{equation}
r = \frac{ \sqrt{\kappa+1} - \sqrt{\kappa+1+\chi}}{ \sqrt{\kappa+1} + \sqrt{\kappa+1+\chi}}
\end{equation}
\item $p = \sqrt{\kappa+1},$ or $\kappa = p^2-1$. 
\begin{align}
E  =& - \frac{\hbar c\sqrt{\pi}}{8\pi^2\Gamma[(D+1)/2+\alpha]\Gamma(\alpha)} \int_1^\infty dp (2p) 
(p^2-1)^{(D-1)/2-\alpha}\int_0^\infty ds\, s^{(D-1)/2}\nonumber\\
&\times \frac{2r^2 e^{-2\sqrt{2s}pd}}{\sqrt{s}p(1-r^2 e^{-2\sqrt{2s}pd})}
\left( d + \frac{\sqrt{2}}{\sqrt{s(p^2+\chi)}} \right),
\end{align}
with 
\begin{equation}
r = \frac{ p - \sqrt{p^2+\chi}}{ p + \sqrt{p^2+\chi}}
\end{equation}
\item $\xi = \sqrt{2s} d$, or $s = \xi^2/2 d^2$.  

\begin{align}
E  % =& - \frac{\hbar c\sqrt{\pi}}{4\pi^2\Gamma[(D+1)/2+\alpha]\Gamma(\alpha)} 
% \int_1^\infty dp\, (p^2-1)^{(D-1)/2-\alpha}\int_0^\infty d\xi\,\frac{\xi}{d^2} \left( \frac{\xi^2}{2d^2}\right)^{(D-1)/2}\nonumber\\
% &\times \frac{2\sqrt{2} dr^2 e^{-2\xi p}}{\xi (1-r^2 e^{-2\xi p})}\left( d + \frac{2 d}{\sqrt{\xi(p^2+\chi)}} \right)\\
 =& - \frac{\hbar c\sqrt{\pi} 2^{3/2}}{2^{D/2}\pi^2\Gamma[(D+1)/2+\alpha]\Gamma(\alpha) d^{D-1}2^{(D-1)/2}}
 \int_1^\infty dp \, (p^2-1)^{(D-1)/2-\alpha}\int_0^\infty d\xi\,\xi^{D-1}\nonumber\\
&\times \frac{ r^2 e^{-2\xi p}}{(1-r^2 e^{-2\xi p})}\left( 1 + \frac{2 }{\sqrt{\xi(p^2+\chi)}} \right),
\end{align}
with 
\begin{equation}
r = \frac{ p - \sqrt{p^2+\chi}}{ p + \sqrt{p^2+\chi}}
\end{equation}
\item Now set $\alpha=1/2, D=4$.
\begin{align}
\Aboxed{E_{TM} =& - \frac{\hbar c }{8\pi^2 d^{D-1}} \int_1^\infty dp \, (p^2-1)\int_0^\infty d\xi\,\xi^{3} 
\frac{ r^2 e^{-2\xi p}}{ (1-r^2 e^{-2\xi p})}\left( 1 + \frac{2 }{\sqrt{\xi(p^2+\chi)}} \right),}
\end{align}

\end{enumerate}

\begin{align}
E_{TE} = & - \frac{\hbar c }{8\pi^2 d^{3}}\int_0^\infty du\,u^{3}\int_1^\infty dp\,(p^2-1)
\frac{ r^2e^{-2pu}}{(1 -r^2 e^{-2pu})}\left[ 1 +\frac{2}{u\sqrt{p^2+\chi}}\right]
\end{align}

Now integrate by parts with $p$.  
Let's try integrating by parts with respect to $p$?    
\begin{equation}
\int_{1}^\infty dp f(p) = pf(p)\bigg|_{p=1}^\infty -\int_1^\infty dp\, p f'(p) 
\end{equation}
We may have to integrate by parts with respect to $u$ to get the correct power there.  

\begin{align}
\frac{dr}{dp} =& \frac{d}{dp} \frac{p-\sqrt{p^2+\chi}}{p+\sqrt{p^2+\chi}}
= \frac{1-\frac{2p}{2\sqrt{p^2+\chi}}}{p+\sqrt{p^2+\chi}} - 
(p-\sqrt{p^2+\chi})\frac{(1+\frac{2p}{2\sqrt{p^2+\chi}})}{(p+\sqrt{p^2+\chi})^2} 
%=& \frac{1}{\sqrt{p^2+\chi}}\frac{\sqrt{p^2+\chi}-p}{p+\sqrt{p^2+\chi}} - (p-\sqrt{p^2+\chi})\frac{p+\sqrt{p^2+\chi}}{\sqrt{p^2+\chi}(p+\sqrt{p^2+\chi})^2} \\
=& -\frac{2r}{\sqrt{p^2+\chi}}
\end{align}
% Intriguingly, I found that 
% \begin{align}
% \frac{d}{dp}[r^{-2} e^{2pu} -1]&= \left( -2\frac{dr}{dp} r^{-3} e^{2pu} + 2u r^{-2}e^{2pu}\right)\\
% &=  \frac{4}{\sqrt{p^2+\chi}} r^{-2} e^{2pu} + 2u r^{-2}e^{2pu}\\
% &=  2u\left(1+\frac{2}{u\sqrt{p^2+\chi}}\right) r^{-2} e^{2pu}
% \end{align}
We can also write:
\begin{align}
\frac{d}{dp}[1-r^{2} e^{-2pu}]&= -\left( 2r\frac{dr}{dp} e^{-pu} - 2u r^{2}e^{-2pu}\right)\\
&= \left(u+\frac{4}{\sqrt{p^2+\chi}}\right) r^{2}e^{-2pu},
\end{align}
which suggests 
\begin{align}
\Aboxed{\frac{d}{dp}\ln[1-r^{2} e^{-2pu}]&= 2u \frac{r^2 e^{-2pu}}{1 - r^2  e^{-2pu}}\left(1+\frac{2}{u\sqrt{p^2+\chi}}\right)}
\end{align}
So our integral becomes
\begin{align}
I_1 & = \int_1^\infty dp\,(p^2-1)\frac{ r^2e^{-2pu}}{(1 -r^2 e^{-2pu})}\left[ 1 +\frac{2}{u\sqrt{p^2+\chi}}\right]\\
& = \frac{1}{2u}\log\left[1-r^2 e^{-2pu}\right](p^2-1)\bigg|_{p=1}^\infty - \int_1^\infty dp \,\frac{p}{u}\log\left[1-r^2 e^{-2pu}\right]
\end{align}
Note that the boundary term vanishes at both limits.  Let's now apply that integration by parts to the $TE$ energy.  
\begin{align}
%E_{TE} = &  \frac{\hbar c }{8\pi^2 d^{3}}\int_0^\infty du\,u^{2}\int_1^\infty dp\,p \log\left[1- r^2e^{-2pu}\right]\\
\Aboxed{E_{TE}= &  \frac{\hbar c }{8\pi^2}\int_0^\infty d\xi\,\xi^{2}\int_1^\infty dp\,p \log\left[1- r^2e^{-2p\xi d}\right]},
\end{align}
this form has the advantage of merely undoing an earlier scaling (which would not work for frequency dependent materials).
  A differentiation with respect to distance then yields the Lifshitz results.  



\section{Finding the TM Casimir energy}

From November 2013 we have
\begin{align}
I_{12}-I_1-I_2 + I_0 =&   \dfrac{r'_2 e^{-2\sqrt{2\lambda}d}}{4\kappa_1(1-r'_1r'_2 e^{-2\sqrt{2\lambda}d})} 
+\frac{2r'_1r'_2 e^{-\sqrt{2\lambda}d}d}{\sqrt{2\lambda}(1-r'_1r'_2 e^{-2\sqrt{2\lambda}d})} 
- (r'_1+r'_2)\frac{e^{-2\sqrt{2\lambda}d}}{4\lambda(1-r'_1r'_2e^{-2\sqrt{2\lambda}d})}\nonumber\\
& +\frac{r'_1 e^{-2\sqrt{2\lambda}d}}{4\kappa_2(1-r'_1r'_2 e^{-2\sqrt{2\lambda}d})}  
 -   \left(\frac{r'_1}{4\kappa_1}-\frac{(r'_1+r'_2)}{4\lambda}  
+ \frac{r'_2}{4\kappa_2}\right)\frac{r'_1r'_2 e^{-2\sqrt{2\lambda}d}}{(1-r'_1r'_2 e^{-2\sqrt{2\lambda}d})},
\end{align}
where $\kappa = \lambda+\chi$, and 
\begin{equation}
r' =  \frac{\epsilon\sqrt{\lambda}-\sqrt{\kappa}}{\epsilon\sqrt{\lambda}+\sqrt{\kappa}}
\end{equation}

\comment{Let's now set $\chi_1=\chi_2=\chi$}
Now factor out the common terms: 
\begin{align}
I_{tot}=&  \frac{e^{-2\sqrt{2\lambda}d}}{(1-r'^2e^{-2\sqrt{2\lambda}d}}\left[
 \frac{2r'^2d}{\sqrt{2\lambda}}+ \dfrac{r'}{2\kappa}  -\frac{ r'}{2\lambda} 
- \left(\frac{r'}{2\kappa}-\frac{r'}{2\lambda}\right)r'^2\right]\\
=&  \frac{e^{-2\sqrt{2\lambda}d}}{(1-r'^2e^{-2\sqrt{2\lambda}d}}\left[
\frac{2r'^2d}{\sqrt{2\lambda}}+ \frac{r'}{2}(1-r'^2)\left(\dfrac{1}{\kappa}  -\frac{1}{\lambda}\right)\right],
\end{align}


\begin{align}
E =& - \partial_\beta \log Z_{TE}\\
 =& -\frac{\hbar c}{8\pi^2}\int_0^\infty \frac{dT}{T^{1+D/2}}\int dx_0 
\dlangle \frac{1}{\langle \epsilon\rangle^{\alpha}}e^{-\int_0^T\mathfrak{M}dt}\drangle\\
=& - \frac{\hbar c}{8\pi^2\Gamma[(D+1)/2+\alpha]\Gamma(\alpha)} \int_0^\infty d\lambda 
\lambda^{(D-1)/2-\alpha}\int dx_0 \int_0^\infty ds\, s^{\alpha-1}\int_0^\infty dT e^{-\lambda T}
\dlangle \frac{e^{-sT\int_0^T dt  \epsilon-\int_0^T\mathfrak{M}dt}}{\sqrt{T}}\drangle,
\end{align}

$I_{tot}$ is the (renormalized) expression for 
\begin{equation}
I_{tot}=\int dx_0\int_0^\infty dT \frac{e^{-\lambda T}}{\sqrt{T}}\dlangle e^{-\mathfrak{M}-\int_0^T dt V(x_0+B(t))  } \drangle,
\end{equation}

\begin{enumerate}
\item We get the energy by making the following replacements \comment{$\lambda\rightarrow \lambda+s$, $\chi\rightarrow s\chi$.  } 
(However, this does not apply to terms in $e^{\Xi}$.  )
 \begin{align}
E=&  C\int_0^\infty d\lambda \lambda^{(D-1)/2-\alpha}\int_0^\infty ds\, s^{\alpha-1}\nonumber\\
&\times  \frac{e^{-2\sqrt{2(\lambda+s)}d}}{1-r'^2e^{-2\sqrt{2(\lambda+s)}d}}
\left[\frac{2r'^2d}{\sqrt{2(\lambda+s)}}+ \frac{r'}{2}(1-r'^2)\left(\dfrac{1}{\lambda+s+s\chi}  -\frac{1}{\lambda+s}\right)\right],
\end{align}
and 
\begin{equation}
r' =  \frac{(1+\chi)\sqrt{\lambda+s}-\sqrt{\lambda+ s + s\chi}}{(1+\chi)\sqrt{\lambda+s}+\sqrt{\lambda+s+s\chi}},
\end{equation}
\begin{equation}
C = -\frac{\hbar c\sqrt{2\pi}}{8\pi^2\Gamma[(D+1)/2+\alpha]\Gamma(\alpha)} 
\end{equation}

\item Now take $\lambda = qs$.  
 \begin{align}
E=&  C\int_0^\infty dq q^{(D-1)/2-\alpha}\int_0^\infty ds\, s^{\alpha-1}s^{1+(D-1)/2-\alpha}\nonumber\\
&\times  \frac{e^{-2\sqrt{2(q+1)s}d}}{1-r'^2e^{-2\sqrt{2(q+1)s}d}}\left[
\frac{2r'^2d}{\sqrt{2s(q+1)}}+ \frac{r'}{2}(1-r'^2)\left(\dfrac{1}{s(q+1+\chi)}  -\frac{1}{s(q+1)}\right)\right],
\end{align}
and 
\begin{equation}
r' =  \frac{(1+\chi)\sqrt{q+1}-\sqrt{q+ 1 + \chi}}{(1+\chi)\sqrt{q+1}+\sqrt{q+1+\chi}},
\end{equation}
\item Now take $p = \sqrt{q+1}$.  $\rightarrow  q = p^2-1, dq = 2p dp$.  
 \begin{align}
E%=&   C\int_1^\infty dp(2p) (p^2-1)^{(D-1)/2-\alpha}\int_0^\infty ds\, s^{\alpha-1}s^{1+(D-1)/2-\alpha}\nonumber\\
% &\times  \frac{e^{-2\sqrt{2s}pd}}{1-r'^2e^{-2\sqrt{2s}pd}}\left[\frac{2r'^2d}{\sqrt{2s}p}+ 
% \frac{r'}{2}(1-r'^2)\left(\dfrac{1}{s(p^2+\chi)}  -\frac{1}{s p^2}\right)\right]\\
=&  2C\int_1^\infty dp (p^2-1)^{(D-1)/2-\alpha}\int_0^\infty ds\, s^{(D-1)/2}\nonumber\\
&\times\frac{e^{-2\sqrt{2s}pd}}{1-r'^2e^{-2\sqrt{2s}pd}}\left[\frac{2r'^2d}{\sqrt{2s}}- 
\frac{r'}{2}(1-r'^2)\left(\dfrac{\chi}{s(p^2+\chi)p}\right)\right],
\end{align}
and 
\begin{equation}
r' =  \frac{(1+\chi)p-\sqrt{p^2 + \chi}}{(1+\chi)p+\sqrt{p^2+\chi}},
\end{equation}
\item Finally, take 
$s = \xi^2$
\item Now take $p = \sqrt{q+1}$.  $\rightarrow  q = p^2-1, dq = 2p dp$.  
 \begin{align}
E% =&  4C\int_1^\infty dp (p^2-1)^{(D-1)/2-\alpha}\int_0^\infty d\xi\,\xi\xi^{D-1}\nonumber\\
% &\times  \frac{e^{-2\sqrt{2}\xi pd}}{1-r'^2e^{-2\sqrt{2}\xi pd}}\left[\frac{2r'^2d}{\sqrt{2}\xi}-
%  \frac{r'}{2}(1-r'^2)\left(\frac{ \chi}{\xi^2(p^2+\chi)p}\right)\right]\\
=&  4C\int_1^\infty dp (p^2-1)^{(D-1)/2-\alpha}\int_0^\infty d\xi\,\xi^{D-1}\nonumber\\
&\times  \frac{r'^2e^{-2\sqrt{2}\xi pd}}{1-r'^2e^{-2\sqrt{2}\xi pd}}\left[\frac{2d}{\sqrt{2}}-
 \frac{1}{2}\left(\frac{1}{r'}-r'\right)\left(\dfrac{\chi}{\xi(p^2+\chi)p}\right)\right],
\end{align}
and 
\begin{equation}
r' =  \frac{(1+\chi)p-\sqrt{p^2 + \chi}}{(1+ \chi)p+\sqrt{p^2+\chi}},
\end{equation}

\item Now take $u = \sqrt{2}\xi d$ 
 \begin{align}
E% =&  4C\frac{1}{2^{5/2}d^4}\int_1^\infty dp (p^2-1)\int_0^\infty du\,u^{3} 
% \frac{r'^2e^{-2u p}}{1-r'^2e^{-2u p}}\left[2d- \left(\frac{1}{r'}-r'\right)\dfrac{\chi d}{u(p^2+\chi)p}\right]\\
=&  C\frac{\sqrt{2}}{d^3}\int_1^\infty dp (p^2-1)\int_0^\infty du\,u^{3} 
 \frac{r'^2e^{-2u p}}{1-r'^2e^{-2u p}}\left[1- \left(\frac{1}{r'}-r'\right)\dfrac{\chi }{2u(p^2+\chi)p}\right],
\end{align}
and 
\begin{equation}
r' =  \frac{(1+\chi)p-\sqrt{p^2 + \chi}}{(1+ \chi)p+\sqrt{p^2+\chi}},
\end{equation}
\comment{proportional to Schwinger}
\end{enumerate}

\subsection{Integrate by parts w.r.t. $p$}
Let's check if this also goes for the TM case.  
\begin{align}
\frac{d}{dp}\ln[1-r'^2 e^{-2pu}] =& \frac{-2r' \frac{dr'}{dp} e^{-2pu} + 2u r'^2 e^{-2pu}}{1-r'^2 e^{-2pu}} \\
% &= \frac{r'^2 e^{-2pu}}{1-r'^2 e^{-2pu}}\left( 2u -\frac{2}{r'} \frac{dr'}{dp}\right)\\
% &= 2u\frac{r'^2 e^{-2pu}}{1-r'^2 e^{-2pu}}\left( 1 -\frac{1}{ur'} \frac{dr'}{dp}\right)\\
&= 2u\frac{r'^2 e^{-2pu}}{1-r'^2 e^{-2pu}}\left( 1 -\frac{1}{u} \frac{d\ln[r']}{dp}\right)
\end{align}

Now use 
\begin{align}
\frac{d}{dp}\ln[r'] =& \frac{d}{dp}\left(\log[e^{2\Xi}p - \sqrt{p^2+\chi}] -\ln[e^{2\Xi}p + \sqrt{p^2+\chi}]\right) \\
=& \frac{e^{2\Xi} - \frac{p}{\sqrt{p^2+\chi}}}{e^{2\Xi}p-\sqrt{p^2+\chi}} -\frac{e^{2\Xi} + \frac{p}{\sqrt{p^2+\chi}}}{e^{2\Xi}p + \sqrt{p^2+\chi}}\\ 
%=& \frac{\left[e^{2\Xi} - \frac{p}{\sqrt{p^2+\chi}}\right]\left[e^{2\Xi}p + \sqrt{p^2+\chi}\right]-\left[e^{2\Xi} + \frac{p}{\sqrt{p^2+\chi}}\right]\left[e^{2\Xi}p-\sqrt{p^2+\chi}\right]}{e^{4\Xi}p^2-(p^2+\chi)}\\
% =& \frac{\left[e^{4\Xi}p  - e^{2\Xi}\frac{p^2}{\sqrt{p^2+\chi}} +e^{2\Xi}\sqrt{p^2+\chi} - p\right]}{e^{4\Xi}p^2-(p^2+\chi)}\nonumber\\
% & - \frac{\left[e^{4\Xi}p + e^{2\Xi}\frac{p^2}{\sqrt{p^2+\chi}} -e^{2\Xi}\sqrt{p^2+\chi}-p\right]}{e^{4\Xi}p^2-(p^2+\chi)}\\
% =& \frac{2e^{2\Xi}\sqrt{p^2+\chi} - 2e^{2\Xi}\frac{p^2}{\sqrt{p^2+\chi}} }{e^{4\Xi}p^2-(p^2+\chi)}\nonumber\\
=& \frac{2\chi e^{2\Xi}}{\sqrt{p^2+\chi}[e^{4\Xi}p^2-(p^2+\chi)]}\label{eq:TM_integration_by_parts}
\end{align}

Then 
\begin{align}
\Aboxed{\frac{d}{dp}\ln[1-r'^2 e^{-2pu}] &= 
2u\frac{r'^2 e^{-2pu}}{1-r'^2 e^{-2pu}}\left( 1 -\frac{2\chi e^{2\Xi}}{u\sqrt{p^2+\chi}[e^{4\Xi}p^2-(p^2+\chi)]}\right)}
\end{align}

% Ultimately, we want this to be proportional to 
 \begin{align}
 C=&\frac{r'^2 e^{-2pu}}{1-r'^2 e^{-2pu}}\left[1 - \left(\frac{1}{r'}-r'\right)\dfrac{\chi }{2u(p^2+\chi)p}\right]\\
% =&\frac{r'^2 e^{-2pu}}{1-r'^2 e^{-2pu}}\left[1 - \left(\frac{e^{2\Xi}p+\sqrt{p^2+\chi}}{e^{2\Xi}p-\sqrt{p^2+\chi}}
% -\frac{e^{2\Xi}p-\sqrt{p^2+\chi}}{e^{2\Xi}p+\sqrt{p^2+\chi}}\right)\dfrac{\chi }{2u(p^2+\chi)p}\right]\\
% =&\frac{r'^2 e^{-2pu}}{1-r'^2 e^{-2pu}}\left[1 - \frac{4pe^{2\Xi}\sqrt{p^2+\chi}}{e^{4\Xi}p^2-(p^2+\chi)}\dfrac{\chi }{2u(p^2+\chi)p}\right]\\
 =&\frac{r'^2 e^{-2pu}}{1-r'^2 e^{-2pu}}\left[1 - \frac{2\chi e^{2\Xi}}{u\sqrt{p^2+\chi}[e^{4\Xi}p^2-(p^2+\chi)]}\right]\label{eq:CTM}\\
 \end{align}
% So this will work.  

\begin{align}
E_{TM} = & -\frac{\hbar c}{8\pi^2 d^3}\int_0^\infty du\,u^{3} \int_1^\infty dp\, (p^2-1) 
\frac{r'^2e^{-2u p}}{1-r'^2e^{-2u p}}\left[1- \left(\frac{1}{r'}-r'\right)\dfrac{\chi }{2u(p^2+\chi)p}\right],
\end{align}
We can use Eqs.~(\ref{eq:CTM}) and (\ref{eq:TM_integration_by_parts}) to integrate with respect to $p$.  
\begin{align}
I_2 =& \int_1^\infty dp\, (p^2-1) \frac{r'^2e^{-2u p}}{1-r'^2e^{-2u p}}
\left[1- \left(\frac{1}{r'}-r'\right)\dfrac{\chi }{2u(p^2+\chi)p}\right]\\
=& \left[(p^2-1)\frac{1}{2u}\log[1-r'^2 e^{-2pu}]\right]_{p=1}^\infty - \int_1^\infty dp\,\frac{p}{u}\log[1-r'^2 e^{-2pu}]
\end{align}
Which gives us 
\begin{align}
\Aboxed{E_{TM}= & \frac{\hbar c}{8\pi^2 }\int_0^\infty d\xi\,\xi^{2} \int_1^\infty dp\, p \log[1-r'^2 e^{-2p\xi d}]}
\end{align}
\comment{Am I off by $1/4$?  That should be a matter of more careful accounting.}


\section{Finite Temperature and Dispersion}
\label{sec:nonzero_temp}
We will handle the finite temperature (and dispersion) in the atom-plane geometry.
  We derive the partition function for finite temperature for both the TE and TM polarizations.
  So far, I have done both Casimir and Casimir-Polder energies for TE, and only Casimir-Polder for TM.
  The zero temperature limits all work out nicely.
  I am stumbling a little over the right way to do the high temperature limit here.  


\subsection{Should really be using the Free Energy?}

So I read Babb's paper\footnote{Babb, J. F. and Klimchitskaya, G. L., and Mostepanenko, V. M., 
``Casimir-Polder interaction between an atom and a cavity wall under the influence of real conditions'',
 Phys. Rev. A, \textbf{70},042901,(2004)} (which Dan cites for thermal Casimir-Polder calculations.~\cite{Babb2004})
  In it they use the free energy, which is $\mathcal{F} = -k_BT\log Z$, as the basis of their calculations.
  I've been trying to use the mean energy, $E= -\partial_\beta\log Z$. 

% Tanmoy's initial calculation emphasized using $\mathcal{F}$ in the limit $\beta\rightarrow \infty$.
%   I subbed in using the mean energy angle because I (for some reason) felt more comfortable with that.
%   In the limit $T\rightarrow 0$ they of course agree - as borne out above.  

% However my finite temperature results using the energy are completely crap.
% If I actually use $\mathcal{F}$, then I think I reproduce Dan's  finite temperature result\footnote{Steck, D. A.,
%  ``Quantum Optics Notes'', Eq. (14.326)}: 
% \begin{equation}
%   V_{CP} = \frac{k_BT}{4\pi\epsilon_0c^2}{\sum_{n=0}^{\infty}}'s_n^2\alpha(is_n)
% \int_0^\infty dk_T \frac{k_T}{\kappa_n}\left[ r_\perp(\theta,is_n)+ 
%   \left(1 +\frac{2k_T^2c^2}{s_n^2}\right)r_\|(\theta,is_n)\right]e^{-2\kappa_nz},
% \end{equation}
% where $\kappa_n = \sqrt{s_n^2/c^2+k_T^2}$, and $r_\|, r_\perp$ still depend on $\cos(\theta)$.
%   I think after variable transformation they become the reflection coefficients we use above.  

% In the limit of zero temperature, then the factor of $\beta^{-1}$ will get eaten - 
% as the derivative has been doing for us. 
% For high temperature, I don't need to take any derivatives, and I get exactly the results I want.       

% \section{TE Polarization: Thermal Partition Function}

% Our partition function is 
% \begin{equation}
% Z_{TE} = \int D\phi \exp\left[ -\frac{\epsilon_0}{2}\int_0^\beta d\beta'\int d^3x\, 
% \left( \frac{\epsilon(x)}{\hbar^2}(\partial_{\beta'}\phi)^2 + c^2|\nabla\phi|^2\right)\right] .
% \end{equation}

% Let us change to using $\tau = \beta \hbar c$ as our temperature coordinate.  Then   
% \begin{equation}
% Z_{TE} = \int D\phi \exp\left[ -\frac{\epsilon_0 c^2 }{2 \hbar c}\int_0^\beta d\tau'\int d^3x\, 
% \left( \epsilon(x)(\partial_{\tau}\phi)^2 + |\nabla\phi|^2\right)\right] .
% \end{equation}

% We will now introduce the Matsubara frequencies $\omega_n = (2\pi n)/(\beta \hbar)$, with 
% \begin{equation}
% \phi(\beta,x) = \sum_{n=-\infty}^{\infty}e^{i\frac{\omega_n}{c}\tau} \phi_n(x),
% \end{equation}
% where $\tau = \beta\hbar c$, and the $\phi_n$ are complex variables.  We will also need to use  
% \begin{equation}
% \int_0^{\beta \hbar c}d\tau e^{i\frac{(\omega_n+\omega_m)}{c}\tau} = \beta\hbar c \delta_{n,-m},
% \end{equation}
% and $\phi_n^* = \phi_{-n}$ since $\phi^*(\beta, x) = \phi(\beta, x)$.  

% \begin{shaded}
% \comment{See Dec 2012 notes for how to handle the variable counting from doubling the number of variables}
% I'm just trying to figure out the factors of 2 here - just think of the transform to Matsubara 
% frequencies as a ordinary change of variables.
%   In December I had completely reduced the problem to a discrete problem in $\beta$ as well.
%   Say I have $N_\beta$ initial time-steps.
%   I have $N_\beta$ real variables.
%   If I introduce a Fourier series, then I now have $N-2$ complex variables for $0<n<N_\beta/2$, and 2 real variables at $n=0, N_\beta/2$.
%   Furthermore we know that $\phi_n^* = \phi_{-n}$.
%   So from the Gaussian structure of the integrals over $\phi_n$, each of these $(N-2)$ integrals over $\phi_n$ is equal.
%   When I carry out the integrals I have 
% \begin{equation}
% \prod_{n=-N_\beta/2}^{N_\beta/2-1}\int D\phi_n e^{-A_n|\phi_n|^2} 
% = C\left( \frac{1}{\sqrt{A_0}}\frac{1}{\sqrt{A_{N_\beta/2}}}\prod_{n>0}\frac{1}{A_n}\right),
% \end{equation}
% where I have used 
% \begin{equation}
% \int D\phi_nD\phi_n^* e^{-A_n|\phi_n|^2} = \int D\phi_r D\phi_ie^{-A_n(\phi_r^2+\phi_n^2)} = \frac{1}{A_n}, 
% \end{equation}
% where $\phi_n = \phi_r + i \phi_i$.
%   We then consider the limit where $N_\beta\rightarrow \infty$.  

% Does this still work with $\epsilon(i\omega)?$  What do the Kramers-Kr\"onig relations have to say?
%   Is $\epsilon^*(i\omega) = \epsilon(-i\omega)$?
%   Or is it just a result of saying this Hamiltonian is real.
%   If so, we're golden.  
% \end{shaded}

% Then we have 
% \begin{equation}
% Z_{TE} = \prod_{n=-\infty}^{\infty} \int D\phi_n\exp\left[ -\frac{\beta \epsilon_0 c^2 }{2}
% \int d^3x\, \phi_n^*\left(\epsilon(x)\frac{\omega_n^2}{c^2} -\nabla^2\right)\phi_n\right] .
% \end{equation}

% Now let's \comment{assume} we just handle dispersion by taking $\epsilon(x)\rightarrow \epsilon(i\omega_n,x)$.
%   This idea is following Rahi's derivation where $\epsilon$ is treated as an effective action 
% where the effect of the electron field has been intregrated out.  
% Then 
% \begin{equation}
% \log Z_{TE} = -{\sum_{n=0}^\infty}'\log\det\left[\frac{1}{2}
% \left(\epsilon(i\omega_n,\vect{x})\frac{\omega_n^2}{c^2} -\nabla^2\right)\right].
% \end{equation}
% where there is only a factor of $\frac{1}{2}$ for $n=0$.
%   We will renormalize this against vacuum, 
% \begin{equation}
% \log Z_{TE} -\log Z_0= -{\sum_{n=0}^\infty}'\left\{\log\det\left[ 
% \frac{1}{2}\left(\epsilon(i\omega_n,\vect{x})\frac{\omega_n^2}{c^2} -\nabla^2\right)\right] 
% - \log\det\left[ \frac{1}{2}\left(\frac{\omega_n^2}{c^2} -\nabla^2\right)\right]\right\}
% \end{equation}
% Note that for $n=0$ the Matsubara frequency $\omega_n=0$, so we have 
% \begin{align}
%   \log\det\left[ \frac{1}{2}\left(\epsilon(0,\vect{x})\frac{\omega_0^2}{c^2} -\nabla^2\right)\right]
%  - \log\det\left[ \frac{1}{2}\left(\frac{\omega_0^2}{c^2} -\nabla^2\right)\right] = 0.
% \end{align}
% This assumes that $\epsilon(\omega)$ has at most a simple pole at zero frequency, 
% such that $\lim_{\omega\rightarrow 0}\omega^2\epsilon(\omega)=0.$    

% Now the renormalized free energy is 
% \begin{align}
% F-F_0 & = -k_BT \log \frac{Z_{TE}}{Z_0} \\
% % & = k_BT {\sum_{n}}'\tr\left\{ \log\left[ \frac{1}{2}
% %     \left(\epsilon(i\omega_n,\vect{x})\frac{\omega_n^2}{c^2} -\nabla^2\right)\right]
% % -\log\left[ \frac{1}{2}\left(\frac{\omega_n^2}{c^2} -\nabla^2\right)\right]\right\}\\
% &=k_BT{\sum_n}'\int_0^\infty \frac{dT}{T}\int d^3x\,\frac{1}{(2\pi T)^{3/2}}
% \dlangle e^{-T\frac{\omega_n^2}{2c^2}} -  e^{-\frac{ T \omega_n^2\langle\epsilon(i\omega_n)\rangle}{2c^2}}\drangle
% \label{eq:TEworldline_partition_function}
% \end{align}
% where we introduced the path integral.
%   In this case the operator $ e^{T\nabla^2}$ only needs a 3-dimensional Hilbert space.
%   So we also only use the normalization for 3D.
%   The remaining factor of $\sqrt{2\pi T}$ will emerge in the zero temperature limit.
%   This scaling with $T$ also reflects the different scaling behaviours in the near-field, 
% thermal and far-field regions, as these will each have different approximations to the Matsubara sum.  

% \begin{shaded}
% The sequence of events we actually followed when we neglected dispersion from the start was
% \begin{enumerate}
% \item Take zero temperature limit, which lets us treat $\omega_n^2$ as $\partial_\tau^2$ in the field partition function.  
% \begin{align}
% \log Z & = -\frac{1}{2}\log\det[-\frac{1}{2}\epsilon\partial_\tau^2 - \frac{1}{2}\nabla^2] \\
% & = \frac{1}{2}\int d^4x_0 \int \frac{dT}{T} \langle x_0| e^{[\epsilon(\hat{x})T+\nabla^2]T/2}|x_0\rangle\\
% & = \frac{1}{8\pi^2}\int_0^{\beta\hbar c}d\tau_0\int d^3x_0 \int \frac{dT}{T^3} \dlangle \frac{1}{\sqrt{\langle \epsilon\rangle}}\drangle
% \end{align}
% \item Take $k_BT \log Z$ - which only actually affects $\int d\tau_0 = \beta\hbar c$, which is now a coordinate
% \begin{equation}
% E = -\frac{\hbar c}{8\pi^2}\int d^3x_0 \int \frac{dT}{T^3} \dlangle\frac{1}{\sqrt{\langle \epsilon\rangle}}\drangle
% \end{equation}
% \item Expand $\epsilon(x)$ to linear order in $\alpha_0/\epsilon_0$.  All functions are just functions of space.  
% \begin{equation}
% E(x') = \frac{\hbar c\alpha_0}{16\pi^2\epsilon_0} \int \frac{dT}{T^3} \dlangle \frac{1}{\langle \epsilon\rangle^{3/2}}\drangle_{x(0)=x'}.
% \end{equation}

% \end{enumerate}
% \end{shaded}

% \begin{shaded}
% \subsection{First thing I tried, and evidently the wrong thing to do}

% (Note that using the free-energy would entirely bypass these concerns : No extra derivatives, no problems.)

% So I think it is correct to start from:
% \begin{equation}
% E=-\partial_\beta \log Z = -\partial_\beta{\sum_n}'\int_0^\infty \frac{dT}{T}\int d^3x\,\frac{1}{(2\pi T)^{3/2}}\dlangle  - e^{-\frac{ T \omega_n^2\langle\epsilon(i\omega_n)\rangle}{2c^2}}\drangle
% \end{equation}

% For clarity, from here on I will suppress the renormalization terms.  They can be restored by subtracting off the same thing but with $\epsilon\rightarrow 1$ everywhere.  

% Our sequence of operations once we got $\log Z$ in this case was: 
% \begin{enumerate}
% \item Take $\partial_\beta\log Z$ to get the energy.  
% \begin{align}
% E&={\sum_n}'\frac{\omega_n^2}{\beta c^2}\int d^3x_0\int_0^\infty dT\,\frac{1}{(2\pi T)^{3/2}}\dlangle  \left( \langle\epsilon(i\omega_n)\rangle +\frac{i}{2}\omega_n\langle\partial_\omega\epsilon(i\omega_n)\rangle \right)e^{-\frac{ T \omega_n^2\langle\epsilon(i\omega_n)\rangle}{2c^2}}\drangle
% \end{align}

% \item Expand to linear order in $\alpha$

% \begin{align}
% E&={\sum_n}'\frac{\omega_n^2}{\epsilon_0\beta c^2}\int_0^\infty dT\,\frac{1}{(2\pi T)^{3/2}}\dlangle  \left(\alpha(i\omega_n)+\frac{i}{2}\omega_n\partial_\omega\alpha(i\omega_n) - \frac{ T \omega_n^2\alpha(i\omega_n)}{2c^2} \right)e^{-\frac{ T \omega_n^2\langle\epsilon(i\omega_n)\rangle}{2c^2}}\drangle_{x_0=x'}
% \end{align}
% \item Take zero temperature limit, which replaces $\omega_n\rightarrow \omega,\sum_n \rightarrow \int_0^\infty d\omega$.  Also take the far-field limit on the frequency integral.  $\alpha(i\omega)\rightarrow\alpha_0, \epsilon(i\omega,x)\rightarrow \epsilon(x)$.  
% \begin{equation}
% E=\frac{\hbar}{2\pi\epsilon_0 c^2}\int_0^\infty d\omega\,\omega^2\int_0^\infty dT\,\frac{1}{(2\pi T)^{3/2}}\dlangle   \left(\alpha_0 - \frac{ T \omega^2\alpha_0}{2c^2} \right)e^{-\frac{ T \omega^2\langle\epsilon\rangle}{2c^2}}\drangle
% \end{equation}
% \item Carry out the Gaussian integral in frequency.  
% \begin{equation}
% E= \frac{\hbar c\alpha_0}{8\pi^2 \epsilon_0}\int_0^\infty \frac{dT}{T^3}\dlangle  \left( \frac{1}{\langle \epsilon\rangle^{3/2}}-\frac{3}{2\langle\epsilon\rangle^{5/2}} \right) \drangle.
% \end{equation}
% \end{enumerate}
% Perhaps the derivative $\partial_\beta$ does not commute with some of these limits and approximations?  Perhaps this should be delayed to the final stage?  A derivative of a sum, is not necessarily equal to the sum of derivatives?  

% Of those steps, I think you can safely swap  3 and 4.  But I think these must precede step 5, otherwise you don't have an integral, or even a Gaussian one.  
% \end{shaded}

\subsection{Casimir-Polder energy}
We can extract the Casimir-Polder energy by introducing a test-particle,
 with $\epsilon(\omega,\vect{x})\rightarrow \epsilon(\omega,\vect{x})
+\frac{\alpha(i\omega)}{\epsilon_0}\delta(\vect{x}-\vect{x}_0)$.
  The factor of $\epsilon_0$ arises since this is the relative permittivity.
  Some care is necessary with the $\delta$-function, and should really be considered the limiting 
result of some function like $f(x,\sigma) = e^{-x^2/(2\sigma^2)}/\sqrt{2\pi\sigma^2}$.
  We will then expand the energy to linear order in $\alpha$.  

\begin{align}
E-E_0&=k_BT\sum'_n\,\int_0^\infty \frac{dT}{T}\int d^3x\,\frac{1}{(2\pi T)^{3/2}}
\dlangle e^{-T\frac{\omega_n^2}{2c^2}} - \left(1 - \frac{ T\omega_n^2}{2\epsilon_0c^2}\alpha(i\omega_n)
\langle \delta(x-x')\rangle\right)e^{-\frac{ T \omega_n^2\langle\epsilon(i\omega_n)\rangle}{2c^2}}\drangle
\end{align}
I think the first two terms the energy due to the dielectrics on their own, without the atom, 
and such can be subtracted from the atom-wall interaction energy.
  The physical renormalization is to expand the energy for just a polarizable particle,
 and consider the energy change with and without the presence of the other dielectric.
  So we subtract off the same energy, but with $\epsilon=1$ everywhere.

We can also simplify this a bit by noting that 
\begin{equation}
\int d^3x_0\, \dlangle \frac{1}{T}\int_0^T dt \delta(x(t)-x')\drangle = \int d^3x_0 \delta(x_0-x'),
\end{equation}
This follows from considering the discrete form of the path integral,
 $\int d^3x_0 \prod_i\int d^3x_i f(x_i-x_{i-1})$, where the integral is cyclic under permutations of indices.  
In this way we can always relabel the point that passes through $x_0$ to be the starting point of the loop.
   So we will then only consider loops that start and return to $x_0 = x'$.
   Mathematically, we will use this delta function to eliminate the integral over $\int d^3x_0$.  

The renormalized result is 
\begin{align}
E-E_0&=-k_BT{\sum_n}\,\frac{ \omega_n^2}{2\epsilon_0c^2}\alpha(i\omega_n)\int_0^\infty dT\,
\frac{1}{(2\pi T)^{3/2}}\dlangle e^{-T\frac{\omega_n^2}{2c^2}} 
-e^{-\frac{ T \omega_n^2\langle\epsilon(i\omega_n)\rangle}{2c^2}}\drangle\label{eq:TE_thermal_energy}
\end{align}

We will focus on the Casimir-Polder result, since I can then compare to Dan's expressions in the notes.
  For the Casimir results, I will collate some results from the literature to compare in the nonzero
 temperature/dispersive cases.  


% \begin{align}
% E-E_0&=-k_BT \frac{\hbar\beta}{2\pi}\int_{-\infty}^\infty d\omega\int_0^\infty \frac{dT}{T}
%\int d^3x\,\frac{1}{(2\pi T)^{3/2}}\dlangle e^{-T\frac{\omega^2}{2c^2}} 
%-  e^{-\frac{ T \omega^2\langle\epsilon(i\omega)\rangle}{2c^2}}\drangle,
% \end{align}


% \subsection{Far-field approximation, zero temperature}

% Let us consider how to take the far-field approximation from these expressions.  Since we are taking an ensemble average over Gaussian random walks, so the loops will intersect all the surfaces when $T\sim d^2$, where $d$ is the distance from the source point $x_0$ to the farthest surface.  Secondly, the frequency integral is dominated by the exponential factors, which will contribute most when $T\omega^2/c^2\sim 1$.  This suggests that frequencies with $d^2\omega^2/c^2<1$ will contribute most.  In the limit where $d/c$ is large, only frequencies near $0$ will contribute, and we can approximate $\epsilon(i\omega) \approx \epsilon(0)$ everywhere.



% \begin{shaded}
% One question is, can we say $\lim_{\omega\rightarrow 0}\partial_\omega\alpha(i\omega)= 0?$  
%If we approximate the atom as a harmonic oscillator \footnotemark
%  with damping $\gamma$, and resonant frequency $\omega_0$   , then 
% \begin{equation}
% \alpha(i\omega)= \frac{e/m}{\omega_0^2+\omega^2+\gamma\omega},
% \end{equation}
% whereas if we do a quantum mechanical perturbation theory calculation \footnotemark  we get
% \begin{equation}
% \alpha(i\omega) =\sum_j \frac{2\omega_{j0}|\langle g | d_z|e_j\rangle|^2}{2\hbar(\omega_{j0}^2+\omega^2)},
% \end{equation}
% where $\omega_{j0}$ are the transition frequencies.
%  Evidently in both cases $\lim_{\omega\rightarrow 0}\partial_\omega\alpha(i\omega)=0$, if $\gamma=0$.   
% \end{shaded}
% \footnotetext{Rosa, F.S.S, Dalvit, D. A. R. and Milonni, P. W., Phys. Rev. A, \textbf{84},053813,(2011), ``Electromagnetic energy, absorption, and Casimir Forces, II. Inhomogenous dielectric media''}
% \footnotetext{Steck, Daniel A. ``Quantum Optics Notes'', Eq.(14.146) for the scalar Kramers-Heisenberg formula, see also Eq.(14.152).}

\subsubsection{Feynman-Kac formula}

From Dan's work (or my re-working of it) we have the Laplace-Mellin transform for the single body Feynman-Kac formula,
% \begin{align}
% \int_0^\infty \frac{dT}{T^{1+z}}\dlangle e^{-s[T+ \chi\int_0^Tdt\Theta(x-d)]}\drangle 
% =& \int_0^\infty \frac{dT}{T^{1+z-1/2}}e^{-sT}\dlangle \frac{e^{-s \chi \int _0^T dt \Theta(x-d)}}{\sqrt{T}}\drangle\\
% =& \frac{1}{\Gamma[z+1/2]}\int_0^\infty d\lambda\, \lambda^{z-1/2}\int_0^\infty dT e^{-(\lambda+s)T}
% \dlangle \frac{e^{-s \chi \int_0^T dt \Theta(x-d)}}{\sqrt{T}}\drangle.
% \end{align}
% In our case $z=1/2$, and $s= \omega^2/(2c^2)$.
%   We also need the actual analytical expression for that path integral,
% \begin{equation}
% \int_0^\infty dT e^{-(\lambda+s) T} \dlangle \frac{e^{-s\chi\int_0^T dt \Theta(x-d)}}{\sqrt{2\pi T}}\drangle  
% =\frac{1}{\sqrt{2(\lambda+s)}}\left[1 - e^{-2\sqrt{2(\lambda+s)}|d|}\frac{\sqrt{\lambda+s(1+\chi)}
% -\sqrt{\lambda+s}}{\sqrt{\lambda+s(1+\chi)}+\sqrt{\lambda+s}}\right].
% \end{equation}
We will need to apply both of these results as 
\begin{align}
\int_0^\infty dT\,\frac{1}{(2\pi T)^{3/2}}\dlangle e^{-sT} - e^{-sT \langle\epsilon(i\omega)\rangle}\drangle 
& =\frac{1}{2\pi}\int_0^\infty d\lambda\, \frac{e^{-2\sqrt{2(\lambda+s)}|d|}}{\sqrt{2(\lambda+s)}}
\frac{\sqrt{\lambda+s[1+\chi(i\omega)]}-\sqrt{\lambda+s}}{\sqrt{\lambda+s[1+\chi(i\omega)]}+\sqrt{\lambda+s}}
\end{align}

On plugging this in to Eq.~(\ref{eq:TE_thermal_energy}) we have
\begin{align}
E-E_0&=-k_BT{\sum_n}'\frac{\omega_n^2\alpha(i\omega_n)}{4\pi\epsilon_0c^2}\int_0^\infty d\lambda\, 
\frac{e^{-2\sqrt{2(\lambda+\omega_n^2/(2c^2))}|d|}}{\sqrt{2(\lambda+\omega_n^2/(2c^2))}}
\frac{\sqrt{\lambda+\omega_n^2/(2c^2)[1+\chi(i\omega_n)]}-\sqrt{\lambda+\omega_n^2/(2c^2)}}
{\sqrt{\lambda+\omega_n^2/(2c^2)[1+\chi(i\omega_n)]}+\sqrt{\lambda+\omega_n^2/(2c^2)}},
\end{align}
Let's make a couple variable changes to put this into a more tractable form.  
First we change integration variable using $\lambda = \kappa \omega_n^2/(2c^2)$.  
\begin{align}
E-E_0&=-k_BT{\sum_n}'\frac{\omega_n^2\alpha(i\omega_n)}{4\pi\epsilon_0c^2}\frac{\omega_n^2}{2c^2}
\int_0^\infty d\kappa\, \frac{e^{-2\omega_n\sqrt{(\kappa+1)}|d|/c}}{\omega_n\sqrt{(\kappa+1)}/c}
\frac{\sqrt{\kappa+1+\chi(i\omega_n)]}-\sqrt{\kappa+1}}{\sqrt{\kappa+1+\chi(i\omega_n)}+\sqrt{\kappa+1}},
\end{align}
Next we define $\kappa +1= p^2$.  
\begin{align}
E-E_0&=-k_BT{\sum_n}'\frac{\omega_n^3\alpha(i\omega_n)}{4\pi\epsilon_0c^3}\int_1^\infty dp\,e^{-2\omega_n p|d|/c}
\frac{\sqrt{p^2+\chi(i\omega_n)}-p}{\sqrt{p^2+\chi(i\omega_n)}+p},
\label{eq:TE_CP_finite_temperature}
\end{align}
This is the general result for finite temperature and dispersion.
  We can also take the zero temperature limit.
  In the limit $\beta\rightarrow \infty$ the difference between Matsubara frequencies approaches zero,
 $\Delta\omega_n =\frac{2\pi}{\beta\hbar}$.  Then we can take $\sum_n\Delta\omega_n \rightarrow \int_0^\infty d\omega$.
\begin{align}
E-E_0&=-\frac{\hbar}{8\pi^2\epsilon_0c^3}\int_0^\infty d\omega\,\omega^3\alpha(i\omega)
\int_1^\infty dp\,e^{-2\omega p|d|/c}\frac{\sqrt{p^2+\chi(i\omega)}-p}{\sqrt{p^2+\chi(i\omega)}+p},\label{eq:TE_CP_zero_temperature}
\end{align}
where now $\omega$ is a continuous variable.  

\subsection{Various Limiting Cases}

Let us now consider the various limits for the Casimir-Polder case.
  Since we are taking an ensemble average over Gaussian random walks,
 the loops will intersect all the surfaces when $T\sim d^2$, 
where $d$ is the distance from the source point $x_0$ to the farthest surface.
  Secondly, the frequency sum/integral is dominated by the exponential factors,
 which will contribute most when $T\omega_n^2/c^2\sim 1$.
  This suggests that frequencies with  $\omega_n^2< c^2/d^2$  will contribute most.   

\subsubsection{Zero temperature, far-field limit}

We start from Eq.~(\ref{eq:TE_CP_zero_temperature}).
  If we also take the limit where $d/c$ is large, only frequencies near $0$ will contribute,
 and we can approximate $\epsilon(i\omega) \approx \epsilon(0)$, $\alpha(i\omega)\approx\alpha_0$ everywhere.  
\comment{Does this also only work up to a certain distance, 
at which point the integral crosses over to High temperature,since there is always some thermal background.
  e.g. infra-red radiation. }
\begin{align}
E-E_0&=-\frac{\hbar}{8\pi^2\epsilon_0c^3}\int_0^\infty d\omega\,\omega^3\alpha(i\omega)
\int_1^\infty dp\,e^{-2\omega p|d|/c}\frac{\sqrt{p^2+\chi(0)}-p}{\sqrt{p^2+\chi(0)}+p}
\end{align}

 Now evaluate the $\omega$ integral, 
\begin{equation}
\int_{0}^\infty d\omega\,\omega^3e^{-2\omega r d/c} = \frac{3 c^4}{8 p^4 d^4}
\end{equation}
which leaves a by now familiar integral:
\begin{align}
E-E_0&= -\frac{3\hbar c\alpha_0}{64\pi^2 \epsilon_0 d^4}\int_1^\infty dp\,p^{-4}\frac{\sqrt{p^2+\chi}-p}{\sqrt{p^2+\chi}+p}
\end{align}

\subsubsection{Near field, low temperature}
Let's now work in the limit where $d<<c/\omega_{j0}$.
  In this case all frequencies contribute, but we can convert the sum into an integral.
  The difference here is that all of the frequency dependence of $\epsilon(\omega)$  will matter.  
\begin{equation}
E-E_0=-\frac{\hbar}{8\pi^2\epsilon_0c^3}\int_0^\infty d\omega\,\omega^3\alpha(i\omega)
\int_1^\infty dp\,e^{-2\omega p|d|/c}\frac{\sqrt{p^2+\chi(i\omega)}-p}{\sqrt{p^2+\chi(i\omega)}+p}
\end{equation}
The integral contributes most when the exponent is order unity.
  The presence of $\alpha$ means that frequencies around $\omega_{j0}$ will dominate the frequency integral.
  Then $p \sim  c/(d\omega_{j0})\gg 1$.
  Let's use that fact to approximate the reflection coefficient, 
and see if we can reproduce the known van der Waals result.  
\begin{align}
  \frac{\sqrt{p^2+\chi(i\omega)}-p}{\sqrt{p^2+\chi(i\omega)}+p}\approx 
& \frac{ p + \frac{\chi}{2p}-p}{2p+\frac{\chi(i\omega)}{2p}} \approx \frac{\chi}{4p^2} 
\end{align}
Plugging this in, we can then evaluate the $p$ integral
\begin{align}
E-E_0=&-\frac{\hbar}{8\pi^2\epsilon_0c^3}\int_0^\infty d\omega\,\omega^3\alpha(i\omega)\chi(i\omega)
\int_1^\infty dp\,\frac{1}{4p^2}e^{-2\omega p|d|/c}
\end{align}

Let's try to work on that integral a bit.  
\begin{align}
\int_1^\infty dp\,\frac{1}{4p^2}e^{-2\omega p|d|/c}%  =& -\frac{1}{4p}e^{-2\omega p d/c}\bigg|_{p=1}^{\infty}
 % + \int_1^\infty dp\, \frac{1}{4p}\times \frac{-2\omega d}{c}e^{-2\omega pd/c}\\
=& \frac{1}{4}e^{-2\omega d/c} - \frac{2\omega d}{c} \int_1^\infty dp\, \frac{e^{-2\omega pd/c}}{p}.
\end{align}

Recall, we are working in the so-called near-field limit where $d\omega_{j0}/c<<1.$  
I think we can get away with approximating this as just the exponential term.  

\begin{align}
E-E_0=&-\frac{\hbar}{32\pi^2\epsilon_0c^3}\int_0^\infty d\omega\,\omega^3\alpha(i\omega)\chi(i\omega)e^{-2\omega d/c}\\
=&-\frac{\hbar}{32\pi^2\epsilon_0 d^3}\int_0^\infty d\omega\,\frac{\omega^3d^3}{c^3}\alpha(i\omega)\chi(i\omega)e^{-2\omega d/c}\approx 0
\end{align}
Since I think we are working wiht $d\omega/c$ is very small, so this term is tiny.  

(From Dan's analysis in the notes, apparently we can drop this term, or rather it is negligible in comparison to the $TM$ contribution.)


\subsubsection{High Temperature, (far field ?)}  

As we noted earlier, the renormalized partition function vanishes for $\omega_0$.
The leading term is $\omega_1$, which will be exponentially suppressed relative to the TM contribution.  


% This is the general case.  

% The presence of $i\langle \partial_\omega\epsilon(i\omega_n)$ in our expression is acceptable,
% since $\epsilon(i\omega_n)$ is in itself a real function, so $k_BT\epsilon(i\omega_n)$ is also real.
%  As it stands, this factor of $i$ will then combine with further factors of $i$ from the form of the derivative.
%  For example, the response of a harmonic oscillator with frequency  $\omega_0$ is $\epsilon(\omega) = A/(\omega^2-\omega_0^2)$.
%  The derivative is $\partial_\omega\epsilon(\omega) = -2A\omega/(\omega^2-\omega_0^2)^2$.  
% Now if we consider imaginary frequencies then $\epsilon(i\omega_n) = -A/(\omega_n^2+\omega_0^2)$, 
%and $i\partial_\omega \epsilon(\omega)\big|_{\omega=i\omega_n} = -i (iA\omega_n)/(\omega_n^2+\omega_0^2)^2$, which is real.
%    We would get the same result in evaluating $k_BT\epsilon(i\omega_n)$ directly.  


% \subsection{TE Polarization: Casimir}

% Let's try to do this for the Casimir energy due to TE as well.  
% We will start from 
% \begin{equation}
% E-E_0=k_BT{\sum_n}'\int_0^\infty \frac{dT}{T}\int dx\,\frac{1}{(2\pi T)^{3/2}}\dlangle e^{-T\frac{\omega_n^2}{2c^2}}
%  -  e^{-\frac{ T \omega_n^2\langle\epsilon(i\omega_n)\rangle}{2c^2}}\drangle\label{eq:casimir_partition_function}
% \end{equation}
% We will need to also subtract off the renormalized one body energies.  
% As before, we need the Feynman-Kac formula,
% \begin{align}
% &\int dx\int_0^\infty dT \frac{e^{-\lambda T}}{\sqrt{2\pi T}}\left[e^{-T\langle\epsilon_{12}\rangle}+
%  e^{-T\langle\epsilon_{0}\rangle} - e^{-T\langle\epsilon_{1}\rangle}- e^{-T\langle\epsilon_{2}\rangle}\right]\nonumber\\ 
% =&  \frac{u_1u_2 e^{-2\sqrt{2\lambda}d}}{\sqrt{2\lambda}(1-u_1u_2 e^{-2\sqrt{2\lambda}d})}
% \left( 2d + \frac{\sqrt{2}}{\sqrt{(\lambda+\chi_1)}} + \frac{\sqrt{2}}{\sqrt{(\lambda+\chi_2)}} \right)
% \end{align}
% where $u_i = (\sqrt{\lambda}-\sqrt{\lambda+\chi})/(\sqrt{\lambda}+\sqrt{\lambda+\chi})$.
%   We will also need 
% \begin{align}
% \int_0^\infty \frac{dT}{T^{1+z}}\dlangle e^{-s[T+ \chi\int_0^Tdt\Theta(x-d)]}\drangle=& 
% \frac{1}{\Gamma[z+1/2]}\int_0^\infty d\lambda\, \lambda^{z-1/2}\int_0^\infty dT e^{-(\lambda+s)T}
% \dlangle \frac{e^{-s \chi \int_0^T dt \Theta(x-d)}}{\sqrt{T}}\drangle.
% \end{align}
% In this case $z=3/2$, and $s= \omega^2/(2c^2)$, so we need to take 
% $\lambda\rightarrow \lambda+ \omega_n^2/(2c^2)$, and $\chi\rightarrow s\omega_n^2/(2c^2)$.
%   Putting these identities together in Eq.~(\ref{eq:casimir_partition_function})yields 
% \begin{align}
% E-E_0=&k_BT{\sum_n}'\frac{1}{\Gamma[2]2\pi}\int_0^\infty d\lambda\, \lambda 
%  \frac{u_1u_2 e^{-2\sqrt{2\lambda+\omega_n^2/c^2}d}}{\sqrt{2\lambda+\omega_n^2/c^2}(1-u_1u_2 e^{-2\sqrt{2\lambda+\omega_n^2/c^2}d})}\nonumber\\
% &\times \left( 2d + \frac{\sqrt{2}}{\sqrt{\lambda+\omega_n^2/(2c^2)(1+\chi_1)}}
%  + \frac{\sqrt{2}}{\sqrt{\lambda+\omega_n^2/(2c^2)/(1+\chi_2)}} \right)
% \end{align}
% with 
% \begin{equation}
% u_i = \frac{\sqrt{\lambda+\omega_n^2/(2c^2)}-\sqrt{\lambda+\omega_n^2/(2c^2)(1+\chi_i)}}
% {\sqrt{\lambda+\omega_n^2/(2c^2)}+\sqrt{\lambda+\omega_n^2/(2c^2)(1+\chi_i)}}
% \end{equation}

% We'll now make some variable changes to put this in a more tractable form.
%   First up, let's define $\lambda = \kappa \omega_n^2/(2c^2)$.  
% \begin{align}
% E-E_0%=&k_BT{\sum_n}'\frac{1}{2\pi}\int_0^\infty d\kappa\,\frac{\omega_n^4}{4c^4} \kappa  \frac{cu_1u_2 e^{-2\omega_n\sqrt{\kappa+1}d/c}}{\omega_n\sqrt{\kappa+1}(1-u_1u_2 e^{-2\omega_n\sqrt{\kappa+1}d/c})}\nonumber\\
% %&\times \left( 2d + \frac{2c}{\omega_n\sqrt{\kappa+1+\chi_1}} + \frac{2 c}{\omega_n\sqrt{\kappa+1+\chi_2}} \right)\\
% =&k_BT{\sum_n}'\frac{1}{2\pi}\int_0^\infty d\kappa\,\frac{\omega_n^2}{2c^2} \kappa 
%  \frac{u_1u_2 e^{-2\omega_n\sqrt{\kappa+1}d/c}}{\sqrt{\kappa+1}(1-u_1u_2 e^{-2\omega_n\sqrt{\kappa+1}d/c})}
% \left( \frac{\omega_nd}{c} + \frac{1}{\sqrt{\kappa+1+\chi_1}} + \frac{1}{\sqrt{\kappa+1+\chi_2}} \right)
% \end{align}
% with
% \begin{equation}
% u_i = \frac{\sqrt{\kappa+1}-\sqrt{\kappa+1+\chi_i}}{\sqrt{\kappa +1}+\sqrt{\kappa+1+\chi_i}}.
% \end{equation}
% Next up define $\kappa+1 = p^2$.  
% \begin{align}
% E-E_0%=&k_BT{\sum_n}'\frac{\omega_n^2}{4\pi c^2}\int_1^\infty dp\,2p (p^2-1)  \frac{u_1u_2 e^{-2\omega_n pd/c}}{p(1-u_1u_2 e^{-2\omega_npd/c})}\left( \frac{\omega_nd}{c} + \frac{1}{\sqrt{p^2+\chi_1}} + \frac{1}{\sqrt{p^2+\chi_2}} \right)\\
% =&k_BT{\sum_n}'\frac{\omega_n^2}{2\pi c^2}\int_1^\infty dp\,(p^2-1)  
% \frac{u_1u_2 e^{-2\omega_n pd/c}}{(1-u_1u_2 e^{-2\omega_npd/c})}
% \left( \frac{\omega_nd}{c} + \frac{1}{\sqrt{p^2+\chi_1}} + \frac{1}{\sqrt{p^2+\chi_2}} \right)
% \end{align}
% with
% \begin{equation}
% u_i = \frac{p-\sqrt{p^2+\chi_i}}{p+\sqrt{p^2+\chi_i}}.
% \end{equation}
% Finally, let's integrate by parts with respect to $p$.  
% \begin{shaded}
%  First up take the derivative of the reflection coefficient, 
% \begin{align}
% \frac{dr}{dp} =& \frac{d}{dp} \frac{p-\sqrt{p^2+\chi}}{p+\sqrt{p^2+\chi}}
% = \frac{1-\frac{2p}{2\sqrt{p^2+\chi}}}{p+\sqrt{p^2+\chi}} - (p-\sqrt{p^2+\chi})
% \frac{(1+\frac{2p}{2\sqrt{p^2+\chi}})}{(p+\sqrt{p^2+\chi})^2} 
% %=& \frac{1}{\sqrt{p^2+\chi}}\frac{\sqrt{p^2+\chi}-p}{p+\sqrt{p^2+\chi}} - (p-\sqrt{p^2+\chi})\frac{p+\sqrt{p^2+\chi}}{\sqrt{p^2+\chi}(p+\sqrt{p^2+\chi})^2} \\
% = -\frac{2r}{\sqrt{p^2+\chi}}
% \end{align}
% We can also write:
% \begin{align}
% \frac{d}{dp}[1-r_1r_2 e^{-2p\omega_n d/c }]% &= -\left( r_1\frac{dr_2}{dp} e^{-2p\omega_n d/c} + r_2\frac{dr_1}{dp} e^{-2p\omega_n d/c} - 2\xi r_1r_2d e^{-2p\omega_n d/c}\right)\\
% % &= -\left( -2 \frac{r_1r_2}{\sqrt{p^2+\chi_2}} -2 \frac{r_1r_2}{\sqrt{p^2+\chi_1}}- 2r_1r_2\frac{\omega_nd}{c} \right)e^{-2p\omega_n d/c}\\
% &= 2\left(\frac{\omega_nd}{c} +\frac{1}{\sqrt{p^2+\chi_1}} +\frac{1}{\sqrt{p^2+\chi_2}}\right) r_1r_2e^{-2p\omega_n d/c},
% \end{align}
% which suggests 
% \begin{align}
% \Aboxed{\frac{d}{dp}\ln[1-r_1r_2 e^{-2p\omega_n d/c}]
% &= \frac{2r_1r_2 e^{-2p\omega_n d/c}}{1 - r_1r_2  e^{-2p\omega_n d/c}}
% \left(\frac{\omega_n d}{c}+\frac{1}{\sqrt{p^2+\chi_1}}+\frac{1}{\sqrt{p^2+\chi_2}}\right)}
% \end{align}
% \end{shaded}

% So after integration by parts our the renormalized Casimir energy becomes
% \begin{align}
% E-E_0 & % = k_BT{\sum_n}'\frac{\omega_n^2}{2\pi c^2}\int_1^\infty dp\,(p^2-1)\frac{ r_1r_2e^{-2\omega_n p d/c}}{(1 -r_1r_2 e^{-2\omega_n pd/c})}\left[ \frac{\omega_n d}{c} +\frac{1}{\sqrt{p^2+\chi_1}}+\frac{1}{\sqrt{p^2+\chi_2}}\right]\\
% % & = k_BT{\sum_n}'\frac{\omega_n^2}{2\pi c^2}\left\{\frac{1}{2}\log\left[1-r_1r_2 e^{-2\omega_n p d/c}\right](p^2-1)\bigg|_{p=1}^\infty - \int_1^\infty dp \,p\log\left[1-r_1r_2 e^{-2\omega_n p d/c}\right]\right\}\\
% & = -k_BT{\sum_n}'\frac{\omega_n^2}{2\pi c^2}\int_1^\infty dp \,p
% \log\left[1-r_1r_2 e^{-2\omega_n p d/c}\right]\label{eq:Casimir_energy_finite_temperature}
% \end{align}
% We can take the zero temperature limit here as well:
% \begin{align}
% E-E_0& = -\frac{\hbar c}{4\pi^2}\int_0^\infty dk\,k^2\int_1^\infty dp \,p
% \log\left[1-r_1r_2 e^{-2k p d}\right]\label{eq:Casimir_energy_zero_temperature}
% \end{align}

\section{TM Polarization:Partition Function}

In this case we are starting from the TM polarization
% \begin{equation}
% Z_{TM} = \int D\psi \exp\left[ -\frac{\epsilon_0}{2}\int_0^\beta d\beta'\int d^3x\,
%  \left( \frac{\epsilon(x)}{\hbar^2}(\partial_{\beta'}\psi)^2 + c^2\frac{1}{\epsilon}|\nabla\sqrt{\epsilon}\psi|^2\right)\right] .
% \end{equation}

Let us change to using $\tau = \beta \hbar c$ as our temperature coordinate.  Then   
\begin{equation}
Z_{TM} = \int D\psi \exp\left[ -\frac{\epsilon_0 c^2 }{2 \hbar c}\int_0^{\hbar\beta c} 
d\tau'\int d^3x\, \psi\left( \epsilon(x)(\partial_{\tau}
  -\sqrt{\epsilon}\nabla \epsilon^{-1}\nabla\sqrt{\epsilon} -\nabla^2\right)\psi\right].
\end{equation}

As before, we introduce the Matsubara frequencies $\omega_n$
% We will now introduce the Matsubara frequencies $\omega_n = (2\pi n)/(\beta \hbar)$, with 
% \begin{equation}
% \psi(\beta,x) = \sum_{n=-\infty}^{\infty}e^{i\frac{\omega_n}{c}\tau} \psi_n(x),
% \end{equation}
% where $\tau = \beta\hbar c$, and the $\psi_n$ are complex variables.  We will also need to use  
% \begin{equation}
% \int_0^{\beta \hbar c}d\tau e^{i\frac{(\omega_n+\omega_m)}{c}\tau} = \beta\hbar c \delta_{n,-m},
% \end{equation}
% and $\psi_n^* = \psi_{-n}$ since $\psi^*(\beta, x) = \psi(\beta, x)$.  
Then we have 
\begin{equation}
Z_{TM} = \prod_{n=-\infty}^{\infty} \int D\psi_n\exp\left[ -\frac{\beta \epsilon_0 c^2 }{2}\int d^3x\, 
\psi_n^*\left(\epsilon(i\omega_n,x)\frac{\omega_n^2}{c^2}+   V_{TM} -\nabla^2\right)\psi_n\right], 
\end{equation}
where $V_{TM} = (\nabla\ln\sqrt{\epsilon})^2-\nabla^2\log\sqrt{\epsilon}$.
  Note that the presence of $V_{TM}$ implies there will be a contribution to the $n=0$ Matsubara term,
 which is good, since we know that the TM energy is the dominant contribution in that case.
This is what should give us a the dominant near-field, and high temperature results.  
   We will renormalize this against vacuum, 
% \begin{equation}
% \log Z_{TE} -\log Z_0= -{\sum_{n=0}^\infty}'\left\{\log\det\left[ \frac{1}{2}
% \left(\epsilon(i\omega_n,\vect{x})\frac{\omega_n^2}{c^2} +V_{TM}-\nabla^2\right)\right]
%  - \log\det\left[ \frac{1}{2}\left(\frac{\omega_n^2}{c^2} -\nabla^2\right)\right]\right\}
% \end{equation}
% Note that for $n=0$ the Matsubara frequency $\omega_n=0$, so we have 
% \begin{align}
% &\log\det\left[ \frac{1}{2}\left(\epsilon(0,\vect{x})\frac{\omega_0^2}{c^2}+V_{TM} -\nabla^2\right)\right]
%  - \log\det\left[ \frac{1}{2}\left(\frac{\omega_0^2}{c^2} -\nabla^2\right)\right] \nonumber\\
% &= \log\det\left[\frac{1}{2}\left(V_{TM} -\nabla^2\right)\right]
%  - \log\det\left[ \frac{1}{2}\left( -\nabla^2\right)\right]\ne  0
% \end{align}
% This assumes that $\epsilon(\omega)$ has at most a simple pole at zero frequency,
%  such that $\lim_{\omega\rightarrow 0}\omega^2\epsilon(\omega)=0.$    
Then renormalized Casimir energy is 
\begin{align}
E-E_0 & = -k_BT \log \frac{Z_{TM}}{Z_0} \\
&=k_BT{\sum_n}'\int_0^\infty \frac{dT}{T}\int d^3x\,\frac{1}{(2\pi T)^{3/2}}\dlangle e^{-T\frac{\omega_n^2}{2c^2}}
 -  e^{-\frac{ T \omega_n^2\langle\epsilon(i\omega_n)\rangle}{2c^2} - \frac{T}{2}\langle V_{TM}\rangle}\drangle\label{eq:TMworldline_partition_function}
\end{align}

\subsection{Casimir-Polder energy}

The Casimir-Polder energy can be recovered by expanding $\epsilon$ to linear order in $\alpha(i\omega)/\epsilon_0\delta(x-x').$
  We will also have to expand $V_{TM}$,
\begin{align}
T\langle V_{TM}\rangle =& \int_0^Tdt\, (\partial_x\log\sqrt{\epsilon})^2 - \partial_x^2\log\sqrt{\epsilon}\\
%=& \int_0^Tdt\, \frac{1}{4}(\partial_x\log\epsilon)^2 - \frac{1}{2}\partial_x^2\log\epsilon\\
=& \int_0^Tdt\, \frac{1}{4}[\partial_x\log(\epsilon + \alpha\delta(x-x_0)/\epsilon_0)]^2 
- \frac{1}{2}\partial_x^2\log(\epsilon + \alpha\delta(x-x_0)/\epsilon_0)\\
%=& \int_0^Tdt\, \frac{1}{4}\{\partial_x[\log(\epsilon) + \alpha\delta(x-x_0)/(\epsilon(x_0)\epsilon_0)]\}^2 - \frac{1}{2}\partial_x^2\{\log(\epsilon) + \alpha\delta(x-x_0)/(\epsilon_0\epsilon(x_0))\}\\
=& \int_0^Tdt\, V_{TM} +\frac{\alpha}{2}\partial_x\log\epsilon\partial_x[\delta(x-x_0)/(\epsilon(x)\epsilon_0)]
 - \frac{\alpha}{2}\partial_x^2[\delta(x-x_0)/(\epsilon_0\epsilon(x))]
\end{align}
We will simplify this a bit by assuming we are only considering the polarizable particle in regions 
where $\epsilon(x)$ is constant in the vicinity of $x_0$.
  Then we can drop any terms in $\partial_x\epsilon|_{x=x_0}$.
  This lets us drop the second term, and any derivatives from expanding out the derivative.
  I think any extra terms from expanding out these derivatives would ultimately get eaten when
 considering the effect of $\delta'$.
  It will be operationally cleaner to just leave the derivatives acting on the products, 
and integrate by parts at the end. 

Let us suppress the renormalization terms for the mean time.  
\begin{align}
E&=-k_BT{\sum_n}'\int_0^\infty \frac{dT}{T}\int d^3x\,\frac{1}{(2\pi T)^{3/2}}
\dlangle e^{-\frac{ T \omega_n^2\langle\epsilon(i\omega_n)\rangle}{2c^2} - T\langle V_{TM}\rangle} \right.\right. \nonumber \\
& \hspace{3cm}\times\left.\left.\left(-T\frac{\omega_n^2}{2c^2}\frac{\alpha(i\omega_n)}{\epsilon_0}
\langle\delta(x-x_0)\rangle  - \frac{\alpha T}{4}\langle\partial_x^2[\delta(x-x_0)/(\epsilon_0\epsilon(x))]\rangle\right)\drangle\\
&=k_BT{\sum_n}'\frac{\alpha(i\omega_n)}{2\epsilon_0}\int_0^\infty dT\,\frac{1}{(2\pi T)^{3/2}}
\dlangle \left(\frac{\omega_n^2}{c^2}  - \frac{1}{2}\partial_x^2\right)
e^{-\frac{ T \omega_n^2\langle\epsilon(i\omega_n)\rangle}{2c^2} - T\langle V_{TM}\rangle}\drangle
\end{align}
Now doing the subtraction of the same energy with $\epsilon=1$ gives: 
\begin{equation}
E-E_0=-k_BT{\sum_n}'\frac{\alpha(i\omega_n)}{2\epsilon_0}\int_0^\infty dT\,
\frac{1}{(2\pi T)^{3/2}}\dlangle \frac{\omega_n^2}{c^2}e^{-T\frac{\omega_n^2}{2c^2}}
-\left(\frac{\omega_n^2}{c^2}  - \frac{1}{2}\partial_x^2\right)e
^{-\frac{ T \omega_n^2\langle\epsilon(i\omega_n)\rangle}{2c^2} - T\langle V_{TM}\rangle}\drangle\label{eq:TM_CP_finite_temperature},
\end{equation}
which is our initial result for the finite-temperature Casimir-Polder energy.
  As before, we can straightforwardly take the zero temperature limit: 
\begin{equation}
E-E_0=-\frac{\hbar}{2\pi}\int_0^\infty d\omega\frac{\alpha(i\omega)}{2\epsilon_0}
\int_0^\infty dT\,\frac{1}{(2\pi T)^{3/2}}\dlangle \frac{\omega^2}{c^2}e^{-T\frac{\omega^2}{2c^2}}-\left(\frac{\omega^2}{c^2}  - \frac{1}{2}\partial_x^2\right)e^{-\frac{ T \omega^2\langle\epsilon(i\omega)\rangle}{2c^2} - T\langle V_{TM}(i\omega)\rangle}\drangle\label{eq:TM_CP_zero_temperature},
\end{equation}

\subsubsection{Laplace-Mellin and Feynman-Kac Formulae}
We will again need to use the Laplace-Mellin transforms, and Feynman-Kac Formulae.
  We quote the results:
The Laplace-Mellin transform is
\begin{align}
\int_0^\infty \frac{dT}{T^{1+z}}\dlangle e^{-sT\langle\epsilon\rangle - T\langle V_{TM}\rangle}\drangle =&
 \frac{1}{\Gamma[z+1/2]}\int_0^\infty d\lambda\, \lambda^{z-1/2}\int_0^\infty dT e^{-(\lambda+s)T}
\dlangle \frac{e^{-\int_0^T dt\,(s\chi+ V_{TM})}}{\sqrt{T}}\drangle.
\end{align}
For Casimir-Polder we need$z=1/2$, and for Casimir we need $z=3/2$.
  In both cases we need $s= \omega^2/(2c^2)$.
  We also need the actual analytical expression for that path integral.

For one body we need:
\begin{equation}
\int_0^\infty dT e^{-(\lambda+s) T} \dlangle \frac{e^{-s\chi\int_0^T dt \Theta(x-d)}}{\sqrt{2\pi T}}\drangle  
=\frac{1}{\sqrt{2(\lambda+s)}}\left[1 - e^{-2\sqrt{2(\lambda+s)}|d|}\frac{\sqrt{\lambda+s(1+\chi)}
-\sqrt{\lambda+s}e^{2\Xi}}{\sqrt{\lambda+s(1+\chi)}+\sqrt{\lambda+s}e^{2\Xi}}\right],
\end{equation}
where $e^{2\Xi} = (1+\chi)$ comes from the contribution of $e^{-V_{TM}}$.
  \comment{Correct signs?} For two macroscopic bodies we will need:
\begin{align}
&\int dx\int_0^\infty dT \frac{e^{-(\lambda +s)T}}{\sqrt{2\pi T}}\left[e^{-s\int_0^T dt\,(\chi_{12} + V_{12,TM})}
 +1 -e^{-s\int_0^T dt\,(\chi_{1} + V_{1,TM})}-e^{-s\int_0^T dt\,(\chi_{2} + V_{2,TM})}\right]\nonumber\\ 
=&  \dfrac{u_1'u'_2e^{-2\sqrt{2\lambda}d}}{1 - u'_1u'_2 e^{-2\sqrt{2\lambda}d}}\left[ \frac{2 d}{\sqrt{2\lambda}}
-\frac{ e^{2\Xi_1}}{\sqrt{\lambda+s}\sqrt{\lambda+s(1+\chi_1)}}
\frac{s\chi_1}{e^{4\Xi_1}(\lambda+s)-[\lambda+s(1+\chi_1)]}  + \{1 \leftrightarrow 2\}  \right].
\end{align}
where 
\begin{equation}
u'_i = \frac{\sqrt{\lambda+s}(1+\chi)-\sqrt{\lambda+s(1+\chi)}}{\sqrt{\lambda+s}(1+\chi)+\sqrt{\lambda+s(1+\chi)}}
\end{equation}
As nasty as that two-body expression may be, exactly the same tricks will work on it, 
and it will simplify down to exactly the same form as the other polarization.  

\subsection{TM Casimir-Polder: Limiting Cases}

We will work with the case of an atom in front of a dielectric surface.
  Let's first plug in the relevant one-body Feynman-Kac formula into the partition function.
 
\begin{align}
E-E_0=& -k_BT{\sum_n}'\frac{\alpha(i\omega_n)}{2\epsilon_0}\int_0^\infty dT\,
\frac{1}{(2\pi T)^{3/2}}\left(\frac{\omega_n^2}{c^2}  - \frac{1}{2}\partial_x^2\right)
\dlangle e^{-T\frac{\omega_n^2}{2c^2}}-e^{-\frac{ T \omega_n^2\langle\epsilon(i\omega_n)\rangle}{2c^2} - T\langle V_{TM}\rangle}\drangle \\
=& -k_BT{\sum_n}'\frac{\alpha(i\omega_n)}{4\pi\epsilon_0}\left(\frac{\omega_n^2}{c^2}  
- \frac{1}{2}\partial_d^2\right)\int_0^\infty d\lambda\, 
\frac{e^{-2\sqrt{2(\lambda+\frac{\omega_n^2}{2c^2})}d}}{\sqrt{2\lambda+\omega_n^2/c^2}}
\frac{\sqrt{\lambda+\frac{\omega_n^2}{2c^2}(1+\chi)}-\sqrt{\lambda+\frac{\omega_n^2}{2c^2}}e^{2\Xi}}
{\sqrt{\lambda+\frac{\omega_n^2}{2c^2}(1+\chi)}+\sqrt{\lambda+\frac{\omega_n^2}{2c^2}}e^{2\Xi}} 
\end{align}
Now make our usual substitutions: $\lambda = \kappa\omega_n^2/(2c^2)$, and then $p^2 = \kappa+1$.  
\begin{align}
E-E_0%=& -k_BT{\sum_n}'\frac{\alpha(i\omega_n)}{4\pi\epsilon_0}\left(\frac{\omega_n^2}{c^2}  - \frac{1}{2}\partial_d^2\right)\int_0^\infty d\kappa\, \frac{\omega_n^2}{2c^2}\frac{e^{-2\sqrt{\kappa+1}\omega_n d/c}c}{\omega_n\sqrt{\kappa+1}}\frac{\sqrt{\kappa+1+\chi}-\sqrt{\kappa+1}e^{2\Xi}}{\sqrt{\kappa+1+\chi}+\sqrt{\kappa+1}e^{2\Xi}} \\
=& -k_BT{\sum_n}'\frac{\omega_n\alpha(i\omega_n)}{4\pi\epsilon_0c}
\left(\frac{\omega_n^2}{c^2}  - \frac{1}{2}\partial_d^2\right)
\int_1^\infty dp\,e^{-2p\omega_n d/c}\frac{\sqrt{p^2+\chi}-pe^{2\Xi}}{\sqrt{p^2+\chi}+p e^{2\Xi}} 
\end{align}

Take the $\partial_d$ derivatives, get 
\begin{align}
E-E_0%=& -k_BT{\sum_n}'\frac{\omega_n\alpha(i\omega_n)}{4\pi\epsilon_0c}\int_1^\infty dp\,\left(\frac{\omega_n^2}{c^2}  - \frac{2\omega_n^2p^2}{c^2}\right)e^{-2p\omega_n d/c}\frac{\sqrt{p^2+\chi}-pe^{2\Xi}}{\sqrt{p^2+\chi}+p e^{2\Xi}} \\
=& -k_BT{\sum_n}'\frac{\omega^3_n\alpha(i\omega_n)}{4\pi\epsilon_0c^3}\int_1^\infty dp\,
\left(1-2p^2\right)e^{-2p\omega_n d/c}\frac{\sqrt{p^2+\chi}-pe^{2\Xi}}{\sqrt{p^2+\chi}+p e^{2\Xi}} 
\end{align}

\subsubsection{Zero temperature, far-field}
In the far-field of the atom, we have $d\omega_{j0}/c>>1$, so replace $\epsilon,\alpha$ by their d.c. values.  
\begin{align}
E-E_0=& -\frac{\hbar}{2\pi}\int_0^\infty d\omega \frac{\omega^3\alpha_0}{4\pi\epsilon_0c^3}
\int_1^\infty dp\,\left(1-2p^2\right)e^{-2p\omega d/c}\frac{\sqrt{p^2+\chi}-p(1+\chi)}{\sqrt{p^2+\chi}+p(1+\chi)}\\
=& -\frac{3\hbar c\alpha_0}{64\pi^2\epsilon_0d^4}\int_1^\infty dp\,p^{-4}
\left(1-2p^2\right)\frac{\sqrt{p^2+\chi}-p(1+\chi)}{\sqrt{p^2+\chi}+p(1+\chi)},
\end{align}
yet another familiar integral (up to a lingering sign on a reflection coefficient?
 I seem to have currently stumbled onto the correct choice. )  

\subsubsection{Zero temperature, near-field}

In this limit you take $d\rightarrow 0$, but all frequencies contribute, as governed by $\alpha(i\omega)$.
    We have frequency integral determined by $\alpha$ so frequencies $w < w_{j0}$ will dominate.
  Alternatively, just take $p\sim c/(d\omega)$.
  Since $d$ is small, then important $p$ are very large?
  I think this implicitly takes $\omega d/c<<1$?
\begin{align}
E-E_0=& -\frac{\hbar}{8\pi^2\epsilon_0c^3}\int_0^\infty d\omega \omega^3\alpha(i\omega)\int_1^\infty dp\,
\left(1-2p^2\right)e^{-2p\omega d/c}\frac{\sqrt{p^2+\chi}-pe^{2\Xi}}{\sqrt{p^2+\chi}+p e^{2\Xi}} 
\end{align}
The reflection coefficient becomes
\begin{equation}
\frac{\sqrt{p^2+\chi}-p(1+\chi)}{\sqrt{p^2+\chi}+p(1+\chi)} \approx \frac{ p-p(1+\chi)}{p+p(1+\chi)} =
 -\frac{\epsilon(i\omega)-1}{\epsilon(i\omega)+1}.
\end{equation}
Plug this in, and evaluate $p$ integral
\begin{align}
E-E_0\approx& \frac{\hbar}{8\pi^2\epsilon_0c^3}\int_0^\infty d\omega \omega^3
\alpha(i\omega)\frac{\epsilon(i\omega)-1}{\epsilon(i\omega)+1}\int_1^\infty dp\,2p^2e^{-2p\omega d/c}\\
=& \frac{\hbar}{8\pi^2\epsilon_0c^3}\int_0^\infty d\omega \omega^3
\alpha(i\omega)\frac{\epsilon(i\omega)-1}{\epsilon(i\omega)+1}\left(-\frac{c^3e^{-2\omega d/c}(1+\omega d/c)^2}{2 d^3\omega^3}\right)\\
&\approx -\frac{\hbar }{16\pi^2\epsilon_0 d^3}\int_0^\infty d\omega 
\alpha(i\omega)\frac{\epsilon(i\omega)-1}{\epsilon(i\omega)+1}.
\end{align}
which is the correct answer ( an expected result since we were angling for this result by making these limits.
  BUt reassuring to see it emerge nonetheless).  

\subsection{High temperature, far-field(?)}

% We are again working in a far-field limit.
%   At high temperature $\beta\rightarrow 0$, so $\omega_i\rightarrow \infty$.
%   $\omega_n = 2\pi/(\beta\hbar)$.  
% The renormalized energy is 
% \begin{align}
% E-E_0=& -\partial_\beta{\sum_n}'\frac{\omega^3_n\alpha(i\omega_n)}{4\pi\epsilon_0c^3}\int_1^\infty dp\,
% \left(1-2p^2\right)e^{-2p\omega_n d/c}\frac{\sqrt{p^2+\chi}-p(1+\chi)}{\sqrt{p^2+\chi}+p (1+\chi)} 
% \end{align}
% Other work tells us that only $\omega_0$ contributes here.
%   Naively taking $\omega_0=0$, we have \emph{no} $\beta$ dependence anywhere.
%   So that just gives us zero?  Perhaps we have to be careful with the order of operations here 
% - or it is ok to do the $\beta$ derivative right now?
%   We will use $\partial_\beta\omega_n = \partial_\beta[2\pi/(\beta\hbar)] = -2\pi/(\beta^2\hbar) = -k_BT \omega_n$.

% \begin{align}
% E-E_0=& -\partial_\beta{\sum_n}'\frac{\omega^3_n\alpha(i\omega_n)}{4\pi\epsilon_0c^3}\int_1^\infty dp\,\left(1-2p^2\right)e^{-2p\omega_n d/c}\frac{\sqrt{p^2+\chi}-p(1+\chi)}{\sqrt{p^2+\chi}+p (1+\chi)} 
% \end{align}
% \subsubsection{Keeping $\omega_1$}
% So if $\omega_0$ does not contribute, let's try the next term.  Since $\omega_1d/c>>1$, we have only small $p$ contributing.  Since our integral's lower bound is $p=1$, only $p$ close to 1 will contribute.  Let's use $p = 1+s$.    
% \begin{align}
% E-E_0=& -\partial_\beta{\sum_n}'\frac{\omega^3_1\alpha(i\omega_1)}{4\pi\epsilon_0c^3}e^{-2\omega_1 d/c}\int_0^\infty ds\,\left(1-2(1-s)^2\right)e^{-2s\omega_1 d/c}\frac{\sqrt{(1+s)^2+\chi}-(1+s)(1+\chi)}{\sqrt{(1+s)p^2+\chi}+(1+s) (1+\chi)}
% \end{align}
% If we approximate the reflection coefficient at $s=0$ weget
% \begin{align}
% E-E_0=& -\partial_\beta\frac{\omega^3_1\alpha(i\omega_1)}{4\pi\epsilon_0c^3}e^{-2\omega_1 d/c}\int_0^\infty ds\,\left(1-2\right)e^{-2s\omega_1 d/c}\frac{\sqrt{1+\chi}-(1+\chi)}{\sqrt{1+\chi}+ (1+\chi)}\\
% =& -\partial_\beta\frac{\omega^3_1\alpha(i\omega_1)}{4\pi\epsilon_0c^3}\frac{\sqrt{1+\chi}-1 }{\sqrt{1+\chi}+1}e^{-2\omega_1 d/c}\frac{c}{2\omega_1 d}.  
% \end{align}
% Which just decays exponentially with distance (rapidly no less).  Hmm.

% \subsubsection{Retrying with free energy}

Let's try this using the free energy, $F = -\beta^{-1}\log Z$.
  The Casimir energy is starts from 
\begin{equation}
F-F_0=-\beta^{-1}{\sum_n}'\frac{\alpha(i\omega_n)}{2\epsilon_0}\int_0^\infty dT\,
\frac{1}{(2\pi T)^{3/2}}\dlangle \frac{\omega_n^2}{c^2}e^{-T\frac{\omega_n^2}{2c^2}}-\left(\frac{\omega_n^2}{c^2}  
- \frac{1}{2}\partial_x^2\right)e^{-\frac{ T \omega_n^2\langle\epsilon(i\omega_n)\rangle}{2c^2} - T\langle V_{TM}\rangle}\drangle
\end{equation}
If we only keep the term with $\omega_0=0$, we have 
\begin{equation}
F-F_0=-\frac{1}{2}\beta^{-1}\frac{\alpha(0)}{2\epsilon_0}\int_0^\infty dT\,\frac{1}{(2\pi T)^{3/2}}
\dlangle -\left(-\frac{1}{2}\partial_x^2\right)e^{ - T\langle V_{TM}\rangle}\drangle
\end{equation}
Previously, we've done the calculations for the Feynman-Kac formula for just the $TM$ potential.
  Since the Laplace-tranform is trivial, we can do it immediately.  We get 
\begin{align}
\dlangle e^{-\int_0^T dt V_{TM}}\drangle &= 1 + \frac{\sinh(\Xi/2)}{\cosh\Xi}[e^{\Xi/2} + e^{-\Xi/2}]e^{-2 d^2/T}\\
%&= 1 + \frac{(e^{\Xi/2} - e^{-\Xi/2})}{(e^{\Xi/2} + e^{-\Xi/2})}{e^\Xi + e^{-\Xi}}e^{-2 d^2/T}\\
%&= 1 + \frac{e^{\Xi} - e^{-\Xi}}{e^\Xi + e^{-\Xi}}e^{-2 d^2/T}\\
&= 1 + \frac{e^{2\Xi} - 1}{e^{2\Xi} + 1} e^{-2 d^2/T}\\
&= 1 + \frac{\epsilon(0) - 1}{\epsilon(0)+1}e^{-2 d^2/T}.
\end{align}
Plugging this in, and taking the derivative 
\begin{align}
F-F_0=&-\frac{k_BT\alpha_0}{16\pi\epsilon_0}\frac{\epsilon(0)-1}{\epsilon(0)+1} 
\int_0^\infty dT\,\partial_d^2\frac{1}{\sqrt{2\pi }T^{3/2}} e^{-2 d^2/T}\\
=&-\frac{k_BT\alpha_0}{16\pi\epsilon_0d^3}\frac{\epsilon(0)-1}{\epsilon(0)+1}.
\end{align}


 
    % \section{TE/TM Zero temperature atom-plane, plane-plane}
    % \section{TE/TM Zero temperature atom-sphere, atom-cylinder}
    % \section{Finite Temperature}

    % \comment{Maybe I need some words here to avoid weirdness?}


%%% Local Variables: 
%%% mode: latex
%%% TeX-master: "thesis_master"
%%% End: 
