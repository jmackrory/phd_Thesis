\chapter{Electromagnetic Worldlines - Analytical Results}
\label{ch:analytical}
\section{Extracting Energies}

Let us think a bit about how to extract the energies we want.
  We will consider two cases: 
 One, finding the interaction energy of a microscopic test particle with some set of bodies,
 and the interaction energy between a set of macroscopic bodies. 

All of these results will start by computing the mean energy based on 
\begin{equation}
\langle E\rangle = -\partial_\beta\log Z
\end{equation}
(Note that this procedure in terms of partition functions and derivatives lends itself to also calculating fluctuations).  

We can try to handle the atom-body case by taking $\epsilon \rightarrow \epsilon + \alpha\delta$.
  The fundamental question in all of these cases is: what energy are you calculating, 
and what physical procedure are you using to generate some configuration.
  The subtleties here tend to introduce various negative signs when you consider slightly different ways of achieving the same result.  

However, let's charge ahead, and use: 
\begin{equation}
V_{CP} = -\frac{1}{2}\alpha_0\langle E^2\rangle,
\end{equation}
which is a result that follows from using second order perturbation theory in the atom-field coupling.
  Given, a partition functions: 
\begin{align}
Z_{TE} = \int D\phi e^{-\frac{1}{2}\int d\beta \frac{\Pi^2}{\epsilon} + c^2\nabla\phi^2+\frac{i}{\hbar}\partial_\beta\phi \pi  }
Z_{TE} = \int D\phi e^{-\int d\beta \frac{\Pi^2}{2} + \epsilon^{-1}\nabla\phi^2+i\partial_\beta\phi \pi  }
\end{align}

\subsection{TE}
Let's now take $\epsilon = \epsilon+\alpha\delta$, and expand to leading order in $\alpha.$  
We need two terms:
\begin{align}
\langle\epsilon+\alpha\delta\rangle^{-1/2} =&  \left(\frac{1}{T}\int_0^Tdt\,\epsilon + \alpha\delta\right)^{-1/2}\\
%=& \left(\frac{1}{T}\int_0^Tdt\,\epsilon + \alpha\frac{1}{T}\int_0^Tdt\,\delta\right)^{-1/2}\\
=& \langle\epsilon\rangle^{-1/2} -\frac{1}{2}\alpha\langle\delta\rangle\langle\epsilon\rangle^{-3/2}
\end{align}
\subsection{TM Potential}
Now take 
\begin{align}
\langle V_{TM}[\epsilon +\alpha\delta] \rangle =& 
\frac{1}{2T}\int_0^Tdt\,\frac{1}{2}\nabla^2\log(\epsilon+\alpha\delta)-\frac{1}{4}|\nabla\log(\epsilon+\alpha\delta)|^2\\
% =& \frac{1}{2T}\int_0^Tdt\,\frac{1}{2}\nabla^2\left( \log\epsilon + \alpha\epsilon^{-1}\delta\right)
% -\frac{1}{4}|\nabla\log(\epsilon) +\alpha\nabla\epsilon^{-1}\delta|^2\\
=& \frac{1}{2T}\int_0^Tdt\,\frac{1}{2}\nabla^2\log\epsilon - \frac{1}{4}|\nabla\log\epsilon|^2 
+ \frac{\alpha}{2}\nabla^2(\epsilon^{-1}\delta)-\frac{\alpha}{2}\nabla\log(\epsilon)\cdot\nabla\epsilon^{-1}\delta\\
=& \langle V_{\text{back}}\rangle + \frac{\alpha}{4}\langle \nabla^2(\epsilon^{-1}\delta)
- \nabla\log(\epsilon)\cdot\nabla\epsilon^{-1}\delta\rangle
\end{align}
\subsection{Functional derivative of $\log Z_{TM}$}
We then get
\begin{align}
\log Z =&\int_0^\infty \frac{dT}{8\pi^2T^3}\int dx_0\bigdlangle\langle\epsilon\rangle^{-1/2}e^{- T\langle V\rangle}\bigdrangle\\
% =& \int_0^\infty \frac{dT}{8\pi^2T^3}\int dx_0\biggdlangle \left(\langle\epsilon\rangle^{-1/2}
% -\frac{\alpha\langle\delta\rangle}{2\langle\epsilon\rangle^{3/2}}\right)
% \left( 1 + \frac{\alpha T}{4}\langle\nabla^2(\epsilon^{-1}\delta)
% -\nabla\log(\epsilon)\cdot\nabla\epsilon^{-1}\delta\rangle\right)e^{-T\langle V\rangle}\biggdrangle\\
=& \int_0^\infty \frac{dT}{8\pi^2T^3}\int dx_0\biggdlangle
\left(\langle\epsilon\rangle^{-1/2}-\frac{\alpha\langle\delta\rangle}{2\langle\epsilon\rangle^{3/2}}  
+ \frac{\alpha T}{4}\langle\epsilon\rangle^{-1/2}\langle \nabla^2(\epsilon^{-1}\delta)-
\nabla\log(\epsilon)\cdot\nabla\epsilon^{-1}\delta\rangle\right)e^{-T\langle V\rangle}\biggdrangle
\end{align}

Now consider $\langle \delta\,f\rangle $
\begin{align}
\int dx_0\int Dx\langle \delta(x)f(x)\rangle P(x) 
=&  \int dx_0\prod_{j=1}^{N}dx_n\delta(x_N-x_0)\prod_k P(x_k) \frac{1}{N}\sum_j \delta(x_j-x')f(x_j) \nonumber\\
= &f(x')\int \prod_{j=1}^N\delta(x_N-x') dx_jP(x_j) = f(x')\int Dx P(x)_{x_0=x(T)=x'}
\end{align}
This is a trace, which is invariant under cylic permutations.
  We can cyclically permute all the loop labels around so that we call the point that passes through $x'$ the loop origin.
  Doing this for every term, means we get $N$ copies of the same integrals.  We can then call $x_0=x'$, and drop that sum over $j$.   
The same logic works for derivatives, and $\delta'$.
  So we can find the linear term in $\alpha$.
  This is effectively the functional derivative with respect to $\epsilon$.  
\begin{align}
\frac{\delta}{\delta\epsilon(x')}\log Z=& \int_0^\infty \frac{dT}{16\pi^2T^3}\int dx_0
\biggdlangle\bigg[ -\frac{\delta(x_0-x')}{\langle\epsilon\rangle^{3/2}}\nonumber\\
&  +  \frac{T}{2}\langle\epsilon\rangle^{-1/2}\left(\nabla^2[\epsilon^{-1}(x_0)\delta(x_0-x')]
-\nabla\log(\epsilon)\cdot\nabla\epsilon^{-1}\delta(x_0-x')\rangle\right)\bigg]e^{-T\langle V\rangle}\biggdrangle
\end{align}
Now integrate by parts on the delta functions, and we get 
\begin{align}
\frac{\delta}{\delta\epsilon(x')}\log \ZTM=& \int_0^\infty \frac{dT}{16\pi^2T^3}\int dx_0\delta(x_0-x')
\biggdlangle-\frac{e^{T\langle V\rangle}}{\langle\epsilon\rangle^{3/2}}
  +  \frac{T}{2}\epsilon^{-1}(x_0)\nabla'^2\left(\langle\epsilon\rangle^{-1/2} e^{T\langle V\rangle}\right)\nonumber\\
& \hspace{3cm}+\epsilon^{-1}\frac{T}{2}\nabla'\cdot 
\frac{\nabla'\epsilon(x')}{\epsilon(x')}\langle\epsilon\rangle^{-1/2}e^{-T\langle V\rangle}\biggdrangle
\end{align}
In cases with piece wise constant media, we can drop the final term, saying $\nabla'\epsilon(x')\approx 0$.
  I will then use $\int dx \delta'(x-y)f(x) = \partial_y f(y) = \partial_y\int dx f(x)$ to pull the derivatives out.  
\begin{align}
\Aboxed{\frac{\delta}{\delta\epsilon(x')}\log \ZTM=& \frac{1}{16\pi^2}\int_0^\infty dT
\left[ \frac{1}{2}\epsilon^{-1}(x_0)\nabla'^2\biggdlangle
\frac{ e^{-\int dt V}}{\langle\epsilon\rangle^{1/2}T^2}\biggdrangle
-\biggdlangle\frac{e^{-\int dt V}}{\langle\epsilon\rangle^{3/2}T^3} \biggdrangle \right]}
\end{align}

\section{Useful Analytical results}

\subsection{ Laplace-Mellin Transforms}

Let's introduce the Laplace transform, 
\begin{equation}
\mathfrak{L}[f](s) = \int_0^\infty dt e^{-st} f(t),
\end{equation}
the Mellin transform
\begin{equation}
\mathcal{M}[f](z)= \int_0^\infty dt\, t^{z-1}f(t),
\end{equation}
and the $\Gamma$ function.  
\begin{equation}
\Gamma(z) = \int_0^\infty ds\, s^{z-1} e^{-s} = \mathcal{M}[e^{-s}](z)
\end{equation}

We will show what we will refer to as the Laplace-Mellin theorem \footnote{Cite IBM dude}
\begin{equation}
\mathcal{M}[f](z) = \frac{1}{\Gamma(1-z)}\mathcal{M}[\mathcal{L}[f]](1-z)\label{eq:Laplace-Mellin}
\end{equation}
This is most easily motivated by starting with the RHS, swapping the order of the integrals, and scaling the $s$ integral.  
\begin{align}
\mathcal{M}[\mathcal{L}[f]](1-z) =& 
\int_0^\infty ds\, s^{-z} \int_0^\infty dt\,e^{-st} f(t)\\ 
% =& \int_0^\infty dt\,\left[\int_0^\infty ds s^{-z} e^{-st}\right] f(t)\\
% =& \int_0^\infty dt\,\left[\int_0^\infty d\frac{u}{t}\, t^zu^{-z} e^{-u}\right] f(t) \\
=&\left[\int_0^\infty du u^{-z} e^{-u}\right]\int_0^\infty dt\,t^{z-1} f(t) \\
=& \Gamma(1-z)\mathcal{M}[f](t)
\end{align}
This is the form we will use most often anyway.  
\subsection{Gamma function/ Moment Theorem}

We will show that 
\begin{equation}
\frac{1}{\Gamma[\alpha]}\int_0^\infty ds\,s^{\alpha-1}\dlangle e^{-s(x+\beta)}\drangle  
= \dlangle \frac{1}{(x+\beta)^\alpha}\drangle\label{eq:moment_theorem}
\end{equation}
We can do this via a straightforward approach: 
\begin{align}
\frac{1}{\Gamma[\alpha]}\int_0^\infty ds\,s^{\alpha-1}\dlangle e^{-s(x+\beta)}\drangle 
% =&\frac{1}{\Gamma[\alpha]}\int_0^\infty ds\,s^{\alpha-1}\int dx f(x) e^{-s(x+\beta)}\\
% =& \frac{1}{\Gamma[\alpha]}\int dx f(x) \int_0^\infty ds\,s^{\alpha-1} e^{-s(x+\beta)} \\
% =&\frac{1}{\Gamma[\alpha]}\int dx f(x) \frac{1}{(x+\beta)^\alpha}\int_0^\infty dt\,t^{\alpha-1} e^{-t} \\
=&\int dx f(x) \frac{1}{(x+\beta)^\alpha} \\
=& \dlangle \frac{1}{(x+\beta)^\alpha}\drangle
\end{align}

\section{Finding the TE CP energy}

In the previous section we found various analytical expressions of the form $ \dlangle e^{-\lambda t - \int V}\drangle$.
  The identities given in the last two sections will let us find analytical solutions.
  We will start with the Casimir-Polder energy above a dielectric surface.
  We will then be treating cases like $\epsilon_r(z) = 1+\chi\Theta(x-d)$.  

Let's put the Casimir results in the form where we can substitute in the analytical results.
  From Eq.~(\ref{eq:log Z TE}), and  the moment theorem Eq.~(\ref{eq:moment_theorem})we can write 
\begin{align}
\int_0^\infty \frac{dT}{T^{1+D/2}}\dlangle\frac{1}{\left(1+\chi \langle \Theta[x(T)-d]\rangle\right)^\alpha} \drangle 
% = &\int_0^\infty \frac{dT}{T^{1+D/2}}\frac{1}{\Gamma[\alpha]}\int_0^\infty ds s^{\alpha-1} 
% \dlangle e^{-s(\chi \int_0^T dt \Theta(x-d) +1)}\drangle \\
% =&\frac{1}{\Gamma[\alpha]}\int_0^\infty \frac{dT}{T^{1+D/2-\alpha}}\int_0^\infty ds s^{\alpha-1} e^{-s T}
% \dlangle e^{-s \chi \int_0^T dt \Theta(x-d)}\drangle \\
=&\frac{1}{\Gamma[\alpha]}\int_0^\infty ds s^{\alpha-1}\int_0^\infty \frac{dT}{T^{1+D/2-\alpha}}
\dlangle e^{-s(T+ \chi \int_0^T dt \Theta(x-d))}\drangle 
\end{align}
where we used the Inverse-moment theorem, rescaled the $\lambda\rightarrow \lambda T$,
 and swapped the order of integration. We can see that that $T$ integral has the form of a Mellin transform.
  Now we can use the Laplace-Mellin transform Eq.~(\ref{eq:Laplace-Mellin}) to write the $T$ integral using
\begin{align}
\int_0^\infty \frac{dT}{T^{1+z}}e^{-sT}\dlangle e^{-s \chi\int_0^Tdt\Theta(x-d)}\drangle 
% =& \int_0^\infty \frac{dT}{T^{1+z-1/2}}e^{-sT}\dlangle \frac{e^{-s \chi \int dt_0^T dt \Theta(x-d)}}{\sqrt{T}}\drangle\\
% =&\mathcal{M}\left[e^{-sT}\dlangle \frac{e^{-s \chi \int_0^T dt \Theta(x-d)}}{\sqrt{T}}\drangle\right]\left(-z+1/2\right) \\
% =& \frac{1}{\Gamma[1+z-1/2]}\mathcal{M}\left[\int_0^\infty dT e^{-(\lambda+s)T}
% \dlangle \frac{e^{-s \chi \int_0^T dt \Theta(x-d)}}{\sqrt{T}}\drangle\right]\left(-z+1/2\right) \\
=& \frac{1}{\Gamma[z+1/2]}\int_0^\infty d\lambda\, \lambda^{z-1/2}\int_0^\infty dT e^{-(\lambda+s)T}
\dlangle \frac{e^{-s \chi \int_0^T dt \Theta(x-d)}}{\sqrt{T}}\drangle,
\end{align}
where we will take $z=D/2$.  
\comment{Comment earlier on factors of $\sqrt{T}$ as normalization for brownian bridge.}
From the Feynman-Kac formula for one step in Eq.~(\ref{eq:Feynman-Kac TE one step}),  we have 
\begin{equation}
\int_0^\infty dT e^{-(\lambda+s) T} \dlangle \frac{e^{-s\chi T_s[B_T;d]}}{\sqrt{T}}\drangle  
= \sqrt{\frac{\pi}{\lambda+s}}\left[1 - e^{-2\sqrt{2(\lambda+s)}|d|}
\frac{\sqrt{\lambda+s(1+\chi)}-\sqrt{\lambda+s}}{\sqrt{\lambda+s(1+\chi)}+\sqrt{\lambda+s}}\right],
\end{equation}
where we substistuted $s\rightarrow s\chi, \lambda\rightarrow \lambda+s$.  

We can now write the renormalized potential that we wanted as 
\begin{align}
V_D(\chi,d)=-\frac{\sqrt{\pi}}{\Gamma[\alpha]\Gamma\left[(D+1)/2-\alpha\right]}
\int_0^\infty ds s^{\alpha-1}\int_0^\infty d\lambda \lambda^{(D-1)/2-\alpha}
\frac{e^{-2\sqrt{2(\lambda+s)}|d|}}{\sqrt{\lambda+s}} 
\frac{\sqrt{\lambda+s(1+\chi)}-\sqrt{\lambda+s}}{\sqrt{\lambda+s(1+\chi)}+\sqrt{\lambda+s}},
\end{align}

\subsection{Variable rescaling to evaluate the integral}
Now we're in a position to try evaluating these integrals.
  We can put this into the same form as the Lifshitz expressions for this case after we make a number of variable changes.
  Similar steps are required for the other energies we calculate.  
First we take $\lambda \rightarrow \lambda/(8d^2), s\rightarrow s/(8d^2)$
\begin{align}
V_D(\chi,d)%=&-\frac{\sqrt{\pi}}{\Gamma[\alpha]\Gamma\left[(D+1)/2-\alpha\right]}(8 d^2)^{\alpha-\alpha-(D+1)/2-1/2}\nonumber \\
%&\times\int_0^\infty ds s^{\alpha-1}\int_0^\infty d\lambda \lambda^{(D-1)/2-\alpha}\frac{e^{-\sqrt{(\lambda+s)}}}{\sqrt{\lambda+s}} \frac{\sqrt{\lambda+s(1+\chi)}-\sqrt{\lambda+s}}{\sqrt{\lambda+s(1+\chi)}+\sqrt{\lambda+s}}\\
=&\frac{\sqrt{\pi}}{2^{3D/2}\Gamma[\alpha]\Gamma\left[(D+1)/2-\alpha\right]d^D}\nonumber \\
&\times(-1)\int_0^\infty ds s^{\alpha-1}\int_0^\infty d\lambda \lambda^{(D-1)/2-\alpha}
\frac{e^{-\sqrt{(\lambda+s)}}}{\sqrt{\lambda+s}} 
\frac{\sqrt{\lambda+s(1+\chi)}-\sqrt{\lambda+s}}{\sqrt{\lambda+s(1+\chi)}+\sqrt{\lambda+s}},
\end{align}
Secondly we will substitute $\lambda +s = u^2$,
\begin{align}
J_{\alpha,D} % =& (-1)\int_0^\infty ds s^{\alpha-1}\int_{\sqrt{s}}^\infty du\,(2u) (u^2-s)^{(D-1)/2-\alpha}\frac{e^{-u}}{u} \frac{\sqrt{u^2+s\chi}-u}{\sqrt{u^2+s\chi  }+u}\\
=& 2\int_0^\infty ds s^{\alpha-1}\int_{\sqrt{s}}^\infty du\, (u^2-s)^{(D-1)/2-\alpha}e^{-u} 
\frac{u-\sqrt{u^2+s\chi}}{u+\sqrt{u^2+s\chi  }}.
\end{align}
Now take $u = v\sqrt{s}$.  
\begin{align}
J_{\alpha,D} % =& 2\int_0^\infty ds s^{\alpha-1}\int_{1}^\infty du\, \sqrt{s}(v^2s-s)^{(D-1)/2-\alpha}e^{-\sqrt{s}v}\frac{v-\sqrt{v^2+\chi}}{v+\sqrt{v^2+\chi  }}\\
% =& 2\int_{1}^\infty du\,(v^2-1)^{(D-1)/2-\alpha}\frac{v-\sqrt{v^2+\chi}}{v+\sqrt{v^2+\chi  }}\int_0^\infty ds s^{\alpha-1+1/2+(D-1)/2-\alpha}e^{-\sqrt{s}v}\\
=& 2\int_{1}^\infty du\,(v^2-1)^{(D-1)/2-\alpha}\frac{v-\sqrt{v^2+\chi}}{v+\sqrt{v^2+\chi  }}
\int_0^\infty ds s^{D/2-1}e^{-\sqrt{s}v}
\end{align}
Lastly take $t= v\sqrt{s}$.  or $s = v^{-2}t^2$, with $ds = 2v^{-2}t dt$.  
\begin{align}
J % =& 2\int_{1}^\infty dv\,(v^2-1)^{(D-1)/2-\alpha}\frac{v-\sqrt{v^2+\chi}}{v+\sqrt{v^2+\chi  }}\int_0^\infty  dt (2v^{-2}t)\left(v^{-2}t^{2}\right)^{D/2-1}e^{-t}\\
% =& 4\int_{1}^\infty dv\,(v^2-1)^{(D-1)/2-\alpha}\frac{v-\sqrt{v^2+\chi}}{v+\sqrt{v^2+\chi  }}\int_0^\infty  dt v^{-D} t^{D-1}e^{-t}\\
=& 4\Gamma[D]\int_{1}^\infty dv\,v^{-D}(v^2-1)^{(D-1)/2-\alpha}\frac{v-\sqrt{v^2+\chi}}{v+\sqrt{v^2+\chi  }}
\end{align}

So we have:
\begin{align}
V_D(\chi,d)=\frac{\sqrt{\pi}\Gamma[D]}{2^{3D/2-2}\Gamma[\alpha]\Gamma\left[(D+1)/2-\alpha\right]d^D}
\int_{1}^\infty dv\,v^{-D}(v^2-1)^{(D-1)/2-\alpha}\frac{v-\sqrt{v^2+\chi}}{v+\sqrt{v^2+\chi  }}\nonumber \\
\end{align}

Let's evaluate this for the case we care about: $D=4,\alpha=3/2$, 
for the energy of an atom above a dielectric half-space.
\begin{align}
V_D(\chi,d)=&\frac{\sqrt{\pi}\Gamma[4]}{2^{4}\Gamma[3/2]d^4}\int_{1}^\infty dv\,v^{-4}\frac{v-\sqrt{v^2+\chi}}{v+\sqrt{v^2+\chi  }}\nonumber \\
% =&\frac{3}{48d^4\chi^{3/2}}\left\{4 \chi^{3/2}+24\chi^{1/2}-12 \sqrt{\chi  (\chi +1)}-3 \log \left[2 \chi +2 \sqrt{\chi  (\chi+1)}+1\right]-6 \text{arcsinh}\left(\sqrt{\chi }\right)\right\}\\
=&\frac{3}{4d^4}\left\{ \frac{1}{3}+2\chi^{-1}- \chi^{-3/2}\sqrt{\chi  (\chi +1)} 
-\frac{\log \left[2 \chi +2 \sqrt{\chi  (\chi+1)}+1\right]}{4 \chi^{3/2}}
-\frac{\text{arcsinh}\left(\sqrt{\chi }\right)}{2\chi^{3/2}}\right\},
\end{align}

% \subsection{Sinh-log}
% Comparison with Dan's expression suggests that 
% \begin{equation}
%  \sinh ^{-1}\left(\sqrt{x}\right)=\frac{1}{2}\log \left(2 x+2 \sqrt{x (x+1)}+1\right)
% \end{equation}

% We can try to verify this by writing 
% \begin{equation}
% \sinh(x) = \frac{1}{2}(e^{x}-e^{-x}),
% \end{equation}
% or 
% \begin{equation}
% x = \frac{1}{2}\left(e^{\text{arcsinh}[x]} - e^{-\text{arcsinh}[x]}\right)
% \end{equation}
% Take $u = e^{\text{arcsinh}[x]}$.  
% \begin{align}
% 2x = u - 1/u,
% \end{align}
% or
% \begin{equation}
% u^2 - 2xu - 1 = 0,
% \end{equation}
% which has solutions, 
% \begin{equation}
% u = \frac{2x \pm \sqrt{4x^2 + 4}}{2}  = x \pm \sqrt{x^2+1}.
% \end{equation}
% Now $u>0,$ so only the postive solution is valid.  
% \begin{align}
% e^{as} =& x \pm \sqrt{x^2+1}\\
% \Aboxed{\text{arcsinh}[x] =& \log(x + \sqrt{x^2+1})}
% \end{align}


\section{Finding the TM CP energy}

Now we can use the Feynman-Kac formula for the combined $TM$ potential,
 and step to calculate the energy of an atom with a wall due to the TM polarization.
  We will again have to use the Laplace-Mellin transform, and the scaled gamma function to develop the formalism.  

We can plug this back in to the potential.  
\begin{align}
\mathcal{V}_D(\chi,d)=-\frac{1}{\Gamma[\alpha]\Gamma\left[(D+1)/2-\alpha\right]}
\int_0^\infty ds s^{\alpha-1}\int_0^\infty d\lambda \lambda^{(D-1)/2-\alpha}\dlangle 
\int_0^\infty dT \frac{e^{-(\lambda+s)T}}{\sqrt{T}}e^{-\int_0^T dt\, V_{TM} + s\chi\Theta(x-d)}\drangle
\end{align}

\subsection{Changing Variables}
So this is what we really want.  Time to start simplifying this integral.  
We will renormalize against $d\rightarrow \infty$, and drop the normalization factor.  
Dan focuses on the case where we normalize against $e^{-c^2/(2t)}/\sqrt{t}$, 
as opposed to $e^{-c^2/(2t)}/\sqrt{2\pi t}$.  \comment{I will multiply my results by $\sqrt{2\pi}$.  }

\begin{align}
\mathcal{V}_D(\chi,d)=-\frac{\sqrt{\pi}}{\Gamma[\alpha]\Gamma\left[(D+1)/2-\alpha\right]}
\int_0^\infty ds s^{\alpha-1}\int_0^\infty d\lambda \lambda^{(D-1)/2-\alpha}
\frac{\left(\sqrt{\lambda+s}e^{2\Xi }- \sqrt{\lambda +s(1 +\chi)}\right) }
{\sqrt{\lambda +s} e^{2 \Xi }+\sqrt{\lambda +s(1+\chi)}}\frac{e^{-2\sqrt{2(\lambda+s) } |d|}}{\sqrt{(\lambda+s)}}
\end{align}
What follows is a lot of changes of variables to turn this into something like an integral against a decaying exponential.  
This follows in exact analogy with the TE case.  
% $u=\lambda+s$.  
% \begin{align}
% \mathcal{V}_D(\chi,d)=-\frac{\sqrt{\pi}}{\Gamma[\alpha]\Gamma\left[(D+1)/2-\alpha\right]}
% \int_0^\infty ds s^{\alpha-1}\int_s^\infty du\, u^{-1/2}(u-s)^{(D-1)/2-\alpha}
% \frac{\left(\sqrt{u}e^{2\Xi }- \sqrt{u + s\chi}\right) }{\sqrt{u} e^{2 \Xi }+\sqrt{u + s\chi}}e^{-2\sqrt{2u } |d|}
% \end{align}
% Now take $u\rightarrow u/8d^2, s\rightarrow s/(8d^2)$.  
% \begin{align}
% \mathcal{V}_D(\chi,d)% =&-\frac{\sqrt{\pi}}{\Gamma[\alpha]\Gamma\left[(D+1)/2-\alpha\right]}\frac{1}{(8d^2)^\alpha}\frac{1}{(8d^2)^{(D-1)/2-\alpha+1-1/2}}\nonumber\\
% % & \times\int_0^\infty ds s^{\alpha-1}\int_s^\infty du\, u^{-1/2}(u-s)^{(D-1)/2-\alpha}\frac{\left(\sqrt{u}e^{2\Xi }- \sqrt{u + s\chi}\right) }{\sqrt{u} e^{2 \Xi }+\sqrt{u + s\chi}}e^{-\sqrt{u }}\\
% =&-\frac{\sqrt{\pi}}{\Gamma[\alpha]\Gamma\left[(D+1)/2-\alpha\right]}\frac{1}{(8d^2)^{D/2}}\nonumber\\
% & \times\int_0^\infty ds s^{\alpha-1}\int_s^\infty du\, u^{-1/2} (u-s)^{(D-1)/2-\alpha}
% \frac{\left(\sqrt{u}e^{2\Xi }- \sqrt{u + s\chi}\right) }{\sqrt{u} e^{2 \Xi }+\sqrt{u + s\chi}}e^{-\sqrt{u }}
% \end{align}
% Now define $u = sv$.  
% \begin{align}
% \mathcal{V}_D=&-\frac{\sqrt{\pi}}{\Gamma[\alpha]\Gamma\left[(D+1)/2-\alpha\right]}\frac{1}{(8d^2)^{D/2}}\nonumber\\
% & \times\int_0^\infty ds s^{(D-2)/2}\int_1^\infty dv \, v^{-1/2}(v-1)^{(D-1)/2-\alpha}
% \frac{\left(\sqrt{v}e^{2\Xi }- \sqrt{v + \chi}\right) }{\sqrt{v} e^{2 \Xi }+\sqrt{v + \chi}}e^{-\sqrt{s v}}
% \end{align}
% Now define $ s = t^2, v = w^2$.  
% \begin{align}
% \mathcal{V}_D%=&-\frac{4\sqrt{\pi}}{\Gamma[\alpha]\Gamma\left[(D+1)/2-\alpha\right]}\frac{1}{(8d^2)^{D/2}}\int_0^\infty dt\, t^{(D-2)+1}\int_1^\infty dw\,(w^2-1)^{(D-1)/2-\alpha}\frac{\left(w e^{2\Xi }- \sqrt{w^2 + \chi}\right) }{w e^{2 \Xi }+\sqrt{w^2 + \chi}}e^{-tw}\\
% =&-\frac{4\sqrt{\pi}}{\Gamma[\alpha]\Gamma\left[(D+1)/2-\alpha\right]}\frac{1}{(8d^2)^{D/2}}
% \int_0^\infty dt\, t^{D-1}\int_1^\infty dw\,(w^2-1)^{(D-1)/2-\alpha}
% \frac{\left(w e^{2\Xi }- \sqrt{w^2 + \chi}\right) }{w e^{2 \Xi }+\sqrt{w^2 + \chi}}e^{-tw}
% \end{align}
% Now take swap the $t,w$ integrals, and turn the $t$ integral into a Gamma function.  
% \begin{align}
% \mathcal{V}_D%=&-\frac{4\sqrt{\pi}}{\Gamma[\alpha]\Gamma\left[(D+1)/2-\alpha\right]}\frac{1}{(8d^2)^{D/2}}\int_1^\infty dw\,(w^2-1)^{(D-1)/2-\alpha}\frac{\left(w e^{2\Xi }- \sqrt{w^2 + \chi}\right) }{w e^{2 \Xi }+\sqrt{w^2 + \chi}}\int_0^\infty dt\, t^{D-1}e^{-tw}\\
% =&-\frac{4\sqrt{\pi}\Gamma[D]}{\Gamma[\alpha]\Gamma\left[(D+1)/2-\alpha\right]}
% \frac{1}{(8d^2)^{D/2}}\int_1^\infty dw\,w^{-D}(w^2-1)^{(D-1)/2-\alpha}
% \frac{\left(w e^{2\Xi }- \sqrt{w^2 + \chi}\right) }{w e^{2 \Xi }+\sqrt{w^2 + \chi}}
% \end{align}

So we finally have
\begin{align}
\Aboxed{\mathcal{V}_D=&-\frac{\Gamma[D]\sqrt{\pi}}{2^{3D/2-2}\Gamma[\alpha]\Gamma\left[(D+1)/2-\alpha\right]}
\frac{1}{d^{D}}\int_1^\infty dw\,w^{-D}(w^2-1)^{(D-1)/2-\alpha}\frac{w(1+\chi)- \sqrt{w^2 + \chi} }{w (1+\chi)+\sqrt{w^2 + \chi}}},
\end{align}
where we used $\Xi = \log\sqrt{1+\chi}$.  

Now for the full atom-wall potential we need 
\begin{align}
V = \mathcal{V}_D(\chi,d;3/2) - \frac{1}{2}\partial_d^2\mathcal{V}_{D-2}(x,d;1/2)
\end{align}
%\comment{The scaled ``Gies'' formulation uses  $V = V_D - V_{D-2}''$}


\subsection{Evaluating integral for specific $D$, and $\alpha$}

We need $D=4$, $\alpha=3/2$.  
\begin{align}
\mathcal{V}_D(\chi;d;3/2)%=&-\frac{\Gamma[D]\sqrt{\pi}}{2^{4}\Gamma[3/2]\Gamma\left[(D+1)/2-3/2\right]}\frac{1}{d^D}\int_1^\infty dw\,w^{-D}(w^2-1)^{(D-1)/2-3/2}\frac{w(1+\chi)- \sqrt{w^2 + \chi} }{w (1+\chi)+\sqrt{w^2 + \chi}}\\
=&-\frac{3}{4d^4}\int_1^\infty dw\,w^{-4}\frac{w(1+\chi)- \sqrt{w^2 + \chi} }{w (1+\chi)+\sqrt{w^2 + \chi}}
\end{align}

We also need $D=2$, $\alpha=1/2$
\begin{align}
\mathcal{V}_{D-2}(\chi;d;1/2)%=&-\frac{\Gamma[D-2]\sqrt{\pi}}{2^{(3(D-2)/2-2)}\Gamma[1/2]\Gamma\left[(D-1)/2-1/2\right]}\frac{1}{d^{D-2}}\int_1^\infty dw\,w^{-(D-2)}(w^2-1)^{(D-3)/2-1/2}\frac{w(1+\chi)- \sqrt{w^2 + \chi} }{w (1+\chi)+\sqrt{w^2 + \chi}}\\
=&-\frac{1}{2d^2}\int_1^\infty dw\,w^{-2}\frac{w(1+\chi)- \sqrt{w^2 + \chi} }{w (1+\chi)+\sqrt{w^2 + \chi}}
\end{align}
\comment{Doesn't $V_2$ go to zero for $D=2$ due to $\Gamma[0]$ in the denominator?}

We finally need 
\begin{align}
\frac{1}{2}\partial_d^2(\mathcal{V}_{D-2}(\chi;d;1/2)=
&-\frac{3}{2d^4}\int_1^\infty dw\,w^{-2}\frac{w(1+\chi)- \sqrt{w^2 + \chi} }{w (1+\chi)+\sqrt{w^2 + \chi}}
\end{align}

Now if we go way back up and compare to averages we needed for the Casimir energy, we wanted: 

\begin{equation}
V(d) = -\frac{1}{2}\alpha_0\langle E^2\rangle = -\frac{1}{2}\hbar c\frac{\delta}{\delta\epsilon(d)}\log Z
\end{equation}
In addition, there is a factor of $-\hbar c\alpha_0/(16\epsilon_0\pi^2)$ 
from the original path integral expression, which includes a $1/2$ from the functional derivative 
($-\delta/(\delta\epsilon)$, and $-1/2$ from the $\det[A]^{-1/2}$ from the initial Gaussian integral.  
So the final Casimir energy is 
\begin{equation}
V = -\frac{3\hbar c\alpha_0}{64\pi^2\epsilon_0d^4} \int_1^\infty dw\,w^{-4}(1-2w^2)
\frac{w(1+\chi)- \sqrt{w^2 + \chi} }{w (1+\chi)+\sqrt{w^2 + \chi}},
\end{equation}
which is the same as the Lifshitz integral.  

\section{Finding the TE Casimir energy}

This section is quite similar to the corresponding Casimir-Polder energy for a test particle placed near a plane.
  In this cae however, we will deal with two planar dielectrics.
  We will thus need the solutions for that Fokker-Planck equation in all regions, and then have to evaluate an integral over space.
  That is the only conceptual difference, otherwise this follows the same pattern.  

I have found a result by Zhou and Spruch, \footnote{Zhou, F, and Spruch, L., ``van der Waals and retardation (Casimir) interactions of an electron or atom with multilayered walls'', Phys. Rev. A, \textbf{52}, 297, (1995)}, where they have in Eq. (3.19),
\begin{align}
\frac{F}{L^2} =& -\frac{\hbar}{2\pi^2c^3}\int_0^\infty d\xi \xi^3 \epsilon_3^{3/2}\int_1^\infty dp\,p^2\left[ \frac{r_{13}r_{23}e^{-2\sqrt{\epsilon_3}p\xi l/c}}{1 - r_{13}r_{23}e^{-2\sqrt{\epsilon_3}p\xi l/c}} + \frac{r'_{13}r'_{23}e^{-2\sqrt{\epsilon_3}p\xi l/c}}{1 - r'_{13}r'_{23}e^{-2\sqrt{\epsilon_3}p\xi l/c}}\right]\\
\Aboxed{\frac{F}{L^2}=& -\frac{\hbar c}{2\pi^2d^4}\int_0^\infty du u^3\int_1^\infty dp\,p^2\left[ \frac{r^2e^{-2pu}}{1 - r^2e^{-2pu}} + \frac{r'^2e^{-2pu}}{1 - r'^2e^{-2pu}}\right]}
\end{align}



\subsection{Rescaling TE again}

Let's try to check our TE results, by rescaling that again.  
\begin{align}
E =& - \partial_\beta \log Z_{TE}\\
 =& -\frac{\hbar c}{8\pi^2}\int_0^\infty \frac{dT}{T^{1+D/2}}\int dx_0 \dlangle \frac{1}{\langle \epsilon\rangle^{\alpha}}\drangle\\
=& - \frac{\hbar c}{8\pi^2\Gamma[(D+1)/2+\alpha]\Gamma(\alpha)} 
\int_0^\infty d\lambda \lambda^{(D-1)/2-\alpha}\int dx_0 \int_0^\infty ds\, s^{\alpha-1}
\int_0^\infty dT e^{-\lambda T}\dlangle \frac{e^{-s\int_0^T dt  \epsilon}}{\sqrt{T}}\drangle\\
=& - \frac{\hbar c}{8\pi^2\Gamma[(D+1)/2+\alpha]\Gamma(\alpha)} 
\int_0^\infty d\lambda \lambda^{(D-1)/2-\alpha}\int_0^\infty ds\, s^{\alpha-1} I(\lambda+s,s\chi)
\end{align}

$I_{tot}$ is the (renormalized) expression for 
\begin{equation}
I_{tot}(\lambda,\chi)=\int dx_0\int_0^\infty dT \frac{e^{-\lambda T}}{\sqrt{T}}\dlangle e^{-\int_0^T dt \chi\Theta(x_0+B(t))  } \drangle,
\end{equation}
we found that 
\begin{align}
I_{tot}(\lambda,\chi) =& I_{12}-I_1-I_2 + I_0 \\
=&  \sqrt{2\pi}\frac{r_1r_2 e^{-2\sqrt{2\lambda}d}}{\sqrt{2\lambda}(1-r_1r_2 e^{-2\sqrt{2\lambda}d})}
\left( 2d + 2\frac{\sqrt{2}}{\sqrt{\lambda+\chi}}\right)
\end{align}

Plugging all this in, we have 
\begin{align}
E  =& - \frac{\hbar c\sqrt{\pi}}{8\pi^2\Gamma[(D+1)/2+\alpha]\Gamma(\alpha)} 
\int_0^\infty d\lambda \lambda^{(D-1)/2-\alpha}\int_0^\infty ds\, s^{\alpha-1}
\frac{2r^2 e^{-2\sqrt{2\lambda}d}}{\sqrt{\lambda+s}(1-r^2 e^{-2\sqrt{2(\lambda+s)}d})}
\left( d + \frac{\sqrt{2}}{\sqrt{\lambda+s+s\chi}} \right),
\end{align}
with 
\begin{equation}
r = \frac{ \sqrt{\lambda+s} - \sqrt{\lambda+s+s\chi}}{ \sqrt{\lambda+s} + \sqrt{\lambda+s+s\chi}}
\end{equation}
Substitution time!
\begin{enumerate}
\item $\lambda = s\kappa$.
\begin{align}
E  =& - \frac{\hbar c\sqrt{\pi}}{8\pi^2\Gamma[(D+1)/2+\alpha]\Gamma(\alpha)} \int_0^\infty d\kappa 
\kappa^{(D-1)/2-\alpha}\int_0^\infty ds\, s^{(D-1)/2-\alpha+1} s^{\alpha-1}\nonumber\\
&\times \frac{2r^2 e^{-2\sqrt{2(\kappa+1)s}d}}{\sqrt{s(\kappa+1)}(1-r^2 e^{-2\sqrt{2s(\kappa+1)}d})}
\left( d + \frac{\sqrt{2}}{\sqrt{s(\kappa+1+\chi)}} \right),
\end{align}
with 
\begin{equation}
r = \frac{ \sqrt{\kappa+1} - \sqrt{\kappa+1+\chi}}{ \sqrt{\kappa+1} + \sqrt{\kappa+1+\chi}}
\end{equation}
\item $p = \sqrt{\kappa+1},$ or $\kappa = p^2-1$. 
\begin{align}
E  =& - \frac{\hbar c\sqrt{\pi}}{8\pi^2\Gamma[(D+1)/2+\alpha]\Gamma(\alpha)} \int_1^\infty dp (2p) 
(p^2-1)^{(D-1)/2-\alpha}\int_0^\infty ds\, s^{(D-1)/2}\nonumber\\
&\times \frac{2r^2 e^{-2\sqrt{2s}pd}}{\sqrt{s}p(1-r^2 e^{-2\sqrt{2s}pd})}
\left( d + \frac{\sqrt{2}}{\sqrt{s(p^2+\chi)}} \right),
\end{align}
with 
\begin{equation}
r = \frac{ p - \sqrt{p^2+\chi}}{ p + \sqrt{p^2+\chi}}
\end{equation}
\item $\xi = \sqrt{2s} d$, or $s = \xi^2/2 d^2$.  

\begin{align}
E  % =& - \frac{\hbar c\sqrt{\pi}}{4\pi^2\Gamma[(D+1)/2+\alpha]\Gamma(\alpha)} 
% \int_1^\infty dp\, (p^2-1)^{(D-1)/2-\alpha}\int_0^\infty d\xi\,\frac{\xi}{d^2} \left( \frac{\xi^2}{2d^2}\right)^{(D-1)/2}\nonumber\\
% &\times \frac{2\sqrt{2} dr^2 e^{-2\xi p}}{\xi (1-r^2 e^{-2\xi p})}\left( d + \frac{2 d}{\sqrt{\xi(p^2+\chi)}} \right)\\
 =& - \frac{\hbar c\sqrt{\pi} 2^{3/2}}{2^{D/2}\pi^2\Gamma[(D+1)/2+\alpha]\Gamma(\alpha) d^{D-1}2^{(D-1)/2}}
 \int_1^\infty dp \, (p^2-1)^{(D-1)/2-\alpha}\int_0^\infty d\xi\,\xi^{D-1}\nonumber\\
&\times \frac{ r^2 e^{-2\xi p}}{(1-r^2 e^{-2\xi p})}\left( 1 + \frac{2 }{\sqrt{\xi(p^2+\chi)}} \right),
\end{align}
with 
\begin{equation}
r = \frac{ p - \sqrt{p^2+\chi}}{ p + \sqrt{p^2+\chi}}
\end{equation}
\item Now set $\alpha=1/2, D=4$.
\begin{align}
\Aboxed{E_{TM} =& - \frac{\hbar c }{8\pi^2 d^{D-1}} \int_1^\infty dp \, (p^2-1)\int_0^\infty d\xi\,\xi^{3} 
\frac{ r^2 e^{-2\xi p}}{ (1-r^2 e^{-2\xi p})}\left( 1 + \frac{2 }{\sqrt{\xi(p^2+\chi)}} \right),}
\end{align}

\end{enumerate}

\begin{align}
E_{TE} = & - \frac{\hbar c }{8\pi^2 d^{3}}\int_0^\infty du\,u^{3}\int_1^\infty dp\,(p^2-1)
\frac{ r^2e^{-2pu}}{(1 -r^2 e^{-2pu})}\left[ 1 +\frac{2}{u\sqrt{p^2+\chi}}\right]
\end{align}

Now integrate by parts with $p$.  
Let's try integrating by parts with respect to $p$?    
\begin{equation}
\int_{1}^\infty dp f(p) = pf(p)\bigg|_{p=1}^\infty -\int_1^\infty dp\, p f'(p) 
\end{equation}
We may have to integrate by parts with respect to $u$ to get the correct power there.  

\begin{align}
\frac{dr}{dp} =& \frac{d}{dp} \frac{p-\sqrt{p^2+\chi}}{p+\sqrt{p^2+\chi}}
= \frac{1-\frac{2p}{2\sqrt{p^2+\chi}}}{p+\sqrt{p^2+\chi}} - 
(p-\sqrt{p^2+\chi})\frac{(1+\frac{2p}{2\sqrt{p^2+\chi}})}{(p+\sqrt{p^2+\chi})^2} 
%=& \frac{1}{\sqrt{p^2+\chi}}\frac{\sqrt{p^2+\chi}-p}{p+\sqrt{p^2+\chi}} - (p-\sqrt{p^2+\chi})\frac{p+\sqrt{p^2+\chi}}{\sqrt{p^2+\chi}(p+\sqrt{p^2+\chi})^2} \\
=& -\frac{2r}{\sqrt{p^2+\chi}}
\end{align}
% Intriguingly, I found that 
% \begin{align}
% \frac{d}{dp}[r^{-2} e^{2pu} -1]&= \left( -2\frac{dr}{dp} r^{-3} e^{2pu} + 2u r^{-2}e^{2pu}\right)\\
% &=  \frac{4}{\sqrt{p^2+\chi}} r^{-2} e^{2pu} + 2u r^{-2}e^{2pu}\\
% &=  2u\left(1+\frac{2}{u\sqrt{p^2+\chi}}\right) r^{-2} e^{2pu}
% \end{align}
We can also write:
\begin{align}
\frac{d}{dp}[1-r^{2} e^{-2pu}]&= -\left( 2r\frac{dr}{dp} e^{-pu} - 2u r^{2}e^{-2pu}\right)\\
&= \left(u+\frac{4}{\sqrt{p^2+\chi}}\right) r^{2}e^{-2pu},
\end{align}
which suggests 
\begin{align}
\Aboxed{\frac{d}{dp}\ln[1-r^{2} e^{-2pu}]&= 2u \frac{r^2 e^{-2pu}}{1 - r^2  e^{-2pu}}\left(1+\frac{2}{u\sqrt{p^2+\chi}}\right)}
\end{align}
So our integral becomes
\begin{align}
I_1 & = \int_1^\infty dp\,(p^2-1)\frac{ r^2e^{-2pu}}{(1 -r^2 e^{-2pu})}\left[ 1 +\frac{2}{u\sqrt{p^2+\chi}}\right]\\
& = \frac{1}{2u}\log\left[1-r^2 e^{-2pu}\right](p^2-1)\bigg|_{p=1}^\infty - \int_1^\infty dp \,\frac{p}{u}\log\left[1-r^2 e^{-2pu}\right]
\end{align}
Note that the boundary term vanishes at both limits.  Let's now apply that integration by parts to the $TE$ energy.  
\begin{align}
%E_{TE} = &  \frac{\hbar c }{8\pi^2 d^{3}}\int_0^\infty du\,u^{2}\int_1^\infty dp\,p \log\left[1- r^2e^{-2pu}\right]\\
\Aboxed{E_{TE}= &  \frac{\hbar c }{8\pi^2}\int_0^\infty d\xi\,\xi^{2}\int_1^\infty dp\,p \log\left[1- r^2e^{-2p\xi d}\right]},
\end{align}
this form has the advantage of merely undoing an earlier scaling (which would not work for frequency dependent materials).
  A differentiation with respect to distance then yields the Lifshitz results.  



\section{Finding the TM Casimir energy}

From November 2013 we have
\begin{align}
I_{12}-I_1-I_2 + I_0 =&   \dfrac{r'_2 e^{-2\sqrt{2\lambda}d}}{4\kappa_1(1-r'_1r'_2 e^{-2\sqrt{2\lambda}d})} 
+\frac{2r'_1r'_2 e^{-\sqrt{2\lambda}d}d}{\sqrt{2\lambda}(1-r'_1r'_2 e^{-2\sqrt{2\lambda}d})} 
- (r'_1+r'_2)\frac{e^{-2\sqrt{2\lambda}d}}{4\lambda(1-r'_1r'_2e^{-2\sqrt{2\lambda}d})}\nonumber\\
& +\frac{r'_1 e^{-2\sqrt{2\lambda}d}}{4\kappa_2(1-r'_1r'_2 e^{-2\sqrt{2\lambda}d})}  
 -   \left(\frac{r'_1}{4\kappa_1}-\frac{(r'_1+r'_2)}{4\lambda}  
+ \frac{r'_2}{4\kappa_2}\right)\frac{r'_1r'_2 e^{-2\sqrt{2\lambda}d}}{(1-r'_1r'_2 e^{-2\sqrt{2\lambda}d})},
\end{align}
where $\kappa = \lambda+\chi$, and 
\begin{equation}
r' =  \frac{\epsilon\sqrt{\lambda}-\sqrt{\kappa}}{\epsilon\sqrt{\lambda}+\sqrt{\kappa}}
\end{equation}

\comment{Let's now set $\chi_1=\chi_2=\chi$}
Now factor out the common terms: 
\begin{align}
I_{tot}=&  \frac{e^{-2\sqrt{2\lambda}d}}{(1-r'^2e^{-2\sqrt{2\lambda}d}}\left[
 \frac{2r'^2d}{\sqrt{2\lambda}}+ \dfrac{r'}{2\kappa}  -\frac{ r'}{2\lambda} 
- \left(\frac{r'}{2\kappa}-\frac{r'}{2\lambda}\right)r'^2\right]\\
=&  \frac{e^{-2\sqrt{2\lambda}d}}{(1-r'^2e^{-2\sqrt{2\lambda}d}}\left[
\frac{2r'^2d}{\sqrt{2\lambda}}+ \frac{r'}{2}(1-r'^2)\left(\dfrac{1}{\kappa}  -\frac{1}{\lambda}\right)\right],
\end{align}


\begin{align}
E =& - \partial_\beta \log Z_{TE}\\
 =& -\frac{\hbar c}{8\pi^2}\int_0^\infty \frac{dT}{T^{1+D/2}}\int dx_0 
\dlangle \frac{1}{\langle \epsilon\rangle^{\alpha}}e^{-\int_0^T\mathfrak{M}dt}\drangle\\
=& - \frac{\hbar c}{8\pi^2\Gamma[(D+1)/2+\alpha]\Gamma(\alpha)} \int_0^\infty d\lambda 
\lambda^{(D-1)/2-\alpha}\int dx_0 \int_0^\infty ds\, s^{\alpha-1}\int_0^\infty dT e^{-\lambda T}
\dlangle \frac{e^{-sT\int_0^T dt  \epsilon-\int_0^T\mathfrak{M}dt}}{\sqrt{T}}\drangle,
\end{align}

$I_{tot}$ is the (renormalized) expression for 
\begin{equation}
I_{tot}=\int dx_0\int_0^\infty dT \frac{e^{-\lambda T}}{\sqrt{T}}\dlangle e^{-\mathfrak{M}-\int_0^T dt V(x_0+B(t))  } \drangle,
\end{equation}

\begin{enumerate}
\item We get the energy by making the following replacements \comment{$\lambda\rightarrow \lambda+s$, $\chi\rightarrow s\chi$.  } 
(However, this does not apply to terms in $e^{\Xi}$.  )
 \begin{align}
E=&  C\int_0^\infty d\lambda \lambda^{(D-1)/2-\alpha}\int_0^\infty ds\, s^{\alpha-1}\nonumber\\
&\times  \frac{e^{-2\sqrt{2(\lambda+s)}d}}{1-r'^2e^{-2\sqrt{2(\lambda+s)}d}}
\left[\frac{2r'^2d}{\sqrt{2(\lambda+s)}}+ \frac{r'}{2}(1-r'^2)\left(\dfrac{1}{\lambda+s+s\chi}  -\frac{1}{\lambda+s}\right)\right],
\end{align}
and 
\begin{equation}
r' =  \frac{(1+\chi)\sqrt{\lambda+s}-\sqrt{\lambda+ s + s\chi}}{(1+\chi)\sqrt{\lambda+s}+\sqrt{\lambda+s+s\chi}},
\end{equation}
\begin{equation}
C = -\frac{\hbar c\sqrt{2\pi}}{8\pi^2\Gamma[(D+1)/2+\alpha]\Gamma(\alpha)} 
\end{equation}

\item Now take $\lambda = qs$.  
 \begin{align}
E=&  C\int_0^\infty dq q^{(D-1)/2-\alpha}\int_0^\infty ds\, s^{\alpha-1}s^{1+(D-1)/2-\alpha}\nonumber\\
&\times  \frac{e^{-2\sqrt{2(q+1)s}d}}{1-r'^2e^{-2\sqrt{2(q+1)s}d}}\left[
\frac{2r'^2d}{\sqrt{2s(q+1)}}+ \frac{r'}{2}(1-r'^2)\left(\dfrac{1}{s(q+1+\chi)}  -\frac{1}{s(q+1)}\right)\right],
\end{align}
and 
\begin{equation}
r' =  \frac{(1+\chi)\sqrt{q+1}-\sqrt{q+ 1 + \chi}}{(1+\chi)\sqrt{q+1}+\sqrt{q+1+\chi}},
\end{equation}
\item Now take $p = \sqrt{q+1}$.  $\rightarrow  q = p^2-1, dq = 2p dp$.  
 \begin{align}
E%=&   C\int_1^\infty dp(2p) (p^2-1)^{(D-1)/2-\alpha}\int_0^\infty ds\, s^{\alpha-1}s^{1+(D-1)/2-\alpha}\nonumber\\
% &\times  \frac{e^{-2\sqrt{2s}pd}}{1-r'^2e^{-2\sqrt{2s}pd}}\left[\frac{2r'^2d}{\sqrt{2s}p}+ 
% \frac{r'}{2}(1-r'^2)\left(\dfrac{1}{s(p^2+\chi)}  -\frac{1}{s p^2}\right)\right]\\
=&  2C\int_1^\infty dp (p^2-1)^{(D-1)/2-\alpha}\int_0^\infty ds\, s^{(D-1)/2}\nonumber\\
&\times\frac{e^{-2\sqrt{2s}pd}}{1-r'^2e^{-2\sqrt{2s}pd}}\left[\frac{2r'^2d}{\sqrt{2s}}- 
\frac{r'}{2}(1-r'^2)\left(\dfrac{\chi}{s(p^2+\chi)p}\right)\right],
\end{align}
and 
\begin{equation}
r' =  \frac{(1+\chi)p-\sqrt{p^2 + \chi}}{(1+\chi)p+\sqrt{p^2+\chi}},
\end{equation}
\item Finally, take 
$s = \xi^2$
\item Now take $p = \sqrt{q+1}$.  $\rightarrow  q = p^2-1, dq = 2p dp$.  
 \begin{align}
E% =&  4C\int_1^\infty dp (p^2-1)^{(D-1)/2-\alpha}\int_0^\infty d\xi\,\xi\xi^{D-1}\nonumber\\
% &\times  \frac{e^{-2\sqrt{2}\xi pd}}{1-r'^2e^{-2\sqrt{2}\xi pd}}\left[\frac{2r'^2d}{\sqrt{2}\xi}-
%  \frac{r'}{2}(1-r'^2)\left(\frac{ \chi}{\xi^2(p^2+\chi)p}\right)\right]\\
=&  4C\int_1^\infty dp (p^2-1)^{(D-1)/2-\alpha}\int_0^\infty d\xi\,\xi^{D-1}\nonumber\\
&\times  \frac{r'^2e^{-2\sqrt{2}\xi pd}}{1-r'^2e^{-2\sqrt{2}\xi pd}}\left[\frac{2d}{\sqrt{2}}-
 \frac{1}{2}\left(\frac{1}{r'}-r'\right)\left(\dfrac{\chi}{\xi(p^2+\chi)p}\right)\right],
\end{align}
and 
\begin{equation}
r' =  \frac{(1+\chi)p-\sqrt{p^2 + \chi}}{(1+ \chi)p+\sqrt{p^2+\chi}},
\end{equation}

\item Now take $u = \sqrt{2}\xi d$ 
 \begin{align}
E% =&  4C\frac{1}{2^{5/2}d^4}\int_1^\infty dp (p^2-1)\int_0^\infty du\,u^{3} 
% \frac{r'^2e^{-2u p}}{1-r'^2e^{-2u p}}\left[2d- \left(\frac{1}{r'}-r'\right)\dfrac{\chi d}{u(p^2+\chi)p}\right]\\
=&  C\frac{\sqrt{2}}{d^3}\int_1^\infty dp (p^2-1)\int_0^\infty du\,u^{3} 
 \frac{r'^2e^{-2u p}}{1-r'^2e^{-2u p}}\left[1- \left(\frac{1}{r'}-r'\right)\dfrac{\chi }{2u(p^2+\chi)p}\right],
\end{align}
and 
\begin{equation}
r' =  \frac{(1+\chi)p-\sqrt{p^2 + \chi}}{(1+ \chi)p+\sqrt{p^2+\chi}},
\end{equation}
\comment{proportional to Schwinger}
\end{enumerate}

\subsection{Integrate by parts w.r.t. $p$}
Let's check if this also goes for the TM case.  
\begin{align}
\frac{d}{dp}\ln[1-r'^2 e^{-2pu}] =& \frac{-2r' \frac{dr'}{dp} e^{-2pu} + 2u r'^2 e^{-2pu}}{1-r'^2 e^{-2pu}} \\
% &= \frac{r'^2 e^{-2pu}}{1-r'^2 e^{-2pu}}\left( 2u -\frac{2}{r'} \frac{dr'}{dp}\right)\\
% &= 2u\frac{r'^2 e^{-2pu}}{1-r'^2 e^{-2pu}}\left( 1 -\frac{1}{ur'} \frac{dr'}{dp}\right)\\
&= 2u\frac{r'^2 e^{-2pu}}{1-r'^2 e^{-2pu}}\left( 1 -\frac{1}{u} \frac{d\ln[r']}{dp}\right)
\end{align}

Now use 
\begin{align}
\frac{d}{dp}\ln[r'] =& \frac{d}{dp}\left(\log[e^{2\Xi}p - \sqrt{p^2+\chi}] -\ln[e^{2\Xi}p + \sqrt{p^2+\chi}]\right) \\
=& \frac{e^{2\Xi} - \frac{p}{\sqrt{p^2+\chi}}}{e^{2\Xi}p-\sqrt{p^2+\chi}} -\frac{e^{2\Xi} + \frac{p}{\sqrt{p^2+\chi}}}{e^{2\Xi}p + \sqrt{p^2+\chi}}\\ 
%=& \frac{\left[e^{2\Xi} - \frac{p}{\sqrt{p^2+\chi}}\right]\left[e^{2\Xi}p + \sqrt{p^2+\chi}\right]-\left[e^{2\Xi} + \frac{p}{\sqrt{p^2+\chi}}\right]\left[e^{2\Xi}p-\sqrt{p^2+\chi}\right]}{e^{4\Xi}p^2-(p^2+\chi)}\\
% =& \frac{\left[e^{4\Xi}p  - e^{2\Xi}\frac{p^2}{\sqrt{p^2+\chi}} +e^{2\Xi}\sqrt{p^2+\chi} - p\right]}{e^{4\Xi}p^2-(p^2+\chi)}\nonumber\\
% & - \frac{\left[e^{4\Xi}p + e^{2\Xi}\frac{p^2}{\sqrt{p^2+\chi}} -e^{2\Xi}\sqrt{p^2+\chi}-p\right]}{e^{4\Xi}p^2-(p^2+\chi)}\\
% =& \frac{2e^{2\Xi}\sqrt{p^2+\chi} - 2e^{2\Xi}\frac{p^2}{\sqrt{p^2+\chi}} }{e^{4\Xi}p^2-(p^2+\chi)}\nonumber\\
=& \frac{2\chi e^{2\Xi}}{\sqrt{p^2+\chi}[e^{4\Xi}p^2-(p^2+\chi)]}\label{eq:TM_integration_by_parts}
\end{align}

Then 
\begin{align}
\Aboxed{\frac{d}{dp}\ln[1-r'^2 e^{-2pu}] &= 
2u\frac{r'^2 e^{-2pu}}{1-r'^2 e^{-2pu}}\left( 1 -\frac{2\chi e^{2\Xi}}{u\sqrt{p^2+\chi}[e^{4\Xi}p^2-(p^2+\chi)]}\right)}
\end{align}

% Ultimately, we want this to be proportional to 
% \begin{align}
% C=&\frac{r'^2 e^{-2pu}}{1-r'^2 e^{-2pu}}\left[1 - \left(\frac{1}{r'}-r'\right)\dfrac{\chi }{2u(p^2+\chi)p}\right]\\
% =&\frac{r'^2 e^{-2pu}}{1-r'^2 e^{-2pu}}\left[1 - \left(\frac{e^{2\Xi}p+\sqrt{p^2+\chi}}{e^{2\Xi}p-\sqrt{p^2+\chi}}
% -\frac{e^{2\Xi}p-\sqrt{p^2+\chi}}{e^{2\Xi}p+\sqrt{p^2+\chi}}\right)\dfrac{\chi }{2u(p^2+\chi)p}\right]\\
% =&\frac{r'^2 e^{-2pu}}{1-r'^2 e^{-2pu}}\left[1 - \frac{4pe^{2\Xi}\sqrt{p^2+\chi}}{e^{4\Xi}p^2-(p^2+\chi)}\dfrac{\chi }{2u(p^2+\chi)p}\right]\\
% =&\frac{r'^2 e^{-2pu}}{1-r'^2 e^{-2pu}}\left[1 - \frac{2\chi e^{2\Xi}}{u\sqrt{p^2+\chi}[e^{4\Xi}p^2-(p^2+\chi)]}\right]\label{eq:CTM}\\
% \end{align}
% So this will work.  

\begin{align}
E_{TM} = & -\frac{\hbar c}{8\pi^2 d^3}\int_0^\infty du\,u^{3} \int_1^\infty dp\, (p^2-1) 
\frac{r'^2e^{-2u p}}{1-r'^2e^{-2u p}}\left[1- \left(\frac{1}{r'}-r'\right)\dfrac{\chi }{2u(p^2+\chi)p}\right],
\end{align}
We can use Eqs.~(\ref{eq:CTM}) and (\ref{eq:TM_integration_by_parts}) to integrate with respect to $p$.  
\begin{align}
I_2 =& \int_1^\infty dp\, (p^2-1) \frac{r'^2e^{-2u p}}{1-r'^2e^{-2u p}}
\left[1- \left(\frac{1}{r'}-r'\right)\dfrac{\chi }{2u(p^2+\chi)p}\right]\\
=& \left[(p^2-1)\frac{1}{2u}\log[1-r'^2 e^{-2pu}]\right]_{p=1}^\infty - \int_1^\infty dp\,\frac{p}{u}\log[1-r'^2 e^{-2pu}]
\end{align}
Which gives us 
\begin{align}
\Aboxed{E_{TM}= & \frac{\hbar c}{8\pi^2 }\int_0^\infty d\xi\,\xi^{2} \int_1^\infty dp\, p \log[1-r'^2 e^{-2p\xi d}]}
\end{align}
\comment{Am I off by $1/4$?  That should be a matter of more careful accounting.}


\section{Finite Temperature and Dispersion}

We will handle the finite temperature (and dispersion) in the atom-plane geometry.
  We derive the partition function for finite temperature for both the TE and TM polarizations.
  So far, I have done both Casimir and Casimir-Polder energies for TE, and only Casimir-Polder for TM.
  The zero temperature limits all work out nicely.
  I am stumbling a little over the right way to do the high temperature limit here.  


\subsection{Should really be using the Free Energy?}

So I read Babb's paper\footnote{Babb, J. F. and Klimchitskaya, G. L., and Mostepanenko, V. M., 
``Casimir-Polder interaction between an atom and a cavity wall under the influence of real conditions'',
 Phys. Rev. A, \textbf{70},042901,(2004)} (which Dan cites for thermal Casimir-Polder calculations.~\cite{Babb2004})
  In it they use the free energy, which is $\mathcal{F} = -k_BT\log Z$, as the basis of their calculations.
  I've been trying to use the mean energy, $E= -\partial_\beta\log Z$. 

% Tanmoy's initial calculation emphasized using $\mathcal{F}$ in the limit $\beta\rightarrow \infty$.
%   I subbed in using the mean energy angle because I (for some reason) felt more comfortable with that.
%   In the limit $T\rightarrow 0$ they of course agree - as borne out above.  

% However my finite temperature results using the energy are completely crap.
% If I actually use $\mathcal{F}$, then I think I reproduce Dan's  finite temperature result\footnote{Steck, D. A.,
%  ``Quantum Optics Notes'', Eq. (14.326)}: 
% \begin{equation}
%   V_{CP} = \frac{k_BT}{4\pi\epsilon_0c^2}{\sum_{n=0}^{\infty}}'s_n^2\alpha(is_n)
% \int_0^\infty dk_T \frac{k_T}{\kappa_n}\left[ r_\perp(\theta,is_n)+ 
%   \left(1 +\frac{2k_T^2c^2}{s_n^2}\right)r_\|(\theta,is_n)\right]e^{-2\kappa_nz},
% \end{equation}
% where $\kappa_n = \sqrt{s_n^2/c^2+k_T^2}$, and $r_\|, r_\perp$ still depend on $\cos(\theta)$.
%   I think after variable transformation they become the reflection coefficients we use above.  

% In the limit of zero temperature, then the factor of $\beta^{-1}$ will get eaten - 
% as the derivative has been doing for us. 
% For high temperature, I don't need to take any derivatives, and I get exactly the results I want.       

\section{TE Polarization: Thermal Partition Function}

Our partition function is 
\begin{equation}
Z_{TE} = \int D\phi \exp\left[ -\frac{\epsilon_0}{2}\int_0^\beta d\beta'\int d^3x\, 
\left( \frac{\epsilon(x)}{\hbar^2}(\partial_{\beta'}\phi)^2 + c^2|\nabla\phi|^2\right)\right] .
\end{equation}

Let us change to using $\tau = \beta \hbar c$ as our temperature coordinate.  Then   
\begin{equation}
Z_{TE} = \int D\phi \exp\left[ -\frac{\epsilon_0 c^2 }{2 \hbar c}\int_0^\beta d\tau'\int d^3x\, 
\left( \epsilon(x)(\partial_{\tau}\phi)^2 + |\nabla\phi|^2\right)\right] .
\end{equation}

We will now introduce the Matsubara frequencies $\omega_n = (2\pi n)/(\beta \hbar)$, with 
\begin{equation}
\phi(\beta,x) = \sum_{n=-\infty}^{\infty}e^{i\frac{\omega_n}{c}\tau} \phi_n(x),
\end{equation}
where $\tau = \beta\hbar c$, and the $\phi_n$ are complex variables.  We will also need to use  
\begin{equation}
\int_0^{\beta \hbar c}d\tau e^{i\frac{(\omega_n+\omega_m)}{c}\tau} = \beta\hbar c \delta_{n,-m},
\end{equation}
and $\phi_n^* = \phi_{-n}$ since $\phi^*(\beta, x) = \phi(\beta, x)$.  

\begin{shaded}
\comment{See Dec 2012 notes for how to handle the variable counting from doubling the number of variables}
I'm just trying to figure out the factors of 2 here - just think of the transform to Matsubara 
frequencies as a ordinary change of variables.
  In December I had completely reduced the problem to a discrete problem in $\beta$ as well.
  Say I have $N_\beta$ initial time-steps.
  I have $N_\beta$ real variables.
  If I introduce a Fourier series, then I now have $N-2$ complex variables for $0<n<N_\beta/2$, and 2 real variables at $n=0, N_\beta/2$.
  Furthermore we know that $\phi_n^* = \phi_{-n}$.
  So from the Gaussian structure of the integrals over $\phi_n$, each of these $(N-2)$ integrals over $\phi_n$ is equal.
  When I carry out the integrals I have 
\begin{equation}
\prod_{n=-N_\beta/2}^{N_\beta/2-1}\int D\phi_n e^{-A_n|\phi_n|^2} 
= C\left( \frac{1}{\sqrt{A_0}}\frac{1}{\sqrt{A_{N_\beta/2}}}\prod_{n>0}\frac{1}{A_n}\right),
\end{equation}
where I have used 
\begin{equation}
\int D\phi_nD\phi_n^* e^{-A_n|\phi_n|^2} = \int D\phi_r D\phi_ie^{-A_n(\phi_r^2+\phi_n^2)} = \frac{1}{A_n}, 
\end{equation}
where $\phi_n = \phi_r + i \phi_i$.
  We then consider the limit where $N_\beta\rightarrow \infty$.  

Does this still work with $\epsilon(i\omega)?$  What do the Kramers-Kr\"onig relations have to say?
  Is $\epsilon^*(i\omega) = \epsilon(-i\omega)$?
  Or is it just a result of saying this Hamiltonian is real.
  If so, we're golden.  
\end{shaded}

Then we have 
\begin{equation}
Z_{TE} = \prod_{n=-\infty}^{\infty} \int D\phi_n\exp\left[ -\frac{\beta \epsilon_0 c^2 }{2}
\int d^3x\, \phi_n^*\left(\epsilon(x)\frac{\omega_n^2}{c^2} -\nabla^2\right)\phi_n\right] .
\end{equation}

Now let's \comment{assume} we just handle dispersion by taking $\epsilon(x)\rightarrow \epsilon(i\omega_n,x)$.
  This idea is following Rahi's derivation where $\epsilon$ is treated as an effective action 
where the effect of the electron field has been intregrated out.  
Then 
\begin{equation}
\log Z_{TE} = -{\sum_{n=0}^\infty}'\log\det\left[\frac{1}{2}
\left(\epsilon(i\omega_n,\vect{x})\frac{\omega_n^2}{c^2} -\nabla^2\right)\right].
\end{equation}
where there is only a factor of $\frac{1}{2}$ for $n=0$.
  We will renormalize this against vacuum, 
\begin{equation}
\log Z_{TE} -\log Z_0= -{\sum_{n=0}^\infty}'\left\{\log\det\left[ 
\frac{1}{2}\left(\epsilon(i\omega_n,\vect{x})\frac{\omega_n^2}{c^2} -\nabla^2\right)\right] 
- \log\det\left[ \frac{1}{2}\left(\frac{\omega_n^2}{c^2} -\nabla^2\right)\right]\right\}
\end{equation}
Note that for $n=0$ the Matsubara frequency $\omega_n=0$, so we have 
\begin{align}
  \log\det\left[ \frac{1}{2}\left(\epsilon(0,\vect{x})\frac{\omega_0^2}{c^2} -\nabla^2\right)\right]
 - \log\det\left[ \frac{1}{2}\left(\frac{\omega_0^2}{c^2} -\nabla^2\right)\right] = 0.
\end{align}
This assumes that $\epsilon(\omega)$ has at most a simple pole at zero frequency, 
such that $\lim_{\omega\rightarrow 0}\omega^2\epsilon(\omega)=0.$    

Now the renormalized free energy is 
\begin{align}
F-F_0 & = -k_BT \log \frac{Z_{TE}}{Z_0} \\
% & = k_BT {\sum_{n}}'\tr\left\{ \log\left[ \frac{1}{2}
%     \left(\epsilon(i\omega_n,\vect{x})\frac{\omega_n^2}{c^2} -\nabla^2\right)\right]
% -\log\left[ \frac{1}{2}\left(\frac{\omega_n^2}{c^2} -\nabla^2\right)\right]\right\}\\
&=k_BT{\sum_n}'\int_0^\infty \frac{dT}{T}\int d^3x\,\frac{1}{(2\pi T)^{3/2}}
\dlangle e^{-T\frac{\omega_n^2}{2c^2}} -  e^{-\frac{ T \omega_n^2\langle\epsilon(i\omega_n)\rangle}{2c^2}}\drangle
\label{eq:TEworldline_partition_function}
\end{align}
where we introduced the path integral.
  In this case the operator $ e^{T\nabla^2}$ only needs a 3-dimensional Hilbert space.
  So we also only use the normalization for 3D.
  The remaining factor of $\sqrt{2\pi T}$ will emerge in the zero temperature limit.
  This scaling with $T$ also reflects the different scaling behaviours in the near-field, 
thermal and far-field regions, as these will each have different approximations to the Matsubara sum.  

% \begin{shaded}
% The sequence of events we actually followed when we neglected dispersion from the start was
% \begin{enumerate}
% \item Take zero temperature limit, which lets us treat $\omega_n^2$ as $\partial_\tau^2$ in the field partition function.  
% \begin{align}
% \log Z & = -\frac{1}{2}\log\det[-\frac{1}{2}\epsilon\partial_\tau^2 - \frac{1}{2}\nabla^2] \\
% & = \frac{1}{2}\int d^4x_0 \int \frac{dT}{T} \langle x_0| e^{[\epsilon(\hat{x})T+\nabla^2]T/2}|x_0\rangle\\
% & = \frac{1}{8\pi^2}\int_0^{\beta\hbar c}d\tau_0\int d^3x_0 \int \frac{dT}{T^3} \dlangle \frac{1}{\sqrt{\langle \epsilon\rangle}}\drangle
% \end{align}
% \item Take $k_BT \log Z$ - which only actually affects $\int d\tau_0 = \beta\hbar c$, which is now a coordinate
% \begin{equation}
% E = -\frac{\hbar c}{8\pi^2}\int d^3x_0 \int \frac{dT}{T^3} \dlangle\frac{1}{\sqrt{\langle \epsilon\rangle}}\drangle
% \end{equation}
% \item Expand $\epsilon(x)$ to linear order in $\alpha_0/\epsilon_0$.  All functions are just functions of space.  
% \begin{equation}
% E(x') = \frac{\hbar c\alpha_0}{16\pi^2\epsilon_0} \int \frac{dT}{T^3} \dlangle \frac{1}{\langle \epsilon\rangle^{3/2}}\drangle_{x(0)=x'}.
% \end{equation}

% \end{enumerate}
% \end{shaded}

% \begin{shaded}
% \subsection{First thing I tried, and evidently the wrong thing to do}

% (Note that using the free-energy would entirely bypass these concerns : No extra derivatives, no problems.)

% So I think it is correct to start from:
% \begin{equation}
% E=-\partial_\beta \log Z = -\partial_\beta{\sum_n}'\int_0^\infty \frac{dT}{T}\int d^3x\,\frac{1}{(2\pi T)^{3/2}}\dlangle  - e^{-\frac{ T \omega_n^2\langle\epsilon(i\omega_n)\rangle}{2c^2}}\drangle
% \end{equation}

% For clarity, from here on I will suppress the renormalization terms.  They can be restored by subtracting off the same thing but with $\epsilon\rightarrow 1$ everywhere.  

% Our sequence of operations once we got $\log Z$ in this case was: 
% \begin{enumerate}
% \item Take $\partial_\beta\log Z$ to get the energy.  
% \begin{align}
% E&={\sum_n}'\frac{\omega_n^2}{\beta c^2}\int d^3x_0\int_0^\infty dT\,\frac{1}{(2\pi T)^{3/2}}\dlangle  \left( \langle\epsilon(i\omega_n)\rangle +\frac{i}{2}\omega_n\langle\partial_\omega\epsilon(i\omega_n)\rangle \right)e^{-\frac{ T \omega_n^2\langle\epsilon(i\omega_n)\rangle}{2c^2}}\drangle
% \end{align}

% \item Expand to linear order in $\alpha$

% \begin{align}
% E&={\sum_n}'\frac{\omega_n^2}{\epsilon_0\beta c^2}\int_0^\infty dT\,\frac{1}{(2\pi T)^{3/2}}\dlangle  \left(\alpha(i\omega_n)+\frac{i}{2}\omega_n\partial_\omega\alpha(i\omega_n) - \frac{ T \omega_n^2\alpha(i\omega_n)}{2c^2} \right)e^{-\frac{ T \omega_n^2\langle\epsilon(i\omega_n)\rangle}{2c^2}}\drangle_{x_0=x'}
% \end{align}
% \item Take zero temperature limit, which replaces $\omega_n\rightarrow \omega,\sum_n \rightarrow \int_0^\infty d\omega$.  Also take the far-field limit on the frequency integral.  $\alpha(i\omega)\rightarrow\alpha_0, \epsilon(i\omega,x)\rightarrow \epsilon(x)$.  
% \begin{equation}
% E=\frac{\hbar}{2\pi\epsilon_0 c^2}\int_0^\infty d\omega\,\omega^2\int_0^\infty dT\,\frac{1}{(2\pi T)^{3/2}}\dlangle   \left(\alpha_0 - \frac{ T \omega^2\alpha_0}{2c^2} \right)e^{-\frac{ T \omega^2\langle\epsilon\rangle}{2c^2}}\drangle
% \end{equation}
% \item Carry out the Gaussian integral in frequency.  
% \begin{equation}
% E= \frac{\hbar c\alpha_0}{8\pi^2 \epsilon_0}\int_0^\infty \frac{dT}{T^3}\dlangle  \left( \frac{1}{\langle \epsilon\rangle^{3/2}}-\frac{3}{2\langle\epsilon\rangle^{5/2}} \right) \drangle.
% \end{equation}
% \end{enumerate}
% Perhaps the derivative $\partial_\beta$ does not commute with some of these limits and approximations?  Perhaps this should be delayed to the final stage?  A derivative of a sum, is not necessarily equal to the sum of derivatives?  

% Of those steps, I think you can safely swap  3 and 4.  But I think these must precede step 5, otherwise you don't have an integral, or even a Gaussian one.  
% \end{shaded}

\subsection{Casimir-Polder energy}
We can extract the Casimir-Polder energy by introducing a test-particle,
 with $\epsilon(\omega,\vect{x})\rightarrow \epsilon(\omega,\vect{x})
+\frac{\alpha(i\omega)}{\epsilon_0}\delta(\vect{x}-\vect{x}_0)$.
  The factor of $\epsilon_0$ arises since this is the relative permittivity.
  Some care is necessary with the $\delta$-function, and should really be considered the limiting 
result of some function like $f(x,\sigma) = e^{-x^2/(2\sigma^2)}/\sqrt{2\pi\sigma^2}$.
  We will then expand the energy to linear order in $\alpha$.  

\begin{align}
E-E_0&=k_BT\sum'_n\,\int_0^\infty \frac{dT}{T}\int d^3x\,\frac{1}{(2\pi T)^{3/2}}
\dlangle e^{-T\frac{\omega_n^2}{2c^2}} - \left(1 - \frac{ T\omega_n^2}{2\epsilon_0c^2}\alpha(i\omega_n)
\langle \delta(x-x')\rangle\right)e^{-\frac{ T \omega_n^2\langle\epsilon(i\omega_n)\rangle}{2c^2}}\drangle
\end{align}
I think the first two terms the energy due to the dielectrics on their own, without the atom, 
and such can be subtracted from the atom-wall interaction energy.
  The physical renormalization is to expand the energy for just a polarizable particle,
 and consider the energy change with and without the presence of the other dielectric.
  So we subtract off the same energy, but with $\epsilon=1$ everywhere.

We can also simplify this a bit by noting that 
\begin{equation}
\int d^3x_0\, \dlangle \frac{1}{T}\int_0^T dt \delta(x(t)-x')\drangle = \int d^3x_0 \delta(x_0-x'),
\end{equation}
This follows from considering the discrete form of the path integral,
 $\int d^3x_0 \prod_i\int d^3x_i f(x_i-x_{i-1})$, where the integral is cyclic under permutations of indices.  
In this way we can always relabel the point that passes through $x_0$ to be the starting point of the loop.
   So we will then only consider loops that start and return to $x_0 = x'$.
   Mathematically, we will use this delta function to eliminate the integral over $\int d^3x_0$.  

The renormalized result is 
\begin{align}
E-E_0&=-k_BT{\sum_n}\,\frac{ \omega_n^2}{2\epsilon_0c^2}\alpha(i\omega_n)\int_0^\infty dT\,
\frac{1}{(2\pi T)^{3/2}}\dlangle e^{-T\frac{\omega_n^2}{2c^2}} 
-e^{-\frac{ T \omega_n^2\langle\epsilon(i\omega_n)\rangle}{2c^2}}\drangle\label{eq:TE_thermal_energy}
\end{align}

We will focus on the Casimir-Polder result, since I can then compare to Dan's expressions in the notes.
  For the Casimir results, I will collate some results from the literature to compare in the nonzero
 temperature/dispersive cases.  


% \begin{align}
% E-E_0&=-k_BT \frac{\hbar\beta}{2\pi}\int_{-\infty}^\infty d\omega\int_0^\infty \frac{dT}{T}
%\int d^3x\,\frac{1}{(2\pi T)^{3/2}}\dlangle e^{-T\frac{\omega^2}{2c^2}} 
%-  e^{-\frac{ T \omega^2\langle\epsilon(i\omega)\rangle}{2c^2}}\drangle,
% \end{align}


% \subsection{Far-field approximation, zero temperature}

% Let us consider how to take the far-field approximation from these expressions.  Since we are taking an ensemble average over Gaussian random walks, so the loops will intersect all the surfaces when $T\sim d^2$, where $d$ is the distance from the source point $x_0$ to the farthest surface.  Secondly, the frequency integral is dominated by the exponential factors, which will contribute most when $T\omega^2/c^2\sim 1$.  This suggests that frequencies with $d^2\omega^2/c^2<1$ will contribute most.  In the limit where $d/c$ is large, only frequencies near $0$ will contribute, and we can approximate $\epsilon(i\omega) \approx \epsilon(0)$ everywhere.



% \begin{shaded}
% One question is, can we say $\lim_{\omega\rightarrow 0}\partial_\omega\alpha(i\omega)= 0?$  
%If we approximate the atom as a harmonic oscillator \footnotemark
%  with damping $\gamma$, and resonant frequency $\omega_0$   , then 
% \begin{equation}
% \alpha(i\omega)= \frac{e/m}{\omega_0^2+\omega^2+\gamma\omega},
% \end{equation}
% whereas if we do a quantum mechanical perturbation theory calculation \footnotemark  we get
% \begin{equation}
% \alpha(i\omega) =\sum_j \frac{2\omega_{j0}|\langle g | d_z|e_j\rangle|^2}{2\hbar(\omega_{j0}^2+\omega^2)},
% \end{equation}
% where $\omega_{j0}$ are the transition frequencies.
%  Evidently in both cases $\lim_{\omega\rightarrow 0}\partial_\omega\alpha(i\omega)=0$, if $\gamma=0$.   
% \end{shaded}
% \footnotetext{Rosa, F.S.S, Dalvit, D. A. R. and Milonni, P. W., Phys. Rev. A, \textbf{84},053813,(2011), ``Electromagnetic energy, absorption, and Casimir Forces, II. Inhomogenous dielectric media''}
% \footnotetext{Steck, Daniel A. ``Quantum Optics Notes'', Eq.(14.146) for the scalar Kramers-Heisenberg formula, see also Eq.(14.152).}

\subsubsection{Feynman-Kac formula}

From Dan's work (or my re-working of it) we have the Laplace-Mellin transform for the single body Feynman-Kac formula,
% \begin{align}
% \int_0^\infty \frac{dT}{T^{1+z}}\dlangle e^{-s[T+ \chi\int_0^Tdt\Theta(x-d)]}\drangle 
% =& \int_0^\infty \frac{dT}{T^{1+z-1/2}}e^{-sT}\dlangle \frac{e^{-s \chi \int _0^T dt \Theta(x-d)}}{\sqrt{T}}\drangle\\
% =& \frac{1}{\Gamma[z+1/2]}\int_0^\infty d\lambda\, \lambda^{z-1/2}\int_0^\infty dT e^{-(\lambda+s)T}
% \dlangle \frac{e^{-s \chi \int_0^T dt \Theta(x-d)}}{\sqrt{T}}\drangle.
% \end{align}
% In our case $z=1/2$, and $s= \omega^2/(2c^2)$.
%   We also need the actual analytical expression for that path integral,
% \begin{equation}
% \int_0^\infty dT e^{-(\lambda+s) T} \dlangle \frac{e^{-s\chi\int_0^T dt \Theta(x-d)}}{\sqrt{2\pi T}}\drangle  
% =\frac{1}{\sqrt{2(\lambda+s)}}\left[1 - e^{-2\sqrt{2(\lambda+s)}|d|}\frac{\sqrt{\lambda+s(1+\chi)}
% -\sqrt{\lambda+s}}{\sqrt{\lambda+s(1+\chi)}+\sqrt{\lambda+s}}\right].
% \end{equation}
We will need to apply both of these results as 
\begin{align}
\int_0^\infty dT\,\frac{1}{(2\pi T)^{3/2}}\dlangle e^{-sT} - e^{-sT \langle\epsilon(i\omega)\rangle}\drangle 
& =\frac{1}{2\pi}\int_0^\infty d\lambda\, \frac{e^{-2\sqrt{2(\lambda+s)}|d|}}{\sqrt{2(\lambda+s)}}
\frac{\sqrt{\lambda+s[1+\chi(i\omega)]}-\sqrt{\lambda+s}}{\sqrt{\lambda+s[1+\chi(i\omega)]}+\sqrt{\lambda+s}}
\end{align}

On plugging this in to Eq.~(\ref{eq:TE_thermal_energy}) we have
\begin{align}
E-E_0&=-k_BT{\sum_n}'\frac{\omega_n^2\alpha(i\omega_n)}{4\pi\epsilon_0c^2}\int_0^\infty d\lambda\, 
\frac{e^{-2\sqrt{2(\lambda+\omega_n^2/(2c^2))}|d|}}{\sqrt{2(\lambda+\omega_n^2/(2c^2))}}
\frac{\sqrt{\lambda+\omega_n^2/(2c^2)[1+\chi(i\omega_n)]}-\sqrt{\lambda+\omega_n^2/(2c^2)}}
{\sqrt{\lambda+\omega_n^2/(2c^2)[1+\chi(i\omega_n)]}+\sqrt{\lambda+\omega_n^2/(2c^2)}},
\end{align}
Let's make a couple variable changes to put this into a more tractable form.  
First we change integration variable using $\lambda = \kappa \omega_n^2/(2c^2)$.  
\begin{align}
E-E_0&=-k_BT{\sum_n}'\frac{\omega_n^2\alpha(i\omega_n)}{4\pi\epsilon_0c^2}\frac{\omega_n^2}{2c^2}
\int_0^\infty d\kappa\, \frac{e^{-2\omega_n\sqrt{(\kappa+1)}|d|/c}}{\omega_n\sqrt{(\kappa+1)}/c}
\frac{\sqrt{\kappa+1+\chi(i\omega_n)]}-\sqrt{\kappa+1}}{\sqrt{\kappa+1+\chi(i\omega_n)}+\sqrt{\kappa+1}},
\end{align}
Next we define $\kappa +1= p^2$.  
\begin{align}
E-E_0&=-k_BT{\sum_n}'\frac{\omega_n^3\alpha(i\omega_n)}{4\pi\epsilon_0c^3}\int_1^\infty dp\,e^{-2\omega_n p|d|/c}
\frac{\sqrt{p^2+\chi(i\omega_n)}-p}{\sqrt{p^2+\chi(i\omega_n)}+p},
\label{eq:TE_CP_finite_temperature}
\end{align}
This is the general result for finite temperature and dispersion.
  We can also take the zero temperature limit.
  In the limit $\beta\rightarrow \infty$ the difference between Matsubara frequencies approaches zero,
 $\Delta\omega_n =\frac{2\pi}{\beta\hbar}$.  Then we can take $\sum_n\Delta\omega_n \rightarrow \int_0^\infty d\omega$.
\begin{align}
E-E_0&=-\frac{\hbar}{8\pi^2\epsilon_0c^3}\int_0^\infty d\omega\,\omega^3\alpha(i\omega)
\int_1^\infty dp\,e^{-2\omega p|d|/c}\frac{\sqrt{p^2+\chi(i\omega)}-p}{\sqrt{p^2+\chi(i\omega)}+p},\label{eq:TE_CP_zero_temperature}
\end{align}
where now $\omega$ is a continuous variable.  

\subsection{Various Limiting Cases}

Let us now consider the various limits for the Casimir-Polder case.
  Since we are taking an ensemble average over Gaussian random walks,
 the loops will intersect all the surfaces when $T\sim d^2$, 
where $d$ is the distance from the source point $x_0$ to the farthest surface.
  Secondly, the frequency sum/integral is dominated by the exponential factors,
 which will contribute most when $T\omega_n^2/c^2\sim 1$.
  This suggests that frequencies with  $\omega_n^2< c^2/d^2$  will contribute most.   

\subsubsection{Zero temperature, far-field limit}

We start from Eq.~(\ref{eq:TE_CP_zero_temperature}).
  If we also take the limit where $d/c$ is large, only frequencies near $0$ will contribute,
 and we can approximate $\epsilon(i\omega) \approx \epsilon(0)$, $\alpha(i\omega)\approx\alpha_0$ everywhere.  
\comment{Does this also only work up to a certain distance, 
at which point the integral crosses over to High temperature,since there is always some thermal background.
  e.g. infra-red radiation. }
\begin{align}
E-E_0&=-\frac{\hbar}{8\pi^2\epsilon_0c^3}\int_0^\infty d\omega\,\omega^3\alpha(i\omega)
\int_1^\infty dp\,e^{-2\omega p|d|/c}\frac{\sqrt{p^2+\chi(0)}-p}{\sqrt{p^2+\chi(0)}+p}
\end{align}

 Now evaluate the $\omega$ integral, 
\begin{equation}
\int_{0}^\infty d\omega\,\omega^3e^{-2\omega r d/c} = \frac{3 c^4}{8 p^4 d^4}
\end{equation}
which leaves a by now familiar integral:
\begin{align}
E-E_0&= -\frac{3\hbar c\alpha_0}{64\pi^2 \epsilon_0 d^4}\int_1^\infty dp\,p^{-4}\frac{\sqrt{p^2+\chi}-p}{\sqrt{p^2+\chi}+p}
\end{align}

\subsubsection{Near field, low temperature}
Let's now work in the limit where $d<<c/\omega_{j0}$.
  In this case all frequencies contribute, but we can convert the sum into an integral.
  The difference here is that all of the frequency dependence of $\epsilon(\omega)$  will matter.  
\begin{equation}
E-E_0=-\frac{\hbar}{8\pi^2\epsilon_0c^3}\int_0^\infty d\omega\,\omega^3\alpha(i\omega)
\int_1^\infty dp\,e^{-2\omega p|d|/c}\frac{\sqrt{p^2+\chi(i\omega)}-p}{\sqrt{p^2+\chi(i\omega)}+p}
\end{equation}
The integral contributes most when the exponent is order unity.
  The presence of $\alpha$ means that frequencies around $\omega_{j0}$ will dominate the frequency integral.
  Then $p \sim  c/(d\omega_{j0})\gg 1$.
  Let's use that fact to approximate the reflection coefficient, 
and see if we can reproduce the known van der Waals result.  
\begin{align}
  \frac{\sqrt{p^2+\chi(i\omega)}-p}{\sqrt{p^2+\chi(i\omega)}+p}\approx 
& \frac{ p + \frac{\chi}{2p}-p}{2p+\frac{\chi(i\omega)}{2p}} \approx \frac{\chi}{4p^2} 
\end{align}
Plugging this in, we can then evaluate the $p$ integral
\begin{align}
E-E_0=&-\frac{\hbar}{8\pi^2\epsilon_0c^3}\int_0^\infty d\omega\,\omega^3\alpha(i\omega)\chi(i\omega)
\int_1^\infty dp\,\frac{1}{4p^2}e^{-2\omega p|d|/c}
\end{align}

Let's try to work on that integral a bit.  
\begin{align}
\int_1^\infty dp\,\frac{1}{4p^2}e^{-2\omega p|d|/c}%  =& -\frac{1}{4p}e^{-2\omega p d/c}\bigg|_{p=1}^{\infty}
 % + \int_1^\infty dp\, \frac{1}{4p}\times \frac{-2\omega d}{c}e^{-2\omega pd/c}\\
=& \frac{1}{4}e^{-2\omega d/c} - \frac{2\omega d}{c} \int_1^\infty dp\, \frac{e^{-2\omega pd/c}}{p}.
\end{align}

Recall, we are working in the so-called near-field limit where $d\omega_{j0}/c<<1.$  
I think we can get away with approximating this as just the exponential term.  

\begin{align}
E-E_0=&-\frac{\hbar}{32\pi^2\epsilon_0c^3}\int_0^\infty d\omega\,\omega^3\alpha(i\omega)\chi(i\omega)e^{-2\omega d/c}\\
=&-\frac{\hbar}{32\pi^2\epsilon_0 d^3}\int_0^\infty d\omega\,\frac{\omega^3d^3}{c^3}\alpha(i\omega)\chi(i\omega)e^{-2\omega d/c}\approx 0
\end{align}
Since I think we are working wiht $d\omega/c$ is very small, so this term is tiny.  

(From Dan's analysis in the notes, apparently we can drop this term, or rather it is negligible in comparison to the $TM$ contribution.)


\subsubsection{High Temperature, (far field ?)}  

As we noted earlier, the renormalized partition function vanishes for $\omega_0$.
The leading term is $\omega_1$, which will be exponentially suppressed relative to the TM contribution.  


% This is the general case.  

% The presence of $i\langle \partial_\omega\epsilon(i\omega_n)$ in our expression is acceptable,
% since $\epsilon(i\omega_n)$ is in itself a real function, so $k_BT\epsilon(i\omega_n)$ is also real.
%  As it stands, this factor of $i$ will then combine with further factors of $i$ from the form of the derivative.
%  For example, the response of a harmonic oscillator with frequency  $\omega_0$ is $\epsilon(\omega) = A/(\omega^2-\omega_0^2)$.
%  The derivative is $\partial_\omega\epsilon(\omega) = -2A\omega/(\omega^2-\omega_0^2)^2$.  
% Now if we consider imaginary frequencies then $\epsilon(i\omega_n) = -A/(\omega_n^2+\omega_0^2)$, 
%and $i\partial_\omega \epsilon(\omega)\big|_{\omega=i\omega_n} = -i (iA\omega_n)/(\omega_n^2+\omega_0^2)^2$, which is real.
%    We would get the same result in evaluating $k_BT\epsilon(i\omega_n)$ directly.  


% \subsection{TE Polarization: Casimir}

% Let's try to do this for the Casimir energy due to TE as well.  
% We will start from 
% \begin{equation}
% E-E_0=k_BT{\sum_n}'\int_0^\infty \frac{dT}{T}\int dx\,\frac{1}{(2\pi T)^{3/2}}\dlangle e^{-T\frac{\omega_n^2}{2c^2}}
%  -  e^{-\frac{ T \omega_n^2\langle\epsilon(i\omega_n)\rangle}{2c^2}}\drangle\label{eq:casimir_partition_function}
% \end{equation}
% We will need to also subtract off the renormalized one body energies.  
% As before, we need the Feynman-Kac formula,
% \begin{align}
% &\int dx\int_0^\infty dT \frac{e^{-\lambda T}}{\sqrt{2\pi T}}\left[e^{-T\langle\epsilon_{12}\rangle}+
%  e^{-T\langle\epsilon_{0}\rangle} - e^{-T\langle\epsilon_{1}\rangle}- e^{-T\langle\epsilon_{2}\rangle}\right]\nonumber\\ 
% =&  \frac{u_1u_2 e^{-2\sqrt{2\lambda}d}}{\sqrt{2\lambda}(1-u_1u_2 e^{-2\sqrt{2\lambda}d})}
% \left( 2d + \frac{\sqrt{2}}{\sqrt{(\lambda+\chi_1)}} + \frac{\sqrt{2}}{\sqrt{(\lambda+\chi_2)}} \right)
% \end{align}
% where $u_i = (\sqrt{\lambda}-\sqrt{\lambda+\chi})/(\sqrt{\lambda}+\sqrt{\lambda+\chi})$.
%   We will also need 
% \begin{align}
% \int_0^\infty \frac{dT}{T^{1+z}}\dlangle e^{-s[T+ \chi\int_0^Tdt\Theta(x-d)]}\drangle=& 
% \frac{1}{\Gamma[z+1/2]}\int_0^\infty d\lambda\, \lambda^{z-1/2}\int_0^\infty dT e^{-(\lambda+s)T}
% \dlangle \frac{e^{-s \chi \int_0^T dt \Theta(x-d)}}{\sqrt{T}}\drangle.
% \end{align}
% In this case $z=3/2$, and $s= \omega^2/(2c^2)$, so we need to take 
% $\lambda\rightarrow \lambda+ \omega_n^2/(2c^2)$, and $\chi\rightarrow s\omega_n^2/(2c^2)$.
%   Putting these identities together in Eq.~(\ref{eq:casimir_partition_function})yields 
% \begin{align}
% E-E_0=&k_BT{\sum_n}'\frac{1}{\Gamma[2]2\pi}\int_0^\infty d\lambda\, \lambda 
%  \frac{u_1u_2 e^{-2\sqrt{2\lambda+\omega_n^2/c^2}d}}{\sqrt{2\lambda+\omega_n^2/c^2}(1-u_1u_2 e^{-2\sqrt{2\lambda+\omega_n^2/c^2}d})}\nonumber\\
% &\times \left( 2d + \frac{\sqrt{2}}{\sqrt{\lambda+\omega_n^2/(2c^2)(1+\chi_1)}}
%  + \frac{\sqrt{2}}{\sqrt{\lambda+\omega_n^2/(2c^2)/(1+\chi_2)}} \right)
% \end{align}
% with 
% \begin{equation}
% u_i = \frac{\sqrt{\lambda+\omega_n^2/(2c^2)}-\sqrt{\lambda+\omega_n^2/(2c^2)(1+\chi_i)}}
% {\sqrt{\lambda+\omega_n^2/(2c^2)}+\sqrt{\lambda+\omega_n^2/(2c^2)(1+\chi_i)}}
% \end{equation}

% We'll now make some variable changes to put this in a more tractable form.
%   First up, let's define $\lambda = \kappa \omega_n^2/(2c^2)$.  
% \begin{align}
% E-E_0%=&k_BT{\sum_n}'\frac{1}{2\pi}\int_0^\infty d\kappa\,\frac{\omega_n^4}{4c^4} \kappa  \frac{cu_1u_2 e^{-2\omega_n\sqrt{\kappa+1}d/c}}{\omega_n\sqrt{\kappa+1}(1-u_1u_2 e^{-2\omega_n\sqrt{\kappa+1}d/c})}\nonumber\\
% %&\times \left( 2d + \frac{2c}{\omega_n\sqrt{\kappa+1+\chi_1}} + \frac{2 c}{\omega_n\sqrt{\kappa+1+\chi_2}} \right)\\
% =&k_BT{\sum_n}'\frac{1}{2\pi}\int_0^\infty d\kappa\,\frac{\omega_n^2}{2c^2} \kappa 
%  \frac{u_1u_2 e^{-2\omega_n\sqrt{\kappa+1}d/c}}{\sqrt{\kappa+1}(1-u_1u_2 e^{-2\omega_n\sqrt{\kappa+1}d/c})}
% \left( \frac{\omega_nd}{c} + \frac{1}{\sqrt{\kappa+1+\chi_1}} + \frac{1}{\sqrt{\kappa+1+\chi_2}} \right)
% \end{align}
% with
% \begin{equation}
% u_i = \frac{\sqrt{\kappa+1}-\sqrt{\kappa+1+\chi_i}}{\sqrt{\kappa +1}+\sqrt{\kappa+1+\chi_i}}.
% \end{equation}
% Next up define $\kappa+1 = p^2$.  
% \begin{align}
% E-E_0%=&k_BT{\sum_n}'\frac{\omega_n^2}{4\pi c^2}\int_1^\infty dp\,2p (p^2-1)  \frac{u_1u_2 e^{-2\omega_n pd/c}}{p(1-u_1u_2 e^{-2\omega_npd/c})}\left( \frac{\omega_nd}{c} + \frac{1}{\sqrt{p^2+\chi_1}} + \frac{1}{\sqrt{p^2+\chi_2}} \right)\\
% =&k_BT{\sum_n}'\frac{\omega_n^2}{2\pi c^2}\int_1^\infty dp\,(p^2-1)  
% \frac{u_1u_2 e^{-2\omega_n pd/c}}{(1-u_1u_2 e^{-2\omega_npd/c})}
% \left( \frac{\omega_nd}{c} + \frac{1}{\sqrt{p^2+\chi_1}} + \frac{1}{\sqrt{p^2+\chi_2}} \right)
% \end{align}
% with
% \begin{equation}
% u_i = \frac{p-\sqrt{p^2+\chi_i}}{p+\sqrt{p^2+\chi_i}}.
% \end{equation}
% Finally, let's integrate by parts with respect to $p$.  
% \begin{shaded}
%  First up take the derivative of the reflection coefficient, 
% \begin{align}
% \frac{dr}{dp} =& \frac{d}{dp} \frac{p-\sqrt{p^2+\chi}}{p+\sqrt{p^2+\chi}}
% = \frac{1-\frac{2p}{2\sqrt{p^2+\chi}}}{p+\sqrt{p^2+\chi}} - (p-\sqrt{p^2+\chi})
% \frac{(1+\frac{2p}{2\sqrt{p^2+\chi}})}{(p+\sqrt{p^2+\chi})^2} 
% %=& \frac{1}{\sqrt{p^2+\chi}}\frac{\sqrt{p^2+\chi}-p}{p+\sqrt{p^2+\chi}} - (p-\sqrt{p^2+\chi})\frac{p+\sqrt{p^2+\chi}}{\sqrt{p^2+\chi}(p+\sqrt{p^2+\chi})^2} \\
% = -\frac{2r}{\sqrt{p^2+\chi}}
% \end{align}
% We can also write:
% \begin{align}
% \frac{d}{dp}[1-r_1r_2 e^{-2p\omega_n d/c }]% &= -\left( r_1\frac{dr_2}{dp} e^{-2p\omega_n d/c} + r_2\frac{dr_1}{dp} e^{-2p\omega_n d/c} - 2\xi r_1r_2d e^{-2p\omega_n d/c}\right)\\
% % &= -\left( -2 \frac{r_1r_2}{\sqrt{p^2+\chi_2}} -2 \frac{r_1r_2}{\sqrt{p^2+\chi_1}}- 2r_1r_2\frac{\omega_nd}{c} \right)e^{-2p\omega_n d/c}\\
% &= 2\left(\frac{\omega_nd}{c} +\frac{1}{\sqrt{p^2+\chi_1}} +\frac{1}{\sqrt{p^2+\chi_2}}\right) r_1r_2e^{-2p\omega_n d/c},
% \end{align}
% which suggests 
% \begin{align}
% \Aboxed{\frac{d}{dp}\ln[1-r_1r_2 e^{-2p\omega_n d/c}]
% &= \frac{2r_1r_2 e^{-2p\omega_n d/c}}{1 - r_1r_2  e^{-2p\omega_n d/c}}
% \left(\frac{\omega_n d}{c}+\frac{1}{\sqrt{p^2+\chi_1}}+\frac{1}{\sqrt{p^2+\chi_2}}\right)}
% \end{align}
% \end{shaded}

% So after integration by parts our the renormalized Casimir energy becomes
% \begin{align}
% E-E_0 & % = k_BT{\sum_n}'\frac{\omega_n^2}{2\pi c^2}\int_1^\infty dp\,(p^2-1)\frac{ r_1r_2e^{-2\omega_n p d/c}}{(1 -r_1r_2 e^{-2\omega_n pd/c})}\left[ \frac{\omega_n d}{c} +\frac{1}{\sqrt{p^2+\chi_1}}+\frac{1}{\sqrt{p^2+\chi_2}}\right]\\
% % & = k_BT{\sum_n}'\frac{\omega_n^2}{2\pi c^2}\left\{\frac{1}{2}\log\left[1-r_1r_2 e^{-2\omega_n p d/c}\right](p^2-1)\bigg|_{p=1}^\infty - \int_1^\infty dp \,p\log\left[1-r_1r_2 e^{-2\omega_n p d/c}\right]\right\}\\
% & = -k_BT{\sum_n}'\frac{\omega_n^2}{2\pi c^2}\int_1^\infty dp \,p
% \log\left[1-r_1r_2 e^{-2\omega_n p d/c}\right]\label{eq:Casimir_energy_finite_temperature}
% \end{align}
% We can take the zero temperature limit here as well:
% \begin{align}
% E-E_0& = -\frac{\hbar c}{4\pi^2}\int_0^\infty dk\,k^2\int_1^\infty dp \,p
% \log\left[1-r_1r_2 e^{-2k p d}\right]\label{eq:Casimir_energy_zero_temperature}
% \end{align}

\section{TM Polarization:Partition Function}

In this case we are starting from the TM polarization
% \begin{equation}
% Z_{TM} = \int D\psi \exp\left[ -\frac{\epsilon_0}{2}\int_0^\beta d\beta'\int d^3x\,
%  \left( \frac{\epsilon(x)}{\hbar^2}(\partial_{\beta'}\psi)^2 + c^2\frac{1}{\epsilon}|\nabla\sqrt{\epsilon}\psi|^2\right)\right] .
% \end{equation}

Let us change to using $\tau = \beta \hbar c$ as our temperature coordinate.  Then   
\begin{equation}
Z_{TM} = \int D\psi \exp\left[ -\frac{\epsilon_0 c^2 }{2 \hbar c}\int_0^{\hbar\beta c} 
d\tau'\int d^3x\, \psi\left( \epsilon(x)(\partial_{\tau}
  -\sqrt{\epsilon}\nabla \epsilon^{-1}\nabla\sqrt{\epsilon} -\nabla^2\right)\psi\right].
\end{equation}

As before, we introduce the Matsubara frequencies $\omega_n$
% We will now introduce the Matsubara frequencies $\omega_n = (2\pi n)/(\beta \hbar)$, with 
% \begin{equation}
% \psi(\beta,x) = \sum_{n=-\infty}^{\infty}e^{i\frac{\omega_n}{c}\tau} \psi_n(x),
% \end{equation}
% where $\tau = \beta\hbar c$, and the $\psi_n$ are complex variables.  We will also need to use  
% \begin{equation}
% \int_0^{\beta \hbar c}d\tau e^{i\frac{(\omega_n+\omega_m)}{c}\tau} = \beta\hbar c \delta_{n,-m},
% \end{equation}
% and $\psi_n^* = \psi_{-n}$ since $\psi^*(\beta, x) = \psi(\beta, x)$.  
Then we have 
\begin{equation}
Z_{TM} = \prod_{n=-\infty}^{\infty} \int D\psi_n\exp\left[ -\frac{\beta \epsilon_0 c^2 }{2}\int d^3x\, 
\psi_n^*\left(\epsilon(i\omega_n,x)\frac{\omega_n^2}{c^2}+   V_{TM} -\nabla^2\right)\psi_n\right], 
\end{equation}
where $V_{TM} = (\nabla\ln\sqrt{\epsilon})^2-\nabla^2\log\sqrt{\epsilon}$.
  Note that the presence of $V_{TM}$ implies there will be a contribution to the $n=0$ Matsubara term,
 which is good, since we know that the TM energy is the dominant contribution in that case.
This is what should give us a the dominant near-field, and high temperature results.  
   We will renormalize this against vacuum, 
% \begin{equation}
% \log Z_{TE} -\log Z_0= -{\sum_{n=0}^\infty}'\left\{\log\det\left[ \frac{1}{2}
% \left(\epsilon(i\omega_n,\vect{x})\frac{\omega_n^2}{c^2} +V_{TM}-\nabla^2\right)\right]
%  - \log\det\left[ \frac{1}{2}\left(\frac{\omega_n^2}{c^2} -\nabla^2\right)\right]\right\}
% \end{equation}
% Note that for $n=0$ the Matsubara frequency $\omega_n=0$, so we have 
% \begin{align}
% &\log\det\left[ \frac{1}{2}\left(\epsilon(0,\vect{x})\frac{\omega_0^2}{c^2}+V_{TM} -\nabla^2\right)\right]
%  - \log\det\left[ \frac{1}{2}\left(\frac{\omega_0^2}{c^2} -\nabla^2\right)\right] \nonumber\\
% &= \log\det\left[\frac{1}{2}\left(V_{TM} -\nabla^2\right)\right]
%  - \log\det\left[ \frac{1}{2}\left( -\nabla^2\right)\right]\ne  0
% \end{align}
% This assumes that $\epsilon(\omega)$ has at most a simple pole at zero frequency,
%  such that $\lim_{\omega\rightarrow 0}\omega^2\epsilon(\omega)=0.$    
Then renormalized Casimir energy is 
\begin{align}
E-E_0 & = -k_BT \log \frac{Z_{TM}}{Z_0} \\
&=k_BT{\sum_n}'\int_0^\infty \frac{dT}{T}\int d^3x\,\frac{1}{(2\pi T)^{3/2}}\dlangle e^{-T\frac{\omega_n^2}{2c^2}}
 -  e^{-\frac{ T \omega_n^2\langle\epsilon(i\omega_n)\rangle}{2c^2} - \frac{T}{2}\langle V_{TM}\rangle}\drangle\label{eq:TMworldline_partition_function}
\end{align}

\subsection{Casimir-Polder energy}

The Casimir-Polder energy can be recovered by expanding $\epsilon$ to linear order in $\alpha(i\omega)/\epsilon_0\delta(x-x').$
  We will also have to expand $V_{TM}$,
\begin{align}
T\langle V_{TM}\rangle =& \int_0^Tdt\, (\partial_x\log\sqrt{\epsilon})^2 - \partial_x^2\log\sqrt{\epsilon}\\
%=& \int_0^Tdt\, \frac{1}{4}(\partial_x\log\epsilon)^2 - \frac{1}{2}\partial_x^2\log\epsilon\\
=& \int_0^Tdt\, \frac{1}{4}[\partial_x\log(\epsilon + \alpha\delta(x-x_0)/\epsilon_0)]^2 
- \frac{1}{2}\partial_x^2\log(\epsilon + \alpha\delta(x-x_0)/\epsilon_0)\\
%=& \int_0^Tdt\, \frac{1}{4}\{\partial_x[\log(\epsilon) + \alpha\delta(x-x_0)/(\epsilon(x_0)\epsilon_0)]\}^2 - \frac{1}{2}\partial_x^2\{\log(\epsilon) + \alpha\delta(x-x_0)/(\epsilon_0\epsilon(x_0))\}\\
=& \int_0^Tdt\, V_{TM} +\frac{\alpha}{2}\partial_x\log\epsilon\partial_x[\delta(x-x_0)/(\epsilon(x)\epsilon_0)]
 - \frac{\alpha}{2}\partial_x^2[\delta(x-x_0)/(\epsilon_0\epsilon(x))]
\end{align}
We will simplify this a bit by assuming we are only considering the polarizable particle in regions 
where $\epsilon(x)$ is constant in the vicinity of $x_0$.
  Then we can drop any terms in $\partial_x\epsilon|_{x=x_0}$.
  This lets us drop the second term, and any derivatives from expanding out the derivative.
  I think any extra terms from expanding out these derivatives would ultimately get eaten when
 considering the effect of $\delta'$.
  It will be operationally cleaner to just leave the derivatives acting on the products, 
and integrate by parts at the end. 

Let us suppress the renormalization terms for the mean time.  
\begin{align}
E&=-k_BT{\sum_n}'\int_0^\infty \frac{dT}{T}\int d^3x\,\frac{1}{(2\pi T)^{3/2}}
\dlangle e^{-\frac{ T \omega_n^2\langle\epsilon(i\omega_n)\rangle}{2c^2} - T\langle V_{TM}\rangle} \right.\right. \nonumber \\
& \hspace{3cm}\times\left.\left.\left(-T\frac{\omega_n^2}{2c^2}\frac{\alpha(i\omega_n)}{\epsilon_0}
\langle\delta(x-x_0)\rangle  - \frac{\alpha T}{4}\langle\partial_x^2[\delta(x-x_0)/(\epsilon_0\epsilon(x))]\rangle\right)\drangle\\
&=k_BT{\sum_n}'\frac{\alpha(i\omega_n)}{2\epsilon_0}\int_0^\infty dT\,\frac{1}{(2\pi T)^{3/2}}
\dlangle \left(\frac{\omega_n^2}{c^2}  - \frac{1}{2}\partial_x^2\right)
e^{-\frac{ T \omega_n^2\langle\epsilon(i\omega_n)\rangle}{2c^2} - T\langle V_{TM}\rangle}\drangle
\end{align}
Now doing the subtraction of the same energy with $\epsilon=1$ gives: 
\begin{equation}
E-E_0=-k_BT{\sum_n}'\frac{\alpha(i\omega_n)}{2\epsilon_0}\int_0^\infty dT\,
\frac{1}{(2\pi T)^{3/2}}\dlangle \frac{\omega_n^2}{c^2}e^{-T\frac{\omega_n^2}{2c^2}}
-\left(\frac{\omega_n^2}{c^2}  - \frac{1}{2}\partial_x^2\right)e
^{-\frac{ T \omega_n^2\langle\epsilon(i\omega_n)\rangle}{2c^2} - T\langle V_{TM}\rangle}\drangle\label{eq:TM_CP_finite_temperature},
\end{equation}
which is our initial result for the finite-temperature Casimir-Polder energy.
  As before, we can straightforwardly take the zero temperature limit: 
\begin{equation}
E-E_0=-\frac{\hbar}{2\pi}\int_0^\infty d\omega\frac{\alpha(i\omega)}{2\epsilon_0}
\int_0^\infty dT\,\frac{1}{(2\pi T)^{3/2}}\dlangle \frac{\omega^2}{c^2}e^{-T\frac{\omega^2}{2c^2}}-\left(\frac{\omega^2}{c^2}  - \frac{1}{2}\partial_x^2\right)e^{-\frac{ T \omega^2\langle\epsilon(i\omega)\rangle}{2c^2} - T\langle V_{TM}(i\omega)\rangle}\drangle\label{eq:TM_CP_zero_temperature},
\end{equation}

\subsubsection{Laplace-Mellin and Feynman-Kac Formulae}
We will again need to use the Laplace-Mellin transforms, and Feynman-Kac Formulae.
  We quote the results:
The Laplace-Mellin transform is
\begin{align}
\int_0^\infty \frac{dT}{T^{1+z}}\dlangle e^{-sT\langle\epsilon\rangle - T\langle V_{TM}\rangle}\drangle =&
 \frac{1}{\Gamma[z+1/2]}\int_0^\infty d\lambda\, \lambda^{z-1/2}\int_0^\infty dT e^{-(\lambda+s)T}
\dlangle \frac{e^{-\int_0^T dt\,(s\chi+ V_{TM})}}{\sqrt{T}}\drangle.
\end{align}
For Casimir-Polder we need$z=1/2$, and for Casimir we need $z=3/2$.
  In both cases we need $s= \omega^2/(2c^2)$.
  We also need the actual analytical expression for that path integral.

For one body we need:
\begin{equation}
\int_0^\infty dT e^{-(\lambda+s) T} \dlangle \frac{e^{-s\chi\int_0^T dt \Theta(x-d)}}{\sqrt{2\pi T}}\drangle  
=\frac{1}{\sqrt{2(\lambda+s)}}\left[1 - e^{-2\sqrt{2(\lambda+s)}|d|}\frac{\sqrt{\lambda+s(1+\chi)}
-\sqrt{\lambda+s}e^{2\Xi}}{\sqrt{\lambda+s(1+\chi)}+\sqrt{\lambda+s}e^{2\Xi}}\right],
\end{equation}
where $e^{2\Xi} = (1+\chi)$ comes from the contribution of $e^{-V_{TM}}$.
  \comment{Correct signs?} For two macroscopic bodies we will need:
\begin{align}
&\int dx\int_0^\infty dT \frac{e^{-(\lambda +s)T}}{\sqrt{2\pi T}}\left[e^{-s\int_0^T dt\,(\chi_{12} + V_{12,TM})}
 +1 -e^{-s\int_0^T dt\,(\chi_{1} + V_{1,TM})}-e^{-s\int_0^T dt\,(\chi_{2} + V_{2,TM})}\right]\nonumber\\ 
=&  \dfrac{u_1'u'_2e^{-2\sqrt{2\lambda}d}}{1 - u'_1u'_2 e^{-2\sqrt{2\lambda}d}}\left[ \frac{2 d}{\sqrt{2\lambda}}
-\frac{ e^{2\Xi_1}}{\sqrt{\lambda+s}\sqrt{\lambda+s(1+\chi_1)}}
\frac{s\chi_1}{e^{4\Xi_1}(\lambda+s)-[\lambda+s(1+\chi_1)]}  + \{1 \leftrightarrow 2\}  \right].
\end{align}
where 
\begin{equation}
u'_i = \frac{\sqrt{\lambda+s}(1+\chi)-\sqrt{\lambda+s(1+\chi)}}{\sqrt{\lambda+s}(1+\chi)+\sqrt{\lambda+s(1+\chi)}}
\end{equation}
As nasty as that two-body expression may be, exactly the same tricks will work on it, 
and it will simplify down to exactly the same form as the other polarization.  

\subsection{TM Casimir-Polder: Limiting Cases}

We will work with the case of an atom in front of a dielectric surface.
  Let's first plug in the relevant one-body Feynman-Kac formula into the partition function.
 
\begin{align}
E-E_0=& -k_BT{\sum_n}'\frac{\alpha(i\omega_n)}{2\epsilon_0}\int_0^\infty dT\,
\frac{1}{(2\pi T)^{3/2}}\left(\frac{\omega_n^2}{c^2}  - \frac{1}{2}\partial_x^2\right)
\dlangle e^{-T\frac{\omega_n^2}{2c^2}}-e^{-\frac{ T \omega_n^2\langle\epsilon(i\omega_n)\rangle}{2c^2} - T\langle V_{TM}\rangle}\drangle \\
=& -k_BT{\sum_n}'\frac{\alpha(i\omega_n)}{4\pi\epsilon_0}\left(\frac{\omega_n^2}{c^2}  
- \frac{1}{2}\partial_d^2\right)\int_0^\infty d\lambda\, 
\frac{e^{-2\sqrt{2(\lambda+\frac{\omega_n^2}{2c^2})}d}}{\sqrt{2\lambda+\omega_n^2/c^2}}
\frac{\sqrt{\lambda+\frac{\omega_n^2}{2c^2}(1+\chi)}-\sqrt{\lambda+\frac{\omega_n^2}{2c^2}}e^{2\Xi}}
{\sqrt{\lambda+\frac{\omega_n^2}{2c^2}(1+\chi)}+\sqrt{\lambda+\frac{\omega_n^2}{2c^2}}e^{2\Xi}} 
\end{align}
Now make our usual substitutions: $\lambda = \kappa\omega_n^2/(2c^2)$, and then $p^2 = \kappa+1$.  
\begin{align}
E-E_0%=& -k_BT{\sum_n}'\frac{\alpha(i\omega_n)}{4\pi\epsilon_0}\left(\frac{\omega_n^2}{c^2}  - \frac{1}{2}\partial_d^2\right)\int_0^\infty d\kappa\, \frac{\omega_n^2}{2c^2}\frac{e^{-2\sqrt{\kappa+1}\omega_n d/c}c}{\omega_n\sqrt{\kappa+1}}\frac{\sqrt{\kappa+1+\chi}-\sqrt{\kappa+1}e^{2\Xi}}{\sqrt{\kappa+1+\chi}+\sqrt{\kappa+1}e^{2\Xi}} \\
=& -k_BT{\sum_n}'\frac{\omega_n\alpha(i\omega_n)}{4\pi\epsilon_0c}
\left(\frac{\omega_n^2}{c^2}  - \frac{1}{2}\partial_d^2\right)
\int_1^\infty dp\,e^{-2p\omega_n d/c}\frac{\sqrt{p^2+\chi}-pe^{2\Xi}}{\sqrt{p^2+\chi}+p e^{2\Xi}} 
\end{align}

Take the $\partial_d$ derivatives, get 
\begin{align}
E-E_0%=& -k_BT{\sum_n}'\frac{\omega_n\alpha(i\omega_n)}{4\pi\epsilon_0c}\int_1^\infty dp\,\left(\frac{\omega_n^2}{c^2}  - \frac{2\omega_n^2p^2}{c^2}\right)e^{-2p\omega_n d/c}\frac{\sqrt{p^2+\chi}-pe^{2\Xi}}{\sqrt{p^2+\chi}+p e^{2\Xi}} \\
=& -k_BT{\sum_n}'\frac{\omega^3_n\alpha(i\omega_n)}{4\pi\epsilon_0c^3}\int_1^\infty dp\,
\left(1-2p^2\right)e^{-2p\omega_n d/c}\frac{\sqrt{p^2+\chi}-pe^{2\Xi}}{\sqrt{p^2+\chi}+p e^{2\Xi}} 
\end{align}

\subsubsection{Zero temperature, far-field}
In the far-field of the atom, we have $d\omega_{j0}/c>>1$, so replace $\epsilon,\alpha$ by their d.c. values.  
\begin{align}
E-E_0=& -\frac{\hbar}{2\pi}\int_0^\infty d\omega \frac{\omega^3\alpha_0}{4\pi\epsilon_0c^3}
\int_1^\infty dp\,\left(1-2p^2\right)e^{-2p\omega d/c}\frac{\sqrt{p^2+\chi}-p(1+\chi)}{\sqrt{p^2+\chi}+p(1+\chi)}\\
=& -\frac{3\hbar c\alpha_0}{64\pi^2\epsilon_0d^4}\int_1^\infty dp\,p^{-4}
\left(1-2p^2\right)\frac{\sqrt{p^2+\chi}-p(1+\chi)}{\sqrt{p^2+\chi}+p(1+\chi)},
\end{align}
yet another familiar integral (up to a lingering sign on a reflection coefficient?
 I seem to have currently stumbled onto the correct choice. )  

\subsubsection{Zero temperature, near-field}

In this limit you take $d\rightarrow 0$, but all frequencies contribute, as governed by $\alpha(i\omega)$.
    We have frequency integral determined by $\alpha$ so frequencies $w < w_{j0}$ will dominate.
  Alternatively, just take $p\sim c/(d\omega)$.
  Since $d$ is small, then important $p$ are very large?
  I think this implicitly takes $\omega d/c<<1$?
\begin{align}
E-E_0=& -\frac{\hbar}{8\pi^2\epsilon_0c^3}\int_0^\infty d\omega \omega^3\alpha(i\omega)\int_1^\infty dp\,
\left(1-2p^2\right)e^{-2p\omega d/c}\frac{\sqrt{p^2+\chi}-pe^{2\Xi}}{\sqrt{p^2+\chi}+p e^{2\Xi}} 
\end{align}
The reflection coefficient becomes
\begin{equation}
\frac{\sqrt{p^2+\chi}-p(1+\chi)}{\sqrt{p^2+\chi}+p(1+\chi)} \approx \frac{ p-p(1+\chi)}{p+p(1+\chi)} =
 -\frac{\epsilon(i\omega)-1}{\epsilon(i\omega)+1}.
\end{equation}
Plug this in, and evaluate $p$ integral
\begin{align}
E-E_0\approx& \frac{\hbar}{8\pi^2\epsilon_0c^3}\int_0^\infty d\omega \omega^3
\alpha(i\omega)\frac{\epsilon(i\omega)-1}{\epsilon(i\omega)+1}\int_1^\infty dp\,2p^2e^{-2p\omega d/c}\\
=& \frac{\hbar}{8\pi^2\epsilon_0c^3}\int_0^\infty d\omega \omega^3
\alpha(i\omega)\frac{\epsilon(i\omega)-1}{\epsilon(i\omega)+1}\left(-\frac{c^3e^{-2\omega d/c}(1+\omega d/c)^2}{2 d^3\omega^3}\right)\\
&\approx -\frac{\hbar }{16\pi^2\epsilon_0 d^3}\int_0^\infty d\omega 
\alpha(i\omega)\frac{\epsilon(i\omega)-1}{\epsilon(i\omega)+1}.
\end{align}
which is the correct answer ( an expected result since we were angling for this result by making these limits.
  BUt reassuring to see it emerge nonetheless).  

\subsection{High temperature, far-field(?)}

% We are again working in a far-field limit.
%   At high temperature $\beta\rightarrow 0$, so $\omega_i\rightarrow \infty$.
%   $\omega_n = 2\pi/(\beta\hbar)$.  
% The renormalized energy is 
% \begin{align}
% E-E_0=& -\partial_\beta{\sum_n}'\frac{\omega^3_n\alpha(i\omega_n)}{4\pi\epsilon_0c^3}\int_1^\infty dp\,
% \left(1-2p^2\right)e^{-2p\omega_n d/c}\frac{\sqrt{p^2+\chi}-p(1+\chi)}{\sqrt{p^2+\chi}+p (1+\chi)} 
% \end{align}
% Other work tells us that only $\omega_0$ contributes here.
%   Naively taking $\omega_0=0$, we have \emph{no} $\beta$ dependence anywhere.
%   So that just gives us zero?  Perhaps we have to be careful with the order of operations here 
% - or it is ok to do the $\beta$ derivative right now?
%   We will use $\partial_\beta\omega_n = \partial_\beta[2\pi/(\beta\hbar)] = -2\pi/(\beta^2\hbar) = -k_BT \omega_n$.

% \begin{align}
% E-E_0=& -\partial_\beta{\sum_n}'\frac{\omega^3_n\alpha(i\omega_n)}{4\pi\epsilon_0c^3}\int_1^\infty dp\,\left(1-2p^2\right)e^{-2p\omega_n d/c}\frac{\sqrt{p^2+\chi}-p(1+\chi)}{\sqrt{p^2+\chi}+p (1+\chi)} 
% \end{align}
% \subsubsection{Keeping $\omega_1$}
% So if $\omega_0$ does not contribute, let's try the next term.  Since $\omega_1d/c>>1$, we have only small $p$ contributing.  Since our integral's lower bound is $p=1$, only $p$ close to 1 will contribute.  Let's use $p = 1+s$.    
% \begin{align}
% E-E_0=& -\partial_\beta{\sum_n}'\frac{\omega^3_1\alpha(i\omega_1)}{4\pi\epsilon_0c^3}e^{-2\omega_1 d/c}\int_0^\infty ds\,\left(1-2(1-s)^2\right)e^{-2s\omega_1 d/c}\frac{\sqrt{(1+s)^2+\chi}-(1+s)(1+\chi)}{\sqrt{(1+s)p^2+\chi}+(1+s) (1+\chi)}
% \end{align}
% If we approximate the reflection coefficient at $s=0$ weget
% \begin{align}
% E-E_0=& -\partial_\beta\frac{\omega^3_1\alpha(i\omega_1)}{4\pi\epsilon_0c^3}e^{-2\omega_1 d/c}\int_0^\infty ds\,\left(1-2\right)e^{-2s\omega_1 d/c}\frac{\sqrt{1+\chi}-(1+\chi)}{\sqrt{1+\chi}+ (1+\chi)}\\
% =& -\partial_\beta\frac{\omega^3_1\alpha(i\omega_1)}{4\pi\epsilon_0c^3}\frac{\sqrt{1+\chi}-1 }{\sqrt{1+\chi}+1}e^{-2\omega_1 d/c}\frac{c}{2\omega_1 d}.  
% \end{align}
% Which just decays exponentially with distance (rapidly no less).  Hmm.

% \subsubsection{Retrying with free energy}

Let's try this using the free energy, $F = -\beta^{-1}\log Z$.
  The Casimir energy is starts from 
\begin{equation}
F-F_0=-\beta^{-1}{\sum_n}'\frac{\alpha(i\omega_n)}{2\epsilon_0}\int_0^\infty dT\,
\frac{1}{(2\pi T)^{3/2}}\dlangle \frac{\omega_n^2}{c^2}e^{-T\frac{\omega_n^2}{2c^2}}-\left(\frac{\omega_n^2}{c^2}  
- \frac{1}{2}\partial_x^2\right)e^{-\frac{ T \omega_n^2\langle\epsilon(i\omega_n)\rangle}{2c^2} - T\langle V_{TM}\rangle}\drangle
\end{equation}
If we only keep the term with $\omega_0=0$, we have 
\begin{equation}
F-F_0=-\frac{1}{2}\beta^{-1}\frac{\alpha(0)}{2\epsilon_0}\int_0^\infty dT\,\frac{1}{(2\pi T)^{3/2}}
\dlangle -\left(-\frac{1}{2}\partial_x^2\right)e^{ - T\langle V_{TM}\rangle}\drangle
\end{equation}
Previously, we've done the calculations for the Feynman-Kac formula for just the $TM$ potential.
  Since the Laplace-tranform is trivial, we can do it immediately.  We get 
\begin{align}
\dlangle e^{-\int_0^T dt V_{TM}}\drangle &= 1 + \frac{\sinh(\Xi/2)}{\cosh\Xi}[e^{\Xi/2} + e^{-\Xi/2}]e^{-2 d^2/T}\\
%&= 1 + \frac{(e^{\Xi/2} - e^{-\Xi/2})}{(e^{\Xi/2} + e^{-\Xi/2})}{e^\Xi + e^{-\Xi}}e^{-2 d^2/T}\\
%&= 1 + \frac{e^{\Xi} - e^{-\Xi}}{e^\Xi + e^{-\Xi}}e^{-2 d^2/T}\\
&= 1 + \frac{e^{2\Xi} - 1}{e^{2\Xi} + 1} e^{-2 d^2/T}\\
&= 1 + \frac{\epsilon(0) - 1}{\epsilon(0)+1}e^{-2 d^2/T}.
\end{align}
Plugging this in, and taking the derivative 
\begin{align}
F-F_0=&-\frac{k_BT\alpha_0}{16\pi\epsilon_0}\frac{\epsilon(0)-1}{\epsilon(0)+1} 
\int_0^\infty dT\,\partial_d^2\frac{1}{\sqrt{2\pi }T^{3/2}} e^{-2 d^2/T}\\
=&-\frac{k_BT\alpha_0}{16\pi\epsilon_0d^3}\frac{\epsilon(0)-1}{\epsilon(0)+1}.
\end{align}




\section{Gradients for Casimir Energies}
\subsection{Surface Pinned Paths}
\label{sec:path-pinning}

\subsubsection{Force}
The force on a body follows from the gradient of the Casimir energy,
where the derivatives are taken with respect to the body's 
position.
For example, the components of the force on body $2$, expressed
in the basis $\hat{r}_i$, are
given by directional derivatives of the path-integral in Eqs.~(\ref{eq:spatial_path_integral}) and 
(\ref{eq:spatial_path_integralE}) with respect to
the components of the body position $\mathbf{R}_2$:
\begin{align}
  &F_{2,i}:=-\frac{\hbar c}{2(2\pi)^{D/2}}\int_0^\infty \!\!\frac{d\cT}{\cT^{1+D/2}}\change{\hat{r}_i\cdot\nablaR2} W
  \nonumber\\
   &\hspace{0.cm}=
   -\frac{\alpha\chi_2\hbar c}{2(2\pi)^{D/2}}
   \hat{r}_i\cdot\!\!
   \int_0^\infty \!\!\!\frac{d\cT}{\cT^{1+D/2}}   \!\int \!d\vect{x}_0\, 
  \Biggdlangle \frac{
  \big\langle 
  \delta(\sigma_2)\,\nabla\sigma_2\big\rangle}
  {\langle\epsr\rangle^{\alpha+1}}\Biggdrangle_{\!\vect{x}(t)}\!\!\!\!,
  \label{eq:forcepathint}
\end{align}
where $\sigma_2 = \sigma_2[\vect{x}(t)-\vect{R}_2]$ in this expression
\change{, and $\nablaR{i}$ denotes the gradient with
respect $\vect{R}_i$.}
The path integration can be simplified by viewing the $\delta$-function,
whose argument involves $\vect{x}(t)$, as a constraint on the
source point $\vect{x}_0$ of the paths.
Writing out the relevant part of the path-integral (\ref{eq:forcepathint}),
the $\delta$-function reduces the $D$-dimensional integration
over $\vect{x}_0$ to a $(D-1)$-dimensional integration over
the surface of body 2:
\comment{futzed with this - tried to restore}
\begin{align}
  &\int d\vect{x}_0  \Biggdlangle \frac{
  \big\langle 
  \delta\big(\sigma_2[\vect{x}(t)-\vect{R}_2]\big)\,\nabla\sigma_2[\vect{x}(t)-\vect{R}_2]\big\rangle}
  {\langle\epsr\rangle^{\alpha+1}}\Biggdrangle_{\vect{x}(t)} \nonumber\\
  &\hspace{0.5cm}= 
  \int d\vect{x}_0  \Biggdlangle \frac{
  \delta\big(\sigma_2[\vect{x}_0-\vect{R}_2]\big)\,\nabla\sigma_2[\x0-\vect{R}_2]}
  {\langle\epsr\rangle^{\alpha+1}}\Biggdrangle_{\vect{x}(t)} \nonumber\\
  &\hspace{0.5cm}= 
  \oint_{\sigma_2(\vect{x}_0-\vect{R}_2)=0}^{}
   \hspace{-8ex}dS\hspace{5ex}\biggdlangle 
  \frac{\nabla\sigma_2(\x0-\vect{R}_2)}
  {\langle\epsr\rangle^{\alpha+1}|\nabla\sigma_2(\x0-\vect{R}_2)|}\biggdrangle_{\vect{x}(t)},
  \label{eq:delta-normal}
\end{align}
where the final integral (over path source points $\vect{x}_0$)
is a surface integral over the 
surface of body 2.  The first equality here can be understood
in terms of a discrete path (i.e., in a ``time-slicing'' regularization
of the path-integral), where the path average in the numerator
amounts to a sum of terms, each involving a $\delta$-function
involving a path coordinate $\vect{x}_j$.  In the average over all
paths and sum over all source points $\vect{x}_0$,
the $\delta$-function is equivalently a function of the
source point, since $\vect{x}_j$ is the source point
of another equivalent path.
The second equality of Eqs~(\ref{eq:delta-normal})
follows from an application of the H\"ormander formula
(see Eq.~(22) in Ref.~\cite{Mackrory2016}).
The renormalized force vector can be found by summing over all force components and subtracting 
the corresponding single-body energy,
\change{used ``limit'' on oint to get limits underneath integral}
\begin{align}
  \vect{F}_{2}&=
  -\frac{\alpha\chi_2\hbar c}{2(2\pi)^{D/2}}
\int\limits_0^\infty \!\frac{d\cT}{\cT^{1+D/2}}    
\hspace{-3ex}
 \oint\limits_{\sigma_2(\vect{x}_0-\vect{R}_2)=0}^{}
  \hspace{-4ex} dS\hspace{1ex} 
  \hat{n}_2(\vect{x}_0) \nonumber\\
  &\hspace{0.5cm}\times 
  \Biggdlangle\frac{1}{\langle\epsilon_{\mathrm{r},12}\rangle^{\alpha+1}}-\frac{1}{\langle\epsilon_{\mathrm{r},2}\rangle^{\alpha+1}}
  \Biggdrangle_{\vect{x}(t)},
  \label{eq:pinning_force}
\end{align}
where the unit-normal vector for the surface of body 2 is defined by
\begin{equation}
  \hat{n}_2(\x0) := -\frac{\nabla \sigma_2(\x0-\vect{R}_2)}{|\nabla \sigma_2(\x0-\vect{R}_2)|}.
\end{equation}
Qualitatively, the Casimir force on a body arises from 
paths that begin (and end) on its surface, with a vector
path weight
in the direction of the surface normal at the path source point. 
Although the surface of an arbitrary body involves surface normals
pointing in all directions, each surface normal obtains a 
different geometry-dependent
weight via the path ensemble.  The result is, in general,
a nonzero net force.

\subsubsection{Potential Curvature}

This method can be easily extended to the second derivative of the worldline energy, which 
computes the potential curvature,  
\begin{equation}
  C_{ij} := (\hat{r}_i\cdot \nablaR2)(\hat{r}_j\cdot \nablaR2)E.
\end{equation}
For a dielectric describing two bodies, the derivatives with respect to $\vect{R}_2$ in direction $\hat{r}_i$, can be rewritten 
in terms of derivatives with respect to the first body's center $\vect{R}_1$, and the loop coordinates $\vect{x}_k$
\begin{align}
  \nablaR2\langle \epsr\rangle  
  =& \bigg(\sum_{k=1}^N\nablaxk-\nablaR1\bigg)\nonumber\\
  & \times[\langle \epsilon_1(\vect{x}-\vect{R}_1)\rangle+\langle\epsilon_2(\vect{x}-\vect{R}_2)\rangle],
% \frac{\partial}{\partial R_{2,i}}\langle \epsr\rangle  
%   =& \bigg(\sum_{k=1}^N\frac{\partial}{\partial {x}_{k,i}}-\frac{\partial}{\partial R_{1,i}}\bigg)\nonumber\\
%   & \times[\langle \epsilon_1(\vect{x}-\vect{R}_1)\rangle+\langle\epsilon_2(\vect{x}-\vect{R}_2)\rangle],
  \label{eq:shift_derivative}
\end{align}
where $\nablaxk$ is the gradient of the path position $\vect{x}_k$.    
The first derivative can be carried out as before,
% \begin{align}
%   C_{ij} = &
% \frac{\alpha\chi_2\hbar c}{2(2\pi)^{D/2}}\intzinf \frac{d\cT}{\cT^{1+D/2}}
% \int d\vect{x}_0 \nonumber\\
% &\hspace{0.05cm}\times\biggdlangle 
% \hat{r}_{i}\!\cdot\!\bigg(\!\sum_k\nablaxk - \nablaR{1}\!\bigg)
%   \!
%   \bigg[
%   \frac{\hat{r}_{j}\cdot\langle \delta(\sigma_2)\nablaR{2}\sigma_2\rangle}
%   {\langle\epsr\rangle^{\alpha+1}}\bigg]\biggdrangle_{\vect{x}(t)}.
% \end{align}
\begin{align}
  C_{ij} = &
\frac{\alpha\chi_2\hbar c}{2(2\pi)^{D/2}}\intzinf \frac{d\cT}{\cT^{1+D/2}}
\int d\vect{x}_0 \nonumber\\
&\hspace{0.05cm}\times\biggdlangle 
\hat{r}_{i}\!\cdot\!\bigg(\!\sum_k\nablaxk - \nablaR{1}\!\bigg)
  \!
(\hat{r}_{j}\cdot\nablaR{2})
  \frac{1}
  {\langle\epsr\rangle^{\alpha+1}}\biggdrangle_{\vect{x}(t)}.
\end{align}
It is possible to integrate by parts on the gradients $\nablaxk$, 
which then act on the Gaussian probability density,
 and which yields a term proportional to $\sum_{k}(2\vect{x}_k-\vect{x}_{k+1}-\vect{x}_{k-1})$.
This sum of path increments vanishes for closed paths, and thus this term can be dropped.  
The remaining gradient in $\nablaR{1}$ can be straightforwardly evaluated, which yields 
a second independent path-averaged $\delta$-function.  
% and using Eq.~(\ref{eq:delta-normal}), there are now two path-averaged delta-functions which 
% pin the paths to lie on both the first and second surfaces
One $\delta$-function can be manipulated as in Eq.~(\ref{eq:delta-normal}) to constrain the the paths to start on
the first body, while the second $\delta$-function-average pins another point of the path to lie on the second body
and then takes a further averages over which point is pinned.
The resulting expression for the potential curvature is 
\comment{kept $\vect{x}(t)$ subscript on ensemble average: Actually what convention do we want with these?
Sometimes we're adding them, other times we are not}
\begin{align}
  C_{ij}&=
  \frac{\alpha(\alpha+1)\chi_1\chi_2}{2(2\pi)^{D/2}}\intzinf \frac{d\cT}{\cT^{1+D/2}}
  \hspace{-2ex}
  \oint\limits_{\sigma_1(\vect{x}_0-\vect{R}_1)=0}^{}
   \hspace{-4ex} dS\hspace{1ex}   \nonumber\\
  &\hspace{0.5cm} \times\sum_{k=1}^{N-1}\frac{1}{N} \biggdlangle  \mathcal{G}(\vect{x}_0,\vect{x}_k,k,\cT)\nonumber\\
  &\hspace{1cm}\times\frac{[\hat{r}_{i}\cdot\hat{n}_1(\vect{x}_0)][\hat{r}_{j}\cdot\hat{n}_2(\vect{x}_k)]}
  {\langle \epsilon_{\mathrm{r},12}\rangle^{\alpha+2}}     \biggdrangle_{\vect{x}(t)|\sigma_2(\vect{x}_k-\vect{R}_2)=0}.
  \label{eq:potential_curvature}
\end{align}
where 
$\dlangle \cdots\drangle_{\vect{x}(t)|\sigma_2(\vect{x}_k-\vect{R}_2)}$ is the ensemble
average over discrete paths $\vect{x}(t)$ subject to the constraint that $\sigma_2(\vect{x}_k-\vect{R}_2)=0$,
and 
\begin{equation}
  \mathcal{G}(\vect{x}_0,\vect{x}_k,k,\cT)=\frac{e^{-N^2(\vect{x}_0-\vect{x}_k)^2/(2 k(N-k) \cT)}}{[2\pi  \cT k (N-k)/N^2]^{D/2}}
  \label{eq:Gauss_normalization}
\end{equation}
is the Gaussian normalization factor from fixing $\vect{x}_k$ after $k$ steps, and returning to $\vect{x}_0$
in $N-k$ steps. 
There is no need for any further renormalization, since this expression is only non-zero in the presence 
of both bodies, and $\mathcal{G}$ exponentially cuts off terms at small $\cT$.    

\begin{shaded}
  Let us double check that normalization constant.  This can be verified by H\"ormander's expression for 
  the delta function.   
  \begin{equation}
    \int \prod_{k=1}^N dq_k\, f(\vect{q}) \delta[h(\vect{q})]= 
    \oint_{h^{-1}(0)} dS f(\vect{q})\frac{1}{\sqrt{|\nabla h(\vect{q})|^2}}
  \end{equation}
  
  We have previously applied this in the context of the path normalization.  
  In that example, we treat the integration variables $q_k$ are the path increments, $\delta x_k$.
  Then the path-closure condition yields
  \begin{align}
    &\int \prod_{k=0}^{N-1} d(\Delta x_k) \delta(\sum_k \Delta x_j)f(\{\Delta x_k\})
    \prod_{j=0}^{N-1}\frac{1}{\sqrt{2\pi\Delta T}}e^{-\Delta x_k^2/(2\Delta T)}\nonumber\\
   &= \frac{1}{\sqrt{2\pi\Delta T N}}\oint_{\sum_k\Delta x_k=0} \prod_{k=0}^{N-2} d(\Delta x_k)f(\{\Delta x_k\})
    \prod_{j=0}^{N-2}\frac{1}{\sqrt{2\pi\Delta T}}e^{-\Delta x_k^2/(2\Delta T)},
  \end{align}
  which leads to the familiar $1/\sqrt{2\pi T}$ normalization for Brownian bridges.  

  In this case we want fixing to the surface, and also d-dimensional path closure.  
  \begin{align}
    S=&\int d\vect{x}_0\int \prod_{k=0}^{N-1} d(\Delta \vect{x}_k) \delta[\sigma_1(\vect{x}_0)]
    \delta[\sigma_r(\sum_{j=1}^k \Delta \vect{x}_j)]
    \delta^{(d)}[\sum_{j} \Delta \vect{x}_j]
    f(\{\Delta \vect{x}_k\})
    \prod_{j=1}^N\frac{1}{\sqrt{2\pi\Delta T}}e^{-\Delta \vect{x}_k^2/(2\Delta T)},
  \end{align}  
  where $\vect{x}_k = \vect{x}_0+\sum_{j=0}^{j-1}\Delta\vect{x}_0$, $\Delta \vect{x}_k=\vect{x}_{k+1}-\vect{x}_k$.
  (Heuristically I argued I get an surface integral over intermediate positions, weighted by 
  the Gaussian probabilities to go to a point on the surface in $k$ steps, and then return in $N-k$.)

  So we can evaluate the integrals over $\vect{x}_0$ and $\Delta \vect{x}_{N-1}$ immediately.  
  \begin{align}
    S=&\frac{1}{\sqrt{2\pi T}}\oint_{\sigma_1(\vect{x}_0)=0} dS_0\frac{1}{|\vect{n}_1(\vect{x}_0)|}
      \int_{\sum_{k}\Delta x_k=0} \prod_{k=0}^{N-2} d(\Delta \vect{x}_k) 
    \delta[\sigma_b(\sum_{j=1}^k \Delta \vect{x}_j)]
    \prod_{j=1}^N\frac{1}{\sqrt{2\pi\Delta T}}e^{-\Delta \vect{x}_k^2/(2\Delta T)}
    f(\{\Delta \vect{x}_k\}),
  \end{align}  


\end{shaded}



\begin{shaded}
Attempt at Alternative double-pinning notation. 
\begin{align}
  C_{ij}&=
  \frac{\alpha(\alpha+1)\chi_1\chi_2}{2(2\pi)^{D/2}}\intzinf \frac{d\cT}{\cT^{1+D/2}}\int_0^\cT \frac{d\tau}{\cT}\nonumber\\
  &\times
  \oint\limits_{\sigma_1[\vect{x}(0)]}^{}   \hspace{-2ex} dS
  \oint\limits_{\sigma_2[\vect{x}(\tau)]}^{}  \hspace{-2ex} dS'\hspace{1ex}   
\mathcal{G}'[\vect{x}(0),\vect{x}(\tau),\tau,\cT]\nonumber\\
  &\hspace{1cm}\times\biggdlangle\frac{\{\hat{r}_{i}\cdot\hat{n}_1\big[\vect{x}(0)\big]\}
    \{\hat{r}_{j}\cdot\hat{n}_2\big[\vect{x}(\tau)\big]\}}
  {\langle \epsilon_{\mathrm{r},12}\rangle^{\alpha+2}}     \biggdrangle_{\vect{x'}(t); \vect{x'}(\tau)},
  \label{eq:potential_curvature}
\end{align}
where the ensemble average is over paths $\vect{x'}(\tau)$ that are 
pinned to lie on surface $\sigma_1, \sigma_2$ at path positions
$\vect{x}(0), \vect{x}(\tau)$ respectively.  The Gaussian normalization is 
\begin{equation}
  \mathcal{G}'[\vect{x}(0),\vect{x}(\tau),\tau,\cT]=\frac{\sqrt{\cT}}{\sqrt{2\pi  \tau(\cT-\tau)}}
  e^{- \cT[\vect{x}(0)-\vect{x}(\tau)]^2/[2 \tau(\cT-\tau)]}
\end{equation}
\begin{itemize}
\item Use $\vect{x'}$ to distinguish these contrained paths?
\item Double subscripts on ensemble average to denote paths going from $x(0)$ to $x(\tau)$?
\item Use only $\sigma_1[\vect{x}]$ to denote surface we seek?  Current notation reflects desire
ato use sign $\sigma$ in step functions for dielectric, so surface is $\sigma=0$.  Also want
to emphasize body center $\vect{R}_i$ for rigid translations.
\item Problems with later occupation number results, where summing over terms leads to explicit factors of 
$N$.  
\end{itemize}
\end{shaded}

\subsubsection{Torque}
The torque on a body can be found from the first order variation in the energy as that body is
infinitesimally rotated about some axis.  
For concreteness, consider perturbing the dielectric by rotating the second body about its center
by angle $\phi$ about axis $\hat{m}$,
% \begin{equation}
%   K_m = -\partial_\phi W,
% \end{equation}
%where the perturbed dielectric is 
\begin{equation}
  \epsr(\vect{x}) = 1+\chi_1\theta[\sigma_1(\vect{x}-\vect{R}_1)]
  +\chi_2\theta\big\{\sigma_2[\mathcal{R}(\phi)(\vect{x}-\vect{R}_2)]\big\}.
\end{equation}
% where the second body has undergone an infinitesimal rotation 
% by an angle $\phi$ about an axis $\hat{m}$ about its center.  
The infinitesimal rotation matrix is given by 
\begin{equation}
  \mathcal{R}_{ij}(\phi) = \delta_{ij} - m_k\epsilon_{ijk}\phi ,
\end{equation}
where $\delta_{ij}$ is the Krocker delta, and $\epsilon_{ijk}$ is the antisymmetric Levi-Civita tensor. 
Throughout this subsection there are implicit sums over repeated indices.    
The torque for a rotation about axis $\hat{m}$ can be written as $K_m:=\hat{m}\cdot\vect{K}=-\partial_\phi E$.
The $\phi$-derivative only acts on the path-averaged dielectric part of the energy integral,
\begin{align}
  \partial_\phi\langle\epsilon\rangle&=
  \chi_2\langle \partial_\phi\mathcal{R}_{ij}(\phi)(\vect{x}-\vect{R}_2)_j[\hat{r}_i\cdot\nabla\theta(\sigma_2)]\rangle\nonumber\\
  &=-\chi_2\langle m_k\epsilon_{kij}(\vect{x}-\vect{R}_2)_j[\hat{r}_i\cdot\nabla\theta(\sigma_2)]\rangle\nonumber\\
  &=\chi_2\hat{m}\cdot\langle (\vect{x}-\vect{R}_2)\wedge\nabla\theta(\sigma_2)\rangle,
\end{align}
where we used the form of the infinitesimal rotation to write the result as a cross-product 
via $(\vect{a}\wedge\vect{b})_i=\epsilon_{ijk}a_jb_k$.  
This derivative can be directly substituted into the full torque path-integral, 
and similar manipulations to Eq.~(\ref{eq:delta-normal}) can be carried out
to pin the paths to start on the surface of the second body.
In addition, given the form of $\partial_\phi\langle\epsr\rangle$ the full torque $\vect{K}$
can be found by identifying $\partial_\phi E=\hat{m}\cdot\vect{K}$.  
The full renormalized torque worldline path-integral is 
\begin{align}
  \vect{K} &= \frac{\alpha\hbar c\chi_2}{2(2\pi)^{D/2}}\intzinf \frac{d\cT}{\cT^{1+D/2}} 
  \hspace{-3ex}
  \oint\limits_{\sigma_2(\vect{x}_0-\vect{R}_2)=0} 
   \hspace{-4ex} dS\hspace{1ex}\!\big[(\vect{x}_0-\vect{R}_{2})\! \wedge \!\hat{n}_2(\vect{x}_0)\big]   \nonumber\\
  &\hspace{0.5cm}\times\biggdlangle 
\frac{1}{\langle \epsilon_{\mathrm{r},12}\rangle^{\alpha+1}}
  -\frac{1}{\langle \epsilon_{\mathrm{r},2}\rangle^{\alpha+1}}\biggdrangle_{\vect{x}(t)}.
\end{align}
This has the intuitive interpretation of finding the total torque on the body by 
integrating over its surface and taking the cross-product of the vector from the body's center to a surface
element with the force density at that surface element.  

\subsubsection{Casimir--Polder Force}

We briefly note that an alternative expression for the Casimir--Polder force on an atom near a surface
can be found in analogy to the potential curvature in Eq.~(\ref{eq:potential_curvature}).
The force on the atom is $F\subCPi = -\hat{r}_i\cdot\nabla_{\rA}E$.
  In Sec.~\ref{sec:partial_average} we took the derivatives of the path-integral immediately,
  with the dominant contribution coming from the Gaussian probability distribution.  
  Alternatively, one can change the coordinates to $\vect{x}(t)=\rA+\vect{y}(t)$, where 
  $\vect{y}(t)$ is a Brownian bridge starting and returning to the origin, $\vect{y}(0)=\vect{y}(\cT)=0$,
  and then take the desired gradients.
  The resulting force expression is
\begin{align}
  F_i\supTE \!=&\! -\frac{\hbar c\alpha_0}{4(2\pi)^{D/2}}\intzinf \frac{d\cT}{\cT^{1+D/2}}\biggdlangle 
  \hat{r}_i\!\cdot\!\nabla_{\rA}\langle\epsr\rangle^{-3/2}
  \biggdrangle_{\vect{y}(t)},
\end{align}
where this path-integral considers the change in energy as the whole path is translated,
while the results in Sec.~\ref{sec:partial_average}
correspond to shifting only the origin of the path, while keeping the rest of the path fixed.
The derivatives can be carried out, which for piece-wise constant media create delta-functions.
In analogy with the potential curvature, since the starting point is fixed, it is necessary to 
average over pinning other path points to lie on the dielectric surface for each of the bodies.  
The Casimir--Polder force, after summing over all force components, is 
\begin{align}
  \vect{F}\supTE\subCP&=-\frac{3\hbar c\alpha_0}{8(2\pi)^{D/2}}
  \sum_{b=1}^{N_b}\sum_{k=1}^{N-1}\frac{\chi_b}{N}\intzinf \frac{d\cT}{\cT^{1+D/2}}
   \nonumber\\
   &\hspace{0.5cm} \times \biggdlangle \mathcal{G}(\vect{x}_A,\vect{x}_k,k,\cT)
   \frac{\hat{n}_b(\vect{x}_k)}
  {\langle \epsr\rangle^{5/2}}     \biggdrangle_{\vect{x}(t)| \sigma_b(\vect{x}_k-\vect{R}_b)=0},
\end{align}
where we have reverted to using $\vect{x}(t)$, $\mathcal{G}$ is given by 
Eq.~(\ref{eq:Gauss_normalization}), and $b$ indexes each of the $N_b$ dielectric bodies.
In this method the paths are constrained to touch the bodies, which must be taken into account numerically
by averaging over paths where each index that is constrained.
In constrast, the Hermite-Gaussian method discussed in Sec.~\ref{sec:partial_average} 
uses the same paths regardless of the dielectric background.
While the path-pinning method requires more complicated path generation,
it does not suffer from diverging fluctuations as the path resolution is increased.
The Gaussian factor $\mathcal{G}$ exponentially suppresses contributions from pinning small indices $k$,
which would be the problematic terms as $\Delta \cT\rightarrow 0$, 
and thus this method does not require careful handling as $N$ increases.  

\subsection{Occupation Number}
\label{sec:occupation}

The preceding methods offer an intuitive picture of the Casimir force,
however they are poorly behaved in the strong-coupling limit.  
For a typical path of $N$ steps pinned to the surface, approximately half 
of the path will lie inside the body.  For $\chi\gg N$, the denominator $\langle\epsr\rangle^{-1/2}$ dominates
the integrand, so the estimated derivatives tend to zero as $\chi^{-1/2}$ for almost all paths.  
Only rare paths which start on the surface, but do not enter the bulk of the body will contribute significantly.  
As a result the estimated force goes to zero in the strong-coupling limit.
In this section we develop alternative expressions which are better behaved in the strong-coupling
limit and makes direct contact with prior work on Dirichlet worldlines.  

The spatial path-integral can be written in exponential form via the Gamma function,
\begin{align}
  W &= \frac{1}{\Gamma[\alpha]}\int d\vect{x}_0 \int ds\, s^{\alpha-1}e^{-s}\nonumber\\
  &\hspace{0.5cm}\times\bigdlangle e^{-\langle \sum_b\chi_b\theta_b(\vect{x})\rangle}
  - e^{-\sum_b\chi_b\theta_b(\vect{x}_0)}\bigdrangle_{\vect{x}(t)}\label{eq:W_exp2}
\end{align}
where we have introduced a shorthand notation $\theta_b(\vect{x}) = \theta(\sigma_b[\vect{x}-\vect{R}_b])$.
% This exponential form is similar to the approach required to account for non-zero temperature and material
% dispersion~\cite{Mackrory2016}.  The exponential expression also places the integrand in a suitable form 
% to exploiting relevant Feynman-Kac formulae.  
The exponential spatial path-integral for two bodies, after the single-body expressions have been factored out can 
be factorized as 
\begin{align}
  W &= \frac{1}{\Gamma[\alpha]}\int d\vect{x}_0 \int ds\, s^{\alpha-1}e^{-s}\nonumber\\
  &\hspace{0.5cm}\times\Bigdlangle 
  (e^{-\langle \chi_1\theta_1(\vect{x})\rangle}-1)(e^{-\langle \chi_2\theta_2(\vect{x})\rangle}-1)\nonumber\\
  &\hspace{1.25cm}
  -(e^{- \chi_1\theta_1(\vect{x}_0)}-1)(e^{-\chi_2\theta_2(\vect{x}_0)}-1)\Bigdrangle_{\vect{x}(t)}\label{eq:W_exp2}
\end{align}
\comment{Pretty sure a Gies paper does a similar factorization.}
%\begin{align}
  % &e^{-s}[1+e^{-s\langle \chi_1(\vect{x})+\chi_2(\vect{x})\rangle } 
  % - e^{-s\langle \chi_1(\vect{x})\rangle }-e^{-s\langle \chi_2(\vect{x})\rangle}]\nonumber\\
The exponential of the path-averaged potential can be written as a product of potentials 
for each increment, and the potential can be simplified for step-function dielectrics,
\begin{align}
  e^{-s \langle\chi_b\theta_b(\vect{x})\rangle}= \prod_{k=0}^{N-1}
  \left[\bar{\theta}_{b,k}+\theta_{b,k}e^{-s\chi_b/N}\right].\label{eq:exp_step_limit1}
\end{align}
where $\theta_{b,k} := \theta[\sigma_b(\vect{x}_k-\vect{R}_b)],$ 
and $\bar{\theta}_{b,k}:=1-\theta_{b,k}$.  This way of regularizing the surface leads to 
a different representation for the gradients of the Casimir energy.
In this section, we are assuming the step-function is arbitrarily sharp, and taking that limit before any gradients are taken.  
The gradient for a single position can be computed from Eq.~(\ref{eq:exp_step_limit1}) as
\begin{align}
  [\nablaxk e^{-s \chi_b\theta_{b,k}/N}]_{\text{occupation}} 
=& \delta_{b,k}(e^{-s\chi_b/N}-1)\vect{n}_b(\vect{x}_k)
    \label{eq:occupation-grad}
\end{align}
where $\delta_{b,k} := \delta[\sigma_b(\vect{x}_k-\vect{R}_b)]$ and the un-normalized surface normal is
$\vect{n}_b(\vect{x}_k):=\nabla\sigma_{b}(\vect{x}_k-\vect{R}_b)$.
The results in Sec.~\ref{sec:path-pinning} can be recovered if the gradient is 
taken before the step-function limit is taken.  
In essence, that calculation assumed the step was the limit of a smooth sigmoidal function 
and that the path had arbitrarily fine resolution on the scale over which the step-function jump occured,
so that the gradient follows from applying the chain-rule, 
\begin{align}
  [\nablaxk e^{-s \chi_b\theta_{b,k}/N}]_{\text{pinning}} 
 =& \frac{s\chi_b}{N} \delta_{b,k}e^{-s \chi_b\theta_{b,k}/N}\vect{n}_b(\vect{x}_k)
  \label{eq:pinning-grad}
\end{align}
However, throughout this section we will treat using Eq.~(\ref{eq:occupation-grad}) to evaluate gradients,
and later show how the earlier results are recovered.

The force on the second body can be computed by differentiating
the energy with respect to the body position $\vect{R}_2$.  The spatial part of the force integral
can be defined as
\begin{align}
  W_{F,2} :=& -\nablaR{2}W\\
  =& -\frac{1}{\Gamma[\alpha]}\int d\vect{x}_0\int ds\,s^{\alpha-1}e^{-s}\biggdlangle 
  \big(e^{-s\chi_2/N}-1\big)\nonumber\\
  &\times\sum_{j=0}^{N-1}[\vect{n}_2(\vect{x}_j)\delta_{2,j}
  \prod_{k\ne j}\left(\bar\theta_{2,k}+\theta_{2,k}e^{-s\chi_2/N}\right)\nonumber\\
  &\times\bigg[1-\prod_{n=0}^{N-1}\left(\bar{\theta}_{1,n}+\theta_{1,n}e^{-s\chi_1/N}\right)\bigg]\biggdrangle_{\vect{x}(t)}.
\end{align}
where the constant term has zero derivative.
The $s$-integral can be carried out more easily if the integrand is re-arranged into terms with 
a definite number of points $n$ inside each body $b$.  
We define the indicator functions
\begin{align}
  \I[b]0&:= \prod_{j=0}^{N-1}\bar{\theta}_{r,j}\\
  \I[b]n&:= \sum_{j_1=1}\sum_{j_2>j_1}\cdots\sum_{j_{n}>j_{n-1}}\theta_{b,j_1}\theta_{b,j_2}\cdots\theta_{b,j_n},
 \quad n\ge 1,
\end{align}
where $\I[b]n=1$ when there are exactly $n$ points inside body $b$, and zero otherwise;  
there are $n$ sums over indices $j_{n}$, each of which terminates at $j_n=N$.  
There are further restrictions on which of these terms contribute in the integrand.
Due to the presence of the $\delta$-functions, only $N-1$ points are free to 
enter the bodies.  This further implies that the number of points inside both bodies must be less than $N-1$. 
Finally, due to the renormalization only paths with at least one point inside the first body contribute.  
Using the indicator functions, the re-arranged spatial path-integral for the force is 
\begin{align}
  W_{F,2} =& (-1)\int d\vect{x}_0\int ds\,s^{\alpha-1}e^{-s}\biggdlangle \sum_{j=0}^{N-1}\hat{n}_2(\vect{x}_j)\delta_{2,j}\nonumber\\
  &\times\sum_{n=0}^{N-1}
  \big(e^{-s(n+1)\chi_2/N}-e^{-sn\chi_2/N}\big)\I[2]n\nonumber\\
  &\times \sum_{m=1}^{N-n-1}\big(1- e^{-s m \chi_1/N} \big)\I[1]m
  \biggdrangle_{\vect{x}(t)},
\end{align}
The $s$-integral can be carried out term by term, the $\delta$-function can be used to pin paths onto the surface,
and the cyclic-permutation invariance of the path can be used to remove the path-average over pinning, 
as in Eq.~(\ref{eq:delta-normal}).
The full force path-integral is given by 
\begin{align}
  \vect{F}_2 =& (-1)\frac{\hbar c N}{2(2\pi)^{D/2}}\intzinf \frac{d\cT}{\cT^{1+D/2}}
  \hspace{-2ex}\oint\limits_{\sigma_2(\vect{x}_0-\vect{R}_2)=0}  \hspace{-4ex} dS
\nonumber\\
  &\hspace{0.5cm}\times \sum_{n=0}^{N-1}\sum_{m=1}^{N-n-1}\bigdlangle\hat{n}_2(\vect{x}_0)
  \I[1]m\I[2]n f_{m,n}\bigdrangle_{\vect{x}(t)}
  \label{eq:occupation_force}
\end{align}
where the material dependence is carried by 
\begin{align}
  f_{m,n}&:=c_{m,n}-c_{m,n+1}-c_{0,n}+c_{0,n+1},\\
  c_{m,n} &:= \bigg( 1 + \frac{m\chi_1+n\chi_2}{N}\bigg)^{-\alpha},
\end{align}
which come from computing the change in the renormalized energy integrand as another point enters
the second body.  When $\chi_2/N\ll 1$, $f_{m,n}$ can be to leading order in $\chi_2/N$, 
to recover our earlier results for the force.
% The ensemble average $\dlangle\cdots\drangle_{\vect{x}(t)|\sigma_{2}(\vect{x}_j-\vect{R}_2)}$ should be understood to only allow 
% paths where $\vect{x}_j$ is restricted to lie on the surface $\sigma_2(\vect{x}_j-\vect{R}_2)=0$.
% This also applies to the term where $\vect{x}_0$ is fixed, and in that case $\vect{x}_0$ should be restricted to the surface.

In this result the indicator functions carry the position dependence based on whether a given number of points are 
within each body, while $f_{n,m}$ carries the dependence on material properties based on the number of points inside each body.  
This expression is well-behaved in the $\chi\rightarrow\infty$ limit, where only 
the $n=0, m>0$ terms contribute.  In the strong-coupling limit, the main contribution to the 
force comes from paths that just graze the second surface, while also entering the first body.  

For completeness we note the analogous expressions for the torque and potential curvature.  
The manipulations and reasoning used in Sec.~\ref{sec:path-pinning} for the torque and potential curvature
apply here --- the only difference is the form chosen for the gradient, 
and using the indicator functions in the integrand.  
The torque path-integral is 
\begin{align}
  \vect{K}_2 =& -\frac{\hbar c N}{2(2\pi)^{D/2}}\intzinf\frac{d\cT}{\cT^{1+D/2}}
  \hspace{-2ex}\oint\limits_{\sigma_2(\vect{x}_0-\vect{R}_2)=0}  \hspace{-4ex} dS
  \nonumber\\
  &\times\sum_{n=0}^{N-1}\sum_{m=1}^{N-n-1}
  \Bigdlangle(\vect{x}_0-\vect{R}_2)\wedge\hat{n}_2(\vect{x}_0) \nonumber\\
  &\hspace{2.5cm} \times\I[1]m \I[2]n  f_{n,m}\Bigdrangle_{\vect{x}(t)}.
\end{align}
and the potential curvature is given by
\begin{align}
  C_{ij} =& \frac{\hbar c N}{2(2\pi)^{D/2}}\intzinf\frac{d\cT}{\cT^{1+D/2}}
  \hspace{-2ex}\oint\limits_{\sigma_2(\vect{x}_0-\vect{R}_2)=0}  \hspace{-4ex} dS\, \hat{n}_1(\vect{x}_0)
  \nonumber\\ 
  &\times\biggdlangle 
  \sum_{k=0}^{N-1}\hat{n}_2(\vect{x}_k)\mathcal{G}(\vect{x}_0,\vect{x}_k,k,\cT)
  \nonumber\\
  &\hspace{0.75cm} \times\sum_{n=0}^{N-2}\sum_{m=0}^{N-n-2}\I[1]n\I[2]m g_{m,n}
  \biggdrangle_{\vect{x}(t)|\sigma_2(\vect{x}_k-\vect{R}_2)=0}
\end{align}
where 
\begin{align}
  g_{m,n}=c_{m+1,n+1}+c_{m,n}-c_{m+1,n}-c_{m,n+1},
\end{align}
accounts for the change in the energy integrand as the number of points in the first and second
bodies increase.
In the strong-coupling limit, the potential curvature is dominated by terms with $n=m=0$,
which correspond to paths that graze both bodies, while not entering either body.  

The formulation for the Casimir force in Eq.~(\ref{eq:occupation_force}) 
is exactly the construction of paths employed by Gies and Weber for computing 
forces in the sphere-plane and cylinder-plane geometries in the Dirichlet limit~\cite{Weber2010}.  
In that work paths are shifted so that the path just grazes the plane.  The force on the planar
surface is computed by integrating the over the times when the path intersects the spherical or cylindrical surface.
The expressions presented here extend their results by accounting for finite $\chi$, 
and are framed in terms of general geometries.  

In general, different classes of paths are important in the finite $\chi$ and strong-coupling 
cases.  At small $\chi$, the most important path statistic is the sojourn time within the bodies,
while in strong-coupling regime, the first-touching time is the most important statistic.    
This is correspondence was previously used to describe the numerical convergence properties of as 
the resolution of the paths was varied~\cite{Mackrory2016}.  
More practically, this makes it difficult to use a single class of loops to evaluate the potential at all $\chi$:
in weak-coupling one wants a path-ensemble that enters all of the bodies, while in strong-coupling
it is the paths that just touch the surfaces that are most important.

The expressions for the force in Eqs.~(\ref{eq:pinning_force}) and (\ref{eq:occupation_force})
reflect taking two limits in different orders, namely the taking the large $N$ limit and differentiation.  
The first derivation assumed an arbitrarily fine path where $N\gg \chi$ for all $\chi$.  Under
differentiation the arbitrarily fine paths can be pinned to the surface, and there is a range of 
$\cT$ where the integrand is non-zero.  However for a discrete path of length $N$, for sufficiently large $\chi$, 
this range of $\cT$ is inaccessible, and thus the naive numerical estimate fails.    
The second derivation instead takes the $N\rightarrow\infty$ expression last, while using well-behaved
expressions as $\chi\rightarrow\infty$, as is better suited to a numerical method based on discrete paths.  
This method instead highlights finding the times when the number of points inside each body 
change.  

We must distinguish between two facets of the different methods.
One is the choice of starting paths, and the form of the integrand.  In either case, path-pinning
or occupation, we are free to consider a single-path $\{B_j\}$ starting at $x_0$: $x_j=x_0+\sqrt{T}B_j$.
The is also an associated family of paths starting at $x_0$ that translate the original Brownian path
by $-\sqrt{T}B_k$: $x^k_j = x_0+\sqrt{T}(B_j-B_k)$.  This effectively changes which point on that bridge
corresponds to zero.  This sampling is essential for strong-coupling limits where only terms with no points
inside the body contribute to the force or potential curvature.  
This choice to average over which point on the path corresponds to zero is independent of the choice 
of the integrand.  

 
    % \section{TE/TM Zero temperature atom-plane, plane-plane}
    % \section{TE/TM Zero temperature atom-sphere, atom-cylinder}
    % \section{Finite Temperature}

    % \comment{Maybe I need some words here to avoid weirdness?}


%%% Local Variables: 
%%% mode: latex
%%% TeX-master: "thesis_master"
%%% End: 
