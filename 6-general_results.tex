\chapter{Electromagnetic Worldlines - General Results}
\label{ch:general}
There are only general ideas at this point.
There are two options- first take the gauge-theory version.
Second treat linear coupling between the variables with rotation based on the current surface normal.

In the same fashion that a path integral finds the global evolution through time by finding the correct short-time
propagator, these methods aim to find the global evolution through space by chaining together
results based on the nearest planar surface.  

% Another potentially fruitful analogy exists between the curved space of general relativity, 
% and the fact that light bends in spatially varying dielectrics~\cite{Leonhardt2000}. 
% Electromagnetism can be quantized on a curved background without much difficulty.
% One merely replaces all derivatives by their covariant counterparts, which include correction terms
% based on the metric~\cite{Carroll2004}.
% The analogy between gravity and dielectrics has been exploited to propose technologies such as optical 
% black holes, cloaking fields, perfect lenses, and more~\cite{Leonhardt2006}.  
% Unfortunately, this analogy is only exact when $\epsilon=\mu$~\cite{Leonhardt2000}, 
% which most materials do not satisfy.
% It is possible to satisfy the $\epsilon=\mu$ condition over a narrow frequency range using metamaterials, 
% as some experiments have demonstrated~\cite{Leonhardt2006}.  
% So while there are a number of similarities between a dielectric medium and a metric,
% those similarities do not extend to an immediate shortcut past all of these issues.

% We choose to avoid this issue by focusing on improved scalar models, 
% that also correspond to the physical degrees of freedom for the field in certain geometries.  
% The scalars we develop correspond with the Hertz potentials for the plane, and in fact the TE/TM decomposition 
% is commonly exploited in planar geometries \comment{Cite Bordag for planar paper and projection}

% We have some proposals for general methods, but I have not seen how to convert between
% basis so as to stitch together a global atlas from local polarization charts.  (The comparison with 
% the language of differential geometry is deliberate.  In differential geometry a manifold can be covered with 
% coordinates by the union of coordinate charts, which are local maps of the manifold to $\mathbb{R}^n$.  
% Where the charts overlap, the coordinate maps must be consistently mapped into one another.)
% While it might not be possible to cover the whole space globally with a single set of polarizations, 
% it may be possible to do so locally.  For example, one could use the nearest surface 
% normal to define a plane to define the TE/TM polarizations.  As the path propagates, it would be 
% necessary to update the polarizations, and convert the amplitudes into the new coordinates.



\section{Coupled Scalar Polarizations}

See paper by Fosco and Remaggi~\cite{Fosco2012}.


\section{Matrix Path Integral}

    \section{Atom between Angled Plates}
    \section{Plane-Sphere}
    \section{Surface Roughness and corrugations}

%%% Local Variables: 
%%% mode: latex
%%% TeX-master: "thesis_master"
%%% End: 
