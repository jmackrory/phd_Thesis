\chapter{Electromagnetic Worldlines - Numerical Methods and Results}

\begin{itemize}
\item Need to develop methods that can easily generalize to more complicated geometries.
\item Even if using them to solve known problems.
\item Like path integral relies on chaining together correct short-time propagator, we 
  are effectively using planar results at each step, under the assumption that the 
\item Need to study convergence effects carefully.  

\end{itemize}


\section{Numerical Methods}

\begin{itemize}
  \item Governing equations we are simulating.  
\end{itemize}


\subsection{Monte-Carlo sampling}

\begin{itemize}
\item Need average over ensemble of paths.
\item For a Dirichlet scalar, can determine the set of times when integral is non-zero and 
  integrate over that.  In this case, the loop potential is either zero or one, with no variation.
  Thus it is more straightforward to integrate over space/loop-time.  
\item For computationally expensive loops, it makes sense to sample only a single position, time.
  Explore a larger ensemble in the same computational time.
  \item Rely on importance sampling to determine most important regions of $x_0$, $T$ to sample from.
  \item For time: For TE loops, integrand is zero before first touching time.  
    So use $P(\cT,\cT_0) = 2\cT_0^2\theta(\cT-\cT_0)/\cT^3$.
    TM loops turn on more softly, due to potential terms (which take into account possible sub-paths
    that touch the surface).
  \item Birth-death method effectively examines a larger ensemble of loops.
    For two-bodies, it is possible to spawn new loops while close to only a single surface,
    that might touch the other surface.  Thus the first-touching time for the parent loop
    is a poor estimate.  Simplest to use an ensemble average estimate for time:
    $P(\cT,d_0) = \exp(-2d_0^2/\cT)/\cT^3$, where exponential factor is the first touching
    time.  
  \item For spatial integrals, sample from uniform distribution within a radius that bounds all objects.
    Then sample from a power-law outside that, where estimate follows from first-touching time.
    May be possible to optimize, but captures important features without biasing
  \item Frequency integrals: Can execute in parallel (like picking different $chi$).
\end{itemize}

\subsection{Loop generation: TE}

\subsubsection{TE}
\begin{itemize}
\item Use v-loop algorithm from Gies.  Loops have Gaussian increments, and closed paths
\item (Derivation for loops)
\item Jacobian gives normalization.
\item Note connection to Brownian bridge SDE.  
\end{itemize}

\subsubsection{TM}
\begin{itemize}
\item TM potential must also be tracked along a given path.
( Explore the potential-only integrand to highlight the issues.)
\item TM potential takes on values $1<V_k<2$ outside a body, and $0<V_k<1$ when crossing
  or inside.  
\item Leads to large fluctuations, for even moderate $N$.  (Histogram of values, and scaling of variance)
\item Large fluctuations indicate a poor choice of probability distribution, and we should
  adjust our choice.
\item Two methods to incorporate the potential into the probability distribution.
  1) Adjust loop increments by sampling from $P_{TM}(x):= e^{-x^2/(2\Delta T)}e^{-\VTM}$.
  Integrand is piece-wise gaussian.  Can pick a particular Gaussian.  FOr differences of Gaussians,
  use rejection to sample the deviates.  
  2) Use regular Gaussian increments, but spawn new trajectories if acculumated potential 
  is large, and terminate if too small.  
\item Birth-death method is required for first method anyway, as there is still a potential
inf the form of the normalizations.  
\item Figure: Growth of fluctuations.
\item Figure: Histograms of normalization.
\end{itemize}

\subsection{Gradient Estimation}

\begin{itemize}
  \item Need gradients to compute forces, torques from potential.  Also need it directly
    for TM Casimir-Polder energy.
  \item Looking for corrections to gravity need curvature.  Similarly, estimating 
    (Cite Cornell expt) change in oscillation frequency.  Need two spatial derivatives.
    For TM potential, this is 4 spatial derivatives.  
\end{itemize}

\subsubsection{Finite Differences}

\begin{itemize}
  \item Simple to implement.  However, larger fluctuations, and biased. 
  \item Best to use common random numbers.  Also provides better error estimate.
  \item Need to balance choosing $N$, $ds$.  
\end{itemize}

\subsubsection{Partial Averaging-Gaussian Paths}

\begin{itemize}
  \item Consider how to evaluate Gradients with respect to source point of Gaussian 
    path integrals of the energy $\epsr$
    
    \begin{align}
      E =& -\frac{\hbar c\alpha_0}{2(2\pi)^{D/2}}\int \frac{d\cT}{\cT^{1+D/2-\alpha}}
      \biggdlangle \frac{1}{\langle\epsr\rangle}\biggdrangle\\
      =& -\frac{\hbar c\alpha_0}{2(2\pi)^{D/2}}\int \frac{d\cT}{\cT^{1+D/2-\alpha}}\int ds\,\frac{s^{\alpha-1}}{\Gamma[\alpha]}
      \biggdlangle e^{-s\cT \langle \epsr\rangle}\biggdrangle
    \end{align}
  \item If we use unshifted integration variables, $\vect{x}_k$, rather than the shifted, scaled
    Brownian motion variables, can differentiate immediately.
    Momentarily focus on just the path integral piece,
    \begin{equation}
      P = \biggdlangle e^{-s\cT \langle \epsr\rangle}\biggdrangle 
      = \int \prod_{j=1}^{N-1}dx_k \prod_{j=0}^{N-1}\frac{1}{(2\pi \Delta T)^{D/2}}e^{-(\vect{x}_{j+1}-\vect{x}_j)^2/2\Delta T-s\Delta T \epsr(\vect{x}_j)}.
    \end{equation}
    Main component are derivatives w.r.t. Gaussian.  Will neglect derivatives of the potential.  
    Our basic idea is to integrate out intermediate coordinates.  Under the assumption the 
    the integrands are approximately stable, or that the steps compose, we can average multiple
    steps.
  \item This is related to choosing a non-uniform loop with less resolution close to the beginning of the 
    loop.  This is justified via switching integration and differentation.  The Gaussian integrals 
    obviously compose.  The derivatives can just be carried out at the very end.  
  \item Derivatives of the Gaussian are Hermite polynomials:
    \begin{equation}
      \frac{d^n}{dx^n} e^{-x^2} = (-1)^n H_n(x)e^{-x^2}
    \end{equation}
    Scaling the variables to $x\rightarrow x/a$ yields
    \begin{equation}
      \frac{d^n}{dx^n} e^{-(x-\mu)^2/a^2} = a^{-n}(-1)^n H_n\big(\frac{x-\mu}{a}\big)e^{-(x-\mu)^2/a^2}
    \end{equation}

    \item Convolution formula.  Let us consider the convolution of Gaussians with a Hermite 
    polynomial.  This naturally emerges from integrating out a coordinate of the path integral
    \begin{equation}
      I = H_n\big(\frac{x-\mu_1}{\sqrt{2\sigma_1^2}}\big) \frac{e^{-(x-\mu_1)^2/(2\sigma_1^2)}}{\sqrt{2\pi \sigma_1^2}} 
      * \frac{e^{-(x-\mu_2)^2/(2\sigma_2^2)}}{\sqrt{2\pi \sigma_2^2}}.
    \end{equation}
    The convolution integral is most naturally carried out using the Fourier transform.
    \begin{align}
      f*g =& \int dx' f(x-x')g(x')\\
      = & \int dx' \frac{dk}{2\pi} \frac{dq}{2\pi} e^{ik(x-x')+iqx'} f(k)g(q)\\
      = & \int \frac{dk}{2\pi} e^{ikx} f(k)g(k)
    \end{align}
    The Fourier transform of the Gaussian is 
    \begin{align}
      \int dx e^{ikx}\frac{e^{-(x-\mu)^2/(2\sigma^2)}}{\sqrt{2\pi \sigma^2}} 
      &= \int dx' \frac{1}{\sqrt{2\pi\sigma^2}}e^{-(x'-ik/2)^2/(2\sigma^2)+ik\mu-\sigma^2k^2/2}\\
      &= e^{-\sigma^2 k^2/2+ik\mu}.
    \end{align}

    This can be straightforwardly extended to include Hermite polynomials via integration by parts.
    \begin{align}
      \int dx e^{ikx}(2\sigma^2)^{-n/2}(-1)^n H_n\big(\frac{x-\mu}{\sqrt{2\sigma^2}}\big)
        \frac{e^{-(x-\mu)^2/2\sigma^2}}{\sqrt{2\pi \sigma^2}} 
      &= \int dx e^{ikx}\frac{d^n}{dx^n}\frac{e^{-(x-\mu)^2/(2\sigma^2)}}{\sqrt{2\pi \sigma^2}}\\
      &= (-ik)^n\int dx' \frac{e^{ik(x+\mu)-x^2/(2\sigma^2)}}{\sqrt{2\pi \sigma^2}}\\
      &= (-ik)^ne^{-\sigma^2 k^2/2+ik\mu}
    \end{align}

    The convolution of the Hermite-Gaussian, and a regular Gaussian is 
    \begin{align}
      I&=\int dx' (2\sigma_1^2)^{-n/2}(-1)^n H_n\left(\frac{x-x'-\mu_1}{\sqrt{2\sigma_1}}\right)
      \frac{e^{-(x-x'-\mu_1)^2/(2\sigma_1^2)-(x'-\mu_2)^2/(2\sigma_2^2)}}{2\pi \sigma_1 \sigma_2}\\
      &=\int dx' \int \frac{dk}{2\pi}\frac{dq}{2\pi} (-ik)^n e^{-ik(x-x'-\mu_1)}e^{-iq(x'-\mu_2)}
      e^{-\sigma_1^2k^2/2-\sigma_2^2q^2/2}\\
      &=\int \frac{dk}{2\pi}(-ik)^ne^{-ik(x-\mu_1-\mu_2)}e^{-(\sigma_1^2+\sigma_2^2)k^2/2}\\
      &=(-1)^n[2(\sigma_1^2+\sigma_2^2)]^{-n/2}H_n\bigg(\frac{x-\mu_1-\mu_2}{\sqrt{2(\sigma_1^2+\sigma_2^2)}}\bigg)
      \frac{e^{-(x-\mu_1-\mu_2)^2/[2(\sigma_1^2+\sigma_2^2)]}}{\sqrt{2\pi(\sigma_1^2+\sigma_2^2)}}
    \end{align}

  \item Now to actually apply to integrating out Gaussians, starting from endpoint,
    we need various variable transformations.  Must switch between Hermite polynomial w.r.t $x_0$
    and convolution form.  
    \begin{align}
      & \partial_{0,i}^n e^{-(x_0-x_1)^2/(2T_1)-(x_0-x_{N-1})^2/(2T_{N-1})}\nonumber\\
      =& 
     \partial_{0,i}^n \exp\left[-\frac{T_1+T_{N-1}}{2T_1T_{N_1}}x_0^2-\frac{T_{N-1}x_1+T_1x_{N-1}}{T_1T_{N-1}}x_0
       -\frac{x_1^2}{2T_1}-\frac{x_{N-1}^2}{2T_{N-1}}\right]\\
     % =&\partial_{0,i}^n \exp\left[-\frac{T_1+T_{N-1}}{2T_1T_{N_1}}
     %   \left(x_0-\frac{T_{N-1}x_1+T_1x_{N-1}}{T_1+T_{N-1}}\right)^2
     %   +\frac{(T_{N-1}x_1+T_1x_{N-1})^2}{2T_1T_{N-1}(T_1+T_{N-1})}
     %   -\frac{x_1^2}{2T_1}-\frac{x_{N-1}^2}{2T_{N-1}}\right]\\
     =&(-1)^n \left(2\sigma^2_{1,N_1}\right)^{-n/2}
     H_n\bigg(\frac{x_0-z_{1,N-1}}{\sqrt{2}\sigma_{1,N-1}}\bigg)
       \exp\left[-\frac{(x_0-z_{1,N-1})^2}{2\sigma^2_{1,N-1}}-\frac{(x_1-x_{N-1})^2}{2(T_1+T_{N_1})}\right]\\
    \end{align}
    where 
    \begin{gather}
      \sigma_{1,N-1} = \sqrt{\frac{T_1+T_{N-1}}{2T_1T_{N_1}}}\\
      z_{1,N-1} = \frac{T_{N-1}x_1+T_1x_{N-1}}{T_1+T_{N-1}}
    \end{gather}
  \item We really need 
    \begin{align}
      C=\int dx_1 \partial_0^n \frac{1}{(2\pi)^{3/2}\sqrt{T_1T_2T_{N_1}}}e^{-(x_0-x_1)^2/(2T_1)-(x_1-x_2)^2/(2T_2)-(x_0-x_{N-1})^2/(2T_{N-1})}
    \end{align}
    The simplest approach is to exchange integration and differentiation, then completing the square in $x_0$,
    and then carry out the derivatives
    \begin{align}
      C=&\partial_0^n\int dx_1  \frac{1}{(2\pi)^{3/2}\sqrt{T_1T_2T_{N_1}}}e^{-(x_0-x_1)^2/(2T_1)-(x_1-x_2)^2/(2T_2)-(x_0-x_{N-1})^2/(2T_{N-1})}\\
      =&\partial_0^n \frac{1}{2\pi\sqrt{(T_1+T_2)T_{N_1}}}e^{-(x_0-x_2)^2/[2(T_1+T_2)]-(x_0-x_{N-1})^2/(2T_{N-1})}\\
      =&\partial_0^n \frac{1}{2\pi\sqrt{(T_1+T_2)T_{N_1}}}e^{-(x_0-\mu_2)^2/[2\sigma_2^2]-(x_2-x_{N-1})^2/(2[T_{N-1}+T_1+T_2])}\\
      =&(-1)^n(\sqrt{2}\sigma_2)^{-n/2}H_n\bigg(\frac{x_0-\mu_2}{\sqrt{2}\sigma_2}\bigg)
      \frac{1}{(2\pi)\sqrt{(T_1+T_2)T_{N_1}}}e^{-(x_0-x_2)^2/[2(T_1+T_2)]-(x_0-x_{N-1})^2/(2T_{N-1})},
    \end{align}
    where 
    \begin{gather}
      \mu_2 = \frac{T_{N-1}x_2+ (T_1+T_2)x_{N-1}}{T_1+T_2+T_{N-1}}
      \sigma_2^2 = \frac{(T_1+T_2)T_{N-1}}{T_{N-1}+T_1+T_2}
    \end{gather}
  \item Now evidently if we integrate out points symmetrically from the loop origin, and we 
    assume all $T_i=\Delta T$, then after integrated out $m$ steps we have 
    \begin{gather}
      \mu_m = \frac{x_{m+1}+ x_{N-m-1}}{2}
      \sigma_2^2 = m\Delta T.
    \end{gather}
    The gradient of the path integral can then be approximately written as 
    \begin{align}
      C\approx
      =&(-1)^n(\sqrt{2 m\Delta T})^{-n/2}H_n\bigg(\frac{x_0-\bar{x}_m}{\sqrt{2 m\Delta T}}\bigg)\nonumber\\
      &\times \frac{1}{(2\pi m \Delta T)}e^{-(x_0-x_{m+1})^2/[2m\Delta T]-(x_0-x_{N-m-1})^2/(2m\Delta T)}.
    \end{align}
  \item We can estimate how many steps to integrate out based on the touching probability in time $T_m$.  
    The probability to touch a plane a distance $d$ away in time $T_m=m\Delta T = (m/N)T$ is 
    \begin{equation}
      P_{\text{touch}} = e^{-2d^2/T_m},
    \end{equation}
    which can be solved for $m$ if we require that $P_{\mathrm{touch}}$ does not exceed a threshold
    $10^{-\rho}$ with $\rho>0$  
    \begin{equation}
      \frac{m}{N} \le \frac{2d^2}{T}\rho\ln(10).
    \end{equation}
    Note that as $N$ increases, the integration point approaches a constant fraction, even as $N$ increases.  
    This in effect reduces the fluctuations by $m^n$.  
    \item This is in effect choosing to make a loop with non-uniform steps, where in particular the first
    and last step are larger than the others.  We choose the size of those steps to be as 
    large as possible, while not interacting much with a the nearest surface.  
    \item While estimating gradients for the stress-tensor it may be necessary to estimate 
    gradients while close to one surface.  For a renormalized energy, only loops that touch both
    surfaces will contribute.   In this case, the above approach based on integrating out Gaussians
    is of limited utility in reducing the fluctuations.  The assumption that the integrals are approximately
    Gaussian can only be explited for a very small number of steps.  
    However, if we assume there is a Feynman-Kac formula available taking into account the 
    interaction with a simple surface, then similar reasoning can be used.  In this case
    one integrates out further steps, since the Feynman-Kac formulae compose with one another,
    where now the threshold is must balance the path wandering sufficiently far that a simple 
    approximation.

    For example, consider using the Feynman-Kac formula for open loops near a Dirichlet surface, 
    \begin{equation}
      \dlangle e^{-V_D(x-d)}\drangle_{x_{j}\rightarrow x_{j+1}} 
      = \theta[(x_j-d)(x_{j+1}-d)]\left(1-e^{-2(x_j-d)(x_{j+1}-d)/T}\right)
    \end{equation}
    
    
\end{itemize}


\begin{itemize}
  \item Gradients for atom
  \item Stress energy tensor 
\item Application to stress-tensor.  (Evaluate derivatives first, then take limit.)
  Problematic for stress-tensor values near one surface in 2-body scenarios. (Use exponential expressions
  to allow non-Gaussian integration.  Must just be stable distribution  
  Thus expact larger fluctuations near surfaces.  
  For example, stress-energy tensor requires 
  \begin{equation}
   T_{zz}=\lim_{\Delta\rightarrow 0} [\partial_{x_{CP}}^2 - \partial_{\Delta}^2]\dlangle \mathfrak{M}\drangle,
  \end{equation}
  where $\Delta = (x_N-x_0)/2, x_{CP} = (x_N+x_0)/2$.  Using the chain-rule, and employing the
  same differentiation this becomes. where $x_{CP}$ is the centre-point, and $\Delta$ is the separation.
  \begin{gather}
    x_{CP} =  \frac{1}{2}(x_{N}+x_0)\\
    \Delta = \frac{1}{2}(x_{N}-x_0)\\
    x_0  = x_{CP}-\Delta\\
    x_N = x_{CP}+\Delta
  \end{gather}
  Then
  \begin{align}
    \frac{\partial}{\partial x_{CP}} 
    %&= \frac{\partial x_0}{\partial x_{CP}}\frac{\partial}{\partial x_0}
    % +\frac{\partial x_N}{\partial x_{CP}}\frac{\partial}{\partial x_N}\\
&= \frac{1}{2}\frac{\partial}{\partial x_N}+\frac{1}{2}\frac{\partial}{\partial x_0}\\
  \frac{\partial}{\partial \Delta} 
%&= \frac{\partial x_0}{\partial \Delta }\frac{\partial}{\partial x_0}
%    +\frac{\partial x_N}{\partial \Delta}\frac{\partial}{\partial x_N}\\
&= \frac{1}{2}\frac{\partial}{\partial x_N}-\frac{1}{2}\frac{\partial}{\partial x_0}
  \end{align}
Then expanding these derivatives out (correct expression?)
\begin{align}
  T_{zz} &= \lim_{x_N\rightarrow x_0}\frac{1}{4}[(\partial_0+\partial_N)^2+(\partial_0+\partial_N)^2]G\\
  &= \lim_{x_N\rightarrow x_0}\frac{1}{2}[\partial_0^2+\partial_N^2]G
\end{align}

\end{itemize}


\subsection{Pinning}

Let's get the $\delta$-function manipulations correct.  
If we have 
\begin{equation}
  \int d^dx \delta(f(\vect{x}))g(\vect{x}) = \int\limits_{f^{-1}=0} d^{d-1}S \frac{1}{|\nabla f(\vect{S})|}g(\vect{x})
\end{equation}
where we split $\vect{x}$ into a surface normal, and tangent integrals, using
coordinates, $\vect{x}=(f,\vect{S})$, where $\vect{S}$ is $(d-1)$-dimensional, and $|\nabla f|$ using the vector norm.  
Now using the delta-function to limit the integration to lie on the surface we find 
\begin{align}
  F_{2,i}&=\frac{\hbar c a}{8\pi^2}\int \frac{dT}{T^3}\int\limits_{S_2} d^{d-1}\vect{x}_0 
\biggdlangle \frac{\chi_2\hat{R_i}\cdot\hat{n}_2}{\langle\epsr\rangle^{\alpha+1}}\biggdrangle
\end{align}
where we defined the normal vector field
\begin{equation}
  n_2 = \frac{-\nabla f_2}{|\nabla f_2|}.
\end{equation}

   Can treat delta-functions as pinning for Brownian bridges.  Then integrate over times.
   Consider only paths starting on one surface (thus potential must be smooth/regular).
    Convergence issues: For $\chi/N\gg 1$, the pinning estimate goes to zero.  Increasing 
    $N$ 
    Probability of appropriate path is low?
    Any one point saturates the integrand.  Region of available, non-zero $T$ goes to zero.  
    Gies method of parameterizing first touching times directly addresses this.  
  
  \subsubsection{Potential Curvature}
    The pinning idea can be extended to a higher derivatives.  
    For a dielectric describing two bodies, we can rewrite derivatives of path-averages using 
    \begin{equation}
      \nabla_1\langle \epsr\rangle  
% = \bigg(\sum_k\nabla_k-\nabla_2\bigg)\frac{1}{N}\sum_j
%       [\epsilon_{\text{r},1}(\vect{x}_j-\vect{R}_1)+\epsilon_{\text{r},2}(\vect{x}_j-\vect{R}_2)],
= \bigg(\sum_k\nabla_k-\nabla_2\bigg)(\langle \epsilon_1\rangle+\langle\epsilon_2\rangle)
    \end{equation}
    where $\nabla_k$ denotes the gradient w.r.t the path-point $\vect{x}_k$, while $\nabla_1,\nabla_2$
    denote the gradients w.r.t. the body positions $\vect{R}_1$ and $\vect{R}_2$.  

    The gradient of the force, or curvature of the potential is given by 
    \begin{align}
      C_{ij} =& -\frac{\partial}{\partial R_{2,j} }F_{2,i}\\
      =&\frac{\hbar c a}{8\pi^2}\int \frac{d\cT}{\cT^3}\int d\vect{x}_0 \nonumber\\
      &\times\biggdlangle \hat{R}_{2,j}\cdot(\sum_k\nabla_k-\nabla_1)
      \frac{\chi_2\hat{R}_{2,i}\cdot\nabla\theta(\sigma_2)}{\langle\epsr\rangle^{a+1}}\biggdrangle
    \end{align}
    If we integrate by parts on the first gradient, the gradients act on the Gaussian probability distribution,
    and pull down a factor proportional to $\sum_{k}(2\vect{x}_k-\vect{x}_{k+1}-\vect{x}_{k-1})$.
    For closed paths this sum of increments vanishes, and thus we can drop this term.  
    \begin{align}
      C_{ij} 
      =&\frac{\hbar c a(a+1)}{8\pi^2}\int \frac{d\cT}{\cT^3}\int d\vect{x}_0 \nonumber\\
      &\times\biggdlangle (-1)\frac{\chi_1[\hat{R}_{2,j}\cdot\langle \nabla_1\theta(\sigma_1)\rangle]
      [\chi_2\hat{R}_{2,i}\cdot\langle \nabla\theta(\sigma_2)\rangle]}{\langle\epsr\rangle^{a+2}}\biggdrangle.
    \end{align}
    By exploiting the delta functions, and restricting to paths that are constrained to touch both surfaces
    we can write the potential curvature as 
\begin{align}
  C_{ij}&=\frac{\hbar c a(a+1)}{8\pi^2}\int \frac{dT}{T^3}
\biggdlangle \int_{S_1} dx_0 \sum_{k=1}^{N-1}\frac{1}{N}\int_{S_2} dx_k
  \nonumber\\
  &  \frac{\chi_1\chi_2\hat{R}_{2,j}\cdot\hat{n}_1(\vect{x}_0)\hat{R}_{2,i}\cdot\hat{n}_2(\vect{x}_k)}
  {\langle \epsr(x)\rangle^{a+2}}\frac{e^{-N(\vect{x}_0-\vect{x}_k)^2/(2 k(N-k)\Delta T)}}
  {(2\pi \Delta T)\sqrt{k (N-k)}}\biggdrangle
\end{align}
    As written, there is no need for any further renormalization, since this is only non-zero in the presence 
    of both bodies.  
    A similar expression for pinned paths can be developed to compute the force on an atom near
    a surface.  In that case, we restrict $\vect{x}_0$ to coincide with the atom's position.  

(To numerical sample from the $k$ distribution, we will take sample uniformly
-despite being sub-optimal, the shape depends strongly on how large $d/T$ is.  
For $d/T$ small, $[k(N-k)]^{-1/2}$ dominates,  While for $d/T\sim 1$ it's more dominated
by $k(N-k)$, or a Gaussian.  
Alternatively, subdivide the region into $N_s$ sub-samples, and uniformly sample from those.  

\subsubsection{Torque}

A similar construction can be made to find the torque.  We will consider rotating 
one of the bodies, an angle $\theta$ about an axis $\vect{n}$ relative to some reference configuration.
Then the torque can be written as 
\begin{equation}
  K_n = \partial_\theta U[\vect{x}-\vect{R}_1,\mathcal{R}(\vect{x}-\vect{R}_2)]
\end{equation}
where $\mathcal{R}= \exp(-\theta n_i \epsilon_{ijk})$ is the rotation operator in 3D, where $\epsilon_{ijk}
$ is the antisymmetric Levi-Civita tensor, and $\vect{R}_i$ are the respective centers of the bodies, 
with surfaces defined by $\sigma_i$. \comment{I think this is the torque in the $\vect{n}$ direction.}

For the worldline path integrals the torque can be written as 
    \begin{equation}
      K_n = \int d\vect{x_0} \biggdlangle \frac{\partial_\theta[\mathcal{R}_{ij}(\theta)(x_j-R_{2,j})]
        \partial_i \sigma_2(\vect{x}'-\vect{R}_2')\delta(\sigma_2)}{\langle 1+\chi_1\theta(\sigma_1)+
        \chi_2\theta(\sigma_2)\rangle^{a+1}}\biggdrangle
    \end{equation}
    introduce rotated coordinates $\vect{x}' = \mathcal{R}\vect{x}$.  
    \comment{Repeated application of the chain rule.  }

 Find that $ \partial_\theta \mathcal{R} = -n_i\epsilon_{ijk}$.
    \begin{equation}
      K_n = \int dS_2 \biggdlangle \frac{-n_i\epsilon_{ijk}(x'_k-R_{2,k}')\partial_j \sigma_2'\delta(\sigma_2)}
      {\langle 1+\chi_1\theta(\sigma_1)+ \chi_2\theta(\sigma_2)\rangle^{a+1}}\biggdrangle
    \end{equation}
  Note that our choice of original configuration is arbitrary.  We can choose to set $\theta=0$.  
    (Derivation Amounts to finding energy shift from infinitesimal rotation.)
    In this case the axis $\hat{n}$ is also arbitrary, and we can use $K_n=\hat{n}\cdot\vect{K}$,
    to write the full torque as 
    \begin{equation}
      \vect{K} = \int\frac{dT}{T^3}\int dS_2 \biggdlangle \frac{(\vect{x}'-\vect{R}_{2}')\times \hat{n}_2}
      {\langle 1+\chi_1\theta(\sigma_1)+ \chi_2\theta(\sigma_2)\rangle^{a+1}}\biggdrangle
    \end{equation}

\section{Strong coupling limits}

  Compute strong coupling potential as based on probability to \emph{not} touch a surface.
    \begin{equation}
      \mathcal{M}(\vect{x}(t)) = (1-\prod_k e^{-V_{1,k}})(1-\prod_j e^{-V_{2,j}})
    \end{equation}
    In strong coupling we can treat the potential along each step as 
    \begin{equation}
    e^{-V_{1,k}} \approx 1 - \theta(\sigma_1(\vect{x}_k-\vect{R}).
  \end{equation}
    The total renormalized potential is then 
    \begin{equation}
      \mathcal{M}(\vect{x}(t)) = [1-\prod_k (1-\theta_{1,k})][1-\prod_j (1-\theta_{2,j})].
    \end{equation}
    This is unity when a given path intersects both body one, and body 2.  This is 
    a representation of the $\mathcal{M}$ functional from Schafer~\etal\cite{Schafer2016}.

    The worldline path integral can then be written as 
    \begin{equation}
      U = \int \frac{dT}{T^3} \int d\vect{x}_0\dlangle \mathcal{M}(\vect{x}) \drangle
    \end{equation}
    Under linearity, there are a few choices available for which order to do integrals and ensemble
    averaging.  Difference shows up when considering gradients.  

  Gies/Weber ordering
    \begin{equation}
      U = \dlangle \int d\vect{x}_0 \int \frac{dT}{T^3} \mathcal{M}(\vect{x}) \drangle
    \end{equation}
    Determine $\mathcal{M}(t)$ for a given path, and parameterize as function of $x,T$.
    Then carry out integrals (analytically for simple geometries).  
    Then average over paths.  
    \begin{equation}
      U = \dlangle \int d\vect{x}_0 \sum_{T_k}\int_{T_k}^{T_{k+1}} \frac{dT}{T^3}\drangle,
    \end{equation}
    where $T_k=T_k(\vect{x}(t),\sigma_1,\sigma_2)$ are set of crossing times, when path is in contact with both bodies.  
    Not all points will touch surface.  
    For planes, this is simple $T_k = \frac{(d-x_0)^2}{B_k^2}$.
    Can find force via differentiation.  In general complicated relationship. 
    
    Note that the prior strong-coupling limit treated the surface as thin plates.  This is recovered 
    from our formulation 

    Direct differentiation w.r.t. the body position is 
    \begin{align}
      \partial_{2,i}\mathcal{M}(\vect{x}(t)) =& (-1)\big[1-\prod_k (1-\theta_{1,k})\big]\nonumber\\
      &\times \sum_j\partial_i\sigma_2\delta[\sigma_2(\vect{x}_j-\vect{R}_2)]\prod_{l\ne j}(1-\theta_{2,l}).
    \end{align}
    Then force worldline path-integral is 
    \begin{align}
      \vect{F}_2 =& (-1)^2\frac{\hbar c}{8\pi^2} \int \frac{d\cT}{\cT^3}\int_{S_2} d\vect{x}_0 \dlangle 
      \big[1-\prod_k (1-\theta_{1,k})\big]\nonumber\\
      &\times \hat{n}_2\prod_{l\ne 0}(1-\theta_{2,l}).
    \end{align}
    \comment{two negative signs: force= -grad U, and $\partial_r \sigma(x-r) = -sigma'(x-r)$}
    If naively construct paths pinned to surface, as advocated above, only rare paths
    that do not enter into body 2 contribute.  Better to think of finding first touching time 
    for paths: want paths that enter the first body, and just graze the second.  
    Easiest to generate paths, and find first touching time via root-finding.  
    This amounts to interpreting $\delta(\vect{x}_j-\vect{R})$ as $\delta[f(T)]$.
    This is exactly the construction of paths employed by Gies and Weber for the sphere-plate 
    and cylinder-plate geometries~\cite{Weber2010}.  In that work, they re-center the paths such 
    that the minimum contact of the path just grazes the plane.

   Contrast to Weber approach\cite{Weber2009, Weber2010}. They take derivatives of worldline properties, and average
    that expression over worldline ensemble, and positions/times. 
    Captures first touching times, that would be otherwise rare events for paths starting on
    surface.  Better suited to Dirichlet Limit. 

    Similar reasoning can be exploited at finite $\chi$.  
  For a particular path and starting position, Can think of 
  $\int dT \epsr = \sum_j\int_{T_j}^{T_{j+1}}dt c_j$, where $j$ indexes set of times when a point on 
  the path lies just on the surface of one of the bodies, with $c_j$ being an appropriate constant.  
  In this discretization, $T_j$ carry the geometric information.  Under differentiation, these are what change.  



\section{Results}
    \subsection{Atom-Plates}
    
    \subsubsection{TE Polarization}
    \begin{itemize}
      \item Simple Trapezoidal rule
      \item Convergence arguments
      \item ``Exact methods''
    \end{itemize}
    
    \subsubsection{TM Polarization}

    \begin{itemize}
      \item Need gradient estimation.  Use Malliavin calculus~\cite{Fournie1999, Chen2007,Kohatsu-Higa2003}.
        Formal introductions \cite{Nualart2006, Malliavin2006, DiNunno2009}.
      \item Gradients required for computing forces.  
      \item Birth-death methods to handle product of increments.  Related to Genealogical 
        methods used in rare-event simulation.  
      \item Histogram of raw values, with/without birth-death.
      \item 
    \end{itemize}
    
    \section{Planar Dielectrics}




    \section{Atom-Sphere}
    \section{Atom-Cylinder}

    \section{Plane-Sphere}

%%% Local Variables: 
%%% mode: latex
%%% TeX-master: "thesis_master"
%%% End: 
