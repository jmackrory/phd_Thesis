\chapter{``Realistic'' Quantum Trajectories for Position Measurements}

      \section{Position Dependent measurement operators}

      \subsection{Initial Setup}
      \begin{itemize}
        \item Atom emitting light.  Assume dipole transition.
        \item Assume atom driven by resonant probe beam, while trapped in far off-resonant
          dipole trap.  This aligns atomic dipoles in particular direction.  
        \item Atomic Hamiltonian
          \begin{equation}
            H = \frac{\vect{p}^2}{2m}+\sum\hbar \omega\sigma_z
            -\sum_i\vect{d}\cdot\vect{E}_i(\vect{x},t)
          \end{equation}
          center-of-mass motion, internal energy levels, and interaction with external fields.  
        \item Operators defined as 
          \begin{gather}
            \sigma = |g\rangle\langle e|\\
            \sigma^\dag = |e\rangle\langle g|\\
            \sigma_z = |e\rangle\langle e|-|g\rangle\langle g|
          \end{gather}
        \item Commutation relations
          \begin{gather}
            [\sigma^\dag,\sigma] 
            %= [|g\rangle \langle e|, |e\rangle \langle g|]
            = \sigma_z\\
            [\sigma, \sigma_z]   = 2\sigma\\
            [\sigma^\dag, \sigma_z]   = -2\sigma
          \end{gather}
        \item Consider spontaneous emission.  
        \item Stochastic Master Equation (approach heavily influenced by Quantum Optics notes).
          No conditioning
          \begin{equation}
            d\rho = -\frac{i}{\hbar}[H,\rho]dt + \frac{\Gamma}{2} \dec[\sigma e^{i\vect{k\cdot x}}] dt,
          \end{equation}
          where decoherence super-operator is 
          \begin{equation}
            \dec[A]\rho = 2A\rho A^\dag - A^\dag A\rho - \rho A^\dag A.
          \end{equation}
          Density matrix $\rho$ for full center of mass, and internal degrees of freedom.  
          \item Stochastic increment $df = f(t+dt)-f(t)$.
          \item For an experiment monitoring the flourescence via angle-resolved emission,
          we can model the atom's evolution via the following stochastic Master equation (SME)
          \begin{align}
            d\rho =& -\frac{i}{\hbar}[H,\rho]dt - \frac{\Gamma}{2}\hom[\sigma^\dag\sigma]\rho dt
            + \int d\Omega_k \jump[\sigma e^{i\vect{k}\cdot\vect{x}}]\rho dN_k
          \end{align}
          where homodyne and jump operators are 
          \begin{align}
            \hom[A]\rho = A\rho + \rho A^\dag - \tr[\rho(A+A^\dag)]\rho\\
            \jump[A]\rho = \frac{ A\rho A^\dag}{\tr[A\rho A^\dag]} - \rho.
          \end{align}
          The Poisson increments $dN_k$ have mean rate
          \begin{equation}
            dN_k = \Gamma\Tr[\sigma^\dag\sigma\rho] dt.
          \end{equation}
          If we account for the dipole emission pattern, assuming the atom's have their dipole preferentially
          aligned via the effective dipole potential, then 
          \begin{equation}
            dN_{k,i} = \Gamma\Tr[\sigma^\dag\sigma\rho]|u_i(\vect{k})|^2 dt.
          \end{equation}
          where $u_i(\vect{k})$ is the dipole emission pattern at wave-vector $\vect{k}$.  Normalized
          so that $\sum_{i=0,\pm}\int d\Omega_k |u_i|^2 = 1$.

          \item Inefficiency handled by introducing a loss-channel, and tracing over events in that 
          channel.  

          \item For pure-states undergoing continouus measurement , we can unravel the density matrix as an ensemble of pure states.  
          \begin{align}
            d|\psi\rangle = -\frac{i}{\hbar} H|\psi\rangle dt -\frac{\Gamma}{2}
            \left[\sigma^\dag\sigma-\langle \sigma^\dag\sigma\rangle\right]|\psi\rangle dt 
            +\bigg( \frac{\sigma e^{-i\vect{k}\cdot\vect{x}}|\psi\rangle}{\langle \sigma^\dag \sigma\rangle ^{1/2}}
            -1\bigg)dN_k,
          \end{align}
          where the Poisson process $dN_k$ has mean rate 
          \begin{equation}
            \dlangle dN_k\drangle = \Gamma \langle \sigma^\dag\sigma\rangle |u_i(\vect{k})|^2 dt
          \end{equation}
        \item Can also consider mixing signal with local oscillator.  Amplifies signal considerably.
          Different unravelling (get dipole phase information instead of excited/ground state info)

      \end{itemize}
      \subsection{Adiabatic Elimination}
      \begin{itemize}
        \item Adiabatically eliminate atoms internal state.  We assume that the 
          fluorescence occurs on a much faster time-scale than the atom's motion.  This 
          is the same logic used in developing the effective atomic-potential.  We assume that 
          there is a strong, far-off resonant field, and a weak resonant probe.  Far-off resonance
          field leads to confining potential
          \begin{equation}
            V_{\text{dipole}}(x) = \frac{\vect{\Omega}_d(\vect{x})\cdot\vect{\Omega}^*_d(\vect{x})}{\Delta}
          \end{equation}
          Formally, can find Heisenberg equations of motion, and solve in ``steady-state''.
          \begin{itemize}
            \item Input-output theory equations\cite{Gardiner1985, GardinerZoller2004}
              \begin{equation}
                da = -\frac{i}{\hbar}[a,H] -[a,c^\dag]\left(\frac{\gamma}{2}c + \sqrt{\gamma}b_{in}(t)\right)
                -\left(\frac{\gamma}{2}c^\dag + \sqrt{\gamma}b^\dag_{in}(t)\right)[a,c]
              \end{equation}
              where $c$ is the operator coupled to the bath, and $b_in$ in the input noise operator.  
              In rotating picture for $\sigma$
              \begin{equation}
                \dot{\sigma} = -i[\sigma,\omega\sigma_z/2+ \sigma^\dag \alpha]
                -[\sigma,\sigma^\dag]\left(\frac{\Gamma}{2}\sigma + \sqrt{\Gamma}b_{in}\right)
              \end{equation}
              Then use $[\sigma,\sigma^\dag]\sigma = -2\sigma_z\sigma = 2\sigma$
              \begin{equation}
                \dot{\sigma} = -(i\omega+\Gamma)\sigma +\sigma_z\alpha + \sigma_z\sqrt{\Gamma}b_{in}
              \end{equation}
              \comment{factors in defining Rabi frequency?}
            \item Also then need equation for $\sigma_z$
              \begin{align}
                \dot{\sigma}_z &= -i\left([\sigma_z, \sigma\alpha^*+\sigma^\dag\alpha   \right)
                -[\sigma_z,\sigma^\dag]\left(\frac{\Gamma}{2}\sigma + \sqrt{\Gamma}b_{in}\right)
                -\left(\frac{\Gamma}{2}\sigma^\dag + \sqrt{\Gamma}b^\dag_{in}\right)[\sigma_z,\sigma]\\
                &= i\left(2\sigma\alpha^*-2\sigma^\dag\alpha\right)
                +(-2)\sigma^\dag\left(\frac{\Gamma}{2}\sigma + \sqrt{\Gamma}b_{in}\right)
                -\left(\frac{\Gamma}{2}\sigma^\dag + \sqrt{\Gamma}b^\dag_{in}\right)2\sigma
              \end{align}
              Now have two coupled SDEs.  

              \begin{align}
                \dot{\sigma} &= -(i\omega+\Gamma)\sigma +\sigma_z\alpha + \sigma_z\sqrt{\Gamma}b_{in}\\
                \dot{\sigma}_z &= -\Gamma(\sigma_z+I)/2 +2i\left(\sigma\alpha^*-\sigma^\dag\alpha\right)
                -2\sqrt{\Gamma}(\sigma^\dag b_{in}  -b^\dag_{in}2\sigma)
              \end{align}
              
              Solve on average, in steady state. $\dot{f}=0$.
              \begin{align}
                \sigma = \frac{\alpha}{i\omega + \Gamma}
              \end{align}
              

          \end{itemize}
          

          More justified approach, to integrate equations of motion, and solve approximately
          Then find effective Hamiltonian that reproduces those effective equations of motion.
          \begin{equation}
            \sigma\rightarrow \frac{|\vect{\Omega}|}{\Gamma}.
          \end{equation}


      \end{itemize}
      \subsection{Effective Position Measurement}
      \begin{itemize}
        \item No-conditioning
          \begin{equation}
            d\rho_{CM} = -i[H_{\text{eff}},\rho_{CM}]dt + \Gamma \int d\Omega_k\dec[|\vect{\Omega} e^{i\vect{k\cdot x}}] dt
          \end{equation}
        \item Measurement operators are transformed copies of electric field emitted by atoms.
          That is combination of atomic dipole pattern, with spatially dependent emission rate.
          Electric field emitted by atoms is
          \begin{equation}
            \mu = e^{i\vect{k}\cdot\vect{x}}E(\vect{x})
          \end{equation}
          Transform electric field via propagation and lens system.  Electric field incident 
          on detector is a linear superposition of emitted fields.  Then apply transformation
          function on that linear sum.  That is the field at the detector.  
        \item Note, for resonance fluorescence, one integrates over all emission directions for 
          full set of emission operators at each position.  
          Consider lens system as a linear map on electric fields.  
          \begin{equation}
            \mu(x) = \int d\Omega_k\,e^{i\vect{k}\cdot\vect{x}}u_i(\vect{k})\alpha(\vect{x})
          \end{equation}
          


      \end{itemize}



      \section{Zeno Effect and Strong Measurements}

      \begin{itemize}
        \item Include figures/simulations showing reflection
        \item Discuss physics of two decay channels: reflection from re-emission into beams, inference from 
          detectin emitted photons.
        \item Similarity to stochastic potential.  
        \item No net force.  Probability of reflection might approach one, but effect on 
          state is larger, so $d\langle p\rangle = 0$
        \item Cite work commenting on our work.  Moving beyond purely perturbative approach.  
        \item Figure: Reflection as function of $\chi$
      \end{itemize}

      \section{Trajectories for EMCCD cameras}

      \begin{itemize}
        \item Handle noise, uncertainty by building model for measurement process
          including all relevant noise, loss mechanisms.  Build a full, pure 
          trajectory picture including random processes.
          Then average over unobservable processes.   Weight each trajectory
          with Bayesian statistics (re picture of evolution of quantum state). 
        \item Warzsawski and Wiseman modelling 2-level atom emitting light onto 
          APD.  Include dark counts, down-time, finite collection efficiency.
      \end{itemize}
      

      \subsection{EMCCD Amplification and Noise Processes}

      \begin{itemize}
        \item Review of useful concepts from Jeremy Thorne's thesis.  
        \item Camera adds clock-induced charge at given Poissonian rate.
        \item Amplification process exponentially stretches out $n$ photons.
        \item Then Gaussian read-out noise.  
      \end{itemize}

      \begin{itemize}
          \item In order to build quantum trajectories based on this, exploit Bayesian statistics
          to weight all trajectories.  
          \begin{equation}
            |\psi\rangle = \sum_{\text{traj}} P(\text{traj}|\text{outcome})|\psi_{\text{traj}}\rangle.
          \end{equation}
          Find the weighting probability via Bayesian statistics
          \begin{equation}
            P(A|B) = \frac{P(B|A)P(A)}{\sum_AP(B|A)P(A)}
          \end{equation}
          \item Furthermore, must trace over unobserved quantities, such as clock-induced 
          charge, spontaneous emission into free-space.  
      \end{itemize}

      

%%% Local Variables: 
%%% mode: latex
%%% TeX-master: "thesis_master"
%%% End: 
