\chapter{``Realistic'' Quantum Trajectories for Position Measurements}
\label{ch:trajectory}

\section{Continuous Quantum Measurements}

Describe open quantum systems, and include information from continuous measurements. 

\begin{enumerate}
\item Quantum measurements are inherently probabilitistic.  
\item Thus, sequences of measurements are also inherently random.  
\item Greater experimental control means we can now probe isolated quantum systems repeatedly and see how they evolve.  
\item Move beyond simple projective measurements, which cannot describe things like position measurements,
    or absorbtive measurements like photon detection.  Once photon is measured it is destroyed.
\item To describe continuously weakly probing a system, also use stochastic processes as part of
    numerical simulation strategy.
\item Quantum tajectories correspond to monitoring a quantum system (like an atom) via a probe,
 for example shining light on the atom. \cite{Carmichael1993}
\item Simulated trajectories are possible trajectories a system could trace out as we observe.
  Also given a particular measurement record, they correspond to our best estimate of the current state of the system.  
\end{enumerate}

\begin{enumerate}
\item Cite Carmichael Rice JOSA paper~\cite{Carmichael1989}
\item Carmichael 1991 ~\cite{Carmichael1991}
\item Open Systems approach to Quantum Optics\cite{Carmichael1993}
\begin{enumerate}
  \item Motivated by photodetection, and modelling experiments developed a new approach to open systems.  
  \item Sample trajectories then correspond to actually results.
  \item Naturally fits Bayesian framework for interpretation of quantum state.  
\end{enumerate}

\item Example: Monitoring State of Two level atom.
\begin{enumerate}
  \item Hamiltonian and operators
  \item  Stochastic Schr\"odinger equation
    \begin{equation}
      d|\psi\rangle = -\frac{i}{\hbar}\op{H}|\psi\rangle dt 
      - \frac{\Gamma}{2}(\sigma^\dag\sigma-\langle \sigma^\dag\sigma\rangle)|\psi\rangle dt
      + \left(\frac{\sigma|\psi\rangle}{\langle \sigma^\dag\sigma\rangle^{1/2}}-|\psi\rangle \right)dN
      \end{equation}
    where $dN$ is Poisson process which is zero or one, with mean rate
    \begin{equation}
      \dlangle dN\drangle = \Gamma \langle \sigma^\dag\sigma\rangle dt.
    \end{equation}
    If Poisson process fires then detected emitted photon, which implies atom emitted photon, and should 
    apply lowering operator.  If no photon detected, then less likely to be in excited state.  
    Not only possible unravelling.  Can do homodyne detection - interfere light from atom
    with local oscillator.  Can extract phase info about atomic dipole.  Often necessary since direct
    detection typically has weak gain.  Interference effect amplifies the signal to detectable levels.
    However note EMCCD cameras can resolve/detect single photons - will form core of that part of thesis.
    Aspect of contextuality - type of measurement apparatus determines what can be inferred about the 
    state of the system.
 \item   Note that using $\dlangle\cdots\drangle$ for ensemble average over stochastic processes,
    and $\langle A\rangle:=\tr[A\rho]=\sum_i p_i \langle \psi_i|A|\rangle\psi_i\rangle$ 
    for quantum mechnical average.  
    \item Interpret photon detection or non-detection as gaining knowledge about state of system.
  \item Can get regular master equation by finding stochastic master equation, and tracing over noise.
  \item Crucial to use stochastic calculus to consistently work to $\order(dt)$.
  \item Path integral interpretation - master equation evolution is given by path integral, sum over all
    paths.  Monte-Carlo sampling of paths given quantum trajectories. 
    Note this was pointed out by Howard.  
\end{enumerate}
    
\item Cite Marte, Zoller, Parkins, Gardiner (MCWF)  \cite{Dalibard1992,Dum1992,Gardiner1992}
\item Cite Holland ~\cite{Holland1996}, Meystre~\cite{Greenwood1997}.
  Applied to position measurements of atoms by detecting photons.
  Detection of photons localizes atoms.  
\item Control Theory.~\cite{Wiseman1993} 
\begin{enumerate}
    \item Cite Wiseman book
   \item Jacobs and Doherty.
  \item Continuously monitoring system to implement closed-loop feedback control.  
  \item Feedback control to generate Fock states (Haroche group)
\end{enumerate}
\item Quantum Chaos
\begin{enumerate}
  \item Idea of exploring quantum-classical transition.
  Strong measurement is more classical.
  Can extract Lyupanov exponents for diverging trajectories.
  \cite{Bhattacharya2000,Habib2002,Habib2006}
  \cite{Scott2001}
  \item Theoretically shown.  As yet, this transition  not yet observed.  Need to develop
    theory in direction of more carefully modelling experiments.  
\end{enumerate}
\item Advantages:
\begin{enumerate}
  \item Computationally efficient as simulating wave functions.
  Take ensemble average at the end to get density matrix.  
  \item Natural form for feedback control and reconstructing trajectory.  
\end{enumerate}
\item Comment: Relationship of measurement with state and process tomography?  Any?  

\item Warshawski and Wiseman.
  Can describe additional uncertainty by including classical Bayesian probabilities for each indistinguishable trajectory.
  Note thesis of J. Thorne describing model for EMCCD camera.
  Given number for each pixel have probabilities for each number of photons.
  Must then incoherently average over all possible detection histories consistent with measured record.
  \cite{Warszawski2002,Warszawski2003a,Warszawski2003b}
\item Also cover results for generalized measurement functions and 
    somewhat surprising notion that particle reflects from a sufficiently
    strong quantum measurement.~\cite{Mackrory2010}
\end{enumerate}

\section{Position Dependent measurement operators}

\subsection{Initial Setup}
\begin{enumerate}
  \item Atom emitting light.  Assume dipole transition.
  \item Assume atom driven by resonant probe beam, while trapped in far off-resonant
    dipole trap.  This aligns atomic dipoles in particular direction.  
  \item Atomic Hamiltonian
    \begin{equation}
      H = \frac{\vect{p}^2}{2m}+\sum\hbar \omega\sigma_z
      -\sum_i\vect{d}\cdot\vect{E}_i(\vect{x},t)
    \end{equation}
    center-of-mass motion, internal energy levels, and interaction with external fields.  
  \item Operators defined as 
    \begin{gather}
      \sigma = |g\rangle\langle e|\\
      \sigma^\dag = |e\rangle\langle g|\\
      \sigma_z = |e\rangle\langle e|-|g\rangle\langle g|
    \end{gather}
  \item Commutation relations
    \begin{gather}
      [\sigma^\dag,\sigma] 
      % = [|g\rangle \langle e|, |e\rangle \langle g|]
      = \sigma_z\\
      [\sigma, \sigma_z]   = 2\sigma\\
      [\sigma^\dag, \sigma_z]   = -2\sigma
    \end{gather}
  \item Consider spontaneous emission.  
  \item Stochastic Master Equation (approach heavily influenced by Quantum Optics notes).
    No conditioning
    \begin{equation}
      d\rho = -\frac{i}{\hbar}[H,\rho]dt + \frac{\Gamma}{2} \dec[\sigma e^{i\vect{k\cdot x}}] dt,
    \end{equation}
    where decoherence super-operator is 
    \begin{equation}
      \dec[A]\rho = 2A\rho A^\dag - A^\dag A\rho - \rho A^\dag A.
    \end{equation}
    Density matrix $\rho$ for full center of mass, and internal degrees of freedom.  
  \item Stochastic increment $df = f(t+dt)-f(t)$.
  \item For an experiment monitoring the flourescence via angle-resolved emission,
    we can model the atom's evolution via the following stochastic Master equation (SME)
    \begin{align}
      d\rho =& -\frac{i}{\hbar}[H,\rho]dt - \frac{\Gamma}{2}\hom[\sigma^\dag\sigma]\rho dt
      + \int d\Omega_k \jump[\sigma e^{i\vect{k}\cdot\vect{x}}]\rho dN_k
    \end{align}
    where homodyne and jump operators are 
    \begin{align}
      \hom[A]\rho = A\rho + \rho A^\dag - \tr[\rho(A+A^\dag)]\rho\\
      \jump[A]\rho = \frac{ A\rho A^\dag}{\tr[A\rho A^\dag]} - \rho.
    \end{align}
    The Poisson increments $dN_k$ have mean rate
    \begin{equation}
      dN_k = \Gamma\Tr[\sigma^\dag\sigma\rho] dt.
    \end{equation}
    If we account for the dipole emission pattern, assuming the atom's have their dipole preferentially
    aligned via the effective dipole potential, then 
    \begin{equation}
      dN_{k,i} = \Gamma\Tr[\sigma^\dag\sigma\rho]|u_i(\vect{k})|^2 dt.
    \end{equation}
    where $u_i(\vect{k})$ is the dipole emission pattern at wave-vector $\vect{k}$.  Normalized
    so that $\sum_{i=0,\pm}\int d\Omega_k |u_i|^2 = 1$.

  \item Inefficiency handled by introducing a loss-channel, and tracing over events in that 
    channel.  

  \item For pure-states undergoing continouus measurement , we can unravel the density matrix as an ensemble of pure states.  
    \begin{align}
      d|\psi\rangle = -\frac{i}{\hbar} H|\psi\rangle dt -\frac{\Gamma}{2}
      \left[\sigma^\dag\sigma-\langle \sigma^\dag\sigma\rangle\right]|\psi\rangle dt 
      +\bigg( \frac{\sigma e^{-i\vect{k}\cdot\vect{x}}|\psi\rangle}{\langle \sigma^\dag \sigma\rangle ^{1/2}}
      -1\bigg)dN_k,
    \end{align}
    where the Poisson process $dN_k$ has mean rate 
    \begin{equation}
      \dlangle dN_k\drangle = \Gamma \langle \sigma^\dag\sigma\rangle |u_i(\vect{k})|^2 dt
    \end{equation}
  \item Can also consider mixing signal with local oscillator.  Amplifies signal considerably.
    Different unravelling (get dipole phase information instead of excited/ground state info)

\end{enumerate}
\subsection{Adiabatic Elimination}
\begin{enumerate}
  \item Adiabatically eliminate atoms internal state.  We assume that the 
    fluorescence occurs on a much faster time-scale than the atom's motion.  This 
    is the same logic used in developing the effective atomic-potential.  We assume that 
    there is a strong, far-off resonant field, and a weak resonant probe.  Far-off resonance
    field leads to confining potential
    \begin{equation}
      V_{\text{dipole}}(x) = \frac{\vect{\Omega}_d(\vect{x})\cdot\vect{\Omega}^*_d(\vect{x})}{\Delta}
    \end{equation}
    Formally, can find Heisenberg equations of motion, and solve in ``steady-state''.
    \begin{enumerate}
      \item Input-output theory equations\cite{Gardiner1985, GardinerZoller2004}
        \begin{equation}
          da = -\frac{i}{\hbar}[a,H] -[a,c^\dag]\left(\frac{\gamma}{2}c + \sqrt{\gamma}b_{in}(t)\right)
          -\left(\frac{\gamma}{2}c^\dag + \sqrt{\gamma}b^\dag_{in}(t)\right)[a,c]
        \end{equation}
        where $c$ is the operator coupled to the bath, and $b_in$ in the input noise operator.  
        In rotating picture for $\sigma$
        \begin{equation}
          \dot{\sigma} = -i[\sigma,\omega\sigma_z/2+ \sigma^\dag \alpha]
          -[\sigma,\sigma^\dag]\left(\frac{\Gamma}{2}\sigma + \sqrt{\Gamma}b_{in}\right)
        \end{equation}
        Then use $[\sigma,\sigma^\dag]\sigma = -2\sigma_z\sigma = 2\sigma$
        \begin{equation}
          \dot{\sigma} = -(i\omega+\Gamma)\sigma +\sigma_z\alpha + \sigma_z\sqrt{\Gamma}b_{in}
        \end{equation}
        \comment{factors in defining Rabi frequency?}
      \item Also then need equation for $\sigma_z$
        \begin{align}
          \dot{\sigma}_z &= -i\left([\sigma_z, \sigma\alpha^*+\sigma^\dag\alpha   \right)
          -[\sigma_z,\sigma^\dag]\left(\frac{\Gamma}{2}\sigma + \sqrt{\Gamma}b_{in}\right)
          -\left(\frac{\Gamma}{2}\sigma^\dag + \sqrt{\Gamma}b^\dag_{in}\right)[\sigma_z,\sigma]\\
          &= i\left(2\sigma\alpha^*-2\sigma^\dag\alpha\right)
          +(-2)\sigma^\dag\left(\frac{\Gamma}{2}\sigma + \sqrt{\Gamma}b_{in}\right)
          -\left(\frac{\Gamma}{2}\sigma^\dag + \sqrt{\Gamma}b^\dag_{in}\right)2\sigma
        \end{align}
        Now have two coupled SDEs.  

        \begin{align}
          \dot{\sigma} &= -(i\omega+\Gamma)\sigma +\sigma_z\alpha + \sigma_z\sqrt{\Gamma}b_{in}\\
          \dot{\sigma}_z &= -\Gamma(\sigma_z+I)/2 +2i\left(\sigma\alpha^*-\sigma^\dag\alpha\right)
          -2\sqrt{\Gamma}(\sigma^\dag b_{in}  -b^\dag_{in}2\sigma)
        \end{align}
        
        Solve on average, in steady state. $\dot{f}=0$.
        \begin{align}
          \sigma = \frac{\alpha}{i\omega + \Gamma}
        \end{align}
        

    \end{enumerate}
    

    More justified approach, to integrate equations of motion, and solve approximately
    Then find effective Hamiltonian that reproduces those effective equations of motion.
    \begin{equation}
      \sigma\rightarrow \frac{|\vect{\Omega}|}{\Gamma}.
    \end{equation}


\end{enumerate}
\subsection{Effective Position Measurement}
\begin{enumerate}
  \item No-conditioning
    \begin{equation}
      d\rho_{CM} = -i[H_{\text{eff}},\rho_{CM}]dt + \Gamma \int d\Omega_k\dec[|\vect{\Omega} e^{i\vect{k\cdot x}}] dt
    \end{equation}
  \item Measurement operators are transformed copies of electric field emitted by atoms.
    That is combination of atomic dipole pattern, with spatially dependent emission rate.
    Electric field emitted by atoms is
    \begin{equation}
      \mu = e^{i\vect{k}\cdot\vect{x}}E(\vect{x})
    \end{equation}
    Transform electric field via propagation and lens system.  Electric field incident 
    on detector is a linear superposition of emitted fields.  Then apply transformation
    function on that linear sum.  That is the field at the detector.  
  \item Note, for resonance fluorescence, one integrates over all emission directions for 
    full set of emission operators at each position.  
    Consider lens system as a linear map on electric fields.  
    \begin{equation}
      \mu(x) = \int d\Omega_k\,e^{i\vect{k}\cdot\vect{x}}u_i(\vect{k})\alpha(\vect{x})
    \end{equation}
    


\end{enumerate}



\section{Zeno Effect and Strong Measurements}

\begin{enumerate}
  \item Include figures/simulations showing reflection
  \item Discuss physics of two decay channels: reflection from re-emission into beams, inference from 
    detectin emitted photons.
  \item Similarity to stochastic potential.  
  \item No net force.  Probability of reflection might approach one, but effect on 
    state is larger, so $d\langle p\rangle = 0$
  \item Cite work commenting on our work.  Moving beyond purely perturbative approach.  
  \item Figure: Reflection as function of $\chi$
\end{enumerate}

\section{Trajectories for EMCCD cameras}

\begin{enumerate}
  \item Handle noise, uncertainty by building model for measurement process
    including all relevant noise, loss mechanisms.  Build a full, pure 
    trajectory picture including random processes.
    Then average over unobservable processes.   Weight each trajectory
    with Bayesian statistics (re picture of evolution of quantum state). 
  \item Warzsawski and Wiseman modelling 2-level atom emitting light onto 
    APD.  Include dark counts, down-time, finite collection efficiency.
\end{enumerate}


\subsection{EMCCD Amplification and Noise Processes}

\begin{enumerate}
  \item Review of useful concepts from Jeremy Thorne's thesis.  
  \item Camera adds clock-induced charge at given Poissonian rate.
  \item Amplification process exponentially stretches out $n$ photons.
  \item Then Gaussian read-out noise.  
\end{enumerate}

\begin{enumerate}
  \item In order to build quantum trajectories based on this, exploit Bayesian statistics
    to weight all trajectories.  
    \begin{equation}
      |\psi\rangle = \sum_{\text{traj}} P(\text{traj}|\text{outcome})|\psi_{\text{traj}}\rangle.
    \end{equation}
    Find the weighting probability via Bayesian statistics
    \begin{equation}
      P(A|B) = \frac{P(B|A)P(A)}{\sum_AP(B|A)P(A)}
    \end{equation}
  \item Furthermore, must trace over unobserved quantities, such as clock-induced 
    charge, spontaneous emission into free-space.  
\end{enumerate}


\subsection{Raw Quantum trajectories}



%%% Local Variables: 
%%% mode: latex
%%% TeX-master: "thesis_master"
%%% End: 
