\chapter{Literature Review}

This is a collection of notes summarizing important papers.  I will give a condensed version of this in the introduction.  

\section{Casimir Books}

Goal: List and reference important equations/chapters.  
I need to figure out the appropriate way to cite chapters/equations/pages within a thesis.  

\subsection{Milonni: Quantum Vacuum}

Milonni~\cite{Milonnibook1994}

\subsection{Milton: Casimir Effect}
Milton~\cite{Miltonbook2001}

\subsection{Bordag}
Bordag~\cite{Bordagbook2009}

\subsection{Dalvit: Casimir Effect}
Dalvit~\cite{Dalvitbook2011}

\subsection{Parsegian: van der Waals forces}

Parsegian~\cite{Parsegian2006}



\subsection{Israelachivili:Molecular Forces}

Israelachivili~\cite{Israelachvili2011}


\section{Field Theory Books}

\subsection{Brown Quantum Field Theory}

\cite{Brown1994}
Path integrals and stat mech

Field theory.  

Field theory Renormalization

\subsection{Peskin and Schroeder}
\cite{Peskin1995}

\subsection{Srednicki: Quantum Field Theory}
\cite{Srednicki2008}

\subsection{Altland and Simons: Condensed Matter Field Theory}
\cite{Altland2011}

Linear Response

Effective theory

\subsection{Abrikosov:Condensed matter field theory}

\cite{Abrikosov1975}
Thermal field theory

Feynman-Diagram approach to Casimir.


\section{Stochastic Books}

\subsection{Oksendahl}

\subsection{Durrett}

\subsection{Gardiner}



\section{Early Papers}

\subsection{Casimir48}


\cite{Casimir1948}

\subsection{CasimirPolder1948}


~\cite{CasimirPolder1948}.  

\subsection{Lifshitz1956}

~\cite{Lifshitz1956}.  

\subsection{Dzyaloshinkii1959}

\cite{Dzyaloshinskii1959}
\subsection{Dzyaloshinksii1961}

\cite{Dzyaloshinskii1961}

\subsection{Mclachlan1963}

\cite{Mclachlan1963}

\section{Experimental Papers:Casimir}

\subsection{Lamoreaux1997}

\cite{Lamoreaux1997}
\subsection{Sushkov2011}

\cite{Sushkov2011}
\subsection{Mohideen1998}
\cite{Mohideen1998} 

\subsection{Chan2001}

\cite{Chan2001}
\subsection{Bressi2002}

\cite{Bressi2002}

\section{Experimental Papers:Casimir-Polder}
\subsection{Harber2005: Cornell BEC}

\cite{Harber2005}
\subsection{Obrecht2007:Cornell BEC}
\cite{Obrecht2007}

\subsection{Antezza chapter}
\cite{Dalvitbook2011}

\subsection{Alton2011:Atoms-Toroid}

\cite{Alton2011}. 

\subsection{Hung2013:Atom-Microcavity}
 Atoms above 1D Microcavity \cite{Hung2013}


\subsection{Atom-Chips:Folman2000}

\subsection{Atom-Chips:Schneider2003}

\cite{Folman2000,Schneider2003}

\subsection{Cronin-Beams}

\cite{Perreault2005,Lonij2009}

\subsection{Sukenik:Atoms through Cavity}

\cite{Sukenik1993}

\subsection{Experimental arguments}

\begin{itemize}
\item Controversies about role of zero temperature pole.  
\item Lamoreaux favors Drude model, Capasso favours plasma model.
\item Seems experiments favour more 
\end{itemize}

\subsection{Geckos: Autumn2002}
\cite{Autumn2002}

\cite{Hawkes2014}

\subsection{Modifications to gravity}

\begin{itemize}
\item Modifications to gravity on $1\mu m$ or $1mm$ scale.  Cite Lamoreaux 2000 Paper.  Gervaci?
Yukawa type forces.  
\item Subtract off Casimir force background.  Tino group.  Use Casimir shield with fairly thick gold to have same Casimir force, and then vary the medium behind it.  Longer range gravity should lead to Requires very careful measurements, on top of carefully extracting Casimir force.   
\end{itemize}

\subsection{Papers: Proximity Force Approximation}

\begin{itemize}
\item Find first use?  Lamoreaux mentions usage.  Derjaguin?\cite{Derjaguin1956} \cite{Blocki1977}
\item Note problem with non-additivity. 
\item Good as order of magnitude estimate?  Useful if very limited curvature, or effectively approximate geometry as planar.  
\end{itemize}

\subsection{Schwinger: Green function methods 1978}

\cite{Schwinger1978, Milton1978}

Scalar green function methods.  Milton book.  
Planes

\subsection{Milton1978}
Spheres

\begin{itemize}
\item Green tensor methods
\item Cite Barton
\item Cite Philbin(?)
\item Cite Vogel and Welsch
\end{itemize}

\section{Papers: Theory- Reflection Matrix}

\subsection{Balian and Duplantier}
\cite{Balian1977} \cite{Balian1978}
\subsection{Lambrecht/MaiaNeto}

\cite{Lambrecht2006}
\cite{MaiaNeto2008}
\cite{Canaguier-Durand2012}

\subsection{Scattering Matrix Path Integral Methods}

\subsection{Stratton}
Surface integral equations (Stratton-chu)
\cite{Stratton1941}

\subsection{Emig/Buscher}

Paper on deformations showing you can use homogenous green functions within bounding surface.  

\subsection{Emig/Rahi}

Multipole expansion
\cite{Buscher2004}
\cite{Emig2004, Emig2007, Rahi2009}

\subsection{Kenneth/Klich}
\cite{Kenneth2006}
\cite{Kenneth2008}

\subsection{Reid/Johnson/Rodriguez}

Early numerical papers.
\cite{Rodriguez2007},\cite{Rodriguez2007a}, \cite{Rodriguez2009}.  Note use of existent analytical methods and similarities to existent numerical FTDT techniques Builds on earlier papers (uses better basis) 

Cite Johnson textbook.\cite{Johnson2011}

\subsection{Reid papers}

\cite{Reid2009},\cite{Reid2011}, \cite{Reid2013} 
Note success, applicability.  \comment{Cite experimental tylenol pill paper}

\section{Repulsion}

\begin{itemize}
\item Cite Milton paper on anisotropy \cite{Milton2012, Milton2012a}
\item Cite work on metamaterials (hydrogen mirror)
\item Cite Milonni/rosa showing broad-band \cite{Rosa2010}
\end{itemize}

\section{Papers: Worldlines}

Note citations in bordag/johnson mostly as dismissive and limited  

\subsection{Effective actions}

\cite{McKeon1993, Strassler1992,Schubert2001}

\comment{Other references - was one contemporaneous with Strassler? Bern-Kosower}

\comment{Cite 1950 Feynman Scalar QED section}

\cite{Schubert2001}.  

\subsubsection{Applied to QED}

For example, the worldline method has been used to compute relativistic field effects for QED such as the Lamb shift~\cite{Schmidt1995}.  It has also been used as a numerical algorithm\cite{Mazur2014}.

\subsection{Feynman Path Integral for QM}

\cite{Feynman1948,Feynman1965,Brown2005}.

\subsection{Papers: Gies Worldline}

\subsubsection{Gies 2003}
\cite{Gies2003}

\begin{itemize}
\item Cite earlier paper by themselves for first worldline numerics?   Found citations for PFA.
\item Action for scalar fields.   Massive scalar field interacting with some potential $V(x)$.  $V$ has dimensions of mass-squared.  (Field theory units $\hbar=c=1$, so $E=mc^2$)
\begin{equation}
\cL = \frac{1}{2}\partial_\mu\phi\partial^\mu\phi +\frac{1}{2}m^2\phi^2+\frac{1}{2}V(x)\phi^2
\end{equation}
\item Complete unrenormalized quantum effective action for $V$ (integrate out fields $\phi$) is 
\begin{align}
  \Gamma[V] &=\frac{1}{2}\tr\ln \left[ \frac{-\partial_\mu\partial^\mu +m^2 + V(x)}{-\partial_\mu\partial^\mu +m^2}\right]\\
&   =-\frac{1}{2}\int \frac{d\cT}{\cT} \int d^D x \left\{ \langle x|e^{-\cT[-\partial_\mu\partial^\mu +m^2 + V(x)]}|x\rangle -\frac{e^{-m^2 \cT}}{(4\pi \cT)^{D/2}}\right\}
\end{align}
Working in $D=d+1$ euclidean space-time dimensions.  
\item Convert matrix element to path integral.  
\begin{equation}
  \int d^D x \langle x|e^{-\cT[-\partial^2+V(x)]}|x\rangle = \int d^D x_{\text{CM}} \mathcal{N}\int_{x(0)=x(\cT)}Dx e^{-\int_0^\cT d\tau \dot{x}^2/4}
\end{equation}
Comments: Fix normalization from limit of zero potential (which is evaluated in momentum basis), and note that choice to drop factors of 2 has left them with an extra $\sqrt{2}$ in the definition  of the loops.
\begin{align}
\langle x| e^{\partial^2T}|x\rangle &= \int d^Dp\frac{1}{(2\pi)^{D/2}} e^{-p^2T}\langle x|p\rangle\langle p|x\rangle\\
&= \frac{1}{(4\pi \cT)^{D/2}}\\
&=\cN \int_{x(0)=x(\cT)}Dx \exp\left\{-\int_0^\cT d\tau \left[\frac{\dot{x}^2}{4}+V(x_{\text{CM}}+x(\tau)\right]\right\}
\end{align}
\item Now interpret path integral as Gaussian integral/ensemble average over closed loops.  
\begin{equation}
\cN \int_{x(0)=x(\cT)}Dx \exp\left\{-\int_0^\cT d\tau \left[\frac{\dot{x}^2}{4}+V(x_{\text{CM}}+x(\tau)\right]\right\} = \frac{1}{(4\pi \cT)^{D/2}}\dlangle e^{-\int_0^\cT d\tau V[x_{\text{CM}}+x(\tau)]}\drangle
\end{equation}
Also use scaled loops: 
\begin{equation}
x_\mu(\cT t) = \sqrt{\cT}y_\mu(t),
\end{equation}
where $t\in [0,1]$, and we view $\cT$ as a parameter.  
\item Plugging in path representation
\begin{align}
\Gamma[V]=&-\frac{1}{2 (4\pi)^{D/2}}\int_{1/\Lambda^2}^\infty \frac{d\cT}{\cT^{1+D/2}} e^{-m^2\cT}\int d^Dx_{\text{CM}} \\
&\times\dlangle e^{-\cT\int_0^1dt V[x_\text{CM} +\sqrt{\cT}y(t)]}-1 \drangle
\end{align}
For time independent backgrounds use $\int dx_0 = L_{x_0}$ as ``length'' in time direction.  \comment{For partition function version Get $\beta\hbar c$ instead.}
\item \textbf{Renormalization}
Consider field theoretic renormalization (I think whereby an analogue of the Integral Trajectory Matching formalism fixes the unknown(divergent) values at some known scale - typically from a low energy experiment).  
\item Heat Kernel Expansion \comment{Cite Gilkey} .  Expand to non-diverging order to regularize UV divergences.  Each power of $V$ amounts to an external leg.  Get a Tadpole $T^{-D/2}\int dx_0 V$, and 
\item Ignore self-energies and focus on interaction energies (which are insentitive to Field theoretic divergence)
\begin{equation}
E_\text{int}= E_{V_1+V_2+\cdots}- E_{V_1} - E_{V_2}-\cdots
\end{equation}
Can carry out renormalization at loop level and thus avoid divergent sums.  
\item They note that this emphasis on interaction energy is \emph{not} a renormalization procedure.  It circumvents that for multiple bodies.  However for isolated bodies (sphere self-stress), one must specify all of the renormalization conditions, and your answers may depend on the procedure used.
\item \textbf{Numerics}
Introduce some methods (most of which are useless).
Important method is the v-loop.  An exact diagonalization of the Gaussians, under the loop closure constraint.  Note that they choose their loop constraint to be $\int d\tau y_\mu(\tau) =0$.  This amounts to making the center of mass of the loop centered on $x_\text{CM}$.  In contrast, our work has typically used $x(0)=x(\cT)=0$.  

The ``v-loop'' algorithm amounts to considering the probability measure,
\begin{equation}
P(\{x_k\}) =  \delta(x_N-x_0)\prod_{j=0}^{N}e^{-\frac{(x_{j+1}-x_j)^2}{2\Delta T}},
\end{equation}
and completing the square in the exponents to decouple the Gaussians.  
Note their version uses $\sum_{j}x_j=0$ to fix $y_N$.  
The Jacobian can be shown to unity, since the matrix is upper diagonal, with unit diagonal.  
\item \textbf{Tests}
\item Compare to parallel plate energy to get convergence. 
\item Give some consideration to finite $\lambda$ in potential.  
\begin{equation}
V(x) = \lambda\int_{\Sigma}d\sigma \delta(x-x\sigma)
\end{equation}
where $\Sigma$ is a $d-1$ dimensional surface, $\sigma$ is a reparameterization invariant measure and $x_\sigma$ points to the surface.  
\item For interaction energies, need $(e^{-V_{1+2}} - 1) -  (e^{-V_1}-1) - (e^{-V_{2}}-1)$.  If only touch one surface (of disconnected bodies) then 
no contribution, since $V_{1+2}=V_1$
\item Then consider massless scalar between sphere/plate and cylinder plate.  
\item Consider breakdown of PFA as function of separation $a$ to sphere radius $R$.  
\item \textbf{Conclusions}
\item Needs no underlying symmetry.  Precision hinges on loop parameters chosen (and presumably discretization of surfaces)
\item Claim their delta potentials are ``hard''.
\item Note finite temperature and roughness required.  
\item Note numerical differentiation to get force is hard, but derivative can be done right off the bat.  (And energy-momentum tensors)
\item Cite Feinberg and Sucher for possible path for EM quantization.  
\end{itemize}

\subsubsection{Gies2001}
\cite{Gies2001}

Use worldlines to compute EM effective action in specified background EM field, modelling interaction with scalar particle.  

\subsubsection{Gies 2006}
\cite{Gies2006}

Very similar?  Advances?  
D-loops?  Note similarity to methods of generating brownian walks by doubling intermediate points to imrpvoe resolution where required.  

\subsubsection{Gies 2006 a}
\cite{Gies2006a}

\subsection{Papers: Thermal worldlines}

\subsubsection{Geothermal2008:Klingm\"uller}
~\cite{Klingmueller2008}.  
\subsubsection{Weber2009:Inclined planes}
~\cite{Weber2009}
\subsubsection{Weber2009:Interplay}
~\cite{Weber2010}
\subsubsection{Weber2009:Spheres}
\cite{Weber2010a}.  

\subsubsection{Schaden:Pistons}

Cite Schaden applying to pistons\cite{Schaden2009}

Why?  What for?  Benefits?

\subsection{Nearby attempts}

\cite{Aehlig2011}

\cite{Maggs2006, Pasquali2008}

\section{Papers:Stochastic Methods}

\subsection{Gradients}

\subsection{Feynman-Kac}

\subsubsection{Hooghiemstra}



\section{Papers:Dielectric Quantization}



\subsection{Vogel and Welsch}


%\cite{Dung1998}
\cite{Raabe2006}
\cite{Raabe2007}

\subsection{Huttner/Barnett}

\cite{Glauber1991}

\cite{Huttner1992}

\cite{Matloob1995}
\cite{Matloob1996}


\subsection{Philbin}

\cite{Philbin2010}
\cite{Philbin2011}

\cite{Drummond2014}

\subsection{Bordag}
\cite{Bordag1998} \cite{Bordag1999}

\subsection{Bechler}

\cite{Bechler1999}
\cite{Bechler2006}




\section{Quantum Trajectories}

\subsection{Carmichael}
Cite Carmichael Rice JOSA paper.

Cite Carmichael 1993 lectures. \cite{Carmichael1993}

\subsection{Others}

Cite Marte
Cite Parkins  
Cite Gardiner \cite{Gardiner1992}
Cite Marte, Zoller, Parkins, Gardiner (MCWF) \cite{Dalibard1992}
\cite{Dum1992}

\subsection{Path integral version}


\subsection{Measuring atom positions}

 Cite Holland, Meystre.  Applied to position measurements of atoms by detecting photons.  Detection of photons localizes atoms.  
\cite{Holland1996}


\subsection{Control theory}
 Control Theory.  Cite Wiseman book.   \cite{WisemanMilburn2010}
 
Cite Steck feedback control paper.  \cite{Steck2004, Steck2006}

\subsection{Quantum chaos}

Cite Bhattacharya quantum paper.  \cite{Bhattacharya2005}

\subsection{Noisy measurements}
Cite Warshawski/Wiseman \cite{Warszawski2003}

Cite Jeremy Thorne.  


%%% Local Variables: 
%%% mode: latex
%%% TeX-master: "thesis_master"
%%% End: 
