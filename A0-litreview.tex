\chapter{Literature Review}

This is a collection of notes summarizing important papers.  I will give a condensed version of this in the introduction.  

\section{Casimir Books}

Goal: List and reference important equations/chapters.  
I need to figure out the appropriate way to cite chapters/equations/pages within a thesis.  

\subsection{Milonni: Quantum Vacuum}

Milonni~\cite{Milonni1994}

\subsection{Milton: Casimir Effect}
Milton~\cite{Milton2001}

\subsection{Bordag}
Bordag~\cite{Bordag2009}

\subsection{Dalvit: Casimir Effect}
Dalvit~\cite{Dalvit2011}

\subsection{Parsegian: van der Waals forces}

Parsegian~\cite{Parsegian2006}.
Covers van der Waals forces at varying levels of detail.  Finally gets to Green function
point of view for various geometries.  Published prior to major theory breakthroughs.  

\subsection{Israelachivili:Molecular Forces}

Israelachivili~\cite{Israelachvili2011}.
Primarily interested in chemistry effects (and other effects that contribute in addition to 
van der Waals)

\section{Field Theory Books}

\subsection{Brown Quantum Field Theory}

\cite{Brown1994}
Path integrals and stat mech

Field theory.  

Field theory Renormalization

\subsection{Peskin and Schroeder}
\cite{Peskin1995}

\subsection{Srednicki: Quantum Field Theory}
\cite{Srednicki2008}

\subsection{Altland and Simons: Condensed Matter Field Theory}
\cite{Altland2011}

Linear Response

Effective theory

\subsection{Abrikosov:Condensed matter field theory}

\cite{Abrikosov1975}
Thermal field theory

Feynman-Diagram approach to Casimir.
This is tightly related to the version published in papers.  In particular using the 
field theoretic Green function approach.  

\section{Stochastic Books}

\subsection{Oksendahl}

\subsection{Durrett}

\subsection{Gardiner}



\section{Early Papers}

\subsection{Casimir48}


\cite{Casimir1948}

\subsection{CasimirPolder1948}


~\cite{CasimirPolder1948}.  

\subsection{Lifshitz1956}

~\cite{Lifshitz1956}.  

\subsection{Dzyaloshinkii1959}

\cite{Dzyaloshinskii1959}
\subsection{Dzyaloshinksii1961}

\cite{Dzyaloshinskii1961}

\subsection{BarashGinzburg1975}

\cite{Barash1975}
Also comments that resuls derived under limited assumptions also work at finite temperature,
 and with dissipation.  

Introduce Argument principle: $D(\vect{r},\omega)= 1-r_1r_2 e^{-2 r l}$ where $r=\sqrt{k_x^2+k_y^2}$.  
Can write sum over energies as $\sum_n \hbar\omega_n$,
 as the integral against an integral in the complex plane whose poles are at integer spacings.  
In this case, (Eq 6) there are solutions when $D=0$.  So 
\begin{equation}
E=\sum_{\alpha}\frac{\hbar\omega_\alpha}{2}=
-\frac{-1}{2\pi i} \int \frac{dk_xdk_y}{(2\pi)^2}\int_{-i\infty}^{i\infty}d\omega
 \frac{\hbar\omega}{2}\partial_\omega\ln[D(\vect{r},\omega)]
\end{equation}

Also gets used/cited by Schaden's 90s papers.

Comment on studying auxiliary system where frequency is treated as parameter.

Stress the approximation that reduces computing van der Waals forces to computing the Green function.  

Show from microscopic theory (and Maxwell's equations) that the macroscopic 
theory applies in general.
  Average interaction hamiltonian over fluctuations.
  (Similar to Milonni 2010 no?)
  Get relations involving Green tensor   Eq.(25-26)
(They use $g$ to model adiabatic turn-on/turn-off of interaction.)
\begin{equation}
\frac{\omega_n^2}{c^2}\int d\vect{r''} 
\epsilon_{il}(\vect{r},\vect{r''},i|\omega_n|,g)D_{lk}(\vect{r''},\vect{r'},\omega_n,g)
+[\nabla\times\nabla\times D]_{ik}(\vect{r},\vect{r'},\omega_n,g)
 = -\frac{4\pi}{c}\delta_{ik}\delta(\vect{r-r'})
\end{equation}
Change in free energy is 
\begin{equation}
\Delta F = -\frac{k_B T}{2\pi c}\sum_{n}'\omega_n^2\int_0^1\frac{dg}{g}\int d\vect{r'}d\vect{r}
\left[\epsilon_{il}(\vect{r},\vect{r'},i\omega_n,g)
-\delta_{il}\delta(\vect{r-r'})\right]D_{li}(\vect{r'},\vect{r},\omega_n,g)
\end{equation}

Split forces into short-range forces which depend strongly on details,
 and more universal fluctuation physics.
 They allow spatial dispersion on small scales, which is more important for short range forces.  

Eq.35 suggests a connection to QED via the polarization operator, 
\begin{equation}
\Pi_{il}(\vect{r},\vect{r'},\omega_n,\zeta)=\frac{\omega_n^2}{4\pi c}
[\epsilon_{il}(\vect{r},\vect{r'},i|\omega_n|,\zeta)-\delta_{il}\delta(\vect{r-r'})]
\end{equation}
where the full green function for the EM field obeys
\begin{equation}
D = D_0+ \zeta^2D_0\Pi D
\end{equation}
(I think $\zeta$ acts to index contribution of long-wavelength photons,
 $\zeta=0$ is long-wavelength, $\zeta=1$ is all?)
So they quote a familiar looking result: 
\begin{equation}
\Delta F_{\text{long}} =-2k_BT \sum_{n}'\int_0^1 \frac{d\zeta}{\zeta}\zeta^2 \tr[\Pi D]
\end{equation}
Ok, they finally argue that the dielectric constant is dominated by short wavelength physics,
 and finally take $\epsilon_{il}(\vect{r},\vect{r'},i|\omega_n|,\zeta)=\epsilon(\vect{r},i\omega_n)\delta(\vect{r-r'})$.

Then finally show that $D$ is the full Green tensor obeying the usual macroscopic Maxwell equations.  

Just above Eq. 50 they comment that the ``simple'' theory cannot be used with the
 full/exact dielectric function, since it is itself derived as an approximation.  

Last few pages seem to be taking the near-field and far-field limits.  

Generalization to multilayer systems is easy from this point of view,
 since you just tack on more reflection coefficients.  

Comment that anisotropic systems tend to also have torque!

Also cover atom-atom forces from both the dilute limit, and also pointwise interactions.  
They also include magnetic interactions (and seem to have to allow electric-magnetic cross coupling?)

\subsection{Mclachlan1963}

\cite{Mclachlan1963}

\section{Experimental Papers:Casimir}

\subsection{Lamoreaux1997}

\cite{Lamoreaux1997}
\subsection{Sushkov2011}

\cite{Sushkov2011}
\subsection{Mohideen1998}
\cite{Mohideen1998} 

\subsection{Chan2001}

\cite{Chan2001}
\subsection{Bressi2002}

\cite{Bressi2002}

\section{Experimental Papers:Casimir-Polder}
\subsection{Harber2005: Cornell BEC}

\cite{Harber2005}
\subsection{Obrecht2007:Cornell BEC}
\cite{Obrecht2007}

\subsection{Antezza chapter}
\cite{Dalvit2011}

\subsection{Alton2011:Atoms-Toroid}

\cite{Alton2011}. 

\subsection{Hung2013:Atom-Microcavity}
 Atoms above 1D Microcavity \cite{Hung2013}


\subsection{Atom-Chips:Folman2000}

\subsection{Atom-Chips:Schneider2003}

\cite{Folman2000,Schneider2003}

\subsection{Cronin-Beams}

\cite{Perreault2005,Lonij2009}

\subsection{Sukenik:Atoms through Cavity}

\cite{Sukenik1993}

\subsection{Experimental arguments}

\begin{itemize}
\item Controversies about role of zero temperature pole.  
\item Lamoreaux favors Drude model, Capasso favours plasma model.
\item Seems experiments favour more 
\end{itemize}

\subsection{Geckos: Autumn2002}
\cite{Autumn2002}

\cite{Hawkes2014}

\subsection{Modifications to gravity}

\begin{itemize}
\item Modifications to gravity on $1\mu m$ or $1mm$ scale.  Cite Lamoreaux 2000 Paper.  Gervaci?
Yukawa type forces.  
\item Subtract off Casimir force background.  Tino group.  Use Casimir shield with fairly thick gold to have same Casimir force, and then vary the medium behind it.  Longer range gravity should lead to Requires very careful measurements, on top of carefully extracting Casimir force.   
\end{itemize}

\subsection{Papers: Proximity Force Approximation}

\begin{itemize}
\item Find first use?  Lamoreaux mentions usage.  Derjaguin?\cite{Derjaguin1956} \cite{Blocki1977}
\item Note problem with non-additivity. 
\item Good as order of magnitude estimate?  Useful if very limited curvature, or effectively approximate geometry as planar.  
\end{itemize}

\subsection{Schwinger: Green function methods 1978}

\cite{Schwinger1978, Milton1978}

Scalar green function methods.  Milton book.  
Planes

\subsection{Milton1978}
Spheres

\begin{itemize}
\item Green tensor methods
\item Cite Barton
\item Cite Philbin(?)
\item Cite Vogel and Welsch
\end{itemize}

\section{Papers: Theory- Reflection Matrix}

\subsection{Balian and Duplantier}
\cite{Balian1977} \cite{Balian1978}
\subsection{Lambrecht/MaiaNeto}

\cite{Lambrecht2006}
This paper seems to emphasize the optical network aspect of things.  
Uses the single mirror results to form result for Fabry-Perot cavity.
Treats loss.  

\cite{MaiaNeto2008}
This paper treats plate-sphere results.  (Hmm, I thought they had more to do with this)

\cite{Canaguier-Durand2012}

\subsection{Stratton}
Surface integral equations (Stratton-chu)
\cite{Stratton1941}
Early Emig paper~\cite{Emig2001} computes Casimir energy for corrugated waveguide.
Apparently field splits into two independent scalars.  (TE/TM)
Perfectly conducting boundary conditions, so one scalar obeys Dirichlet, 
the other scalar obeys Neumann BC.
  Interesting for that thought, but otherwise not relevant.  

\subsection{Emig2004}

Paper on deformations showing you can use homogenous green functions within bounding surface.  
\cite{Emig2004}

Quantize EM in general in Lorentz gauge with Fadeev-Popov gauge fixing.  
Introduce restricted partition function that ( I think) enforces EM boundary
conditions on the surfaces. (Note that this idea is exploited in Reids derivation)
\begin{equation}
Z^2(H)=\int D(A^*A)\prod_{\alpha,j}\prod_{\zeta_n}\prod_{\vect{x}\in R_\alpha}
\delta\left[\int_{\vect{x'}\in S_\alpha} \mathcal{L}^\alpha_{j\mu}(\zeta_n;\vect{x},\vect{x'})
  A^\mu(\zeta_n,\vect{x})\right]e^{-S_E[A,A^*]}
\end{equation}
where $R_\alpha$ are auxiliary surfaces, $S_\alpha$ denote surfaces.
$\mathcal{L}^\alpha_{j\mu}$ is differential operator depending on $\epsilon$, normal $\vect{n}_\alpha$.  

Boundary conditions are introduced by appeal to the extinction theorem.  
Tangential components of $B, E$ are continuous across surface.
Normal components of $B, D$ are continuous.

Treat surface constraints at functional delta functions defined on surfaces.
Use Fourier Representation of the delta.  Given quadratic  nature of path integral,
can integrate out fields.

Use Green's theorem to relate full Green function to integrals over surfaces
of of \emph{homogenous} Green functions, where $\epsilon(\omega)$ is constant.  

(Relation to Surface Integral equations?)

\subsection{Emig 2007}
\cite{Emig2007}
\comment{Note: Feinberg and Sucher extended Casimir-Polder to magnetic effects.
Should also include Haakh's work}

Ref 9 is for scalar coupled to dielectric (in terms of scattering matrices).
This is Kenneth/Klich.

Emphasizes a Schwinger like point of view, which attributes field to fluctuating
sources inside bodies.  
Focusing on gauge-invariant quantities.  

Use multipole expansion to derive quantities (Reid and others represent this
with a boundary-element method, which expands fields over surfaces.)

(This paper is the prelude to the 2009 Rahi paper.)

\subsection{Rahi 2009}

Multipole expansion
\cite{Buscher2004}
\cite{Rahi2009}

Extensive bibliography noting connection to Kenneth/Klich, Balian and Duplantier.
Physics connection to Schwinger in terms of fluctuating currents as source.
(Choice amounts to which you emphasize our integrate out first.)

\subsubsection{EM Path Integral}

Start from Lagrangian density
\begin{equation}
\mathcal{L} =\frac{1}{2}(\vect{E\cdot D}-\vect{B\cdot H}).
\end{equation}
Physical fields given in terms of gauge fields
\begin{gather}
\vect{E}=-\nabla\phi-\partial_t\vect{A}\\
\vect{B} = \nabla\times\vect{A}.
\end{gather}
Constitutive relations,
\begin{gather}
\vect{B}(t,\vect{x})=\int_{-\infty}^0 dt' \mur(t-t',\vect{x})\vect{H}(t',\vect{x})
\vect{D}(t,\vect{x})=\int_{-\infty}^0 dt' \epsr(t-t',\vect{x})\vect{E}(t',\vect{x})
\end{gather}
Neglecting spatial dispersion, so get local dielectric.  Also assumed to be 
isotropic.  
Assume action integral is given by up to some time $T$. (They're totally going
to go to imaginary time and temperature with this).
Fourier Expand the fields
\begin{equation}
  \vect{E}(\vect{x},t)=\sum_{n=-\infty}^{\infty} e^{-i\omega_nT}\vect{E}(\omega_n,\vect{x}),
\end{equation}
with $\omega_n=2\pi n/T$.

The electromagnetic action is then
\begin{equation}
S(T) = \frac{T}{2}\sum_{n=-\infty}^\infty \int d\vect{x}
\left(\vect{E_n^*}\frac{1}{\epsr}\vect{E}_n-\vect{B}_n^*\frac{1}{\mur}\vect{B}_n\right)
\end{equation}

Now use Classical equations of motion ($\nabla\times\vect{E}=-\partial_t\vect{B}$)
to eliminate $\vect{B}$ from equations.  This is all still at the classical level.
Action becomes
\begin{align}
S(T) = \frac{T}{2}\sum_{n=-\infty}^\infty \int d\vect{x}
\left[\vect{E_n^*}\epsr\vect{E}_n-\frac{1}{\omega_n^2}
(\nabla\times\vect{E}_n^*)\frac{1}{\mur}(\nabla\times\vect{E}_n)\right]
\end{align}
Now use tensor identity manipulation, where the $k^{th}$ element of $(\nabla\times\vect{A})=$ is
\begin{align}
&\int d^3x (\nabla\times\vect{A}) \cdot f(\nabla\times\vect{B})\\
&=\int d^3x f\epsilon_{ijk}\epsilon_{imn}\partial_jA_k\partial_mB_n\\
%&=-\int d^3x A_k \epsilon_{ijk}\partial_j\epsilon_{imn} f(\partial_mB_n)\\
&=\int d^3x A_k \epsilon_{kji}\partial_j\epsilon_{imn} f(\partial_mB_n)\\
&=\int d^3x \vect{A}\cdot\nabla\times f \nabla\times\vect{B}
\end{align}
Using this integration by parts, the action can be written purely in terms of the electric field
\begin{align}
S(T) = \frac{T}{2}\sum_{n=-\infty}^\infty \int d\vect{x}\vect{E_n^*}
\left[\epsr-\frac{1}{\omega_n^2}\nabla\times\frac{1}{\mur}\nabla\times\right]\vect{E}_n
\end{align}
(Note that this paper uses relative permitivitties.) If you factor out appropriate factors of $\epsilon_0$
you get $c^2$ on magnetic/gradient term.  
This can be further decomposed as
\begin{align}
S(T) = \frac{T}{2}\sum_{n=-\infty}^\infty \int d\vect{x}\vect{E_n^*}
\left[D -\frac{c^2}{\omega_n^2}V\right]\vect{E}_n
\end{align}
where the free space wave operator $D$, and a scattering potential $V$, are given by
\begin{gather}
D=I-\frac{c^2}{\omega_n^2}\nabla\times\nabla\times\\
V=I\frac{\omega_n^2}{c^2}[1-\epsr]-\nabla\times\left(\frac{1}{\mur}-1\right)\nabla\times.
\end{gather}
Note that $D$ has simple expressions for the Green function, since this is just the free-space
EM wave operator.  $V$ can then be treated as a perturbation and its effect solved for.

\subsubsection{Path Integrals}
This is promoted to quantum theory by using the fac that or quadratic Hamiltonians, 
the transition amplitude can be written as 
\begin{equation}
\langle \vect{E}_b|e^{-iHT/\hbar}|\vect{E}_b\rangle = \int D\vect{A} e^{\frac{i}{\hbar}S[T]}
\bigg|_{\vect{E}(t=0)=\vect{E}_a,\vect{E}(t=T)=\vect{E}_b}
\end{equation}
Note that they choose $A_0=0$ gauge-fixing.  ( I think there is an additional implicit
gauge this is fixed by this choice, or that they then use the resulting Classical equations
of motion to produce the result).  

Can get Euclidean path integral by setting $\vect{E}_b=\vect{E}_a$, and tracing over.  Then taking
$T\rightarrow -i\hbar \beta$ yields the path integral expression for the partition function
$Z=\tr[e^{-\beta H}]$ for the EM field.  

Interested in ground state energy at zero temperature.  
\begin{equation}
E=-\frac{1}{\beta}\log Z
\end{equation}
Formally divergent, must be regularized.  If consider energy differences to some reference configuration,
these cutoff dependent contributions drop out.  

Renormalize by considering some reference case.
\begin{equation}
E-E_0=-\frac{1}{\beta}[\log Z-\log Z_0]
\end{equation}

Rotate action from Minkowski to Euclidean, by taking $\epsilon(\omega)\rightarrow\epsilon(i\kappa_n)$
where $\kappa_n=2\pi n/(\hbar\beta c)$.  The traditional Matsubara frequencies are defined as $s_n=c\kappa_n$.
  For other gauges apparently must also rotate $A_0$.  

Exploit the facts that:
1) EM field is real, so $\vect{E}^*(\omega)=\vect{E}(-\omega)$ on the real axis.  
2) Can show response functions also obey $\epsilon(ic\kappa)=\epsilon(-ic\kappa)$.  So only need
to consider integration over $\kappa>0$.  
3) However, this means $\vect{E},\vect{E}^*$ are now treated as independent variables (?)
4) In $\beta\rightarrow \infty$ limit, replace sum with integral:
$\sum_{n\ge 0} \rightarrow \frac{\hbar c\beta}{2\pi}\int d\kappa$
So energy can be written
\begin{equation}
E = -\frac{\hbar c}{2\pi}\int_0^\infty d\kappa \log Z(\kappa).
\end{equation}
where path integral $Z$ is
\begin{align}
Z(\kappa)=\int D\vect{A}D\vect{A}^*\exp\left\{-\beta\int d\vect{x}\vect{E}^*\left[
\left(I+\frac{1}{\kappa^2}\nabla\times\nabla\times\right)+\frac{1}{\kappa^2}V\right]\vect{E}\right\}
\end{align}
where the potential operator is
\begin{equation}
V=I\kappa[1-\epsr(ic\kappa,\vect{x}]-\nabla\times\left(\frac{1}{\mur(ic\kappa,\vect{x})}-1\right)\nabla\times.
\end{equation}

\subsubsection{Green function expansions}
Now an aside on setting up the classical Green functions.
The wave equation associated with the classical action in the absence of media is
\begin{equation}
\left(-I\frac{\omega^2}{c^2}+\nabla\times\nabla\times\right)\vect{E}(\omega,\vect{x})=0
\end{equation}
The Green's function dyadic satisfies the associated equation with delta function source,
\begin{equation}
\left(-I\frac{\omega^2}{c^2}+\nabla\times\nabla\times\right)G^0(\omega,\vect{x},\vect{x'})=I\delta(\vect{x-x'})
\label{eq:A0-Green-x}
\end{equation}
This equation can be solved via a Fourier transform,
\begin{equation}
  G^0(\omega;\vect{k},x')=\int d\vect{x} e^{-i\vect{k\cdot x}} G^0(\omega;\vect{x},\vect{x'})
\end{equation}
Using the Fourier representation of the delta function, the Green function must satisfy
\begin{equation}
\left(-I\frac{\omega^2}{c^2}-\vect{k}\times\vect{k}\times\right)G^0(\omega,\vect{k},\vect{x'})
=Ie^{-i\vect{k}\cdot(\vect{x'})}\label{eq:A0-Green-k}
\end{equation}
This is now a linear matrix equation, which can be solved readily.  We will use 
\begin{equation}
(\vect{k}\times\vect{k}\times)_{ij}=-k^2\left(\delta_{ij}-\frac{k_ik_j}{k^2}\right)=-k^2(P_T)_{ij}
\end{equation}
where we used
\begin{equation}
  \epsilon_{iab}k_a\epsilon_{bcd}k_c = \epsilon_{bia}\epsilon_{bcd}k_ak_c 
= (\delta_{ic}\delta_{ad}-\delta_{id}\delta_{ac})k_ak_c
= k_ik_d-\delta_{id}k^2
\end{equation}
We can use the transverse/longitudinal projectors to write the inverse of the matrix on the LHS. 
\begin{equation}
  I = P_L + P_T \quad P_L^2=P_L\quad P_T^2=P_T\quad P_LP_T=0
\end{equation}
The matrix inverse of the operator in Eq.~(\ref{eq:A0-Green-k}) must satisfy
\begin{equation}
  (aP_L + b P_T)[-\frac{\omega^2}{c^2}(P_L+P_T)-k^2P_T]=I
\end{equation}
for some constants $a$,$b$.  Expanding the above relation out, and equating terms on each side,
we find that 
\begin{align}
  (aP_L + b P_T)[-\frac{\omega^2}{c^2}P_L+(k^2-\omega^2/c^2)P_T]=I
\end{align}
Then $a = -\frac{c^2}{\omega^2}$, and $b=\frac{1}{k^2-\omega^2/c^2}$.  
If we write Eq.~(\ref{eq:A0-Green-k}) as $A G = I e^{-i\vect{k\cdot x'}}$, then the Green function 
can be written as 
\begin{align}
  G^0_{ij}& = A^{-1}_{ik}I_{kj} e^{-i\vect{k\cdot x'}} =e^{-i\vect{k\cdot x'}} \left[-\frac{c^2}{\omega^2}\frac{k_ik_j}{k^2}
    +\frac{1}{k^2-\omega^2/c^2}\left(\delta_{ij}-\frac{k_ik_j}{k^2}\right)\right]
\end{align}
Now simplify the term in front of longitudinal projector (briefly using $\alpha=\omega/c$)
\begin{equation}
  -\frac{1}{\alpha^2}+\frac{1}{k^2-\alpha^2} =  -\frac{k^2-\alpha^2+\alpha^2}{\alpha^2(k^2-\alpha^2)}
=-\frac{k^2}{\alpha^2(k^2-\alpha^2)}
\end{equation}
If we invert the Fourier transform in $k$, then the Green function is written as
\begin{equation}
G^0_{ij}(\omega;\vect{x},\vect{x'})=\int \frac{d\vect{k}}{(2\pi)^3} \frac{e^{i\vect{k}\cdot(\vect{x-x'})}}
{k^2-(\omega/c+i\epsilon)^2}\left(\delta_{ij}-\frac{c^2}{\omega^2}k_ik_j\right),
\end{equation}
where that choice of pole prescription corresponds to choosing outgoing waves, i.e. $e^{ik x}$
The Fourier transform can be evaluated (simple exercise in spherical integrals, and reasoning about 
which way to complete the contour.  
This can also be written as 
\begin{align}
G^0_{ij}(\omega;\vect{x},\vect{x'})&=\left(\delta_{ij}-\frac{c^2}{\omega^2}\partial_i\partial'_j\right)
\int \frac{d\vect{k}}{(2\pi)^3} \frac{e^{i\vect{k}\cdot(\vect{x-x'})}}
{k^2-(\omega/c+i\epsilon)^2}\\
&=\left(\delta_{ij}-\frac{c^2}{\omega^2}\partial_i\partial'_j\right)
\frac{e^{i\omega/c|\vect{x-x'}|}}{4\pi|\vect{x-x'}|}
\end{align}
So far, so standard.  

Can write Green function in terms of states, using $\vect{E}_\alpha(\omega)=
\langle x|\vect{E}_\alpha(\omega)\rangle$, where $\alpha$ is labelling index.
Then the Green function can be expanded in Hilbert space as 
\begin{equation}
  G^0 = \int d\omega' \sum_{\alpha} C_\alpha(\omega')
  \frac{|\vect{E}^\mathrm{reg}_\alpha(\omega')\rangle\langle \vect{E}^\mathrm{reg}_\alpha(\omega')|}
  {\omega'^2/c^2-(\omega/c+i\epsilon)^2}
\end{equation}
where $\vect{E}^\mathrm{reg}$ is a ``regular'' non-singular solution to equations of motion.  
Note these are normalized for EM scattering theory, \emph{not} quantum mechanics,
\begin{equation}
\int d\omega' \sum_{\alpha} C_\alpha(\omega')
  |\vect{E}^\mathrm{reg}_\alpha(\omega')\rangle\langle \vect{E}^\mathrm{reg}_\alpha(\omega')| = I
\end{equation}

Distinguish ``regular'', (solutions to source-free equations), from ``outgoing'' solutions
which must obey outgoing boundary conditions as ``outward'' variable, $\xi_1$ goes to infinity.
Must pick one of the coordinates $\xi_i$ to denote a notion of outward.  
They use the example from spherical scattering of Hankel functions being outgoing, $h_l^{(1)}(kr)\sim \frac{e^{ikr}}{r}$
which is singular at the origin.  This is contrasted with the regular Bessel solution, $j_l^{(1)}$.  
The full outgoing solution is the outgoing solution in $\xi_1$ multiplied by the regular solutions
in $\xi_2,\xi_3$.  

The Green function (carrying out the contour integral in $\omega'$) is:
\begin{equation}
G^0_{ij}=\sum_\alpha C_\alpha(\omega)\left\{\begin{array}{cr}
E_i^\mathrm{out}(\omega,\vect{\xi}) E_j^{\mathrm{reg}*}(\omega,\vect{\xi'}) & \xi_1(\vect{x})>\xi_1'(\vect{x'})\\
E_i^\mathrm{reg}(\omega,\vect{\xi}) E_j^{\mathrm{in}*}(\omega,\vect{\xi'}) & \xi_1(\vect{x})<\xi_1'(\vect{x'})
\end{array}
\right.
\end{equation}
Set $C_\alpha$ using the Wronskian of the regular and outgoing solutions, while the completeness
relation for $\xi_2,\xi_3$ sets the correct ``jump'' condition at $\vect{x=x'}$.  

For imaginary $k$, find incoming and regular solutions increase exponentially with $\xi_1$,
which outgoing solutions descrease exponentially.  

(They  use on-shell to denote only equal frequency components contribute, i.e. $\omega'=\omega$.)

\subsubsection{Translation matrices}

In order to use this formalism, one must be able to translate the matrices to different origins on
different bodies.  

You can use the regular solutions as a basis for expansion - provided you omit the possibly 
singular origin for the outgoing solutions.  
\begin{equation}
\vect{E}^\text{reg}_\alpha(\omega,\vect{x}_i) = \sum_{\beta}\mathcal{V}^{ij}_{\beta,\alpha}(\kappa,\vect{x}_i-\vect{x}_j)
\vect{E}^\text{reg}_\beta(\omega,\vect{x}_j)
\end{equation}
Note that in this formulation, $\vect{x}_i$ is the vector from origin $\mathcal{O}_i$ to a particular 
point, and $\vect{x}_j$ is a vector to the \emph{same} point, but with a different origin.  

So for the outgoing solutions, 
\begin{equation}
\vect{E}^\text{out}_\alpha(\omega,\vect{x}_i) = \sum_{\beta}\mathcal{U}^{ij}_{\beta,\alpha}(\kappa,\vect{x}_i-\vect{x}_j)
\vect{E}^\text{reg}_\beta(\omega,\vect{x}_j),
\end{equation}
provided $x_i$ is not near the origin of its coordinate system.  

(They use $X_{ij}=\vect{x}_i-\vect{x}_j$)

This can be applied to express the Green function when $\vect{x}$ lies on one object,
and $\vect{x'}$ lies on another object.  
\begin{equation}
G^0_{ij}(ic\kappa,\vect{x},\vect{x'})=\sum_{\alpha,\beta} C_\beta(\omega)\left\{\begin{array}{cr}
E_{i,\alpha}^\mathrm{reg}(\kappa,\vect{x}_i) \mathcal{U}^{ij}_{\alpha\beta} E_{j,\beta}^{\mathrm{reg}*}(\omega,\vect{x}_j)
& \text{if $i$ and $j$ are outside each other}\\
E_{i,\alpha}^\mathrm{reg}(\kappa,\vect{x}_i) \mathcal{V}^{ij}_{\alpha\beta} E_{j,\beta}^{\mathrm{in}*}(\omega,\vect{x}_j)
& \text{if $i$ is inside $j$, or $i$ is below $j$ (plane)}\\
E_{i,\alpha}^\mathrm{out}(\kappa,\vect{x}_i) \mathcal{W}^{ji}_{\alpha\beta} E_{j,\beta}^{\mathrm{reg}*}(\omega,\vect{x}_j)
& \text{if $j$ is inside $i$, or $j$ is below $i$ (plane)},
\end{array}\right.\label{eq:G0_expansion}
\end{equation}
where $\mathcal{W}^{ij}_{\alpha\beta}=\frac{C_\alpha}{C_\beta}{\mathcal{V}^{ij}_{\alpha\beta}}^\dag$.

This can be further written in matrix form as 
\begin{equation}
G^0_{ij}(ic\kappa,\vect{x},\vect{x'})=\sum_{\alpha,\beta}C_\beta(\omega)
\left( E_{i,\alpha}^\mathrm{reg}(\kappa,\vect{x}_i), E_{i,\alpha}^\mathrm{out}(\kappa,\vect{x}_i)\right)
\Bigg(
\begin{array}{cc}
\mathcal{U}^{ij}_{\alpha\beta} & \mathcal{V}^{ij}_{\alpha\beta}\\
\mathcal{W}^{ji}_{\alpha\beta} & 0
\end{array}\Bigg)
\Bigg(\begin{array}{c}
 E_{j,\alpha}^{\mathrm{reg}*}(\kappa,\vect{x}_j)\\
E_{j,\alpha}^{\mathrm{in}*}(\kappa,\vect{x}_j)
\end{array}
\Bigg)
\end{equation}
For a given geometry only one of these translation matrices is non-zero, and thus picks out 
one element of the resulting matrix.  

This can be further written in bra-ket notation, where we will define the negative
of the central matrix as $X^{ij}$. (They use BB letters).

(This choice of basis expansion does not work well for intersecting objects - however the 
Johnson choice in terms of triangles works just fine)

\subsubsection{Classical Scattering  theory}

Maxwell's equations can be written in Fourier space as 
\begin{gather}
\nabla\times\vect{E}(\omega,\vect{x}) = i\omega \vect{B}(\omega,\vect{x})\\
\nabla\times\frac{1}{\mur}\vect{B}(\omega,\vect{x}) = -i\frac{\omega}{c^2}\epsr\vect{E}(\omega,\vect{x})
\end{gather}
If we combine the equations, the resulting wave equation can be written as
\begin{equation}
  [H_0+V(\omega,\vect{x})]\vect{E} = \frac{\omega^2}{c^2}\vect{E}(\omega,\vect{x})
\end{equation}
where 
\begin{align}
H_0 = \nabla\times\nabla\times \\
V_0 = I\frac{\omega^2}{c^2}[1-\epsr(\omega,\vect{x})]+\nabla\times\left(\frac{1}{\mur}-1\right)\nabla\times,
\end{align}
and $V_0$ is the same potential operator that showed up in the quantum theory.  

The Lippmann-Schwinger equation says
\begin{equation}
|\vect{E}\rangle = |\vect{E}_0\rangle - G_0V|\vect{E}\rangle,
\end{equation}
where $G_0$ is the free Green operator and $|\vect{E}_0\rangle$ is a solution of the free 
wave equation, i.e. $(H_0-\omega^2/c^2I)|\vect{E}_0\rangle=0$.  (I think This follows from
 defining $|\vect{E}\rangle$ as the original free solution, and some scattered part. Then plug this in
and solve for the implicit equation for $\vect{E}$.)

If you plug this into itself repeatedly, then 
\begin{equation}
|\vect{E}\rangle = |\vect{E}_0\rangle - G_0V(|\vect{E}_0\rangle - G_0V|\vect{E}\rangle)\cdots
=|\vect{E}_0\rangle -G_0T|\vect{E}_0\rangle,
\end{equation}
where the electromagnetic T-operator is
\begin{equation}
  T = V\frac{I}{I+G_0V}= VGG_0^{-1},
\end{equation}
where $G$ is the Green function of the full Hamiltonian: $(-\frac{\omega^2}{c^2}I+H_0+V)G=I$
The free Green function $G_0$ obeys $(-\frac{\omega^2}{c^2}I+H_0)G_0=I$.
  
The $T$ operator is related to the scattering amplitude $\mathcal{F}$.  So if we scatter a wave 
from the exterior of an object, then $|\vect{E}_0\rangle$ is taken to be $|\vect{E}^{\text{reg}}_\alpha\rangle$.
Then using the Green function (\ref{eq:G0_expansion}), the full state can be written as
\begin{equation}
  \vect{E}(\omega,\vect{x}) = \vect{E}^{\text{reg}}_\alpha(\omega,\vect{x}) - \sum_\beta \vect{E}^{\text{out}}_\beta(\omega,\vect{x})
  \int d\vect{x'}d\vect{x''} C_\beta(\omega)\vect{E}^{\text{reg}*}_\beta(\omega,\vect{x'})T(\omega,\vect{x'},\vect{x''})
\vect{E}^{\text{reg}}_\beta(\omega,\vect{x''}).
\end{equation}
This can be further written in Dirac notation (reg-out), 
\begin{equation}
  |\vect{E}(\omega)\rangle = |\vect{E}^{\text{reg}}_\alpha(\omega)\rangle+
  \sum_\beta\mathcal{F}_{\beta\alpha}^{ee}|\vect{E}^{\text{out}}_\beta(\omega)\rangle
\end{equation}
where the external-external scattering amplitude is defined as
\begin{equation}
  \mathcal{F}^{ee}_{\beta,\alpha}=(-1)C_\beta(\omega)\langle \vect{E}^{\text{reg}}_\beta|(\omega)T(\omega)
|\vect{E}^{\text{reg}}_\alpha(\omega)\rangle.
\end{equation}
Similary, for internal-external scattering one has (reg-reg) 
\begin{equation}
  |\vect{E}(\omega)\rangle = |\vect{E}^{\text{reg}}_\alpha(\omega)\rangle+
  \sum_\beta\mathcal{F}_{\beta\alpha}^{ie}|\vect{E}^{\text{reg}}_\beta(\omega)\rangle
\end{equation}
where the internal-external scattering amplitude is defined as
\begin{equation}
  \mathcal{F}^{ie}_{\beta,\alpha}=(-1)C_\beta(\omega)\langle \vect{E}^{\text{in}}_\beta|(\omega)T(\omega)
|\vect{E}^{\text{reg}}_\alpha(\omega)\rangle.
\end{equation}
This corresponds to some outside wave being scattered inwards. 

The converse in internal scattering (hard to actually realize).
In this case $E_0\rightarrow E^{\text{out}}$.  (I'll skip this)

The whole set of scattering amplitudes can be written in matrix form,
\begin{equation}
F = \left(\begin{array}{cc}
    F^{ee} & F^{ei} \\ F^{ie} & F^{ii}
  \end{array}
  \right)
\end{equation}
where the amplitudes are 
\begin{align}
  F^{ee}& = C_\alpha \langle \vect{E}^{\text{reg}}_\alpha|T|\vect{E}^{\text{reg}}_\beta\rangle\\
  F^{ie}& = C_\alpha \langle \vect{E}^{\text{in}}_\alpha|T|\vect{E}^{\text{reg}}_\beta\rangle\\
  F^{ei}& = C_\alpha \langle \vect{E}^{\text{reg}}_\alpha|T|\vect{E}^{\text{out}}_\beta\rangle\\
  F^{ii}& = C_\alpha \langle \vect{E}^{\text{in}}_\alpha|T|\vect{E}^{\text{out}}_\beta\rangle
\end{align}
Seems that you don't actually find $T$, but rather make an ansatz for the solutions appropriate
 to the scattering scenario you are considering, and match the appropriate boundary conditions.
(Well what was the point of all that then?)



\subsubsection{Applying to Partition function}

\subsubsection{Translation Matrices}


\subsection{Kenneth/Klich}
\cite{Kenneth2006}
\cite{Kenneth2008}

\subsection{Reid/Johnson/Rodriguez}

Early numerical papers.
\cite{Rodriguez2007},\cite{Rodriguez2007a}, \cite{Rodriguez2009}.
  Note use of existent analytical methods and similarities to existent numerical 
FTDT techniques Builds on earlier papers (uses better basis) 

Cite Johnson textbook chapter.\cite{Johnson2011}

\subsection{Reid papers}

\cite{Reid2009},\cite{Reid2011}, \cite{Reid2013} 
Note success, applicability.  \comment{Cite experimental tylenol pill paper}

This numerical approach is related to prior multipole scattering approach~\cite{Rahi2009}.
Earlier work chose to expand the scattering amplitudes in terms of mode function
solutions.  In contrast, this work expands the fields at the surfaces using 
boundary elements - in particular triangles.  
This avoids some of the issues of poorly converging summations of solutions that occured in
the multipole method.  Get very similar results, but different choice of basis for calculations.
Multipoles exploit symmetry, while boundary elements are generally applicable, make no assumptions 
about surface.  Other than the bodies are piece-wise continuous.  

Naturally decomposes separate frequencies into independent integral.  Could parallelize that way.
Still have to handle large matrices.  Freely available as ScuffEM package maintained by
 Homer Reid~\cite{ScuffEM2016}.  

\section{Repulsion}

\begin{itemize}
\item Cite Milton paper on anisotropy \cite{Milton2012, Milton2012a}
\item Cite work on metamaterials (hydrogen mirror)
\item Cite Milonni/rosa showing broad-band \cite{Rosa2010}
\end{itemize}

\section{Papers: Worldlines}

Note citations in Bordag/Johnson mostly as dismissive and limited.  

\subsection{Effective actions}

\cite{McKeon1993, Strassler1992,Schubert2001}

\comment{Other references - was one contemporaneous with Strassler? Bern-Kosower}

\comment{Cite 1950 Feynman Scalar QED section}

\cite{Schubert2001}.  

\subsubsection{Applied to QED}

For example, the worldline method has been used to compute relativistic field effects for QED such
 as the Lamb shift~\cite{Schmidt1995}.  It has also been used as a numerical algorithm\cite{Mazur2014}.

\subsection{Feynman Path Integral for QM}

\cite{Feynman1948,Feynman1965,Brown2005}.

\subsection{Papers: Gies Worldline}

\subsubsection{Gies 2003}
\cite{Gies2003}

\begin{itemize}
\item Cite earlier paper by themselves for first worldline numerics?   Found citations for PFA.
\item Action for scalar fields.   Massive scalar field interacting with some potential $V(x)$.
  $V$ has dimensions of mass-squared.  (Field theory units $\hbar=c=1$, so $E=mc^2$)
\begin{equation}
\cL = \frac{1}{2}\partial_\mu\phi\partial^\mu\phi +\frac{1}{2}m^2\phi^2+\frac{1}{2}V(x)\phi^2
\end{equation}
\item Complete unrenormalized quantum effective action for $V$ (integrate out fields $\phi$) is 
\begin{align}
  \Gamma[V] &=\frac{1}{2}\tr\ln \left[ \frac{-\partial_\mu\partial^\mu +m^2 + V(x)}
    {-\partial_\mu\partial^\mu +m^2}\right]\\
&   =-\frac{1}{2}\int \frac{d\cT}{\cT} \int d^D x \left\{ 
\langle x|e^{-\cT[-\partial_\mu\partial^\mu +m^2 + V(x)]}|x\rangle -\frac{e^{-m^2 \cT}}{(4\pi \cT)^{D/2}}\right\}
\end{align}
Working in $D=d+1$ euclidean space-time dimensions.  
\item Convert matrix element to path integral.  
\begin{equation}
  \int d^D x \langle x|e^{-\cT[-\partial^2+V(x)]}|x\rangle = \int d^D x_{\text{CM}} \mathcal{N}\int_{x(0)=x(\cT)}Dx e^{-\int_0^\cT d\tau \dot{x}^2/4}
\end{equation}
Comments: Fix normalization from limit of zero potential (which is evaluated in momentum basis), and note that choice to drop factors of 2 has left them with an extra $\sqrt{2}$ in the definition  of the loops.
\begin{align}
\langle x| e^{\partial^2T}|x\rangle &= \int d^Dp\frac{1}{(2\pi)^{D/2}} e^{-p^2T}\langle x|p\rangle\langle p|x\rangle\\
&= \frac{1}{(4\pi \cT)^{D/2}}\\
&=\cN \int_{x(0)=x(\cT)}Dx \exp\left\{-\int_0^\cT d\tau \left[\frac{\dot{x}^2}{4}+V(x_{\text{CM}}+x(\tau)\right]\right\}
\end{align}
\item Now interpret path integral as Gaussian integral/ensemble average over closed loops.  
\begin{equation}
\cN \int_{x(0)=x(\cT)}Dx \exp\left\{-\int_0^\cT d\tau \left[\frac{\dot{x}^2}{4}+V(x_{\text{CM}}+x(\tau)\right]\right\} = \frac{1}{(4\pi \cT)^{D/2}}\dlangle e^{-\int_0^\cT d\tau V[x_{\text{CM}}+x(\tau)]}\drangle
\end{equation}
Also use scaled loops: 
\begin{equation}
x_\mu(\cT t) = \sqrt{\cT}y_\mu(t),
\end{equation}
where $t\in [0,1]$, and we view $\cT$ as a parameter.  
\item Plugging in path representation
\begin{align}
\Gamma[V]=&-\frac{1}{2 (4\pi)^{D/2}}\int_{1/\Lambda^2}^\infty \frac{d\cT}{\cT^{1+D/2}} e^{-m^2\cT}\int d^Dx_{\text{CM}} \\
&\times\dlangle e^{-\cT\int_0^1dt V[x_\text{CM} +\sqrt{\cT}y(t)]}-1 \drangle
\end{align}
For time independent backgrounds use $\int dx_0 = L_{x_0}$ as ``length'' in time direction.  \comment{For partition function version Get $\beta\hbar c$ instead.}
\item \textbf{Renormalization}
Consider field theoretic renormalization (I think whereby an analogue of the Integral Trajectory Matching formalism fixes the unknown(divergent) values at some known scale - typically from a low energy experiment).  
\item Heat Kernel Expansion \comment{Cite Gilkey} .  Expand to non-diverging order to regularize UV divergences.  Each power of $V$ amounts to an external leg.  Get a Tadpole $T^{-D/2}\int dx_0 V$, and 
\item Ignore self-energies and focus on interaction energies (which are insentitive to Field theoretic divergence)
\begin{equation}
E_\text{int}= E_{V_1+V_2+\cdots}- E_{V_1} - E_{V_2}-\cdots
\end{equation}
Can carry out renormalization at loop level and thus avoid divergent sums.  
\item They note that this emphasis on interaction energy is \emph{not} a renormalization procedure.  It circumvents that for multiple bodies.  However for isolated bodies (sphere self-stress), one must specify all of the renormalization conditions, and your answers may depend on the procedure used.
\item \textbf{Numerics}
Introduce some methods (most of which are useless).
Important method is the v-loop.  An exact diagonalization of the Gaussians, under the loop closure constraint.  Note that they choose their loop constraint to be $\int d\tau y_\mu(\tau) =0$.  This amounts to making the center of mass of the loop centered on $x_\text{CM}$.  In contrast, our work has typically used $x(0)=x(\cT)=0$.  

The ``v-loop'' algorithm amounts to considering the probability measure,
\begin{equation}
P(\{x_k\}) =  \delta(x_N-x_0)\prod_{j=0}^{N}e^{-\frac{(x_{j+1}-x_j)^2}{2\Delta T}},
\end{equation}
and completing the square in the exponents to decouple the Gaussians.  
Note their version uses $\sum_{j}x_j=0$ to fix $y_N$.  
The Jacobian can be shown to unity, since the matrix is upper diagonal, with unit diagonal.  
\item \textbf{Tests}
\item Compare to parallel plate energy to get convergence. 
\item Give some consideration to finite $\lambda$ in potential.  
\begin{equation}
V(x) = \lambda\int_{\Sigma}d\sigma \delta(x-x\sigma)
\end{equation}
where $\Sigma$ is a $d-1$ dimensional surface, $\sigma$ is a reparameterization invariant measure and $x_\sigma$ points to the surface.  
\item For interaction energies, need $(e^{-V_{1+2}} - 1) -  (e^{-V_1}-1) - (e^{-V_{2}}-1)$.  If only touch one surface (of disconnected bodies) then 
no contribution, since $V_{1+2}=V_1$
\item Then consider massless scalar between sphere/plate and cylinder plate.  
\item Consider breakdown of PFA as function of separation $a$ to sphere radius $R$.  
\item \textbf{Conclusions}
\item Needs no underlying symmetry.  Precision hinges on loop parameters chosen (and presumably discretization of surfaces)
\item Claim their delta potentials are ``hard''.
\item Note finite temperature and roughness required.  
\item Note numerical differentiation to get force is hard, but derivative can be done right off the bat.  (And energy-momentum tensors)
\item Cite Feinberg and Sucher for possible path for EM quantization.  
\end{itemize}

\subsubsection{Gies2001}
\cite{Gies2001}

Use worldlines to compute EM effective action in specified background EM field, modelling interaction with scalar particle.  

\subsubsection{Gies 2006}
\cite{Gies2006}

Very similar?  Advances?  
D-loops?  Note similarity to methods of generating brownian walks by doubling intermediate points to imrpvoe resolution where required.  

\subsubsection{Gies 2006 a}
\cite{Gies2006a}

\subsection{Papers: Thermal worldlines}

\subsubsection{Geothermal2008:Klingm\"uller}
~\cite{Klingmueller2008}.  
\subsubsection{Weber2009:Inclined planes}
~\cite{Weber2009}
\subsubsection{Weber2009:Interplay}
~\cite{Weber2010}
\subsubsection{Weber2009:Spheres}
\cite{Weber2010a}.  

\subsubsection{Schaden:Pistons}

Cite Schaden applying to pistons\cite{Schaden2009}

Why?  What for?  Benefits?

\subsection{Nearby attempts}

\cite{Aehlig2011}

\cite{Maggs2006, Pasquali2008}

\section{Papers:Stochastic Methods}

\subsection{Gradients}

\subsection{Feynman-Kac}

\subsubsection{Hooghiemstra}



\section{Papers:Dielectric Quantization}



\subsection{Vogel and Welsch}


%\cite{Dung1998}
\cite{Raabe2006}
\cite{Raabe2007}

\subsection{Huttner/Barnett}

\cite{Glauber1991}

\cite{Huttner1992}

\cite{Matloob1995}
\cite{Matloob1996}


\subsection{Philbin}

\cite{Philbin2010}
\cite{Philbin2011}

\cite{Drummond2014}

\subsection{Bordag}
\cite{Bordag1998} \cite{Bordag1999}

\subsection{Bechler}

\cite{Bechler1999}
\cite{Bechler2006}

\section{Quantization in dielectric}


\subsection{Rosa2010}

Justifies/derives why it is okay to use neglect dispersion in derivations
of the Casimir energy.  i.e. okay to set things up with real $\epsilon$,
then take $\epsilon\rightarrow\epsilon(i\omega_n)$.  \cite{Rosa2010}

The classical energy density in a dispersive medium is (Eq. 1)
\begin{equation}
  \bar{u}(\vect{r},\omega) = 
  \left[\frac{d(\omega\epsilon_R)}{d\omega}|\vect{E}_\omega(\vect{r})|^2 +
    \frac{d(\omega\mu_R)}{d\omega}|\vect{H}_\omega(\vect{r})|^2\right],
\end{equation}
where $\epsilon_R,\mu_R$ are the real parts of the permittivity and magnetic 
permeability.  This is neglecting dispersion. 
Apparently this also holds in the quantum theory, at finite temperature,
with or without dispersion \emph{in uniform media}.  
(They note relation to Barash and Ginzburg)

Use harmonic model and carefully add up contributions from each dipole.
Define energy based on integral of Poynting's theorem.  



\section{Quantum Trajectories}

\subsection{Carmichael}
Cite Carmichael Rice JOSA paper.\cite{Carmichael1989,Carmichael1991}

Cite Carmichael 1993 lectures. \cite{Carmichael1993}

\subsection{Others}

Cite Marte
Cite Parkins  
Cite Gardiner \cite{Gardiner1992}
Cite Marte, Zoller, Parkins, Gardiner (MCWF) \cite{Dalibard1992}
\cite{Dum1992}

\subsection{Path integral version}


\subsection{Measuring atom positions}

Measurement of atom's\cite{Caves1987}

 Cite Holland, Meystre.  Applied to position measurements of atoms by detecting photons.  Detection of photons localizes atoms.  
\cite{Holland1996}

Motion in optical lattice
\cite{Greenwood1997}

Lose spatial coherence from single emission.
\cite{Pfau1994}

Which-way \cite{Gagen1993}

\subsection{Control theory}
 Control Theory.  Cite Wiseman book.   \cite{WisemanMilburn2010}

Doherty and Jacobs \cite{Doherty1999}

Cite Steck feedback control paper.  \cite{Steck2004, Steck2006}

\subsection{Quantum chaos}

Cite Bhattacharya quantum paper.  \cite{Bhattacharya2000}

\subsection{Noisy measurements}
Cite Warshawski/Wiseman \cite{Warszawski2003a,Warszawski2003b}

Cite Jeremy Thorne.  


%%% Local Variables: 
%%% mode: latex
%%% TeX-master: "thesis_master"
%%% End: 
