\chapter{Detailed Calculations}
\label{app:nasty_calc}
This Appendix collects a number of lengthy, but tedious calculations required in the main text.  

% \section{Path Integrals in Curved Space}

% In this section we will show that the potential $\VTE$ naturally emerges if we
% work in a curved space.  
% The problem of computing the path integral for a 
% particle in curved space was first considered by deWitt~\cite{deWitt1957}.  
% In a curved space, the position and momentum  operators take on slightly 
% different forms reflecting the differing inner product.  
% Let us follow Pauli to see one way towards handling this~\cite{Pauli1958}.  

% Let us consider a curved space, with generalized coordinates $q_i$.  The metric
% tensor $g_{ij}$ then gives the notion of distance, $ds^2 = g_{ij}dq^idq^j$.  
% In a curved space, the volume element is $\sqrt{|g|}\prod_idq_i$, where $g=\det(g_{ij})$.   

% The identity on the coordinates is 
% \begin{equation}
%  I_q := \int d\vect{q}\sqrt{|g|} |\vect{q}\rangle\langle {q}|
% \end{equation}
% This implies that the wavefunction overlap for two states $|\phi\rangle,|\psi\rangle$
% is 
% \begin{equation}
% \langle \phi|\psi\rangle = \int d\vect{q}\sqrt{|g|} \phi^*(\vect{q})\psi(\vect{q}).
% \end{equation}
% In order for the identity to be idempotent ($I_q^2 = I_q$), this requires that
% the overlap between coordinate eigenstates is 
% \begin{equation}
% \langle \vect{q}|\vect{q'}\rangle = \frac{1}{\sqrt{|g|}}\delta(\vect{q}-\vect{q}')
% \end{equation}

% The position and momentum operators should be hermitian, and obey the 
% canonical commutation relations, 
% \begin{equation}
% [\op{q}_i,\op{p}_j]=i\hbar\delta_{ij}.
% \end{equation}
% If we define the position operators such that 
% \begin{equation}
% \langle \vect{q'}|\op{q}|\psi\rangle=\vect{q'}\langle \vect{q'}|\psi\rangle,
% \end{equation}
% then we can infer from the commutation relations, and the hermiticity condition
% that the momentum operators are given by
% \begin{align}
% \langle\phi|\op{p}_i|\psi\rangle = \int d\vect{q}\sqrt{|g|}\phi^*(\vect{q})
% \frac{1}{\sqrt{|g|}}\frac{\partial}{\partial q_i}\sqrt{|g|}\psi(\vect{q}).  
% \end{align}
% The presence of the extra factors of $\sqrt{|g|}$ ensures that the 
% $\langle\phi|(\op{p}_i|\psi\rangle = (\langle \phi|\op{p}_i^\dag)|\psi\rangle$,
% which follows from the integral represenation via an integration by parts.  

% An alternative method of setting up the path integral in curved space is to 
% transform that flat-space path integral to curvilinear coordinates~\cite{Gervais1976,Girotti1983}.  
% The main concern in this approach is to consistently work to $\order(\Delta T)$.  
% Given the typical Gaussian path measure, which implies $\Delta x^2 \sim \Delta T$, 
% this requires us to also Taylor expand up to fourth order in spatial 
% functions~\cite{McLaughlin1971}.    

% Let us consider the following action
% \begin{equation}
%   S = \frac{m}{2}g_{ij}(\vect{q})\dot{q}_i\dot{q}_j - V(x)
% \end{equation}
% We will find the 
% \begin{enumerate}
% \item Lagrangian.
% \item Momentum
% \item Hamiltonian.  Note $pgp$ structure.  Operator ordering problem in curved space.  
%   Have classical equations, with ambiguous ordering (and tiny, tiny consequences)
%   for different choices.  
% \item Wave equation is Laplace-Beltrami operator (what you get from changing coords
%   -motivates Gervais approach.  
% \item DeWitt: Path integral is kernel of appropriate Schrodinger equation.
% \item Kleinert, transformation useful in curvilinear coordinates. (path integral
%   for hydrogen atom.
% \end{enumerate}



% \section{Path Integral in Curved Space}

% \begin{enumerate}
%   \item Path integral construction on metric-affine space is surprisingly complicated 
%     and error-fraught.  
%   \item Care is required in construction to get all terms.  
%   \item Similar to multiplicative noise in SDE.
% \end{enumerate}

% \begin{enumerate}
%   \item Given analogy of a medium to a curved space. Cite Leonhardt, Gordon  
%   \item Note general requirement for $\mu=\epsilon$.  Hard to achieve, especially broadband.
%   \item Application to TM path integral with rescaling $\epsilon$.
% \end{enumerate}

% \subsection{Operators in Curved Space}

% \begin{enumerate}
%   \item Quote Position Operator and inner product for spatial wavefunctions.
%   \item Note only momentum operator consistent with that.
%   \item Develop path integral
%   \item Show it obeys the Schrodinger equation (Grosche's test)
% \end{enumerate}

% \subsection{Transformation}

% \begin{enumerate}
%   \item Start with flat-space path integral.\cite{Gervais1976, Kleinert2012}
%   \item Introduce coordinate transformation
%   \item Choose expansion point, expand consistently to $\order(\Delta T)$.
%   \item Note connection to choice of stochastic calculus.
%   \item Expand Jacobian factors
%   \item Expand Gaussian factors
%   \item Convert to terms involving curvature tensors.
%   \item Simplify down to $1D$
%   \item Note typically small value of quantum correction.  $\hbar^2$.
% \end{enumerate}


% \section{Operator Quantization in Curved Space}

% This follows from Bryce DeWitt's early papers \footnote{
% DeWitt, B. S. \textit{Point Transformations in Quantum Mechanics}, 
% {Phys. Rev.}, \textbf{85}, 653, 1952.\\
% deWitt, B. S. \textit{Dynamical Theory in Curved Spaces I: A Review of the 
% Classical and Quantum Action Principles}, {Rev. Mod. Phys.}, \textbf{29}, 377,(1957). } .
% See also a review by Pauli\footnote{
% Pauli, W., \textit{General Principles of Quantum Mechanics}, (1980), 
% translated by P. Achuthan and K Venkatesan} 

% There are two ways of deriving the path integral in curved space.  In the first 
% formulation, we derive the relevant path integral in curved space
% by starting with the curved space Hamiltonian and quantized appropriately.  
% This requires changing the form of the momentum operators, which gain corrections.
% This is in contrast to the second approach, which starts with a flat-space path
%  integral, and transforms that path integral to curvilinear coordinates.
%  We will use $x$ to denote the curvilinear coordinates, and $q_i$ to denote the flat-space coordinates.  

% Consider the following particle Lagrangian,
% \begin{equation}
% L = \frac{1}{2}g_{ij}(x)\dot{x}^i\dot{x}^j
% \end{equation}
% with line element, $ds^2 = g_{ij} dx^i dx^j$, and volume element, 
% $dV = \sqrt{|\det[g_{ij}]|}\prod_idx_i$.  

% This has canonical momenta 
% \begin{equation}
% p_i := \frac{\partial L}{\partial \dot{x}^i} = g_{ij}\dot{x}^j.
% \end{equation}

% The equivalent Hamiltonian is 
% \begin{align}
% H & = p_i\dot{x}^i - L \\
% & = \frac{1}{2} p_i g^{-1}_{ij}p_j = \frac{1}{2}p_i g^{ij}p_j,
% \end{align}
% where  ${g^{-1}}_{ij} = g^{ij}$ is the inverse metric.  

% The choice of where to place the metric (which is a function of position)
%  relative to the momentum operators is sometimes called the 
% \textit{operator-ordering problem}.  We have no classical reason for picking
% any one of the myriad quantum operator ordering choices.  
% We will see that this operator ordering is related to our choice of stochastic calculus.  

% \subsection{Quantization}

% When we quantize this we require that the position and momentum obey the usual commutation relations
% \begin{equation}
% [x_i,p_j] = i\hbar\delta_{ij}.
% \end{equation}
% The other piece we need is representation of the identity for states.
%  We will represent the spatial and momentum identity operators as 
% \begin{gather}
% \int d\vect{x} \sqrt{g} |\vect{x}\rangle \langle \vect{x}| = 1\\
% \int \frac{d\vect{p}}{(2\pi\hbar)^d} |\vect{p}\rangle \langle \vect{p}| = 1
% \end{gather}
% This representation of the spatial identity implies the inner product between
% is
% \begin{equation}
% \langle \phi |\psi\rangle = \int d\vect{x} \sqrt{g} \phi^*(x)\psi(x),
% \end{equation}
% where we have introduced the shorthand notation, $g = \det[g_{ij}]$.
% The position space representation of the momentum operator can be de derived 
% by seeking consistency with the commutation relations and ensuring that the 
% momentum is a hermitian operator
% \begin{equation}
% \langle x| \op{p}_i|\psi\rangle = -i \frac{1}{g^{1/4}} \partial_i\left[ g^{1/4}\psi(x)\right] 
% = -i\left(\partial_i +\frac{1}{4}\frac{\partial_i g}{g}\right)\psi(x).
% \end{equation}
% We can check the hermiticity by requiring 
% $\langle\phi |\op{p}|\psi\rangle = \langle \psi |\op{p}\phi\rangle^{*}$.
%   In position space this becomes 
% \begin{align}
% \langle \phi|\op{p}|\psi\rangle & = -i\int dx\,\sqrt{g}\phi^*(x) 
% \frac{1}{g^{1/4}} \partial_x\left[ g^{1/4}\psi(x)\right]\\
% & = i\int dx\,\partial_i\left[g^{1/4}\phi^*(x)\right] g^{1/4}\psi(x)\\
% & = i\int dx\,\sqrt{g}\psi(x) \frac{1}{g^{1/4}}\partial_i\left[g^{1/4}\phi^*(x)\right]\\
% & = \langle \psi |\op{p}|\phi\rangle^*
% \end{align}
% The spatial representation of the momentum operator can also be written
% in terms of the Christoffel symbols,
% \begin{equation}
%  \frac{1}{g^{1/4}}\partial_i g^{1/4} f = \partial_i f + \frac{1}{4}\partial_i\ln(g)f.
% \end{equation}
% The Christoffel symbols are defined as
% \begin{equation}
% \Gamma_{ij}^k = \frac{1}{2}g^{kl}\left(\partial_ig_{jl}+\partial_jg_{li} 
%   - \partial_lg_{ij}\right)
% \end{equation}
% We can trace over the Christoffel symbols, 
% \begin{align}
% \Gamma_{i} &= \Gamma_{ik}^k=\frac{1}{2}g^{kl}\left(\partial_ig_{kl}+\partial_kg_{li} 
%   - \partial_lg_{ik}\right)\\
% %&=\frac{1}{2}\left(g^{kl}\partial_ig_{kl}+\partial^lg_{li} 
% %  - \partial_kg_{ik}\right)\\
% &=\frac{1}{2}g^{kl}\partial_ig_{kl} = \frac{1}{2}{g^{-1}}_{kl}\partial_ig_{kl}
% \end{align}
% This can be rewritten using $\tr-\log(A) = \log\det(A)$.  Our expression
% for the derivative involves,
% \begin{equation}
% \partial_i\log\det(g)=\partial_i\tr\log(A)=\tr\partial_i\log(A) = \tr[g^{-1}\partial_ig]
% \end{equation}
% We can then write the derivative operator as 
% \begin{equation}
% \langle \vect{q}|p_i|\psi\rangle 
% = -i\hbar\left(\partial_i +\frac{1}{2}\Gamma_i\right)\psi(\vect{q})
% \end{equation}

% This is not quite the full covariant derivative one might anticipate,
% \begin{equation}
% \nabla_iV^j = \partial_iV^j + \Gamma_{ik}^jV^k
% \end{equation}
% However, the wavefunction is a scalar, rather than a vector, and thus
% we are not tracking how the vector components are changed in a curved space. 
% Perhaps this would emerge naturally for spinors (or spin-1 particles - say for
% the photon?), where the wavefunction should be decomposed in terms of irreducible
% represenations of whatever group we are working with?

% We also need to specify the form of the matrix elements.  The matrix elements
% are:
% \begin{gather}
% \langle \vect{q}|\vect{q'}\rangle = \frac{1}{\sqrt{g}}\delta(\vect{q}-\vect{q'})\\
% \langle \vect{p}|\vect{p'}\rangle = \delta(\vect{p}-\vect{p'})\\
% \langle \vect{q}|\vect{p}\rangle = \frac{e^{ip_iq^i/\hbar}}{\sqrt{2\pi\hbar }g^{1/4}}.
% \end{gather}
% The first two relations ensure that $I_x^2=I_x$, $I_p^2=I_p$ by returning delta-functions
% with appropriate factors of the matrix.  
% The third ensures the overlap integral is independent of the basis used.  


% If we assume we have quantum Hamiltonian,
% \begin{equation}
% H = \frac{1}{2}\op{p}_ig^{ij}(\op{q})\op{p}_j
% \end{equation}
% then in position space, the Schrodinger equation is 
% \begin{equation}
% i\hbar \partial_t\psi = 
% -\frac{\hbar^2}{2}\frac{1}{g^{1/4}}\partial_i g^{1/4}g^{ij}g^{1/4}\partial_j \frac{1}{g^{1/4}}\psi(q)
%  = -\frac{\hbar^2}{2}\Delta_{\text{LB}}\psi
% \end{equation}
% This can be reordered into the form of the Laplace-Beltrami operator\footnote{
% Kleinert, H. G. \textit{Path Integrals in Quantum Mechanics, Statistics, 
% Polymer Physics and Financial Markets}, $5^{\text{th}}$ edition, Sec 1.13}
% where the differential operator is the Laplace-Beltrami operator.  
% The Laplace-Beltrami operator is the natural curved-space analogue of the 
% flat-space Laplacian, $\Delta_{\text{LB}}f = \nabla^\mu \nabla_\mu f$.  
% In curved space the divergence, and gradient are modified by the metric.
% The Wikipedia article on the Laplace-Beltrami operator defines 
% \begin{equation}
% \Delta_{LB} = \frac{1}{g^{1/2}}\partial_i g^{1/2}g^{ij} \partial_j
% \end{equation}
% which differs by some ordering terms from the proposed quantum operator.  
% The Laplace-Beltrami operator is given by 
% \begin{align}
% \nabla_i\nabla^i f &= \nabla_i(g^{in}\partial_n f)\\
% &= [\partial_m+\Gamma_{im}^i](g^{mn}\partial_n f)\\
% &= [\partial_m+(\partial_m\ln\sqrt{g})](g^{mn}\partial_n f)\\
% &= \frac{1}{\sqrt{g}}\partial_m[\sqrt{g}(g^{mn}\partial_n f)]
% \end{align}

% \section{Point transformations as effective potentials}

% I have found a paper by Gervais and Jevicki\footnote{Gervais, J.-L and  Jevicki,
%  A. \textit{Point Canonical Transformations in the Path Integral},
%  Nuclear Physics B, \textbf{110}, 93, (1976)} which covers similar ground 
% to what we are treading.  
% The following is my attempt to follow their work.
% They also cite a relevant paper by McLaughlin and Schulman\footnote{
% McLaughlin, D. W. and Schulman, L. S. \textit{Path Integrals in Curved Spaces}, 
% J. Math. Phys. \textbf{12}, 2520, (1971)}.

% \section{Following DeWitt}

% This section is where I will try to reproduce Bryce DeWitt's results
% \footnote{deWitt, B. S. \textit{Dynamical Theory in Curved Spaces I: A Review of the Classical and Quantum Action Principles}, 
% \\{Rev. Mod. Phys.}, \textbf{29}, 377,(1957)}.
%   I'll skip the material on transformation theory, etc, and just leap to the curved spaces.
%   We will need some of those results, but the classical material is a touch irrelevant.  

% \section{1D Example}

% Path integrals in curved spaces are a contentious topic.  
% Most authors who end up strayin into the field either missed out prior
% literature, or made mistakes.  Given the tedious algebra required by calculations
% this is understandable.  There are numerous books inveighing against all other
% approaches, and declaring their version to be the received truth on how to 
% properly write the path integral in curved space.  

% We don't need the full glory of those results, so we will focus our attention
% on just one dimension, and the TM potential directly.  We will look at a couple
% different approaches.    
% Let us apply these results to the 1D TM potential example.
% The single particle 
% \begin{equation}
%   \langle x|H_{TM}|\psi\rangle = \frac{1}{2}(-\nabla\frac{1}{\epsilon}\nabla+\omega^2)\psi(x)
% \end{equation}

% \subsection{Operator view}

% In this version, we interpret $\epsilon$ as a metric.  This implies the
% form of the momentum operators, and spatial identities are changed.  
% The spatial identity is
% \begin{equation}
%   I_X = \int dx\sqrt{\epsilon}|x\rangle\langle x|
% \end{equation}
% where the metric is $g=\epsilon$.
% The conjugate momentum operator is 
% \begin{equation}
%   \langle x|\op{p}|\psi\rangle = -i(\partial_x+\frac{1}{2}\partial_x\ln\sqrt{\epsilon})\psi
% \end{equation}
% The correction is related to the trace of the Christoffel symbols.
% In one dimension we will just use $\Gamma_x=\partial_x\ln\sqrt{\epsilon}$.
% The differential can then be written in operator form as
% \begin{align}
%   H &=\frac{1}{2}[-\partial_x\epsilon^{-1}\partial_x]\\
% %  &=\frac{1}{2}[(\op{p}-\frac{i}{2}\Gamma_x)\epsilon^{-1}(\op{p}-\frac{i}{2}\Gamma_x)]\\
%   &=\op{p}\epsilon^{-1}\op{p}-\frac{i}{2}\op{p}\epsilon^{-1}\Gamma_x
% -\frac{i}{2}\Gamma_x\epsilon^{-1}\op{p}-\frac{\Gamma_x^2}{4\epsilon}
%   \label{eq:H_op}
% \end{align}

% In order to evaluate the matrix elements we must operator-order the Hamiltonian.
% The three basic orderings are anti-standard, Weyl and standard orderings,
% each of which correspond to Ito, Stratonovich and anticipating stochastic
% calculus, which further correspond to expanding the path integral about the
% pre-point, mid-point and post-point.

% We will use the commutation relations $[x,p]=i$.  Prior works were considering
% work in a quantum context, where explicit factors of $\hbar$ are required.  
% Since we will be working with the 

% \subsubsection{Standard ordering}

% First, let us put the Hamiltonian (\ref{eq:H_op}) into anti-standard ordering
% with all position operators to the left of momentum operators.  

% We will have ample opportunity to use: $[f(x),p]=if'$, which implies
% $fp = pf +if'$ and $pf = fp-if'$
% \begin{align}
%   H&=(\op{p}-\frac{i}{2}\Gamma_x)\epsilon^{-1}(\op{p}-\frac{i}{2}\Gamma_x)\\
%    % &=(\frac{1}{\epsilon}p+i\frac{\epsilon'}{\epsilon^2})(p-\frac{i}{2}\Gamma_x)
%    % -\frac{i}{2\epsilon}\Gamma_x(p-\frac{i}{2}\Gamma_x)\\
%    % &=\frac{1}{\epsilon}p^2-\frac{i}{2\epsilon}(\Gamma_xp-i\Gamma_x')+i\frac{\epsilon'}{\epsilon^2}(p-\frac{i}{2}\Gamma_x)
%    % -\frac{i\Gamma_x}{2\epsilon}p-\frac{\Gamma_x^2}{4\epsilon}\\
%    &=\frac{1}{\epsilon}p^2
%    +i\frac{\Gamma_x}{\epsilon}p+\frac{3\Gamma_x^2}{4\epsilon}-\frac{\Gamma_x'}{2\epsilon}
% \end{align}
% Retrying from different starting point
% \begin{align}
% H&=p\epsilon^{-1}  p-p\epsilon^{-1}\frac{i}{2}\Gamma_x
% -\frac{i}{2}\Gamma_x\epsilon^{-1}p-\frac{\Gamma_x^2}{4\epsilon}\\
% % &=(\epsilon^{-1}p +i\frac{\epsilon'}{\epsilon^2})p
% % -\frac{i}{2}[\epsilon^{-1}\Gamma_xp -i(\frac{\Gamma_x'}{\epsilon}-\frac{\Gamma\epsilon'}{\epsilon^2})]
% % -\frac{i}{2}\Gamma_x\epsilon^{-1}p-\frac{\Gamma_x^2}{4\epsilon}\\
% &=(\epsilon^{-1}p +i\frac{\Gamma_x}{\epsilon})p
%  -\frac{\Gamma_x'}{2\epsilon}+\frac{3\Gamma_x^2}{\epsilon}
% \end{align}


% \subsubsection{Anti-Standard ordering}
% Now move momentum operators to the left.  $fp = pf+if'$
% \begin{align}
%   H&=(p-\frac{i}{2}\Gamma_x)\epsilon^{-1}(p-\frac{i}{2}\Gamma_x)\\
% % &=p\epsilon^{-1}p-p\epsilon^{-1}\frac{i}{2}\Gamma_x
% % -\frac{i}{2}\Gamma_x\epsilon^{-1}p-\frac{\Gamma_x^2}{4\epsilon}\\
% % &=p\left(p\epsilon^{-1}-\frac{i\epsilon'}{\epsilon^2}\right)-p\epsilon^{-1}\frac{i}{2}\Gamma_x
% % -\frac{i}{2}\left[p\Gamma_x\epsilon^{-1}
% %   +i\left(\frac{\Gamma_x'}{\epsilon}-\Gamma_x\frac{\epsilon'}{\epsilon^2}\right) \right]
% % -\frac{\Gamma_x^2}{4\epsilon}\\
% &=p\left(p\epsilon^{-1}-\frac{3i\Gamma_x}{\epsilon}\right)
% +\left(\frac{\Gamma_x'}{2\epsilon}-\frac{5\Gamma_x^2}{4\epsilon}\right)
% \end{align}

% \subsubsection{Weyl ordering}

% $fp-pf = if'\rightarrow fp = pf+if'$
% We can then use these pieces to symmetrically order the total as: 
% \begin{align}
% H_W&=\frac{1}{4}\left[p(p\epsilon^{-1}-i\frac{\epsilon'}{\epsilon^2})+ 2p\epsilon^{-1}p +
%   (\epsilon^{-1}p+i\frac{\epsilon'}{\epsilon^2})p\right]
% -\frac{i}{2}\left(p\frac{\Gamma_x}{\epsilon}+\frac{\Gamma_x}{\epsilon}p\right)
% -\frac{\Gamma_x^2}{4\epsilon}\\
% % &=(p^2\epsilon^{-1})_W 
% % -i\left(p\frac{\Gamma_x}{\epsilon}\right)_W
% % -\frac{\Gamma_x^2}{4\epsilon}+i\left[\frac{\epsilon'}{\epsilon^2},p\right] \\
% &=[(p+\frac{i}{2}\Gamma_x)^2\epsilon^{-1}]_W 
% +\left(\frac{2(\epsilon')^2}{\epsilon^3}-\frac{\epsilon''}{\epsilon^2}\right)
% \end{align}
% The potential term can be rewritten as
% \begin{align}
%   (\partial_x\ln\sqrt{\epsilon}) = \frac{\epsilon'}{2\epsilon}\\
%   (\partial_x^2\ln\sqrt{\epsilon}) = \frac{\epsilon''}{2\epsilon}-\frac{\epsilon'^2}{2\epsilon^2}
% \end{align}
% Accounting for the factor of two we have suppressed, the potential can be written
% \begin{align}
%  V'&= \left(\frac{(\epsilon')^2}{\epsilon^3}-\frac{\epsilon''}{2\epsilon^2}\right)\\
%  &=\frac{1}{\epsilon}\left[2(\partial_x\ln\sqrt{\epsilon})^2-\partial_x^2\ln\sqrt{\epsilon}\right],
% \end{align}
% which has an additional factor of $\epsilon$, and $\ln\sqrt{\epsilon}$ relative to the
% TM potential.  Maybe this will be eaten in the path integral? 

% We want to compute the following path integral.  
% \begin{equation}
%   E = \log Z = -\int \frac{d\cT}{\cT}\tr[\exp(-H\cT)]
% \end{equation}
% If we assume that the metric term is only on one dimension, then this becomes a path integral,
% \begin{align}
%   E &= -\int \frac{d\cT}{\cT}\int \prod_k\frac{dx_k dp_k}{2\pi}
% \exp\left[-\frac{\Delta T}{2\epsilon}(p_k-i\Gamma_k/2)^2 -V'_k\Delta T+ip_k\Delta x_k\right]\\
% % &= -\int \frac{d\cT}{\cT}\int \prod_k\frac{dx_k dp_k}{2\pi}\sqrt{\epsilon} 
% % \exp\left[-\frac{\Delta T}{2\epsilon}\left(p_k^2-2i\frac{\Gamma_k}{2}p_k-2i\frac{\epsilon_k}{\Delta T}\Delta x_kp_k\right)
% % +\Gamma_k^2\frac{\Delta T}{8\epsilon} -V'_k\Delta T\right]\\
% &= -\int \frac{d\cT}{\cT}\int \prod_kdx_k\sqrt{\epsilon} 
% \sqrt{\frac{\epsilon_k}{2\pi\Delta T}}
% \exp\left[-\frac{\epsilon_k}{2\Delta T}\left(\Delta x_k+\frac{\Gamma_k\Delta T}{2\epsilon_k}\right)^2
% +\Gamma_k^2\frac{\Delta T}{8\epsilon} -V'_k\Delta T\right]
% \end{align}
% where we have used the Weyl correspondence, $[a(\op{x})b(\op{p})]_W\rightarrow a(\bar{x})b(p)$
% Note that the overlap between spatial and momentum states also changes. 
% \begin{equation}
% \langle x|p\rangle = \frac{e^{ipx}}{\epsilon^{1/4}}
% \end{equation}

% \section{2013-April}

% We will follow work\footnote{Girotti, H.O. and Simoes, T.J.M. 
% \textit{A Generalized Treatment of Point Canonical Transformations in the Path Integral}, Il Nuovo Cimento, \textbf{74}, 59, (1983)} 
% extending the point transformation approach for handling arbitrary orderings.
%   This covers all orderings, but also offers proof for the higher order moments.  

% We start from a flat space, with 
% \begin{equation}
% H = \frac{1}{2}\sum_j p_j^2 + V(q),
% \end{equation}
% with $[q_i,p_j]=i\hbar\delta_{ij}$.  The propagator is 
% \begin{equation}
% K(q_f,t_f; q_i,t_i) = \int \prod_k \frac{d^nq_k}{(2\pi i \hbar \Delta T)^{n/2}} 
% \exp\left[ \frac{i}{\hbar}\left(\frac{(q_{i,k+1}-q_{i,k})^2}{2\Delta T} -\Delta T V(q_k)\right)\right]
% \end{equation}



% \section{Transforming a One-Dimensional path integral}
% \subsection{Transformation}

% Let's now introduce some nonlinear point transformation where 
% \begin{equation}
% q(t) = F[x(t)],
% \end{equation}
% where $x$ are the new ``curvilinear'' coordinates.  We have metric tensor 
% \begin{equation}
% g = \frac{\partial F}{\partial x}\frac{\partial F}{\partial x} = (\partial_xF)^2
% \end{equation}
% Now we normally need the Jacobian determinant for the change of variables.  
% \begin{equation}
% J = \sqrt{g}
% \end{equation}
% We have a flat space propagator which satisfies 
% \begin{align}
% \psi(q_n,t_n) &= \int dq_0 K(q_n,t_n; q_0,t_0)\psi(q_0,t_0)\\
% &= \int dx \sqrt{g(x_0)}K(x_n,t_n; x_0,t_0)\psi(x_0,t_0)
% \end{align}
% So the path integral in these new coordinates is
% \begin{equation}
% K(x_n,t_n; x_0,t_0) = \frac{1}{\sqrt{g(x_0)}}\int \prod_{k=1}^{n-1} dx_k\frac{\sqrt{g_k}}{\sqrt{2\pi i \hbar \Delta T}} \exp\left[ \frac{i}{\hbar}\left(\frac{(F_{k+1}-F_{k})^2}{2\Delta T} -\Delta T V[F(x_k)]\right)\right],
% \end{equation}
% where we had to multiply and divide by $\sqrt{g(x_0)}$. 
% \subsection{Ordering and Expanding.}

% We will expand these operators around 
% \begin{equation}
% x_\alpha(k) = \left(\alpha+\frac{1}{2}\right)x(k+1) + \left(\frac{1}{2}-\alpha\right)x(k)
% \end{equation}
% with $-1/2\le \alpha \le 1/2$.  Then we can expand in Taylor series 
% \begin{gather}
% \boxed{x(k)-x_\alpha   = -\left(\alpha+\frac{1}{2}\right)\Delta x}\\
% \boxed{x(k+1)-x_\alpha = \left(\frac{1}{2}-\alpha\right)\Delta x}
% \end{gather}

% For notational ease, let's define $\alpha_\pm = \frac{1}{2} \pm \alpha$, with $\alpha_++\alpha_- =1, \alpha_+-\alpha_- = 2\alpha$. So we have 
% \begin{gather}
% x_\alpha = \alpha_+ x(k+1) + \alpha_-x(k)\\
% x(k)   = x_\alpha-\alpha_+\Delta x \\
% x(k+1) = x_\alpha+\alpha_-\Delta x
% \end{gather}

% \subsubsection{Kinetic term}
% We now need to expand out to order $\Delta T$.  We will do this first in the exponential.  
% \begin{align}
% F[x(k+1)] = &F(x_\alpha) + \alpha_-\Delta x\partial_xF +\frac{1}{2}\alpha_-^2\Delta x^2\partial_x^2F(x_\alpha) +\frac{1}{6}\alpha_-^3\Delta x^3\partial_x^3F(x_\alpha) \\
% F[x(k)] = &F(x_\alpha) - \alpha_+\Delta x\partial_xF +\frac{1}{2}\alpha_+^2\Delta x^2\partial_x^2 F(x_\alpha) -\frac{1}{6}\alpha_+^3\Delta x^3\partial_x^3 F(x_\alpha)
% \end{align}
% Now we anticipate that $\Delta x\sim\Delta T$.  We will have to treat terms like $(\Delta x)^2\sim \Delta T,$ $(\Delta x)^3/\Delta T\sim \sqrt{\Delta T},$ $(\Delta x)^4/\Delta T\sim \Delta T,$ and $(\Delta x)^6/\Delta T^2\sim \Delta T$.  
% So we have 
% \begin{align}
% F(k+1)-F(k)&  = (\alpha_-+\alpha_+)\Delta x F' +\frac{1}{2}(\alpha_-^2-\alpha_+^2)\Delta x^2 F'' +\frac{1}{6}(\alpha_-^3+\alpha_+^3)\Delta x^3 F''' \\
% &  = \Delta xF' -\alpha\Delta x^2F''(x_\alpha) +\left(\frac{1}{2}\alpha^2 + \frac{1}{24}\right)\Delta x^3  F'''(x_\alpha) 
% \end{align}
% Now square it, and divide by $\Delta T$, and work to order $\Delta T$.  Let us now use $F' = \sqrt{g}$.  
% \begin{align}
% \frac{[F^r(k+1)-F^r(k)]^2}{\Delta T} & = \frac{1}{\Delta T}\left[\Delta xF' -\alpha\Delta x^2F''(x_\alpha) +\left(\frac{1}{2}\alpha^2 + \frac{1}{24}\right)\Delta x^3  F'''(x_\alpha) \right]^2\\
%  % =& \frac{1}{\Delta T}\bigg[\Delta x^2(F')^2 -2\alpha\Delta x^3F' F'' + \alpha^2\Delta x^4F''F'' +\left(\alpha^2 + \frac{1}{12}\right)\Delta x^4F' F'''\bigg]\\
%  % =& \frac{1}{\Delta T}\bigg[\Delta x^2 g -2\alpha\Delta x^3\sqrt{g}\sqrt{g}' + \alpha^2\Delta x^4(\sqrt{g}')^2 +\left(\alpha^2 + \frac{1}{12}\right)\Delta x^4\sqrt{g}\sqrt{g}''\bigg]\\
%  =& \frac{1}{\Delta T}\bigg[\Delta x^2 g -\alpha\Delta x^3g' + \alpha^2\Delta x^4\frac{(g')^2}{4g} +\left(\alpha^2 + \frac{1}{12}\right)\Delta x^4\left(\frac{g''}{2}  - \frac{(g')^2}{4g}\right)\bigg]
% \end{align}

% \subsubsection{Normalization}
% In addition, we must also carry out the expansion for the normalization factor.  Let's expand this out as 
% \begin{align}
% \sqrt{g[x(k)]} & = \sqrt{g(x_\alpha)} + (x(k)-x_\alpha)\frac{g'}{2\sqrt{g}} + \frac{1}{2}(x(k)-x_\alpha)^2\left(\frac{g''}{2\sqrt{g}} - \frac{(g')^2}{4 g^{3/2}}\right)\\
% & = \sqrt{g(x_\alpha)}\left[1 -\alpha_+\Delta x\frac{g'}{2g} + \frac{1}{2}\alpha_+^2\Delta x^2\left(\frac{g''}{2g} - \frac{(g')^2}{4 g^{2}}\right)\right]
% \end{align}

% \subsubsection{Expanding the exponential and normalization}

% So the exponential expansion is 
% \begin{align}
% K & = \frac{1}{\sqrt{g(x_0)}}\int \prod_{k=1}^{n-1} dx_k\frac{\sqrt{g_k}}{\sqrt{2\pi i \hbar \Delta T}} \exp\left[ \frac{i}{\hbar}\left(\frac{(F_{k+1}-F_{k})^2}{2\Delta T} -\Delta T V[F(x_k)]\right)\right]\\
% & = \int \prod_{j=1}^{n-1} dx_j \sqrt{\frac{g(x_\alpha)}{2\pi i \hbar\Delta T}}   e^{i\frac{g(x_\alpha)\Delta x^2}{2\hbar\Delta T} -iV(x_\alpha)\Delta T}\left[1 -\alpha_+\Delta x\frac{g'}{2g} + \frac{1}{2}\alpha_+^2\Delta x^2\left(\frac{g''}{2g} - \frac{(g')^2}{4 g^{2}}\right)\right]\nonumber\\
% &\times \exp\bigg\{ -\frac{i\alpha}{2\hbar}g'\frac{\Delta x^3}{\Delta T} + \frac{i\alpha^2}{8\hbar}\frac{(g')^2}{g}\frac{\Delta x^4}{\Delta T} +\frac{i}{2\hbar}\left(\alpha^2 + \frac{1}{12}\right)\frac{\Delta x^4}{\Delta T}\left(\frac{g''}{2}  - \frac{(g')^2}{4g}\right)\bigg]\bigg\}
% \end{align}
% Let's now expand that exponential 
% \begin{align}% 
% K& = \frac{1}{\sqrt{g(x_0)}}\int \prod_{j=1}^{n-1} dx_j \sqrt{\frac{g(x_\alpha)}{2\pi i \hbar\Delta T}}   e^{i\frac{g(x_\alpha)\Delta x^2}{2\hbar\Delta T} -iV(x_\alpha)\Delta T}\left[1 -\alpha_+\Delta x\frac{g'}{2g} + \frac{1}{2}\alpha_+^2\Delta x^2\left(\frac{g''}{2g} - \frac{(g')^2}{4 g^{2}}\right)\right]\nonumber\\
% &\times \bigg\{1 -\frac{i\alpha}{2\hbar}g'\frac{\Delta x^3}{\Delta T}  + \frac{i\alpha^2}{8\hbar}\frac{(g')^2}{g}\frac{\Delta x^4}{\Delta T}+\frac{i}{2\hbar}\left(\alpha^2 + \frac{1}{12}\right)\frac{\Delta x^4}{\Delta T}\left(\frac{g''}{2}  - \frac{(g')^2}{4g}\right) - \frac{\alpha^2}{8\hbar^2}(g')^2\frac{\Delta x^6}{(\Delta T)^2}\bigg]\bigg\}\\
% & = \int \prod_{j=1}^{n-1} dx_j \sqrt{\frac{g(x_\alpha)}{2\pi i \hbar\Delta T}}   e^{i\frac{g(x_\alpha)\Delta x^2}{2\hbar\Delta T} -iV(x_\alpha)\Delta T} C,
% \end{align}
% where the prefactor is 
% \begin{align}
% C =&1 -\alpha_+\Delta x\frac{g'}{2g} + \frac{1}{2}\alpha_+^2\Delta x^2\left(\frac{g''}{2g} - \frac{(g')^2}{4 g^{2}}\right)-\frac{i\alpha}{2\hbar}g'\frac{\Delta x^3}{\Delta T}\nonumber\\
% &   + \left[\frac{i\alpha^2}{8\hbar}\frac{(g')^2}{g}+\frac{i}{2\hbar}\left(\alpha^2 + \frac{1}{12}\right)\left(\frac{g''}{2}  - \frac{(g')^2}{4g}\right) +\frac{i\alpha_+\alpha}{4\hbar}\frac{(g')^2}{g}\right]\frac{\Delta x^4}{\Delta T} \nonumber\\
% &- \frac{\alpha^2}{8\hbar^2}(g')^2\frac{\Delta x^6}{(\Delta T)^2}.
% \end{align}


% \subsection{Replacing higher moments with their averages}

% Using the above moment theorems We can then make the following replacements which are correct to $\order(\Delta T)$.  

% \begin{align}
% \Delta x^2 \dot{=} & i\hbar\frac{1}{g}\Delta T\\
% \frac{\Delta x^3}{\Delta T} \dot{=}& 3i\hbar \frac{1}{g} \Delta x\\
% \frac{\Delta x^4}{\Delta T} \dot{=}& 3(i\hbar)^2\frac{1}{g^2}\Delta T\\
% \frac{\Delta x^6}{\Delta T^2} \dot{=}& 15(i\hbar)^3\frac{1}{g^3}\Delta T
% \end{align}

% The new averaged prefactor is 
% \begin{align}
% C =&1 -\alpha_+\Delta x\frac{g'}{2g} + \frac{1}{2}\alpha_+^2\left(\frac{g''}{2g} - \frac{(g')^2}{4 g^{2}}\right)\left(i\hbar\frac{\Delta T}{g}\right)-\frac{i\alpha}{2\hbar}g'\left(3i\hbar \frac{\Delta x}{g}\right)\nonumber\\
% &   + \left[\frac{i\alpha^2}{8\hbar}\frac{(g')^2}{g}+\frac{i}{2\hbar}\left(\alpha^2 + \frac{1}{12}\right)\left(\frac{g''}{2}  - \frac{(g')^2}{4g}\right) +\frac{i\alpha_+\alpha}{4\hbar}\frac{(g')^2}{g}\right]\left(3(i\hbar)^2\frac{1}{g^2}\Delta T\right) \nonumber\\
% &- \frac{\alpha^2}{8\hbar^2}(g')^2\left(15(i\hbar)^3\frac{1}{g^3}\Delta T\right)\\
% % =&1 +\alpha\Delta x\frac{g'}{g}-\frac{g'}{4g}\Delta x + \frac{i\hbar}{2}\left(\alpha +\frac{1}{2}\right)^2\left(\frac{g''}{2g^2} - \frac{(g')^2}{4 g^{3}}\right)\Delta T\nonumber\\
% % & -  3i\hbar\Delta T \left[\frac{3\alpha^2}{8}\frac{(g')^2}{g^3}+\frac{1}{2}\left(\alpha^2 + \frac{1}{12}\right)\left(\frac{g''}{2 g^2}  - \frac{(g')^2}{4g^3}\right) +\frac{\alpha}{8}\frac{(g')^2}{g^3}\right]+15i\hbar\frac{(g')^2}{g^3} \frac{\alpha^2}{8}\Delta T\\
% % =&1 +\left(\alpha-\frac{1}{4}\right)\Delta x\frac{g'}{g} + \frac{i\hbar}{2}\left[ \left(\alpha +\frac{1}{2}\right)^2 -3\left(\alpha^2 + \frac{1}{12}\right)\right]\left(\frac{g''}{2g^2} - \frac{(g')^2}{4 g^{3}}\right)\Delta T\nonumber\\
% % & +  \frac{3i}{4}\hbar\Delta T \left(\alpha^2-\frac{\alpha}{2}\right)\frac{(g')^2}{g^3}\\
% =&1 +\left(\alpha-\frac{1}{4}\right)\Delta x\frac{g'}{g} + i\hbar\left(\alpha^2 -\frac{\alpha}{2}\right)\left( \frac{(g')^2}{ g^{3}}-\frac{g''}{2g^2}\right)\Delta T 
% \end{align}


% \subsection{Moment theorems}
% This follows (and fixed a typo in) Dan's work.  
% Let us now consider evaluating Gaussian moments of the form,
% \begin{equation}
% E_{\sigma(x)}[x^n] = \int dx\, x^n \frac{e^{-\frac{x^2}{2\sigma^2(x)\Delta T}}}{\sqrt{2\pi\sigma^2(x)\Delta T}}
% \end{equation}
% We will treat $x$ as $\order(\sqrt{\Delta T})$
% Now expand about the origin
% \begin{gather}
% \sigma(x) \approx \sigma(\mu) + (x-\mu)\sigma'(\mu)\\
% \frac{1}{\sigma(x)} \approx \frac{1}{\sigma(\mu)} -\frac{\sigma'(\mu)}{\sigma^2(\mu)}(x-\mu)\\
% \frac{1}{\sigma^2(x)} \approx \frac{1}{\sigma^2(\mu)} -\frac{2\sigma'(\mu)}{\sigma^3(\mu)}(x-\mu)
% \end{gather}
% We have to deal with moments like $\Delta x^{2n}/\Delta T^{2(n-1)}$ and $\Delta x^{2n+1}/\Delta x^{2n-1}$.  
% Now note that our correction is already $\Delta T$ for even moments, or $\order(\sqrt{\Delta T})$ for the odd moments.  So for even moments, we can drop the corrections, whereas for odd moments we will have to keep the first order correction.  

% \subsubsection{Normal Gaussian Moment theorem}

% Let's now think about the recursion relations for the Gaussian moment theorem.  
% \begin{align}
% E[x^n] =& \int dx\,x^n \frac{e^{-\frac{x^2}{2\sigma^2}}}{\sqrt{2\pi\sigma^2}}\\
%  =&  -\sigma^2x^{n-1}e^{-\frac{x^2}{2\sigma^2}}\bigg|_{x=-\infty}^{\infty} + \int dx\,(n-1)\sigma^2x^{n-2} \frac{e^{-\frac{x^2}{2\sigma^2}}}{\sqrt{2\pi\sigma^2}}\\
%  =&  (n-1)\sigma^2E[x^{n-2}]
% \end{align}

% Now for $\mu = 0$, only $m=n$ contributes, and 
% \begin{align}
% E[x^n] = (n-1)\sigma^2E[x^{n-2}]
% \end{align}

% So for even moments you get $n = 2k$
% \begin{equation}
% E[x^{2k}] = (2k-1)!!\sigma^{2k},\quad k =1,2,\ldots
% \end{equation}
% where $n!! = n(n-2)(n-4)(n-6)\ldots 1$.  (The recursion truncates at $2k-2n = 0$, or $n = k$).  For odd moments we get $n=2k-1$
% \begin{equation}
% E[x^{2k-1}] = (2k-2)\sigma^2E[x^{2k-3}] = (2k-2)!!\sigma^{2k-2}E[x]
% \end{equation}
% Repeat for $m$ steps until $2k-1-2m = 1$, or $m=k-1$.  

% \subsubsection{Odd moments}

% Note that we want to simplify $x^3/\Delta T$, which is order $\Delta T^{1/2}$.  
% \begin{align}
%   E_{\sigma(x)}\left[\frac{x^{2n+1}}{\Delta T^{n}}\right] =& \int dx\, \frac{x^{2n+1}}{\Delta T^{n}} \frac{e^{-\frac{x^2}{2\sigma^2(x)\Delta T}}}{\sqrt{2\pi\sigma^2(x)\Delta T}}\\
% % \approx&  \int dx\, \frac{x^{2n+1}}{\Delta T^{n}}\frac{e^{-\frac{x^2}{2\sigma^2\Delta T}}}{\sqrt{2\pi\sigma^2\Delta T}} \left[1 - x\frac{\sigma'}{\sigma}  \right]\left[1 + \frac{2\sigma'}{\sigma}\frac{x^3}{2\sigma^2\Delta T}\right]\\
%  \approx&  \int dx\, \frac{e^{-\frac{x^2}{2\sigma^2\Delta T}}}{\sqrt{2\pi\sigma^2\Delta T}} \left[\frac{x^{2n+1}}{\Delta T^{n}} - \frac{x^{2n+2}}{\Delta T^{n}}\frac{\sigma'}{\sigma}  + \frac{x^{2n+4}}{\Delta T^{n+1}}\frac{\sigma'}{\sigma^3}\right]
% \end{align}
% Where we expanded everything to $\order(\Delta T^{1/2})$, since the coefficient is already $\order\Delta T^{1/2}$.  Now we integrate by parts 
% \begin{align}
%  E_{\sigma(x)}\left[\frac{x^{2n+1}}{\Delta T^{n}}\right]=&-\sigma^2\Delta T\frac{e^{-\frac{x^2}{2\sigma^2\Delta T}}}{\sqrt{2\pi\sigma^2\Delta T}} \left[\frac{x^{2n}}{\Delta T^{n}} - \frac{x^{2n+1}}{\Delta T^{n}}\frac{\sigma'}{\sigma}  + \frac{x^{2n+3}}{\Delta T^{n+1}}\frac{\sigma'}{\sigma^3}\right]_{x=-\infty}^{\infty}\nonumber\\
% &+\sigma^2\int dx\, \frac{e^{-\frac{x^2}{2\sigma^2\Delta T}}}{\sqrt{2\pi\sigma^2\Delta T}} \left[2n\frac{x^{2n-1}}{\Delta T^{n-1}} - \frac{(2n+1)x^{2n}}{\Delta T^{n-1}}\frac{\sigma'}{\sigma}  + \frac{(2n+3)x^{2n+2}}{\Delta T^{n}}\frac{\sigma'}{\sigma^3}\right]\\
% =&\sigma^2\int dx\, \frac{x^{2n-1}}{\Delta T^{n-1}}\frac{e^{-\frac{x^2}{2\sigma^2\Delta T}}}{\sqrt{2\pi\sigma^2\Delta T}} \left[2n - (2n+1)x\frac{\sigma'}{\sigma}  + \frac{(2n+3)x^{3}}{\Delta T}\frac{\sigma'}{\sigma^3}\right]
% \end{align}

% Now isolate the coefficient of the lower moment we want.  Integrate by parts once more on the extra term.  
% \begin{align}
%  E_{\sigma(x)}\left[\frac{x^{2n+1}}{\Delta T^{n}}\right]=&(2n+1)\sigma^2\int dx\, \frac{x^{2n-1}}{\Delta T^{n-1}}\frac{e^{-\frac{x^2}{2\sigma^2\Delta T}}}{\sqrt{2\pi\sigma^2\Delta T}} \left[1 - x\frac{\sigma'}{\sigma}  + \frac{x^{3}}{\Delta T}\frac{\sigma'}{\sigma^3}\right] \nonumber \\
% &+\int dx\, \frac{x^{2n-1}}{\Delta T^{n-1}}\frac{e^{-\frac{x^2}{2\sigma^2\Delta T}}}{\sqrt{2\pi\sigma^2\Delta T}} \left[-\sigma^2+ \frac{2x^{3}}{\Delta T}\frac{\sigma'}{\sigma}\right] \\
% % =&(2n+1)\sigma^2E_{\sigma(x)}\left[\frac{x^{2n-1}}{\Delta T^{n-1}}\right]+2\sigma'\sigma \int dx\, \frac{e^{-\frac{x^2}{2\sigma^2\Delta T}}}{\sqrt{2\pi\sigma^2\Delta T}} \frac{x^{2n+1}}{\Delta T^{n}}\\
% % =&(2n+1)\int dx \frac{e^{-\frac{x^2}{2\sigma^2(x)\Delta T}}}{\sqrt{2\pi\sigma^2(x)\Delta T}}\frac{x^{2n-1}}{\Delta T^{n-1}}\left[\sigma^2 + 2\sigma'\sigma\right]\\
%  \approx&(2n+1)\int dx \frac{e^{-\frac{x^2}{2\sigma^2(x)\Delta T}}}{\sqrt{2\pi\sigma^2(x)\Delta T}}\frac{x^{2n-1}}{\Delta T^{n-1}}\sigma^2(x)\\
% %=&(2n+1)E_{\sigma(x)}\left[\sigma^2(x)\frac{x^{2n-1}}{\Delta T^{n-1}}\right]\\
% =&(2n+1)!!E_{\sigma(x)}\left[\sigma^{2n}(x)x\right]
% \end{align}
% In the last line we used the recursion relation $n$ times.  

%  \subsection{Dan's log trick}

% We have terms, which we can exponentiate as 
% \begin{equation}
% 1 + \left(\alpha-\frac{1}{4}\right)\frac{g'}{g}\Delta x = \exp\left[\left(\alpha - \frac{1}{4}\right)\frac{g'}{g}\Delta x-\frac{1}{2}\left(\alpha-\frac{1}{4}\right)^2\frac{g'^2}{g^2}\Delta x^2\right]
% \end{equation}
% We will rewrite this as the variation of a logarithm.  We will then be able to relate these linear terms to a normalization factor.  We can expand the log out to second order as 
% \begin{equation}
% \log(y +dy) = \log(y) +\frac{dy}{y} - \frac{1}{2}\frac{dy^2}{y^2}.
% \end{equation}
% We will expand functions of $x_{k+1}$ and $x_k$ around  $x_\alpha$.  
% \begin{align}
% \log\left[\frac{g(x_{k+1})}{g(x_k)}\right] =& \log[g(x_\alpha+\alpha_-\Delta x)] - \log[ g(x_\alpha -\alpha_+\Delta x)]\\
% % =& \log\left[g(x_{\alpha})+\alpha_-\Delta x g' + \alpha_-^2\frac{\Delta x^2}{2} g''\right] - \log\left[g(x_{\alpha})-\alpha_+\Delta x g' + \alpha_+^2\frac{\Delta x^2}{2} g''\right]\\
% % =& \left(\log[g(x_\alpha)] + \alpha_-\Delta x \frac{g'}{g} + \frac{\alpha^2_-\Delta x^2}{2}\frac{g''}{g} - \alpha_-^2\frac{\Delta x^2}{2}\frac{(g')^2}{g^2}\right) \nonumber\\
% % &- \left(\log[g(x_\alpha)] - \alpha_+\Delta x \frac{g'}{g} + \frac{\alpha^2_+\Delta x^2}{2}\frac{g''}{g} - \alpha_+^2\frac{\Delta x^2}{2}\frac{(g')^2}{g^2}\right)\\
% % =& (\alpha_+ +\alpha_-)\frac{g'}{g}\Delta x -(\alpha^2_+-\alpha^2_-)\frac{g''}{2g}\Delta x^2 + (\alpha_+^2-\alpha_-^2)\frac{(g')^2}{2g^2}\\  
% =&   \Delta x \frac{g'}{g} -\alpha\Delta x^2\frac{g''}{g} + \alpha\Delta x^2\frac{(g')^2}{g^2}
% \end{align}
% So we can write 
% \begin{align}
% \frac{g'}{g}\Delta x = \log\left[\frac{g(x_{k+1})}{g(x_k)}\right]  +\alpha\Delta x^2\left(\frac{g''}{g} -\frac{(g')^2}{g^2}\right)\label{eq:velocity_log}
% \end{align}
% Plugging in this expansion for the exponent we have
% \begin{align}
% 1 + \left(\alpha-\frac{1}{4}\right)\frac{g'}{g}\Delta x  =& \exp\left[\left(\alpha - \frac{1}{4}\right)\frac{g'}{g}\Delta x-\frac{1}{2}\left(\alpha-\frac{1}{4}\right)^2\frac{g'^2}{g^2}\Delta x^2\right]\\
% %  =& \exp\left[\left(\alpha - \frac{1}{4}\right)\left[\log\left(\frac{g(x_{k+1})}{g(x_k)}\right)  +\alpha\Delta x^2\left(\frac{g''}{g} - \frac{(g')^2}{g^2}\right)\right]-\frac{1}{2}\left(\alpha-\frac{1}{4}\right)^2\frac{g'^2}{g^2}\Delta x^2\right]\\
% %  =& \exp\left[\left(\alpha - \frac{1}{4}\right)\log\left(\frac{g(x_{k+1})}{g(x_k)}\right)  +\alpha\left(\alpha - \frac{1}{4}\right)\Delta x^2\frac{g''}{g}\right]\nonumber\\
% % &\times\exp\left\{-\Delta x^2\frac{(g')^2}{g^2}\left[\frac{1}{2}\left(\alpha-\frac{1}{4}\right)^2+\alpha\left(\alpha - \frac{1}{4}\right)\right]\right\}\\
%  =& \exp\left[\left(\alpha - \frac{1}{4}\right)\log\left(\frac{g(x_{k+1})}{g(x_k)}\right)  +\alpha\left(\alpha - \frac{1}{4}\right)\Delta x^2\frac{g''}{g}\right]\nonumber\\
% &\times\exp\left[-\Delta x^2\frac{(g')^2}{g^2}\left(\frac{3\alpha^2}{2} -\frac{\alpha}{2}+\frac{1}{32}\right)\right]
% %=& \exp\left[\left(\alpha - \frac{1}{4}\right)\log\left(\frac{g(x_{k+1})}{g(x_k)}\right) +i\hbar\Delta T\left(\alpha^2 - \frac{\alpha}{4}\right)\frac{g''}{g^2} - i\hbar\Delta T\left[\alpha ^3+\frac{\alpha^2}{4}  - \frac{1}{32}\right]\frac{g'^2}{g^3}\right],
% \end{align}
% where in the third line we expanded the exponential to order $\Delta T$, used the moment theorem, and re-exponentiated the result.  So this is a bit unwieldy. 

% \subsection{Restoring the rest: The effective potential}

% Now we are in a position to restore the rest of this mess.  
% We can now exponentiate the remaining $\Delta T$ terms, which will give a potential, (recalling $e^{\frac{i}{\hbar}S}=e^{\frac{i\Delta T}{\hbar}(K-V)}$)
% \begin{align}
%   \frac{-V_{\text{tot}}+V_{\text{ext}}}{\hbar} &= \left(\alpha^2 - \frac{\alpha}{4}\right)\frac{g''}{g^2} -\left(\frac{3}{2}\alpha ^2-\frac{\alpha}{2}  + \frac{1}{32}\right)\frac{g'^2}{g^3}+\left(\alpha^2 -\frac{\alpha}{2}\right)\left( \frac{(g')^2}{ g^{3}}-\frac{g''}{2g^2}\right)\\
% &= \left(\alpha^2 - \frac{\alpha}{4} - \frac{\alpha^2}{2} +\frac{\alpha}{4}\right)\frac{g''}{g^2} +\left(\alpha^2 -\frac{\alpha}{2}-\frac{3}{2}\alpha ^2+\frac{\alpha}{2}- \frac{1}{32}\right)\frac{g'^2}{g^3}\\
% &=  \frac{\alpha^2}{2}\left(\frac{g''}{g^2}-\frac{g'^2}{g^3}\right) - \frac{1}{32}\frac{g'^2}{g^3}
% \end{align}
% So we have the following effective potential:
% \begin{equation}
% \boxed{V_{\text{eff}} = \frac{\hbar}{32}\frac{g'^2}{g^3} -\hbar\frac{\alpha^2}{2}\left(\frac{g''}{g^2}-\frac{g'^2}{g^3}\right) }
% \end{equation}
% \comment{Agrees with Dan's cases}

% \subsection{Propagator}

% So pulling all of this together we have 
% \begin{align}
% K(q_n,t_n; q_0,t_0) = \frac{1}{\sqrt{g(x_0)}}\int \prod_{k=1}^{n-1} dx_k\frac{\sqrt{g_k}}{\sqrt{2\pi i \hbar \Delta T}} \exp\left[ \frac{i}{2\Delta T\hbar}g_{ij}\Delta x^i\Delta x^j +\left(\alpha - \frac{1}{4}\right)\log\left(\frac{g(x_{k+1})}{g(x_k)}\right)-\frac{i}{\hbar}\Delta T [V+V_{\text{eff}}]\right]
% \end{align}
% Now we can use deal with the log by taking (recall we used this trick for all $\Delta x_k = x_{k+1}-x_k, k=0,1,....n-1$)
% \begin{align}
% \prod_{k=0}^{n-1}\exp\left[\left(\alpha - \frac{1}{4}\right)\log\left(\frac{g(x_{k+1})}{g(x_k)}\right)\right] & = \exp\left[\left(\alpha - \frac{1}{4}\right)\sum_{k=0}^{n-1}\left(\log[g(x_k)]-\log[g(x_{k+1})]\right)\right]\\
% & = \exp\left[\left(\frac{1}{4}-\alpha\right)\log\left(\frac{g(x_{n})}{g(x_0)}\right)\right]\\
% & = \left[\frac{g(x_{n})}{g(x_0)}\right]^{ \frac{1}{4}-\alpha}
% \end{align}
% so we have
% \begin{align}
% K(q_n,t_n; q_0,t_0) = \left[\frac{g(x_n)}{g(x_0)}\right]^{\frac{1}{4}-\alpha}\frac{1}{\sqrt{g(x_0)}}\int \prod_{k=1}^{n-1} dx_k\frac{\sqrt{g(x_{k,\alpha})}}{\sqrt{2\pi i \hbar \Delta T}} \exp\left[ \frac{i\Delta x^i\Delta x^j}{2\hbar\Delta T}g_{ij}(x_{k,\alpha}) -\frac{i}{\hbar}\Delta T [V+V_{\text{eff}}]\right]
% \end{align}
% \comment{Agrees with Dan}

% \subsection{Operator Orderings DeWitt's results}
% DeWitt claims that when setting up the path integral in curved coordinates, you find the relevant differential operator in the schrodinger equation is 
% \begin{equation}
% H = -g^{-1/2}\partial_i g^{1/2}g^{ij}\partial_j,
% \end{equation}
% where that is the Laplace-Beltrami operator

% Now using 
% \begin{equation}
% \langle x| p_i |\psi\rangle = -i\hbar g^{-1/4}\partial_i g^{1/4}
% \end{equation}
% Then we can write 
% \begin{align}
% H &= g^{-1/2}g^{1/4}(i\hbar g^{-1/4}\partial_i g^{1/4}) g^{-1/4} g^{1/2}g^{ij} g^{1/4}(i\hbar g^{-1/4} \partial_j g^{1/4} g^{-1/4}
% &= g^{-1/4}p_i  g^{1/2}g^{ij}p_j g^{-1/4}
% \end{align}

% Now if we assume 
% \begin{equation}
% [f(x),p] = i\hbar f'
% \end{equation}
% we can set about ordering this.  

% \subsection{Ito Ordering}

% For Ito order we want all $x$ operators to the left of their counterparts.  
% \begin{equation}
% H = p_ip_j g^{ij}
% \end{equation}

% So let's start commuting things.  
% \begin{align}
% H &= g^{-1/4}p_i  g^{1/2}g^{ij}p_j g^{-1/4}\\
% % & = \left[ p_i g^{-1/4} + i\hbar\partial_i(g^{-1/4})\right] g^{1/2}g^{ij}p_j g^{-1/4}\\
% % & = \left[ p_i g^{-1/4}  -i\frac{\hbar}{4}\frac{\partial_ig}{g^{5/4}}\right]g^{1/2}g^{ij}p_j g^{-1/4}\\
% % & =  p_i g^{1/4}g^{ij}p_j g^{-1/4}  -i\frac{\hbar}{4} \frac{\partial_ig}{g^{3/4}}g^{ij}p_j g^{-1/4}\\
% % & =  p_i [p_j g^{ij}g^{1/4} + i\hbar\partial_j(g^{1/4}g^{ij})]g^{-1/4}  -ip_j\frac{\hbar}{4} \frac{\partial_ig}{g^{3/4}}g^{ij} g^{-1/4} - \frac{\hbar^2}{4}\partial_j\left( \frac{\partial_ig}{g^{3/4}}g^{ij} \right)g^{-1/4}\\
% % & =  p_i p_j g^{ij} + i\hbar p_i\left( \frac{1}{4}\frac{\partial_jg}{g^{3/4}}g^{ij} + g^{1/4}\partial_j g^{ij}\right)g^{-1/4} -ip_j\frac{\hbar}{4} \frac{\partial_ig}{g^{3/4}}g^{ij} g^{-1/4} - \frac{\hbar^2}{4}\partial_j\left( \frac{\partial_ig}{g^{3/4}}g^{ij} \right)g^{-1/4}\\
% % & =  p_i p_j g^{ij} + i\hbar p_i \partial_j g^{ij} - \frac{\hbar^2}{4}\left(\frac{g^{ij}}{g^{3/4}}\partial_j\partial_i g - \frac{3}{4}g^{ij}\partial_ig \frac{\partial_jg}{g^{7/4}} + \frac{\partial_i g}{g^{3/4}}\partial_jg^{ij} \right)g^{-1/4}\\
% & =  p_i p_j g^{ij} + i\hbar p_i \partial_j g^{ij} - \frac{\hbar^2}{4}\left(g^{ij}\frac{\partial_j\partial_i g}{g} - \frac{3}{4}g^{ij}\frac{\partial_ig \partial_jg}{g^{2}} + \frac{\partial_i g}{g}\partial_jg^{ij} \right)
% \end{align}
% Let's now consider 1D, where $g^{ij} = 1/g, \det{g^{ij}} = g, g_{ij} = g$.  We then have 
% \begin{align}
% H & =  p^2g - i\hbar p \frac{g'}{g^2} - \frac{\hbar^2}{4}\left(\frac{g''}{g^2} 
%   - \frac{3}{4}\frac{(g')^2}{g^{3}} - \frac{(g')^2}{g^3} \right)
% \end{align}

% \subsubsection{One Dimension}

% So let's start commuting things.  
% \begin{align}
% H &= g^{-1/4}p  g^{-1/2}p g^{-1/4}\\
% % & = \left[ p g^{-1/4} + i\hbar(g^{-1/4})'\right] g^{-1/2}p_j g^{-1/4}\\
% % & = \left[ p g^{-1/4}  -i\frac{\hbar}{4}\frac{g'}{g^{5/4}}\right]g^{-1/2}p g^{-1/4}\\
% % & =  p g^{-3/4}p g^{-1/4}  -i\frac{\hbar}{4} \frac{g'}{g^{7/4}}p g^{-1/4}\\
% % & =  p(p g^{-1} - i\hbar \frac{3}{4}\frac{g'}{g^{2}})  -i\frac{\hbar}{4}p \frac{g'}{g^{2}} +\frac{\hbar^2}{4}\left(\frac{g''}{g^{7/4}}- \frac{7(g')^2}{4g^{11/4}}\right)'g^{-1/4}\\
% % & =  p\left(p g^{-1} - i\hbar \frac{3}{4}\frac{g'}{g^{2}}\right)  -i\frac{\hbar}{4}p \frac{g'}{g^{2}} +\frac{\hbar^2}{4}\left(\frac{g''}{g^{2}}- \frac{7(g')^2}{4g^{3}}\right)\\
% & =  p^2 g^{-1} - i\hbar p\frac{g'}{g^{2}}  +\frac{\hbar^2}{4}\left(\frac{g''}{g^{2}}- \frac{7(g')^2}{4g^{3}}\right)
% \end{align}

% On constructing the path integral we then have to evaluate 
% \begin{align}
% \langle x_f| e^{-i\frac{H}{\hbar}t} |x_i\rangle =& g^{-1/4}(x_f)g^{-1/4}(x_i)\int\prod_k \frac{dx_k dp_k}{2\pi\hbar}\, e^{\frac{i}{\hbar}p_k(x_{k+1}-x_k)}\nonumber\\
% &\times\exp\left[ -\frac{i}{2\hbar}\Delta t\left(  p^2g^{-1} - i\hbar p g^{-1}\partial_x \ln g  +\frac{\hbar^2}{4}\left(\frac{g''}{g^{2}}- \frac{7(g')^2}{4g^{3}}\right)\right)\right]
% \end{align}

% Complete the square in the momentum
% \begin{equation}
% -\frac{i\Delta T}{2\hbar}\left(\frac{p^2}{g} - i\hbar p\frac{g'}{g^2}  -2p_k\frac{(x_{k+1}-x_k)}{\Delta T} \right) = -\frac{i\Delta T}{2\hbar g}\left(p - g\frac{x_{k+1}-x_k}{\Delta T} - i\hbar \frac{g'}{2g}\right) +\frac{i\Delta T}{2\hbar g}\left(g\frac{x_{k+1}-x_k}{\Delta T} + i\hbar\frac{g'}{2g}\right)^2
% \end{equation}
% Let's now expand out that quadratic term, and use dan's log trick.  
% \begin{align}
% \langle x_f| e^{-i\frac{H}{\hbar}t} |x_i\rangle =& g_f^{-1/4}g_i^{-1/4}\int\prod_k dx_k\sqrt{\frac{g}{2\pi i\hbar\Delta T}}
% \exp\left[\frac{i\Delta T g }{2\hbar }\left(\frac{x_{k+1}-x_k}{\Delta T} + i\hbar\frac{g'}{2g^2}\right)^2 -\frac{i\hbar\Delta T}{8} \left(\frac{g''}{g^{2}}- \frac{7(g')^2}{4g^{3}}\right)\right]\\
% % =& g_f^{-1/4}g_i^{-1/4}\int\prod_k dx_k\sqrt{\frac{g}{2\pi i\hbar\Delta T}}
% % \exp\left[\frac{ig (x_{k+1}-x_k)^2}{2\hbar\Delta T } -\frac{i g}{2}(x_{k+1}-x_k)\frac{g'}{g^2}\right]\nonumber\\
% % &\times\exp\left[ -\frac{i\hbar\Delta T g}{8}\frac{(g')^2}{g^4} -\frac{i\hbar\Delta T}{8} \left(\frac{g''}{g^{2}}- \frac{7(g')^2}{4g^{3}}\right)\right]\\
% =& g_f^{-1/4}g_i^{-1/4}\int\prod_k dx_k\sqrt{\frac{g}{2\pi i\hbar\Delta T}}
% \exp\left[\frac{ig (x_{k+1}-x_k)^2}{2\hbar\Delta T } -(x_{k+1}-x_k)\frac{g'}{2g}\right]\nonumber\\
% &\times\exp\left[ -\frac{i\hbar\Delta T}{8} \left(\frac{g''}{g^{2}}- \frac{3(g')^2}{4g^{3}}\right)\right]
% \end{align}

% Now we can use 
% \begin{equation}
% \frac{g'}{g}\Delta x = \log\left[\frac{g(x_{k+1})}{g(x_k)}\right]  -\frac{1}{2}\Delta x^2\left(\frac{g''}{g} -\frac{(g')^2}{g^2}\right) = \log\left[\frac{g(x_{k+1})}{g(x_k)}\right]  -\frac{i\hbar\Delta T}{2}\left(\frac{g''}{g^2} -\frac{(g')^2}{g^3}\right)
% \end{equation}
% The propagator is 
% \begin{align}
% \langle x_f| e^{-i\frac{H}{\hbar}t} |x_i\rangle =& g_f^{-1/4}g_i^{-1/4}\int\prod_k dx_k\sqrt{\frac{g}{2\pi i\hbar\Delta T}}
% \exp\left[\frac{ig (x_{k+1}-x_k)^2}{2\hbar\Delta T } -\frac{1}{2}\log\left[\frac{g(x_{k+1})}{g(x_k)}\right] \right]\nonumber\\
% &\times\exp\left[ \frac{i\hbar\Delta T}{4}\left(\frac{g''}{g^2} -\frac{(g')^2}{g^3}\right) -\frac{i\hbar\Delta T}{8} \left(\frac{g''}{g^{2}}- \frac{3(g')^2}{4g^{3}}\right)\right]\\
% =& g_f^{-3/4}g_i^{1/4}\int\prod_k dx_k\sqrt{\frac{g}{2\pi i\hbar\Delta T}}
% \exp\left[\frac{ig (x_{k+1}-x_k)^2}{2\hbar\Delta T }+ \frac{i\hbar\Delta T}{8}\left(\frac{g''}{g^2} -\frac{(g')^2}{g^3}\right) -\frac{i\hbar\Delta T}{32}\frac{(g')^2}{g^3}\right],
% \end{align}
% which agrees with the transformed path integral picture.  


% \section{Dielectric as  a Metric}

% How about we force that differential operator into Laplace-Beltrami form
% \begin{align}
% H(x) &= -\frac{1}{\sqrt{\epsilon}}\nabla^2\frac{1}{\sqrt{\epsilon}} \\
% &= -\frac{1}{\sqrt{\epsilon}}\partial_x\left[\frac{1}{\sqrt{\epsilon}}\partial_x  + \left(\partial_x\frac{1 }{\sqrt{\epsilon} }\right)\right] 
% \end{align}
% Let's now try to put this in operator form, using $p = -ig^{-1/4}\partial_x g^{1/4}$.   Let us also choose $g = \epsilon$.
% \begin{align}
% H(x)&= -\frac{1}{\sqrt{g}}ig^{1/4}pg^{-1/4}\left[\frac{1}{\sqrt{g}}ig^{1/4}p g^{-1/4}  -i\left[\frac{1}{\sqrt{g}},p\right]\right]\\
% %&= g^{-1/4}pg^{-1/2}p g^{-1/4}  -g^{-1/4}pg^{-1/4}\left[\frac{1}{\sqrt{g}},p\right]\\
% &= g^{-1/4}p^2g^{-3/4}
% \end{align}

% Intriguingly, the coordinate transform viewpoint says:
% \begin{align}
% \Delta_{LB} =& \frac{1}{\sqrt{g}}\partial_x \frac{1}{\sqrt{g}}\partial_x\\
% =& \frac{\partial x}{\partial q}\partial_x \frac{\partial x}{\partial q}\partial_x\\
% =& \partial_q^2
% \end{align}
% where $\sqrt{g} = \frac{\partial q}{\partial x}$, if $q$ are the initial flat coordinates.  In which case, Dan's claim that the Laplace-Beltrami operator is ``trivial'' stands quite readily.  
% So if we take $\sqrt{g} = \sqrt{\epsilon}$ this says that lengths are increased by $\sqrt{\epsilon(x)} = n(x)$ in this space.   The coordinate transformation would have to be $q(x) = \int_{x_0}^x dx' \sqrt{\epsilon(x')}$.(\comment{where $q$ are the initial flat coordinates})


% Let's now try to develop this as a path integral.  
% \subsection{Ito ordering}
% If we Ito order we have
% \begin{align}
% H =& p^2 g^{-1} + [g^{-1/4},p^2]g^{-3/4}\\
% % =& p^2 g^{-1} -i\frac{1}{4}(g^{-5/4}g'p + p g^{-5/4}g')g^{-3/4}\\
% % =& p^2 g^{-1} -\frac{i}{4}pg^{-2}g' -\frac{i}{4}[g^{-5/4}g',p]g^{-3/4}-\frac{i}{4} p g^{-2}g'\\
% =& p^2 g^{-1} -\frac{i}{2}pg^{-2}g' +\frac{1}{4}\left(-\frac{5}{4}g^{-3}g'^2 + g^{-2}g''\right)
% \end{align}

% We then have a path integral
% \begin{align}
% \Tr[e^{-HT}] =& \int \prod_{k=1}^N\frac{dx_k dp_k}{2\pi}\, \exp\left\{ -\left[\frac{p_k^2}{g} -\frac{i}{2}p\frac{g'}{g^2}+\frac{1}{4}\left(-\frac{5g'^2}{4g^{3}}+ \frac{g''}{g^2}\right)\right]\Delta T+ip_k(x_{k+1}-x_k)  \right\}\\
% % =& \int \prod_{k=1}^N\frac{dx_k dp_k}{2\pi}\, \exp\left[ -\frac{\Delta T}{g}p_k^2 +ip_k\left(x_{k+1}-x_k\frac{g'}{2g^2}\Delta T\right) - \frac{\Delta T}{4}\left(-\frac{5g'^2}{4g^{3}}+ \frac{g''}{g^2}\right) \right]\\
% % =& \int \prod_{k=1}^Ndx_k \sqrt{\frac{g_k}{4\pi \Delta T}}\, \exp\left[ -\frac{g}{4\Delta T}\left(x_{k+1}-x_k\frac{g'}{2g^2}\Delta T\right)^2 + \frac{\Delta T}{4}\left(\frac{5g'^2}{4g^{3}}- \frac{g''}{g^2}\right) \right]\\
% % =&  \int\prod_{k=1}^Ndx_k \sqrt{\frac{g_k}{4\pi \Delta T}}\, \exp\left[ -\frac{g(x_{k+1}-x_k)^2}{4\Delta T} + (x_{k+1}-x_k)\frac{g'}{4g} - \frac{g'^2}{4g^3}\Delta T+ \frac{\Delta T}{4}\left(\frac{5g'^2}{4g^{3}}- \frac{g''}{g^2}\right) \right]\\
% =&  \int\prod_{k=1}^Ndx_k \sqrt{\frac{g_k}{4\pi \Delta T}}\, \exp\left[ -\frac{g(x_{k+1}-x_k)^2}{4\Delta T} + (x_{k+1}-x_k)\frac{g'}{4g} + \frac{\Delta T}{4}\left(\frac{g'^2}{4g^{3}}- \frac{g''}{g^2}\right) \right]
% \end{align}

% Now we write the linear velocity term as a logarithm using
% \begin{equation}
% \frac{g'}{g}\Delta x = \log\left[\frac{g(x_{k+1})}{g(x_k)}\right]  -\frac{1}{2}\Delta x^2\left(\frac{g''}{g} -\frac{(g')^2}{g^2}\right) = \log\left[\frac{g(x_{k+1})}{g(x_k)}\right]  +\frac{\Delta T}{g}\left(\frac{g''}{g^2} -\frac{(g')^2}{g^3}\right),
% \end{equation}
% where we used $\langle\langle \Delta x^2\rangle\rangle = 2\Delta T g^{-1}$.  

% The path integral is now
% \begin{align}
% \tr[e^{-HT}]=&  \int\prod_{k=1}^Ndx_k \sqrt{\frac{g_k}{4\pi \Delta T}}\, \exp\left\{ -\frac{g(x_{k+1}-x_k)^2}{4\Delta T} + \frac{1}{4}\log\left[\frac{g(x_{k+1})}{g(x_k)}\right] \right\} \nonumber\\
% &\times \exp\left[\frac{\Delta T}{4g}\left(\frac{g''}{g^2} -\frac{(g')^2}{g^3}\right) + \frac{\Delta T}{4}\left(\frac{g'^2}{4g^{3}}- \frac{g''}{g^2}\right) \right]\\
% =&  \int\prod_{k=1}^Ndx_k \sqrt{\frac{g_k}{4\pi \Delta T}}\, \exp\left\{ -\frac{g(x_{k+1}-x_k)^2}{4\Delta T} + \frac{1}{4}\log\left[\frac{g(x_{k+1})}{g(x_k)}\right]-\frac{3\Delta T}{4g}\frac{(g')^2}{g^3} \right\}
% \end{align}
% Now since we have a loop integral, the log does yields 
% \begin{equation}
% \prod_{k=0}^{N-1} \log\left[ \frac{g(x_{k+1})}{g(x_k)}\right] = \log\left[ \frac{g(x_{N})}{g(x_0)}\right] = 0
% \end{equation}

% So we get the Ito ordered path integral
% \begin{equation}
% \boxed{\tr[e^{-g^{-1/4}p^2g^{-3/4}T}]=  \int\prod_{k=1}^Ndx_k \sqrt{\frac{g_k}{4\pi \Delta T}}\, \exp\left[ -\frac{g(x_{k+1}-x_k)^2}{4\Delta T}-\frac{3\Delta T}{4g}\frac{(g')^2}{g^3} \right]}\label{eq:Ito_PI_metric}
% \end{equation}

% \subsection{Weyl Ordered Path Integral}

% Let's set about Weyl ordering that Hamiltonian.  
% \subsubsection{Various permutations of operators}

% We will need the anti-standard order expression
% \begin{equation}
% H = p^2 g^{-1} -\frac{i}{2}pg^{-2}g' +\frac{1}{4}\left(-\frac{5}{4}g^{-3}g'^2 + g^{-2}g''\right).
% \end{equation}

% In addition we need the corresponding expression in standard order
% \begin{align}
% H =& g^{-1/4}p^2g^{-3/4}\\
% % =& g^{-1}p^2 + g^{-1/4}[p^2,g^{-3/4}]\\
% % =& g^{-1}p^2 +\frac{3i}{4}g^{-1/4}\left(pg'g^{-7/4} + g'g^{-7/4}p\right)\\
% % =& g^{-1}p^2 +\frac{3i}{2} \frac{g'}{g^2}p  + \frac{3i}{4}g^{-1/4}[p,g'g^{-7/4}]\\
% =& g^{-1}p^2 +\frac{3i}{2} \frac{g'}{g^2}p  + \frac{3}{4}\left( \frac{g''}{g^2} -\frac{7g'^2}{4g^3}\right)
% \end{align}

% And finally, $g$ in the middle.  
% \begin{align}
% H =& g^{-1/4}p^2g^{-3/4}\\
% % =& \left(pg^{-1/4}+ [g^{-1/4},p]\right)\left( g^{-3/4}p  + [p,g^{-3/4}]\right)\\
% % =& \left(pg^{-1/4}-\frac{i}{4}g'g^{-5/4}\right)\left( g^{-3/4}p  +i\frac{3}{4}g'g^{-7/4}\right)\\
% =& pg^{-1}p -\frac{i}{4}g'g^{-2}p  +i\frac{3}{4}pg'g^{-2} + \frac{3}{16}\frac{g'^2}{g^3}
% \end{align}


% So we have 
% \begin{align}
% H =& [pg^{-1}p]_W + \frac{1}{4}\left[ -\frac{i}{2}p\frac{g'}{g^{2}} +\frac{1}{4}\left(\frac{g''}{g^2}-\frac{5g'^2}{4g^3}\right) +\frac{3i}{2} \frac{g'}{g^2}p  + \frac{3}{4}\left( \frac{g''}{g^2} -\frac{7g'^2}{4g^3}\right) -\frac{i}{2}g'g^{-2}p  +i\frac{3}{2}pg'g^{-2} + \frac{3}{8}\frac{g'^2}{g^3}\right]\\
% %=& [pg^{-1}p]_W + \frac{1}{4}\left[ ip\frac{g'}{g^2} +i\frac{g'}{g^2}p +\frac{g''}{g^2}-\frac{5g'^2}{4g^3}\right]\\
% =& [pg^{-1}p]_W + \frac{i}{2}\left[p\frac{g'}{g^2}\right]_W +\frac{1}{4}\left(\frac{g''}{g^2}-\frac{5g'^2}{4g^3}\right)
% \end{align}
% So we get a velocity in this case, due to the asymmetry of the operators.  

% Using the midpoint rule for  Weyl ordered operators, we then can form the path integral as 
% \begin{align}
% \tr[e^{-HT}_W] =& \int \prod_k \frac{dx_kdp_k}{2\pi}\exp\left[-\left(p^2\bar{g}^{-1} + \frac{i}{2}\left[p\frac{\bar{g}'}{\bar{g}^2}\right]_W +\frac{1}{4}\left(\frac{g''}{g^2}-\frac{5g'^2}{4g^3}\right)\right)\Delta T + ip_k(x_{k+1}-x_k)\right]\\
% % =& \int \prod_k \frac{dx_kdp_k}{2\pi}\exp\left[-p^2\bar{g}^{-1}\Delta T - ip_k\left(x_{k+1}-x_k-\frac{\bar{g}'}{2\bar{g}^2}\Delta T\right) -\frac{\Delta T}{4}\left(\frac{g''}{g^2}-\frac{5g'^2}{4g^3}\right) \right]\\
% % =& \int \prod_k dx_k\sqrt{\frac{\bar{g}}{4\pi\Delta T}}\exp\left[-\frac{\bar{g}}{4\Delta T}\left(x_{k+1}-x_k-\frac{\bar{g}'}{2\bar{g}^2}\Delta T\right)^2 -\frac{\Delta T}{4}\left(\frac{g''}{g^2}-\frac{5g'^2}{4g^3}\right) \right]\\
% % =& \int \prod_k dx_k\sqrt{\frac{\bar{g}}{4\pi\Delta T}}\exp\left[-\frac{\bar{g}(x_{k+1}-x_k)^2}{4\Delta T} +(x_{k+1}-x_k)\frac{\bar{g}'}{4\bar{g}} -\frac{\bar{g}'^2}{16 g^3}\Delta T  -\frac{\Delta T}{4}\left(\frac{g''}{g^2}-\frac{5g'^2}{4g^3}\right) \right]\\
% =& \int \prod_k dx_k\sqrt{\frac{\bar{g}}{4\pi\Delta T}}\exp\left[-\frac{\bar{g}(x_{k+1}-x_k)^2}{4\Delta T} +(x_{k+1}-x_k)\frac{\bar{g}'}{4\bar{g}}  -\frac{\Delta T}{4}\left(\frac{g''}{g^2}-\frac{g'^2}{g^3}\right) \right]
% \end{align}

% So now we need to carry out the log expansion for $\alpha=0$.  Fortunately, to order $\Delta T$ we have 
% \begin{equation}
% \log\left[\frac{g(x_{k+1})}{g(x_k)}\right] = \frac{\bar{g}'}{\bar{g}}(x_{k+1}-x_k)
% \end{equation}
% So we have no corrections to the potential from eliminating the velocity, since the log drops out for a trace.  
% \begin{equation}
% \boxed{\tr[e^{-HT}]_W= \int \prod_k dx_k\sqrt{\frac{\bar{g}}{4\pi\Delta T}}\exp\left[-\frac{\bar{g}(x_{k+1}-x_k)^2}{4\Delta T}   -\frac{\Delta T}{4}\left(\frac{g''}{g^2}-\frac{g'^2}{g^3}\right) \right]}
% \end{equation}


% \section{Dielectric as in flat space}
% \subsection{Ito ordering}


% Let's consider trying to find
% \begin{equation}
% E = \int \frac{dT}{T\sqrt{4\pi T}}  \tr e^{-\sigma p^2\sigma T}
% \end{equation}

% We will choose to adopt anti-standard order(this gives an Ito calculus).
% \begin{align}
% \sigma p^2\sigma   %= (p\sigma + i\sigma')p\sigma \\
% & = p\sigma p\sigma + i\sigma' p \sigma\\
% %& = p(p\sigma + i\sigma')\sigma + i(p\sigma' + i \sigma'') \sigma\\
% & = p^2\sigma^2 + 2ip\sigma'\sigma- \sigma''\sigma
% \end{align}


% \subsection{Configuration Space}

% We can carry out the momentum integrals, which gives us 
% \begin{align}
% \tr e^{-\sigma p^2\sigma T} &= \int dx_k dp_k \delta(x_N-x_0)e^{-\frac{\Delta T}{2} p_k^2\sigma^2_{k} - ip(\sigma'\sigma \Delta T) + \frac{\Delta T}{2}\sigma''\sigma+ i p_k(x_{k+1}-x_{k})}\\
% &= \int dx_k \delta(x_N-x_0)\frac{1}{\sqrt{4\pi \Delta T\sigma^2_{k}}} e^{-\frac{[x_{k+1}-x_{k}-(\sigma'\sigma)_{k}\Delta T]^2}{2\Delta T\sigma^2_{k}}  + \frac{1}{2}\Delta T(\sigma''\sigma)_{k}}\\
% &= \int dx_k \delta(x_N-x_0)\frac{1}{\sqrt{4\pi \Delta T\sigma^2_{k}}} e^{-\frac{(x_{k+1}-x_{k})^2}{2\sigma^2_{k}\Delta T}  +(x_{k+1}-x_k)\frac{\sigma'}{\sigma}   + \frac{1}{2}\Delta T(\sigma''\sigma - \sigma'^2)_{k}}.
% \end{align}

% We will now explicitly deal with the linear velocity term.  This will produce a normalization constant.   
% We can then expand a logarithm as 
% \begin{align}
% \log\left[\frac{\sigma(x_{k+1})}{\sigma(x_k)}\right] =& \log\left[  1 + \Delta x \frac{\sigma'}{\sigma}  + \frac{1}{2}\Delta x^2\frac{\sigma''}{\sigma}\right]= \Delta x\frac{\sigma'}{\sigma} + \frac{1}{2}\Delta x^2\left(\frac{\sigma''}{\sigma} - \frac{\sigma'^2}{\sigma^2}\right)
% \end{align}
% If we use this log substitution immediately inside the exponential, we get
% \begin{equation}
% \exp\left[\Delta x\frac{\sigma'}{\sigma}\right] = \exp\left[\log\left[\frac{\sigma(x_{k+1})}{\sigma(x_k)}\right] - \frac{1}{2}\sigma^2\Delta T\left(\frac{\sigma''}{\sigma} - \frac{\sigma'^2}{\sigma^2}\right)\right],
% \end{equation}
% which would cancel out the potential.  

% We will also need
% \begin{equation}
% \prod_{k=0}^{N-1} \exp\left[\log\left(\frac{\sigma(x_{k+1})}{\sigma(x_k)}\right)\right] =  \exp\left[\prod_{k=0}^{N-1}\log\left(\frac{\sigma(x_{k+1})}{\sigma(x_k)}\right)\right] = \frac{\sigma(x_N)}{\sigma(x_0)} = 1,
% \end{equation}
% where we employed $x_N = x_0$.  

% Then we just have 
% \begin{equation}
% Z= \int dx_k \delta(x_N-x_0)\frac{1}{\sqrt{4\pi \Delta T\sigma^2(x_{k})}} e^{-\frac{(x_{k+1}-x_{k})^2}{2\sigma^2(x_k)\Delta T}}.
% \end{equation}


% \section{Putting into Numerical form-Ito$g$} 

% So let's put a bit of thought into how to handle 
% \begin{equation}
% Z = \int \prod_{k=1}^{N-1}dx_k \frac{\sqrt{g_k}}{\sqrt{2\pi\Delta T}} \exp\left[- \frac{g(x_k)(x_{k+1}-x_k)^2}{2\Delta T}\right]
% \end{equation}

% \subsection{Transforming to a Gaussian}
% We can transform variable from positions $x_k$ to increments
% \begin{equation}
% u_k = x_{k+1} - x_k, k = 0,N-1,
% \end{equation}
% with
% \begin{equation}
% x_{k} = \sum_{j=0}^{k-1} u_j.
% \end{equation}
% This transformation has unit Jacobian.  
% \begin{equation}
% Z= \int du_k \delta\left(\sum u_k \right)\sqrt{\frac{g[x_k(u)]}{2\pi \Delta T}} e^{-\frac{g[x(u)]u_k^2}{2\Delta T}}.
% \end{equation}

% Let's now try to decouple the random variables.  Let us introduce 
% \begin{equation}
% dW_k = \sqrt{g_k}u_k
% \end{equation}

% We will need to calculate a Jacobian, and a euclidean norm of gradients.  In both cases we will need $\frac{\partial u_k}{\partial dW_j}$.  
% So we can iteratively define the increments from the Gaussians as 
% \begin{equation}
% u_k = \frac{1}{\sqrt{g_k[x_k(u_{j<k})]}}dW_k.
% \end{equation}

% We then have 
% \begin{align}
% \frac{\partial u_j}{\partial dW_k} &= \frac{\partial}{\partial dW_k}\frac{1}{\sqrt{g_j[(u_{i<j})]}}dW_j\\
% &= \frac{1}{\sqrt{g_j}}\delta_{jk} -\frac{g_j'}{2g^{3/2}_j}dW_j\sum_{i<j}\frac{d u_i}{dW_k}.
% \end{align}
% So we start at $j,k=0$.  We can solve this laboriously, or quite quickly by starting the recursion at zero.  
% We have 
% \begin{gather}
% \frac{\partial u_0}{\partial dW_0} = \frac{1}{\sqrt{g_0}}, \quad \frac{\partial u_0}{\partial dW_{j>0}} = 0
% \end{gather}
% Similarly the second increment is 
% \begin{align}
% \frac{\partial u_1}{\partial dW_0}& = -\frac{1}{\sqrt{g_0}}\frac{g'_1}{2g_1^{3/2}}dW_1, \quad \frac{\partial u_1}{\partial dW_1} = \frac{1}{\sqrt{g_1}} ,\quad \frac{\partial u_1}{\partial dW_{j>1}} = 0.
% \end{align}
% The third increment is 
% \begin{align}
% \frac{\partial u_2}{\partial dW_0}& = -\frac{g'_2}{2g_2^{3/2}}dW_2\left(1-\frac{g'_1}{2g_1^{3/2}}dW_1\right)\frac{1}{\sqrt{g_0}}\, \quad \frac{\partial u_2}{\partial dW_1} = -\frac{1}{\sqrt{g_1}}\frac{g'_2}{2g_2^{3/2}}dW_2, \quad \frac{\partial u_2}{\partial dW_2} = \frac{1}{\sqrt{g_2}}, \quad \frac{\partial u_2}{\partial dW_{j>2}} = 0.
% \end{align}

% For the diagonal, we just get $g_k^{-1/2}$. 
% So if we're off the diagonal we multiply by $g_j' dW_j$ for the current row $j$, and sum up all of the preceding entries in this column.   

% This is enough to infer the pattern
% \begin{align}
% &\frac{\partial u_j}{\partial dW_k} \nonumber\\
% =& \left( 
% \begin{array}{lllllll}
% \frac{1}{\sqrt{g_0}}      & 0     &   \\
% -\frac{1}{\sqrt{g_0}}dW_1\frac{g'_{1}}{2g^{3/2}_1}    & \frac{1}{\sqrt{g_1}}     & 0 \\
% -\frac{1}{\sqrt{g_0}}\left(1-\frac{g'_{1}}{2g^{3/2}_1}dW_1\right)\frac{g'_{2}}{2g^{3/2}_2}dW_2 & -\frac{1}{\sqrt{g_1}}\frac{g'_{2}}{2g^{3/2}_2}dW_2    & \frac{1}{\sqrt{g_2}}   & 0 & 0 \\
% -\frac{1}{\sqrt{g_0}}\left(1-\frac{g'_{1}}{2g^{3/2}_1}dW_1\right)\left(1-\frac{g'_{2}}{2g^{3/2}_2}dW_2\right)\frac{g'_{3}}{2g^{3/2}_3}dW_3 &  -\frac{1}{\sqrt{g_1}}(1-\frac{g'_{2}}{2g^{3/2}_2}dW_2)\frac{g'_{3}}{2g^{3/2}_3}dW_3     &  -\frac{1}{\sqrt{g_2}}\frac{g'_{3}}{2g^{3/2}_3}dW_3 &\frac{1}{\sqrt{g_3}} & 0 \\
% \vdots & & & \ddots
% \end{array}
% \right)
% \end{align}
% Evidently a generic term off the diagonal will have components (row is $u_j$, column is $dW_k$ increment)
% \begin{equation}
% \boxed{\frac{\partial u_j}{\partial dW_k} =
%  \frac{1}{\sqrt{g_k}}\delta_{jk} - \frac{1}{\sqrt{g_k}}\frac{g'_j}{2g^{3/2}_j}dW_j\delta_{j,k+1}
%  - \frac{1}{\sqrt{g_k}}\frac{g'_j}{2g^{3/2}_j}dW_j\prod_{m=k+1}^{j-1} \left(1- \frac{g'_m}{2g^{3/2}_m}dW_m\right)\Theta(j\ge k+2),}
% \end{equation}
% with $j,k=0,1,\ldots N-1$. If we further define $dW_k = \sqrt{\Delta T}z_k$, then we should multiply this by $\sqrt{\Delta T}$.  

% \subsection{Jacobian}

% The first place we need this result is in calculating the jacobian:
% \begin{equation}
% \left|\frac{\partial u_j}{\partial dz_k}\right| = \prod_{k=0}^{N-1}\frac{\sqrt{\Delta T}}{\sqrt{g_k}},
% \end{equation}
% which follows because the matrix is in lower-triangular form.
%   The Jacobian from changing from positions to increments is also lower-triangular, and just gives unit determinant.
%   This factor will eat all of the prefactor normalizations.  

% \subsection{Euclidean norm of gradient}

% Next up we need to calculate the change in the normalization when constrained to loops that close.
%   The loop must close, or the increments sum to zero.
%    Our constraint is 
% \begin{align}
%   h(d\vect{W}) = x_n-x_0 = \sum_{j=0}^{N-1} u_j = \sum_{j=0}^{N-1}u_j\left[\sum_{k<j} u_k[dW] \right]=0
% \end{align}
% The Euclidean norm of the gradient is defined as 
% \begin{align}
% \|\nabla_{\vect{z}}h(\vect{z})\|^2 = &  \Delta T \|\nabla_{d\vect{W}}h(d\vect{W})\|^2\\
% %  = & \Delta T \sum_{k=0}^{N-1} \left(\sum_{j=0}^{N-1} \frac{\partial u_j}{\partial dW_k} \right)^2\\
% %  = & \Delta T \sum_{k=0}^{N-1} \left(\sum_{j=0}^{N-1}\frac{1}{\sqrt{g_k}}\delta_{jk} - \frac{1}{\sqrt{g_k}}\frac{g'_j}{2g^{3/2}_j}dW_j\delta_{j,k+1} \right. \nonumber\\
% % &\left.\qquad - \frac{1}{\sqrt{g_k}}\frac{g'_j}{2g^{3/2}_j}dW_j\prod_{m=k+1}^{j-1} \left(1- \frac{g'_m}{2g^{3/2}_m}dW_m\right)\Theta(j\ge k+2)\right)^2\\
% %  = & \Delta T  \sum_{k=0}^{N-1}\left[\frac{1}{\sqrt{g_k}} - \frac{1}{\sqrt{g_k}}\frac{g'_{k+1}}{2g^{3/2}_{k+1}}dW_{k+1} - \sum_{j=k+2}^{N-1}\frac{1}{\sqrt{g_k}}\frac{g'_j}{2g^{3/2}_j}dW_j\prod_{m=k+1}^{j-1} \left(1- \frac{g'_m}{2g^{3/2}_m}dW_m\right)\right]^2\\
%  = & \Delta T  \sum_{k=0}^{N-1}\frac{1}{g_k}\left[1 - \frac{g'_{k+1}}{2g^{3/2}_{k+1}}dW_{k+1} - \sum_{j=k+2}^{N-1}\frac{g'_j}{2g^{3/2}_j}dW_j\prod_{m=k+1}^{j-1} \left(1- \frac{g'_m}{2g^{3/2}_m}dW_m\right)\right]^2
% \end{align}
% where we carried out the sum over $j$.    Now we have to start guessing a bit - using some inspiration to approximate this thing.  
% It turns out that we can simplify the bracketered term.  We have a sum like: 
% \begin{align}
% 1- f_{k+1}  - \sum_{j={k+2}}^{N-1} f_j\prod_{m=k+1}^{j-1}(1-f_m) =& 1 - f_{k+1}  - f_{k+2}(1-f_{k+1}) - f_{k+3}(1-f_{k+2})(1-f_{k+1}) + \ldots\\
% =& (1-f_{k+2})(1-f_{k+1}) - f_{k+3}(1-f_{k+2})(1-f_{k+1}) + \ldots \\
% =& \prod_{j=k+1}^{N-1}(1-f_j)
% \end{align}
% So our complicated factor is just an exponential.  
% So we get 
% \begin{equation}
%   {\|\nabla_{\vect{z}}h(\vect{z})\|^2 = 
%     \Delta T \sum_{k=0}^{N-1}\frac{1}{g_k}\exp\left[
%       -\sum_{j=k+1}^{N-1}\frac{g_j'}{2g_j^{3/2}}dW_j - \Delta T\frac{g_j'^2}{8g_j^3}\right]^2}
% \end{equation}
% Now we can use the logarithmic expansion to rewrite the Wiener increment term
% \begin{align}
% d\log\left[g\right]=& \frac{g'}{g}dx +\frac{1}{2}\left(\frac{g''}{g} -\frac{g'^2}{g^2}\right)dx^2   \\ 
% =& \frac{g'}{g^{3/2}} dW +\frac{1}{2}\left(\frac{g''}{g^2} -\frac{g'^2}{g^3}\right)dT,
% \end{align}
% to write 
%  \begin{align}
%  \|\nabla_{\vect{z}}h(\vect{z})\|^2 = & \Delta T  \sum_{k=0}^{N-1}\frac{1}{g_k}
%  \exp\left[\sum_{j=k+1}^{N-1}-\frac{1}{2}\log\frac{g_{j+1}}{g_j}+\frac{\Delta T}{4}\left(\frac{g''}{g^2}
%   -\frac{g'^2}{g^3}\right) - \Delta T\frac{g_j'^2}{8g_j^3}\right]^2\\
%   = & \Delta T  \sum_{k=0}^{N-1}\frac{1}{g_k}\exp\left[\log\frac{g_{k+1}}{g_{N}}
%   +\frac{\Delta T}{2}\sum_{j=k+1}^{N-1}\left(\frac{g''}{g^2}-\frac{3g'^2}{2g^3}\right)\right]
% \\
% = & \Delta T  \sum_{k=0}^{N-1}\frac{g_{k+1}}{g_k}\frac{1}{g_N}\exp\left[
% \frac{\Delta T}{2}\sum_{j=k+1}^{N-1}\left(\frac{g''}{g^2} -\frac{3g'^2}{2g^3}\right)\right]
%  \end{align}
% Now we can use the fact that the trace is invariant under cyclic permutations to write, $g_N$ as a loop average.
%   This follows since we cyclically permute all labels.
%   However, I think the integration over partial times will remain.  
% Can we drop the ``small'' terms?

% In continuum language we have
% \begin{equation}
% \|\nabla_{\vect{z}}h(\vect{z})\|^2 =\int_0^T dt\, \frac{g(t+dt)}{g(t)}\frac{1}{g(T)} \exp\left[\int_t^Tds\left(\frac{g''}{2g^2} -\frac{3g'^2}{4g^3}\right)\right]
% \end{equation}
% I think we can then argue that $g(t+dt)/g(t)\sim 1 + \order(N^{-1/2})$, and we can drop the correction at this point.  We can also write $g(T)$ as a sum over all permutations, since this is just a label.  

% So at the end of the day we get:
% \begin{equation}
% Z = \frac{1}{\sqrt{2\pi T}} \bigg<\bigg< e^{-\int dt V}[\langle g^{-1}\rangle_{\text{loop}}]^{-1/2}\left[\frac{1}{T}\int dt \exp\left(\frac{g''}{2g^2} -\frac{3g'^2}{4g^3}\right)\right]^{-1/2}\bigg>\bigg>
% \end{equation}

% If we use the cyclic permutation argument we can rewrite that extra exponential.  We can also pull in the other potential to write this as:
% \begin{equation}
% Z = \frac{1}{\sqrt{2\pi T}}\bigg<\bigg< [\langle \sigma^2\rangle_{\text{loop}}]^{-1/2}\left[\frac{1}{T}\int_0^T dt \exp\left(-\int_{0}^tds \sigma''(s) \sigma(s) + \int_0^T dt\, 2V(t)\right)\right]^{-1/2}\bigg>\bigg>
% \end{equation}

% From our work above: if we use the metric form of the partition function in Eq.~(\ref{eq:Ito_PI_metric}), we have 
% \begin{equation}
% V = \frac{3}{8}\frac{g'^2}{g^3},
% \end{equation}
% where we took $g\rightarrow 2g$ to get from the earlier work creating the path integral to the current convention.  
% So the partition function is 
% \begin{align}
% Z =& \frac{1}{\sqrt{2\pi T}}\bigg<\bigg< [\langle g^{-1}\rangle_{\text{loop}}]^{-1/2}\left[\frac{1}{T}\int_0^T dt \exp\left(\int_{0}^tds \frac{g''}{2g^2}-\frac{3g'^2}{4g^3} + \int_0^T dt\, \frac{3 g'^2}{4g^3}\right)\right]^{-1/2}\bigg>\bigg>\\
% =& \frac{1}{\sqrt{2\pi T}}\bigg<\bigg< [\langle g^{-1}\rangle_{\text{loop}}]^{-1/2}\left[\frac{1}{T}\int_0^T dt \exp\left(\int_{0}^tds \frac{g''}{g^2}+ \int_t^T dt\, \frac{3 g'^2}{2g^3}\right)\right]^{-1/2}\bigg>\bigg>
% \end{align}
% If however, we operator order in flat space there is no potential, since it cancels with the change in variables from the mean velocity.  


% \section{Putting into Numerical form-Weyl} 

% So let's put a bit of thought into how to handle 
% \begin{equation}
% Z = \int \prod_{k=1}^{N-1}dx_k \frac{\sqrt{\bar{g}_k}}{\sqrt{2\pi\Delta T}} \exp\left[- \frac{g(\bar{x}_k)(x_{k+1}-x_k)^2}{2\Delta T}\right],
% \end{equation}
% where we will use $\bar{x}_k = (x_{k+1}+x_k)/2$, and $\bar{g} = g(\bar{x})$.  
% \subsection{Transforming to a Gaussian}
% We can transform variable from positions $x_k$ to increments
% \begin{equation}
% u_k = x_{k+1} - x_k, k = 0,N-1,
% \end{equation}
% with
% \begin{equation}
% x_{k} = \sum_{j=0}^{k-1} u_j.
% \end{equation}
% This transformation has unit Jacobian.  
% \begin{equation}
% Z= \int du_k \delta\left(\sum u_k \right)\sqrt{\frac{\bar{g}_k}{2\pi \Delta T}} e^{-\frac{\bar{g}_ku_k^2}{2\Delta T}}.
% \end{equation}
% now where 
% \begin{equation}
% \bar{g}_k = g\left(\bar{x}_k\right) =  g\left(\frac{u_k}{2}+\sum_{j<k} u_j\right) 
% \end{equation}

% Let's now try to decouple the random variables.  Let us introduce 
% \begin{equation}
% dW_k = \sqrt{\bar{g}_k}u_k
% \end{equation}
% \comment{I amy miss some bars - all functions are understood to be functions of $\bar{x}_j$.  }
% We will need to calculate a Jacobian, and a euclidean norm of gradients.  In both cases we will need $\frac{\partial u_k}{\partial dW_j}$.  
% So we can iteratively define the increments from the Gaussians as 
% \begin{equation}
% u_k = \frac{1}{\sqrt{\bar{g}_k}}dW_k.
% \end{equation}

% \subsection{Partial derivatives of increments}
% We then have 
% \begin{align}
% \frac{\partial u_j}{\partial dW_k} &= \frac{\partial}{\partial dW_k}\frac{1}{\sqrt{g\left(\frac{u_j}{2} +\sum_{i<j}u_i\right)}}dW_j\\
% &= \frac{1}{\sqrt{\bar{g}_j}}\delta_{jk} -\frac{\bar{g}_j'}{2\bar{g}^{3/2}_j}dW_j\left(\frac{1}{2}\frac{du_j}{dW_k} + \sum_{i<j}\frac{d u_i}{dW_k}\right)\\
% \left(1 + \frac{\bar{g}_j'}{4\bar{g}^{3/2}_j}dW_j\right)\frac{du_j}{dW_k} &= \frac{1}{\sqrt{\bar{g}_j}}\delta_{jk} -\frac{\bar{g}_j'}{2\bar{g}^{3/2}_j}dW_j\sum_{i<j}\frac{d u_i}{dW_k}\\
% \frac{du_j}{dW_k} &= \left(1 + \frac{\bar{g}_j'}{4\bar{g}^{3/2}_j}dW_j\right)^{-1}\left(\frac{1}{\sqrt{\bar{g}_j}}\delta_{jk} -\frac{\bar{g}_j'}{2\bar{g}^{3/2}_j}dW_j\sum_{i<j}\frac{d u_i}{dW_k}\right).
% \end{align}
% Now approximate that leading coeffient to order $\Delta T$.  
% \begin{align}
% \left(1 + \frac{\bar{g}_j'}{4\bar{g}^{3/2}_j}dW_j\right)^{-1} \approx& 1 - \frac{\bar{g}_j'}{4\bar{g}_j^{3/2} }dW_j + \frac{\bar{g}_j'^2}{16 \bar{g}_j^3}\Delta T\\
% \approx& \exp\left(-\frac{\bar{g}_j'}{4\bar{g}_j^{3/2} }dW_j+ \frac{\bar{g}_j'^2}{32 \bar{g}_j^3}\Delta T\right) \\
% =& e^{U_j}\label{eq:defn_exp_U}
% \end{align}
% where 
% \begin{equation}
% U_j = -\frac{\bar{g}_j'}{4\bar{g}_j^{3/2} }dW_j+ \frac{\bar{g}_j'^2}{32 \bar{g}_j^3}\Delta T
% \end{equation}
% \subsubsection{Logarithm}
% Now we can expand the increment in $\log g$ to order $\Delta T$.
% \begin{align}
% \log[g(x_{j+1})] -\log[g(x_{j})] =& \log[g\left(\bar{x}_j+ \frac{dx_j}{2}\right)] -\log[g\left(\bar{x}_j- \frac{dx_j}{2}\right)]\\
% =& \log\left[\bar{g}_j+ \frac{dx_j}{2}\bar{g}_j' + \frac{dx_j^2}{4}\bar{g}''_j\right]-\log\left[\bar{g}_j- \frac{dx_j}{2}\bar{g}_j' + \frac{dx_j^2}{4}\bar{g}''_j\right]\\
% =& \log\left[\bar{g}_j\right] + \frac{dx_j}{2}\frac{\bar{g}_j'}{g_j} + \frac{dx_j^2}{4}\frac{\bar{g}''_j}{g_j} - \frac{1}{2}\frac{dx_j^2}{4}\frac{\bar{g}_j'^2}{g^2}\nonumber\\
% &-\left(\log\left[\bar{g}_j\right] - \frac{dx_j}{2}\frac{\bar{g}_j'}{g_j} + \frac{dx_j^2}{4}\frac{\bar{g}''_j}{g_j} - \frac{1}{2}\frac{dx_j^2}{4}\frac{\bar{g}_j'^2}{g^2}\right)\\
% =& dx_j\frac{\bar{g}'_j}{\bar{g}_j}.\label{eq:log_strat}
% \end{align}
% So we can write 
% \begin{equation}
% \boxed{U_k=-\frac{1}{4}\log\left(\frac{\bar{g}_{j+1}}{\bar{g}_j}\right)+ \frac{\bar{g}_j'^2}{32 \bar{g}_j^3}\Delta T}
% \end{equation}

% \subsubsection{Putting derivative matrix into closed form}

% Let us now return to finding the derivative of u w.r.t w:
% \begin{equation}
% \boxed{\frac{du_j}{dW_k} = \frac{e^{U_j}}{\sqrt{\bar{g}_j}}\delta_{jk} -e^{U_j}\frac{\bar{g}_j'}{2\bar{g}^{3/2}_j}dW_j\sum_{i<j}\frac{d u_i}{dW_k}.}
% \end{equation}
% We can solve this by starting the recursion at $j,k=0$.  We have 
% \begin{gather}
% \frac{\partial u_0}{\partial dW_0} = \frac{e^{U_0}}{\sqrt{\bar{g}_0}}, \quad \frac{\partial u_0}{\partial dW_{j>0}} = 0
% \end{gather}
% Similarly the second increment is 
% \begin{align}
% \frac{\partial u_1}{\partial dW_0}& = -\frac{e^{U_0}}{\sqrt{\bar{g}_0}}\frac{e^{U_1}\bar{g}'_1}{2\bar{g}_1^{3/2}}dW_1, \quad \frac{\partial u_1}{\partial dW_1} = \frac{e^{U_1}}{\sqrt{\bar{g}_1}} ,\quad \frac{\partial u_1}{\partial dW_{j>1}} = 0.
% \end{align}
% The third increment is 
% \begin{align}
% \frac{\partial u_2}{\partial dW_0}& = -\frac{e^{U_2}\bar{g}'_2}{2\bar{g}_2^{3/2}}dW_2\left(1-\frac{e^{U_1}\bar{g}'_1}{2\bar{g}_1^{3/2}}dW_1\right)\frac{e^{U_0}}{\sqrt{\bar{g}_0}}\, \quad \frac{\partial u_2}{\partial dW_1} = -\frac{e^{U_1}}{\sqrt{\bar{g}_1}}\frac{e^{U_2}\bar{g}'_2}{2\bar{g}_2^{3/2}}dW_2, \quad \frac{\partial u_2}{\partial dW_2} = \frac{e^{U_2}}{\sqrt{\bar{g}_2}}, \quad \frac{\partial u_2}{\partial dW_{j>2}} = 0.
% \end{align}

% For the diagonal, we just get $\bar{g}_k^{-1/2}$. 
% So if we're off the diagonal we multiply by $\bar{g}_j' dW_j$ for the current row $j$, and sum up all of the preceding entries in this column.   
% Evidently a generic term off the diagonal will have components (row is $u_j$, column is $dW_k$ increment)
% \begin{equation}
% \boxed{\frac{\partial u_j}{\partial dW_k} = \frac{e^{U_k}}{\sqrt{\bar{g}_k}}\delta_{jk} - \frac{e^{U_k}}{\sqrt{\bar{g}_k}}\frac{e^{U_j}\bar{g}'_j}{2\bar{g}^{3/2}_j}dW_j\delta_{j,k+1} - \frac{e^{U_k}}{\sqrt{\bar{g}_k}}\frac{e^{U_j}\bar{g}'_j}{2\bar{g}^{3/2}_j}dW_j\prod_{m=k+1}^{j-1} \left(1- \frac{e^{U_m}\bar{g}'_m}{2\bar{g}^{3/2}_m}dW_m\right)\Theta(j\ge k+2),}
% \end{equation}
% with $j,k=0,1,\ldots N-1$. If we further define $dW_k = \sqrt{\Delta T}z_k$, then we should multiply this by $\sqrt{\Delta T}$.  

% \subsection{Jacobian}

% The first place we need this result is in calculating the jacobian:
% \begin{equation}
% \left|\frac{\partial u_j}{\partial dz_k}\right| = \prod_{k=0}^{N-1}\frac{e^{U_k}\sqrt{\Delta T}}{\sqrt{\bar{g}_k}}
% \end{equation}
% which follows because the matrix is in lower-triangular form.  The Jacobian from changing from positions to increments is also lower-triangular, and just gives unit determinant.  This factor will eat all of the prefactor normalizations, and slightly shift the potential.  

% After cancelling out the normalization prefactors we then have to deal with 
% \begin{equation}
% \exp\left[\sum_k U_k\right] = \prod_{j=0}^{N-1}\left( \frac{\bar{g}_j}{\bar{g}_{j+1}}\right)^{1/4}\exp\left( \sum_j\frac{\bar{g}_j'^2}{32 \bar{g}_j^3}\Delta T\right) = \exp\left( \sum_j\frac{\bar{g}_j'^2}{32 \bar{g}_j^3}\Delta T\right),
% \end{equation}
% since we have closed loops, so that prefactor just gives us unity.  


% \subsection{Euclidean norm of gradient}

% Next up we need to calculate the change in the normalization when constrained to loops that close.  The loop must close, or the increments sum to zero.   Our constraint is 
% \begin{align}
%   h(d\vect{W}) = x_n-x_0 = \sum_{j=0}^{N-1} u_j = \sum_{j=0}^{N-1}u_j\left[\sum_{k<j} u_k[dW] \right]=0
% \end{align}
% We will use the $z_k$, to ensure we carry out all of the transformations we need.  
% The Euclidean norm of the gradient is defined as 
% \begin{align}
% \|\nabla_{\vect{z}}h(\vect{z})\|^2 = &  \Delta T \|\nabla_{d\vect{W}}h(d\vect{W})\|^2\\
%  = & \Delta T \sum_{k=0}^{N-1} \left(\sum_{j=0}^{N-1} \frac{\partial u_j}{\partial dW_k} \right)^2
% \end{align}
% We will plug in our form for $\frac{\partial u_j}{\partial dW_k} $, and simplify the sums.  
% \begin{align}
% \|\nabla_{\vect{z}}h(\vect{z})\|^2 = & \Delta T \sum_{k=0}^{N-1} \left(\sum_{j=0}^{N-1}\frac{e^{U_k}}{\sqrt{\bar{g}_k}}\delta_{jk} - \frac{e^{U_k}}{\sqrt{\bar{g}_k}}\frac{e^{U_j}\bar{g}'_j}{2\bar{g}^{3/2}_j}dW_j\delta_{j,k+1} \right. \nonumber\\
% &\left.\qquad - \frac{e^{U_k}}{\sqrt{\bar{g}_k}}\frac{\bar{g}'_j}{2\bar{g}^{3/2}_j}dW_j\prod_{m=k+1}^{j-1} \left(1- \frac{e^{U_m}\bar{g}'_m}{2\bar{g}^{3/2}_m}dW_m\right)\Theta(j\ge k+2)\right)^2\\
%  = & \Delta T  \sum_{k=0}^{N-1}\frac{e^{2U_k}}{\bar{g}_k}\left[1 - \frac{e^{U_{k+1}}\bar{g}'_{k+1}}{2\bar{g}^{3/2}_{k+1}}dW_{k+1} - \sum_{j=k+2}^{N-1}\frac{e^{U_j}\bar{g}'_j}{2\bar{g}^{3/2}_j}dW_j\prod_{m=k+1}^{j-1} \left(1- \frac{e^{U_m}\bar{g}'_m}{2\bar{g}^{3/2}_m}dW_m\right)\right]^2
% \end{align}
% where we carried out the sum over $j$.    Now we have to start guessing a bit - using some inspiration to approximate this thing.  It turns out that we can simplify the bracketed term.  We will just start writing out the sum starting at $k+2$, and then factoring terms.  It will be clear that wecan then proceed to just factor all of the terms up.  
% \begin{align}
% 1- f_{k+1}  - \sum_{j={k+2}}^{N-1} f_j\prod_{m=k+1}^{j-1}(1-f_m) =& 1 - f_{k+1}  - f_{k+2}(1-f_{k+1}) - f_{k+3}(1-f_{k+2})(1-f_{k+1}) + \ldots\\
% =& (1-f_{k+2})(1-f_{k+1}) - f_{k+3}(1-f_{k+2})(1-f_{k+1}) + \ldots \\
% =& \prod_{j=k+1}^{N-1}(1-f_j)
% \end{align}
% Now that is in a form that begs to be exponentiated.  We can plug in $e^{U_m}$ using the definition in Eq.~(\ref{eq:defn_exp_U}), and simplify this a bit.  
% So we get 
% \begin{align}
% \|\nabla_{\vect{z}}h(\vect{z})\|^2 =& \Delta T  \sum_{k=0}^{N-1}\frac{e^{2U_k}}{\bar{g}_k}\left[\prod_{m=k+1}^{N-1}\left(1 -\frac{e^{U_m}\bar{g}'_m}{2\bar{g}^{3/2}_m}dW_m\right)\right]^2\\
% =& \Delta T  \sum_{k=0}^{N-1}\frac{e^{U_k}}{\bar{g}_k}\prod_{m=k+1}^{N-1}\left[1 -\left(1 - \frac{\bar{g}_j'}{4\bar{g}_j^{3/2} }dW_j + \frac{\bar{g}_j'^2}{16 \bar{g}_j^3}\Delta T\right)\frac{\bar{g}'_m}{2\bar{g}^{3/2}_m}dW_m\right]^2\\
% \approx& \Delta T  \sum_{k=0}^{N-1}\frac{e^{U_k}}{\bar{g}_k}\prod_{m=k+1}^{N-1}\left(1 -\frac{\bar{g}'_m}{2\bar{g}^{3/2}_m}dW_m + \frac{\bar{g}_j'^2}{8\bar{g}_j^{3} }\Delta T\right)\\
% \approx& \Delta T  \sum_{k=0}^{N-1}\frac{e^{2U_k}}{\bar{g}_k}\prod_{m=k+1}^{N-1}\exp\left( -\frac{\bar{g}'_m}{\bar{g}^{3/2}_m}dW_m \right)\\
% \approx& \Delta T  \sum_{k=0}^{N-1}\frac{1}{\bar{g}_k}\exp\left(-\frac{\bar{g}_k'}{2\bar{g}_k^{3/2} }dW_k+ \frac{\bar{g}_k'^2}{16 \bar{g}_k^3}\Delta T\right)\prod_{m=k+1}^{N-1}\exp\left( -\frac{\bar{g}'_m}{\bar{g}^{3/2}_m}dW_m \right)
% \end{align}
% Now we can use the logarithmic expansion in Eq.(~\ref{eq:log_strat}) to rewrite the Wiener increment term
% \begin{align}
% -\frac{g'_m}{g_m^{3/2}} dW_m = -\frac{g'_m}{g_m'} dx = \ln\frac{g_m}{g_{m+1}}
% \end{align}
% We have a sum of these: 
% \begin{equation}
% \prod_{m=k+1}^{N-1}\exp\left( -\frac{\bar{g}'_m}{\bar{g}^{3/2}_m}dW_m \right) = \prod_{m=k+1}^{N-1} \frac{g_m}{g_{m+1}} = \frac{g_{k+1}}{g_N}.
% \end{equation}
% If this came out with a different sign we could be dancing?  
% So our normalization factor is 
% \begin{equation}
% \boxed{\|\nabla_{\vect{z}}h(\vect{z})\|^2 = \Delta T \sum_{k=0}^{N-1}\frac{1}{\bar{g}_k}\frac{\bar{g}_{k+1}}{\bar{g}_{N}} \approx \frac{ T}{g_N} + \order(N^{-1/2})}
% \end{equation}

% Fuck.  So that potential has to do something.   Or this is really just wrong -> check logic for extracting a mean energy.  
% Maybe we forgot an extra term?  

% So if we pull together the Jacobian and Euclidean norm factors we get 
% \begin{align}
% \boxed{Z= \left<\left< \sqrt{\frac{g_N}{2\pi T}}  e^{\Delta T\sum_{m}\frac{\bar{g}'^2}{32g^3}-V}\right>\right>_{g}},
% \end{align}
% where the loops are to be taken with respect to the $g$ loops.  There may be an additional potentital arising rom operator ordering.  
% \comment{replacing constants with their loop average? what order should this be done in?  average of functions, or function of averages?  }

% \section{Classical Gauge Theory - Potentials}

% Consider a Lagrangian density
% \begin{equation}
%   \cL = \epsr(\vect{x})(\nabla\phi+\partial_t\vect{A})^2-c^2(\nabla\times\vect{A})^2.
% \end{equation}
% The equations of motion reproduce the two remaining Maxwell equations:
% \begin{gather}
%   \nabla\cdot\vect{D} = 0
%   \rightarrow \nabla\cdot[\epsilon(\nabla\phi+\partial_t\vect{A})]=0\\
%   \nabla\times\vect{H} - \partial_t\vect{D} = 0
%   \rightarrow \nabla\times\nabla\times\vect{A} + \partial_t[\epsilon(\nabla\phi+\partial_t\vect{A})]=0.
% \end{gather}
% In the path integral we typically impose gauge conditions.  
% \begin{align}
%   G_C &= \nabla\cdot\epsilon\vect{A} = 0\\
%   G_L &= \epsilon^{-\alpha}\nabla\cdot\epsilon\vect{A} +\epsilon^\alpha\partial_t\phi= 0
% \end{align}
% The latter choice is essentially a generalized Lorenz gauge, which cancels off the coupling between $A$, $\phi$.
% We have left $\alpha$ as a parameter to be varied --- but we are most interested in the case when $\alpha=1$.  
% The first equations of motion in generalized Lorenz gauge is
% \begin{align}
% \nabla\cdot[\epsilon(\nabla\phi+\partial_t\vect{A})]
% %&=\nabla\cdot\epsilon\nabla\phi+ \partial_t\nabla\cdot\epsilon\vect{A}\\
% &=\nabla\cdot\epsilon\nabla\phi- \partial_t\epsilon^{2\alpha}\partial_t\phi=0
% \end{align}
% If we further scale the field to $\phi_1 = \epsilon^{-1/2}\phi$, the equation of motion is
% \begin{equation}
%   \boxed{\frac{1}{\sqrt{\epsilon}}\nabla\cdot\epsilon\nabla\frac{1}{\sqrt{\epsilon}}\phi_1- \epsilon^{2\alpha-1}\partial^2_t\phi_1=0}
% \end{equation}

% The second equation of motion is 
% \begin{align}
%   \boxed{\nabla\times\nabla\times\vect{A} + \epsilon\partial^2_t\vect{A} - 
%   \epsilon\nabla\epsilon^{-2\alpha}\nabla\cdot\epsilon\vect{A}=0.}
% \end{align}

% Let's just set $\alpha=1$.  Following our work with the TM potential, the gradient can be
% expanded as 
% \begin{align}
%   \sqrt{f}\partial_i\frac{1}{f}\partial_j\sqrt{f} 
%   &= f^{1/2}\partial_if^{-1/2}[f_j+\partial_j]\\
%   &= (-f_i+\partial_i)[f_j+\partial_j]\\
%   &= \partial_i\partial_j - f_i\partial_j + f_j\partial_i -f_if_j + (\partial_i f_j).
% \end{align}
% The extra terms induce a boundary condition at jump discontinuities.  For the TM boundaries with 
% $f=\ln\sqrt{\epsilon}$, the boundary condition (for a Helmholtz equation) was 
% \begin{equation}
%   f(d-\epsilon) = e^{\Xi}f(d+\epsilon), \qquad
%   f'(d -\epsilon)= e^{-\Xi}f'(d+\epsilon)
% \end{equation}
% at discontinuities from $\epsr=1$ to $\epsr=1+\chi$, and $\Xi = \ln\sqrt{1+\chi}$.

% After expanding out the gradients, the wave equations for the potentials are 
% \begin{gather}
%    (\nabla^2-\epsilon\partial_t^2 + |\nabla \ln\epsilon^{-1/2}|^2 + \nabla^2\ln\epsilon^{-1/2})\phi_1=0\\
%    % [\partial_i\partial_j-\nabla^2\delta_{ij}) + \epsilon\partial^2_t\delta_{ij} - 
%    % (\partial_i\partial_j - e_i\partial_j+e_j\partial_i - e_ie_j+\partial_if_j)]A_j=0.
%    [(\nabla^2-\epsilon\partial^2_t)\delta_{ij} -e_i\partial_j+e_j\partial_i - e_ie_j+\partial_ie_j]A_j=0.
% \end{gather}
% where $e_i=\partial_i\ln\epsilon$.
% These are the same results one finds from the path integral, with Fadeev-Popov gauge fixing, in the 
% ``Feynman gauge'', where the gauge-fixing parameter $\xi=1$.

% The primary questions we wish to answer are: 1)Do the fields found from these equations of motion, 
% agree with the electric fields derived elsewhere? 2) Do you recover the same Casimir--Polder force?

% \subsection{Mode Functions at Planar Interface}

% Let the dielectric function only vary in one dimension, $\epsr(\vect{x})=1+\chi\theta(z)$.
% The terms involving $e_i$ are only non-zero on the boundary, and only $e_z$ is non-zero. 
% Away from the boundaries, the potentials obey: 
% \begin{gather}
%    (\nabla^2-\epsilon\partial_t^2)\phi_1=0\\
%    (\nabla^2-\epsilon\partial^2_t)A_i=0.
% \end{gather}
% Let us expand the fields as plane waves, 
% \begin{equation}
%   \phi_1(\vect{x},t) = \phi_1(\vect{k},\omega) e^{i\vect{k}\cdot\vect{x}-i\omega t}\\
%   A_i(\vect{x},t) = A_i(\vect{k},\omega) e^{i\vect{k}\cdot\vect{x}-i\omega t}\\
% \end{equation}
% Evidently $|\vect{k}|^2=\epsilon(z)\omega^2$ for both fields on either side of the interface. 
% If we are interested in Casimir--Polder shifts from the one side, we must also consider reflected waves.  
% We will follow Dan's notation and define $\vect{k}_- = (k_x,k_y,-k_z)$ to denote the wave reflecting
% off the interface.  

% \begin{gather}
%    (\nabla^2-\epsilon\partial_t^2 + |\partial_z \ln\epsilon^{-1/2}|^2 + \partial_z^2\ln\epsilon^{-1/2})\phi_1=0\\
%    % [\partial_i\partial_j-\nabla^2\delta_{ij}) + \epsilon\partial^2_t\delta_{ij} - 
%    % (\partial_i\partial_j - e_i\partial_j+e_j\partial_i - e_ie_j+\partial_if_j)]A_j=0.
%    [(\nabla^2-\epsilon\partial^2_t)\delta_{ij} -\delta_{iz}e_z\partial_j+\delta_{jz}e_z\partial_i 
%    - \delta_{iz}\delta_{jz}(e_z^2+\partial_ze_z)]A_j=0.
% \end{gather}



\section{Integrated Renormalized Two-Body Feynman-Kac Formula}

  The spatial integral over the solution $f_{12}$ in region I is
  \begin{align}
    J_I  &= \int_{-\infty}^{d_1}dx_0\,\big(f_{12}(\vect{x}_0)-f_{12}\sup0\big)\nonumber\\
    &=\int_{-\infty}^{d_1}dx_0
    e^{-2\sqrt{2(\lambda+\chi_1)}(d_1-x_0)}\frac{u\supTE_2 e^{-2\sqrt{2\lambda}d} - u\supTE_1}
    {\sqrt{2(\lambda+\chi_1)}(1-u\supTE_1u\supTE_2 e^{-2\sqrt{2\lambda}d})}   \nonumber\\
    % &\hspace{1cm}- e^{-2\sqrt{2(\lambda+\chi_1)}(d_1-x_0)}\dfrac{ - u\supTE_1}{\sqrt{2(\lambda+\chi_1)}} -  e^{-2\sqrt{2\lambda}(d_2-x_0)}
    % \frac{u\supTE_2}{\sqrt{2(\lambda)}}\bigg]\\
    &=\frac{u\supTE_2 e^{-2\sqrt{2\lambda}d} - u\supTE_1}{4(\lambda+\chi_1)(1-u\supTE_1u\supTE_2 e^{-2\sqrt{2\lambda}d})}%   \nonumber\\
    \label{eq:J1}
%    &\hspace{1cm}
%    + \dfrac{ u\supTE_1}{4(\lambda+\chi_1)} -  e^{-2\sqrt{2\lambda}d}\frac{u\supTE_2}{4\lambda}
  \end{align}
  The equivalent one-body expressions can be found by setting the one of the susceptibilities to zero.  
  The spatial integrals over the other regions are 
  \begin{align}
    J_{II} &= \int_{d_1}^{d_2}dx_0\,\big(f_{12}(\vect{x}_0)-f_{12}\sup0\big)\nonumber\\
    &=\int_{d_1}^{d_2}dx_0\bigg[\dfrac{2u\supTE_1u\supTE_2 e^{-2\sqrt{2\lambda}d} + u\supTE_1 e^{2\sqrt{2\lambda}(d_1-x_0)} 
    +u\supTE_2 e^{-2\sqrt{2\lambda}(d_2-x_0)}}{\sqrt{2\lambda}(1-u\supTE_1u\supTE_2 e^{-2\sqrt{2\lambda}d})}\bigg]\nonumber\\
    &=\frac{2d\,u\supTE_1u\supTE_2 e^{-2\sqrt{2\lambda}d}}{\sqrt{2\lambda}(1-u\supTE_1u\supTE_2 e^{-2\sqrt{2\lambda}d})}
    +\frac{(u\supTE_1+u\supTE_2)(1-e^{-2\sqrt{2\lambda}d})}{4\lambda(1-u\supTE_1u\supTE_2 e^{-2\sqrt{2\lambda}d})},
    \label{eq:J2}
  \end{align}
  and
  \begin{align}
    J_{II} &= \int_{d_2}^{\infty}dx_0\,\big(f_{12}(\vect{x}_0)-f_{12}\sup0\big)\nonumber\\
    &=\int_{d_2}^\infty dx_0\,e^{2\sqrt{2(\lambda+\chi_2)}(d_2-x_0)}\dfrac{(u\supTE_1 e^{-2\sqrt{2\lambda}d}-u\supTE_2)}
    {\sqrt{2(\lambda+\chi_2)}(1-u\supTE_1u\supTE_2 e^{-2\sqrt{2\lambda}d})}    \\
    &=\dfrac{(u\supTE_1 e^{-2\sqrt{2\lambda}d}-u\supTE_2)}
    {4(\lambda+\chi_2)(1-u\supTE_1u\supTE_2 e^{-2\sqrt{2\lambda}d})}    \label{eq:J3}
  \end{align}
  The total spatial integral for the fully renormalized two-body solution is found by adding together Eqs.(\ref{eq:J1})-(\ref{eq:J3}),
  and subtracting off the one-body integrals.  The result is
  \begin{align}
    &\int_{-\infty}^\infty dx_0\bigg[\big(f_{12}(\vect{x}_0)-f_{12}\sup0\big) -\big(f_{1}(\vect{x}_0)-f_{1}\sup0\big)
    -\big(f_{2}(\vect{x}_0)-f_{2}\sup0\big)\bigg]\\
    % 
   =&\frac{u\supTE_2 e^{-2\sqrt{2\lambda}d} - u\supTE_1}{4(\lambda+\chi_1)(1-u\supTE_1u\supTE_2 e^{-2\sqrt{2\lambda}d})} 
    +\frac{u\supTE_1}{4(\lambda+\chi_1)}- \frac{u\supTE_2 e^{-2\sqrt{2\lambda}d}}{4\lambda} 
    \nonumber\\
    &+\frac{2d\,u\supTE_1u\supTE_2 e^{-2\sqrt{2\lambda}d}}{\sqrt{2\lambda}(1-u\supTE_1u\supTE_2 e^{-2\sqrt{2\lambda}d})}
    +\frac{(u\supTE_1 +u\supTE_2)(1-e^{-2\sqrt{2\lambda}d})}
    {4\lambda(1-u\supTE_1u\supTE_2 e^{-2\sqrt{2\lambda}d})}\nonumber\\
    & -\frac{(u\supTE_1+u\supTE_2) (1-e^{-2\sqrt{2\lambda}d})}{4\lambda}\nonumber\\
    &+\dfrac{u\supTE_1 e^{-2\sqrt{2\lambda}d}-u\supTE_2}{4(\lambda+\chi_2)(1-u\supTE_1u\supTE_2 e^{-2\sqrt{2\lambda}d})}
    -\dfrac{u\supTE_1 e^{-2\sqrt{2\lambda}d}}{4\lambda}    +\dfrac{u\supTE_2}{4(\lambda+\chi_2)}
  \end{align}
  After some algebra \comment{ALGEBRA DEATH TRAP!}
  Then simplifying a little using $a/(1-x) -a = ax/(1-x)$.
\begin{align*}
  J=&+\frac{2d\,u\supTE_1u\supTE_2 e^{-2\sqrt{2\lambda}d}}{\sqrt{2\lambda}(1-u\supTE_1u\supTE_2 e^{-2\sqrt{2\lambda}d})}\nonumber\\
  &\frac{u\supTE_2 e^{-2\sqrt{2\lambda}d} - u\supTE_1}{4(\lambda+\chi_1)(1-u\supTE_1u\supTE_2 e^{-2\sqrt{2\lambda}d})} 
    +\frac{u\supTE_1}{4(\lambda+\chi_1)}   \\
    &   +\frac{(u\supTE_1 +u\supTE_2)(1-e^{-2\sqrt{2\lambda}d})}{4\lambda(1-u\supTE_1u\supTE_2 e^{-2\sqrt{2\lambda}d})}
    -\frac{(u\supTE_1+u\supTE_2)}{4\lambda}\\
    &+\dfrac{u\supTE_1 e^{-2\sqrt{2\lambda}d}-u\supTE_2}{4(\lambda+\chi_2)(1-u\supTE_1u\supTE_2 e^{-2\sqrt{2\lambda}d})}
       +\dfrac{u\supTE_2}{4(\lambda+\chi_2)}
  \end{align*}

\begin{align*}
  J=&+\frac{2d\,u\supTE_1u\supTE_2 e^{-2\sqrt{2\lambda}d}}{\sqrt{2\lambda}(1-u\supTE_1u\supTE_2 e^{-2\sqrt{2\lambda}d})}\nonumber\\
  &+\frac{u\supTE_2 e^{-2\sqrt{2\lambda}d}[1 - (u\supTE_1)^2]}{4(\lambda+\chi_1)(1-u\supTE_1u\supTE_2 e^{-2\sqrt{2\lambda}d})} 
    \\
    & +\frac{(u\supTE_1 +u\supTE_2)e^{-2\sqrt{2\lambda}d}[-1+u\supTE_1u\supTE_2]}
    {4\lambda(1-u\supTE_1u\supTE_2 e^{-2\sqrt{2\lambda}d})}\\
    &+\frac{u\supTE_1 e^{-2\sqrt{2\lambda}d}[1-(u\supTE_2)^2]}{4(\lambda+\chi_2)(1-u\supTE_1u\supTE_2 e^{-2\sqrt{2\lambda}d})}
  \end{align*}

%Rearranging central fraction, to cancel.  And factoring out common terms.
% \begin{align*}
%   J=&+\frac{e^{-2\sqrt{2\lambda}d}}{(1-u\supTE_1u\supTE_2 e^{-2\sqrt{2\lambda}d})}\bigg[\frac{2d\,u\supTE_1u\supTE_2}{\sqrt{2\lambda}}\nonumber\\
%   &+\frac{u\supTE_2[1 - (u\supTE_1)^2]}{4(\lambda+\chi_1)} +\frac{u\supTE_1 [1-(u\supTE_2)^2]}{4(\lambda+\chi_2)}\\
%     & -\frac{[u\supTE_1(1-(u\supTE_2)^2) +u\supTE_2(1-(u\supTE_1)^2)]}{4\lambda}\bigg]\\
%     %
%   =&+\frac{e^{-2\sqrt{2\lambda}d}}{(1-u\supTE_1u\supTE_2 e^{-2\sqrt{2\lambda}d})}\bigg[\frac{2d\,u\supTE_1u\supTE_2}{\sqrt{2\lambda}}\nonumber\\
%   -\frac{u\supTE_1u\supTE_1}{4}\left(\frac{u\supTE_1}{\lambda+\chi_1}+\frac{u\supTE_2}{\lambda+\chi_2}
%     -\frac{
%   &+\frac{u\supTE_2 - u\supTE_2 (u\supTE_1)^2}{4(\lambda+\chi_1)} +\frac{u\supTE_1-u\supTE_1(u\supTE_2)^2]}{4(\lambda+\chi_2)}\\
%     & -\frac{[u\supTE_1(1-(u\supTE_2)^2) +u\supTE_2(1-(u\supTE_1)^2)]}{4\lambda}\bigg]\\
%   \end{align*}
%Hoping to get 
% \begin{align*}
%   J=&+\frac{e^{-2\sqrt{2\lambda}d}}{(1-u\supTE_1u\supTE_2 e^{-2\sqrt{2\lambda}d})}
%   \bigg[\frac{2d\,u\supTE_1u\supTE_2}{\sqrt{2\lambda}}\nonumber\\
%    &+u\supTE_2(1 -u\supTE_1)(1+u\supTE_1)\frac{1}{4}\left( \frac{1}{\lambda+\chi_1} - \frac{1}{\lambda}\right)\\
%    &+u\supTE_1(1-u\supTE_2)(1+u\supTE_2)\frac{1}{4}\left(\frac{1}{\lambda+\chi_2}-\frac{1}{\lambda}\right)\bigg]
%   \end{align*}
All action is happening in second lines now.
% \begin{align*}
%   J_1=&+u\supTE_2\left(1-\frac{\sqrt{\lambda}-\sqrt{\lambda+\chi_1}}{\sqrt{\lambda}+\sqrt{\lambda+\chi_1}}\right)
%   \left(1+\frac{\sqrt{\lambda}-\sqrt{\lambda+\chi_1}}{\sqrt{\lambda}+\sqrt{\lambda+\chi_1}}\right)
%   \frac{1}{4}\left(\frac{\chi_1}{(\lambda+\chi_1)\lambda}\right)\\
%   =&+u\supTE_2\left(\frac{\sqrt{\lambda+\chi_1}}{\sqrt{\lambda}+\sqrt{\lambda+\chi_1}}\right)
%   \left(\frac{\sqrt{\lambda}}{\sqrt{\lambda}+\sqrt{\lambda+\chi_1}}\right)
%   \left(\frac{\chi_1}{(\lambda+\chi_1)\lambda}\right)\\
%   \end{align*}

  For comparison what is 
  \begin{align}
    \frac{1}{r}- r &= \frac{\sqrt{\lambda}+\sqrt{\lambda+\chi}}{\sqrt{\lambda}-\sqrt{\lambda+\chi}}
    - \frac{\sqrt{\lambda}-\sqrt{\lambda+\chi}}{\sqrt{\lambda}+\sqrt{\lambda+\chi}}\\
&= \frac{(\sqrt{\lambda}+\sqrt{\lambda+\chi})^2 - (\sqrt{\lambda}-\sqrt{\lambda+\chi})^2}
    {(\sqrt{\lambda}-\sqrt{\lambda+\chi})(\sqrt{\lambda}+\sqrt{\lambda+\chi})}\\
&= \frac{(2\sqrt{\lambda}\sqrt{\lambda+\chi})}
    {-\chi}\\
  \end{align}
  
  So we have something like 
  \begin{equation}
    (r^{-1}-r)^{-1} = \frac{r}{1-r^2}
  \end{equation}

\section{Integrated Renormalized TM Casimir Interaction Energy}

The integrated, renormalized Casimir interaction energy is
\begin{align}
I_{tot}% -I_1-I_2 + I_0 
=& 
  \frac{2u\supTM_1u\supTM_2 e^{-2\sqrt{2\lambda}d}d}{\sqrt{2\lambda}(1-u\supTM_1u\supTM_2 e^{-2\sqrt{2\lambda}d})} 
 - (u\supTM_1+u\supTM_2)\frac{e^{-2\sqrt{2\lambda}d}}{4\lambda(1-u\supTM_1u\supTM_2e^{-2\sqrt{2\lambda}d})}
\nonumber\\
&+\dfrac{u\supTM_2 e^{-2\sqrt{2\lambda}d}}{4(\lambda+\chi_1)(1-u\supTM_1u\supTM_2 e^{-2\sqrt{2\lambda}d})} 
+\frac{u\supTM_1 e^{-2\sqrt{2\lambda}d}}{4(\lambda+\chi_2)(1-u\supTM_1u\supTM_2 e^{-2\sqrt{2\lambda}d})}  \nonumber\\
& -   \left(\frac{u\supTM_1}{4(\lambda+\chi_1)}-\frac{(u\supTM_1+u\supTM_2)}{4\lambda}  
+ \frac{u\supTM_2}{4(\lambda+\chi_2)}\right)\frac{u\supTM_1u\supTM_2 e^{-2\sqrt{2\lambda}d}}{(1-u\supTM_1u\supTM_2 e^{-2\sqrt{2\lambda}d})}.
\end{align}
\comment{I know this works, so let's try it out}

First factor out the obvious common denominator. 
\begin{align}
I_{tot}% -I_1-I_2 + I_0 
=& \frac{e^{-2\sqrt{2\lambda}d}}{(1-u\supTM_1u\supTM_2 e^{-2\sqrt{2\lambda}d})} \bigg[
  \frac{2u\supTM_1u\supTM_2d}{\sqrt{2\lambda}} 
 - (u\supTM_1+u\supTM_2)\frac{1}{4\lambda}
\nonumber\\
&+\dfrac{u\supTM_2 }{4(\lambda+\chi_1)} 
 +\frac{u\supTM_1}{4(\lambda+\chi_2)}  \nonumber\\
& -   \left(\frac{u\supTM_1}{4(\lambda+\chi_1)}-\frac{(u\supTM_1+u\supTM_2)}{4\lambda}  
+ \frac{u\supTM_2}{4(\lambda+\chi_2)}\right)u\supTM_1u\supTM_2 \bigg]
\end{align}

Now try to group terms in $u\supTM_i$.  Also just focus on the square-bracketed term.
\begin{align}
 J&=  \frac{2u\supTM_1u\supTM_2d}{\sqrt{2\lambda}} 
 +\left(\frac{1}{\lambda+\chi_2}-\frac{1}{\lambda}\right)\frac{u\supTM_1}{4}
+\left(\frac{1}{\lambda+\chi_1}-\frac{1}{\lambda}\right)\frac{u\supTM_2}{4}
  \nonumber\\
& -   \left(\frac{u\supTM_1}{4(\lambda+\chi_1)}-\frac{(u\supTM_1+u\supTM_2)}{4\lambda}  
+ \frac{u\supTM_2}{4(\lambda+\chi_2)}\right)u\supTM_1u\supTM_2 
\end{align}


%%% Local Variables: 
%%% mode: latex
%%% TeX-master: "thesis_master"
%%% End: 
