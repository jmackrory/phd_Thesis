\chapter{Detailed Calculations}
\label{app:nasty_calc}
This appendix collects a couple lengthy, but tedious calculations required in the main text.  

\section{Integrated Renormalized Two-Body Feynman-Kac Formula}

The following calculation is required for the renormalized two-body energies for both TE and 
TM.  Since TE and TM mimic each other in the form of their solutions, most of this can proceed in 
parallel.  Only at the end is the exact form of the solution required.  The form of the two-body
solutions are given in Eqs.~(\ref{eq:Feynman-Kac TE two step}) and (\ref{eq:Feynman-Kac TM two step}).

  The spatial integral over the solution $f_{12}$  in region I is
  \begin{align}
    J_I  &= \int_{-\infty}^{d_1}dx_0\,\big(f_{12}(\vect{x}_0)-f_{12}\sup0\big)\nonumber\\
    &=\int_{-\infty}^{d_1}dx_0\,
    e^{-2\sqrt{2(\lambda+\chi_1)}(d_1-x_0)}\frac{r_2 e^{-2\sqrt{2\lambda}d} - r_1}
    {\sqrt{2(\lambda+\chi_1)}(1-r_1r_2 e^{-2\sqrt{2\lambda}d})}   \nonumber\\
    % &\hspace{1cm}- e^{-2\sqrt{2(\lambda+\chi_1)}(d_1-x_0)}\dfrac{ - r_1}{\sqrt{2(\lambda+\chi_1)}} -  e^{-2\sqrt{2\lambda}(d_2-x_0)}
    % \frac{r_2}{\sqrt{2(\lambda)}}\bigg]\\
    &=\frac{r_2 e^{-2\sqrt{2\lambda}d} - r_1}{4(\lambda+\chi_1)(1-r_1r_2 e^{-2\sqrt{2\lambda}d})}.
    \label{eq:J1}
  \end{align}
  The equivalent one-body expressions can be found by setting one of the susceptibilities to zero.  
  The spatial integrals over the other regions are 
  \begin{align}
    J_{II} &= \int_{d_1}^{d_2}dx_0\,\big[f_{12}(\vect{x}_0)-f_{12}\sup0\big]\nonumber\\
    &=\int_{d_1}^{d_2}dx_0\bigg[\dfrac{2r_1r_2 e^{-2\sqrt{2\lambda}d} + r_1 e^{2\sqrt{2\lambda}(d_1-x_0)} 
    +r_2 e^{-2\sqrt{2\lambda}(d_2-x_0)}}{\sqrt{2\lambda}(1-r_1r_2 e^{-2\sqrt{2\lambda}d})}\bigg]\nonumber\\
    &=\frac{2d\,r_1r_2 e^{-2\sqrt{2\lambda}d}}{\sqrt{2\lambda}(1-r_1r_2 e^{-2\sqrt{2\lambda}d})}
    +\frac{(r_1+r_2)(1-e^{-2\sqrt{2\lambda}d})}{4\lambda(1-r_1r_2 e^{-2\sqrt{2\lambda}d})},
    \label{eq:J2}
  \end{align}
  and
  \begin{align}
    J_{II} &= \int_{d_2}^{\infty}dx_0\,\big(f_{12}(\vect{x}_0)-f_{12}\sup0\big)\nonumber\\
    &=\int_{d_2}^\infty dx_0\,e^{2\sqrt{2(\lambda+\chi_2)}(d_2-x_0)}\dfrac{(r_1 e^{-2\sqrt{2\lambda}d}-r_2)}
    {\sqrt{2(\lambda+\chi_2)}(1-r_1r_2 e^{-2\sqrt{2\lambda}d})}    \\
    &=\dfrac{(r_1 e^{-2\sqrt{2\lambda}d}-r_2)}
    {4(\lambda+\chi_2)(1-r_1r_2 e^{-2\sqrt{2\lambda}d})}.    \label{eq:J3}
  \end{align}
  The total spatial integral for the fully renormalized two-body solution is found by adding together Eqs.~(\ref{eq:J1})--(\ref{eq:J3}),
  and subtracting off the one-body integrals.  The result is
  \begin{align}
   \int_{-\infty}^\infty &dx_0\bigg[\big(f_{12}(\vect{x}_0)-f_{12}\sup0\big) -\big(f_{1}(\vect{x}_0)-f_{1}\sup0\big)
    -\big(f_{2}(\vect{x}_0)-f_{2}\sup0\big)\bigg]\\
    % 
   =&\frac{r_2 e^{-2\sqrt{2\lambda}d} - r_1}{4(\lambda+\chi_1)(1-r_1r_2 e^{-2\sqrt{2\lambda}d})} 
    +\frac{r_1}{4(\lambda+\chi_1)}- \frac{r_2 e^{-2\sqrt{2\lambda}d}}{4\lambda} 
    \nonumber\\
    &+\frac{2d\,r_1r_2 e^{-2\sqrt{2\lambda}d}}{\sqrt{2\lambda}(1-r_1r_2 e^{-2\sqrt{2\lambda}d})}
    +\frac{(r_1 +r_2)(1-e^{-2\sqrt{2\lambda}d})}
    {4\lambda(1-r_1r_2 e^{-2\sqrt{2\lambda}d})}\nonumber\\
    & -\frac{(r_1+r_2) (1-e^{-2\sqrt{2\lambda}d})}{4\lambda}\nonumber\\
    &+\dfrac{r_1 e^{-2\sqrt{2\lambda}d}-r_2}{4(\lambda+\chi_2)(1-r_1r_2 e^{-2\sqrt{2\lambda}d})}
    -\dfrac{r_1 e^{-2\sqrt{2\lambda}d}}{4\lambda}    +\dfrac{r_2}{4(\lambda+\chi_2)}.
  \end{align}
  Pairs of common terms can be simplified by using $a/(1-x) -a = ax/(1-x)$, with $a=r_i$ and $x=r_1r_2e^{-2\sqrt{2\lambda}d}$.
\begin{align}
  J=&\frac{2d\,r_1r_2 e^{-2\sqrt{2\lambda}d}}{\sqrt{2\lambda}(1-r_1r_2 e^{-2\sqrt{2\lambda}d})}
  +\frac{r_2 e^{-2\sqrt{2\lambda}d} - r_1}{4(\lambda+\chi_1)(1-r_1r_2 e^{-2\sqrt{2\lambda}d})} 
    +\frac{r_1}{4(\lambda+\chi_1)}   \nonumber\\
    &   +\frac{(r_1 +r_2)(1-e^{-2\sqrt{2\lambda}d})}{4\lambda(1-r_1r_2 e^{-2\sqrt{2\lambda}d})}
    -\frac{(r_1+r_2)}{4\lambda}
    +\dfrac{r_1 e^{-2\sqrt{2\lambda}d}-r_2}{4(\lambda+\chi_2)(1-r_1r_2 e^{-2\sqrt{2\lambda}d})}
       +\dfrac{r_2}{4(\lambda+\chi_2)}.
  \end{align}
% \begin{align*}
%   J=&+\frac{2d\,r_1r_2 e^{-2\sqrt{2\lambda}d}}{\sqrt{2\lambda}(1-r_1r_2 e^{-2\sqrt{2\lambda}d})}\nonumber\\
%   &+\frac{r_2 e^{-2\sqrt{2\lambda}d}[1 - (r_1)^2]}{4(\lambda+\chi_1)(1-r_1r_2 e^{-2\sqrt{2\lambda}d})} 
%     \\
%     & +\frac{(r_1 +r_2)e^{-2\sqrt{2\lambda}d}[-1+r_1r_2]}
%     {4\lambda(1-r_1r_2 e^{-2\sqrt{2\lambda}d})}\\
%     &+\frac{r_1 e^{-2\sqrt{2\lambda}d}[1-(r_2)^2]}{4(\lambda+\chi_2)(1-r_1r_2 e^{-2\sqrt{2\lambda}d})}
%   \end{align*}
The exponential pieces are common to all terms and can be factored out.  
The terms can be grouped by their denominators 
\begin{align}
  J=\frac{e^{-2\sqrt{2\lambda}d}}{(1-r_1r_2 e^{-2\sqrt{2\lambda}d})}
    \bigg[\frac{2d\,r_1r_2 }{\sqrt{2\lambda}}
  +\frac{r_2 [1 - (r_1)^2]}{4(\lambda+\chi_1)} 
     +\frac{(r_1 +r_2)[-1+r_1r_2]}
    {4\lambda}
    +\frac{r_1[1-(r_2)^2]}{4(\lambda+\chi_2)}\bigg].
  \end{align}
The middle term with denominator $\lambda$ can then paired with the terms in $(\lambda+\chi_1)^{-1}$
and $(\lambda+\chi_2)^{-1}$,
\begin{align}
  J=\frac{e^{-2\sqrt{2\lambda}d}}{(1-r_1r_2 e^{-2\sqrt{2\lambda}d})}
    \bigg[&\frac{2d\,r_1r_2 }{\sqrt{2\lambda}}
    +r_2 [1 - (r_1)^2]\left(\frac{1}{4(\lambda+\chi_1)}-\frac{1}{4\lambda} \right)\nonumber\\
    &+r_1[1-(r_2)^2]\left(\frac{1}{4(\lambda+\chi_2)}-\frac{1}{4\lambda}\right)\bigg]
  \end{align}
After factoring out $r_1r_2$,  the result is
\begin{align}
  J=\frac{r_1r_2e^{-2\sqrt{2\lambda}d}}{(1-r_1r_2 e^{-2\sqrt{2\lambda}d})}
    \bigg[\frac{2d}{\sqrt{2\lambda}}
    -[r_1^{-1} - (r_1)]\frac{\chi_1}{4\lambda(\lambda+\chi_1)}
    -[r_2^{-1}-(r_2)]\frac{\chi_2}{4\lambda(\lambda+\chi_2)}\bigg].
    \label{eq:Jfinal}
  \end{align}
At this point, the exact form of the reflection coefficients must be used to proceed any further.  

\subsection{TE Reflection Coefficients}
For TE reflection coefficents~(\ref{eq:TE_coeff}), the integrated, renormalized two body solution is
% \begin{align*}
%   J=\frac{r\supTE_1r\supTE_2e^{-2\sqrt{2\lambda}d}}{(1-r\supTE_1r\supTE_2 e^{-2\sqrt{2\lambda}d})}
%     \bigg[\frac{2d}{\sqrt{2\lambda}}
%     -[\frac{\sqrt{\lambda}+\sqrt{\lambda+\chi_1}}{\sqrt{\lambda}-\sqrt{\lambda+\chi_1}}
%       -\frac{\sqrt{\lambda}-\sqrt{\lambda+\chi_1}}{\sqrt{\lambda}+\sqrt{\lambda+\chi_1}}]\frac{\chi_1}{4\lambda(\lambda+\chi_1)}
%     +\{1\leftrightarrow 2\}\bigg]
%   \end{align*}
% \begin{align*}
%   J=\frac{r\supTE_1r\supTE_2e^{-2\sqrt{2\lambda}d}}{(1-r\supTE_1r\supTE_2 e^{-2\sqrt{2\lambda}d})}
%     \bigg[\frac{2d}{\sqrt{2\lambda}}
%     -\bigg(\frac{(\sqrt{\lambda}+\sqrt{\lambda+\chi_1})^2-(\sqrt{\lambda}-\sqrt{\lambda+\chi_1})^2}
%     {\lambda-(\lambda+\chi_1)}\bigg)
%     \frac{\chi_1}{4\lambda(\lambda+\chi_1)}    +\{1\leftrightarrow 2\}\bigg]
%   \end{align*}
\begin{align}
  J\supTE&=\frac{r\supTE_1r\supTE_2e^{-2\sqrt{2\lambda}d}}{(1-r\supTE_1r\supTE_2 e^{-2\sqrt{2\lambda}d})}
    \bigg[\frac{2d}{\sqrt{2\lambda}}
    -\sum_{i=1}^2\bigg(\frac{4\sqrt{\lambda}\sqrt{\lambda+\chi_i}}{\lambda-(\lambda+\chi_i)}\bigg)
    \frac{\chi_i}{4\lambda(\lambda+\chi_i)}\bigg]\\
%   \end{align*}
% \begin{align*}
  &=\frac{r\supTE_1r\supTE_2e^{-2\sqrt{2\lambda}d}}{\sqrt{2\lambda}(1-r\supTE_1r\supTE_2 e^{-2\sqrt{2\lambda}d})}
    \bigg(2d
    +\frac{\sqrt{2}}{\sqrt{\lambda+\chi_1}} 
    +\frac{\sqrt{2}}{\sqrt{\lambda+\chi_2}}  \bigg).\label{eq:int_2body_soln_TE}
  \end{align}
This will be used to compute the Casimir energy between two dielectric half-spaces.  
The extra terms will be allow an integration by parts to occur, that considerably simplifies the expressions.


\subsection{TM Reflection Coefficients}
In contrast, for the TM polarization, the $\sqrt{\lambda}\rightarrow e^{2\Xi}\sqrt{\lambda}$, but $\sqrt{\lambda+\chi}$ is unchanged.
Note that the post-factor of $\chi_i/[4\lambda(\lambda+\chi_i)]$ in Eq.~(\ref{eq:Jfinal}) came from the integrating the  
the exponentials, rather than the reflection coefficients.  The TM reflection coefficients are given by Eq.~(\ref{eq:TM_coeff}).
After combining $r\supTM_i-1/r\supTM_i$, the result is
\begin{align}
  J\supTM&=\frac{r\supTM_1r\supTM_2e^{-2\sqrt{2\lambda}d}}{(1-r\supTM_1r\supTM_2 e^{-2\sqrt{2\lambda}d})}
  \bigg[\frac{2d}{\sqrt{2\lambda}}
  -\sum_{i=1}^2\bigg(\frac{4e^{2\Xi_i}\sqrt{\lambda}\sqrt{\lambda+\chi_i}}{\lambda\,e^{4\Xi_i}-(\lambda+\chi_i)}
  \frac{\chi_i}{4\lambda (\lambda+\chi_i)}\bigg)\bigg]\\
% \end{align*}
% \begin{align*}
  &=\frac{r\supTM_1r\supTM_2e^{-2\sqrt{2\lambda}d}}{\sqrt{2\lambda}(1-r\supTM_1r\supTM_2 e^{-2\sqrt{2\lambda}d})}
  \bigg[2d
  -\sum_{i=1}^2\frac{\sqrt{2}e^{2\Xi_i}\chi_i}{\sqrt{\lambda+\chi_i}[\lambda\,e^{4\Xi_i}-(\lambda+\chi_i)]}
 \bigg]\label{eq:int_2body_soln_TM}
\end{align}
Again, the extra terms allow an integration by parts to proceed, and for the Lifshitz form of the energy
to be recovered.  

% \subsection{Working backwards}
% Can also use
% \begin{align}
%   \frac{d}{dp}\log u &= \frac{d}{dp}\log(e^{2\Xi}p-\sqrt{p^2+\chi})-\frac{d}{dp}\log(e^{2\Xi}p+\sqrt{p^2+\chi}) \\
%   &= \frac{2\chi e^{2\Xi}}{\sqrt{p^2+\chi}[e^{4\Xi}p^2-(p^2+\chi)]}.
% \end{align}
% Also know
% \begin{align}
%   \frac{d}{dp}(1 - r_1r_2 e^{-2p t d}) & = r_1r_2e^{-2p t d}(2 t d - \frac{d}{dp}\log r_1-\frac{d}{dp}\log r_2)
% \end{align}
% So 
% \begin{align}
%   \frac{d}{dp}\log(1-r_1r_2 e^{-2p t d}) &= \frac{r_1r_2 e^{-2p t d}}{1-r_1r_2 e^{-2 p t d}}\left( 
%     2 t d -  \frac{2\chi_1 e^{2\Xi_1}}{\sqrt{p^2+\chi_1}[e^{4\Xi_1}p^2-(p^2+\chi_1)]}
%     -\frac{2\chi_2 e^{2\Xi_2}}{\sqrt{p^2+\chi_2}[e^{4\Xi_2}p^2-(p^2+\chi_2)]}.\right)
% \end{align}




%%% Local Variables: 
%%% mode: latex
%%% TeX-master: "thesis_master"
%%% End: 
