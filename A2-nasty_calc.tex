\chapter{Detailed Calculations}
\label{app:nasty_calc}
This Appendix collects a number of lengthy, but tedious calculations required in the main text.  

\section{Integrated Renormalized Two-Body Feynman-Kac Formula}

The following calculation is required for the renormalized two-body energies for both TE and 
TM.  Since TE and TM mimic each other in the form of their solutions, most of this can proceed in 
parallel.  Only at the end is the exact form of the solution required.

For this initial section, the working applies to both TE and TE polarizations. 

  The spatial integral over the solution $f_{12}$ in region I is
  \begin{align}
    J_I  &= \int_{-\infty}^{d_1}dx_0\,\big(f_{12}(\vect{x}_0)-f_{12}\sup0\big)\nonumber\\
    &=\int_{-\infty}^{d_1}dx_0
    e^{-2\sqrt{2(\lambda+\chi_1)}(d_1-x_0)}\frac{u_2 e^{-2\sqrt{2\lambda}d} - u_1}
    {\sqrt{2(\lambda+\chi_1)}(1-u_1u_2 e^{-2\sqrt{2\lambda}d})}   \nonumber\\
    % &\hspace{1cm}- e^{-2\sqrt{2(\lambda+\chi_1)}(d_1-x_0)}\dfrac{ - u_1}{\sqrt{2(\lambda+\chi_1)}} -  e^{-2\sqrt{2\lambda}(d_2-x_0)}
    % \frac{u_2}{\sqrt{2(\lambda)}}\bigg]\\
    &=\frac{u_2 e^{-2\sqrt{2\lambda}d} - u_1}{4(\lambda+\chi_1)(1-u_1u_2 e^{-2\sqrt{2\lambda}d})}%   \nonumber\\
    \label{eq:J1}
%    &\hspace{1cm}
%    + \dfrac{ u_1}{4(\lambda+\chi_1)} -  e^{-2\sqrt{2\lambda}d}\frac{u_2}{4\lambda}
  \end{align}
  The equivalent one-body expressions can be found by setting the one of the susceptibilities to zero.  
  The spatial integrals over the other regions are 
  \begin{align}
    J_{II} &= \int_{d_1}^{d_2}dx_0\,\big(f_{12}(\vect{x}_0)-f_{12}\sup0\big)\nonumber\\
    &=\int_{d_1}^{d_2}dx_0\bigg[\dfrac{2u_1u_2 e^{-2\sqrt{2\lambda}d} + u_1 e^{2\sqrt{2\lambda}(d_1-x_0)} 
    +u_2 e^{-2\sqrt{2\lambda}(d_2-x_0)}}{\sqrt{2\lambda}(1-u_1u_2 e^{-2\sqrt{2\lambda}d})}\bigg]\nonumber\\
    &=\frac{2d\,u_1u_2 e^{-2\sqrt{2\lambda}d}}{\sqrt{2\lambda}(1-u_1u_2 e^{-2\sqrt{2\lambda}d})}
    +\frac{(u_1+u_2)(1-e^{-2\sqrt{2\lambda}d})}{4\lambda(1-u_1u_2 e^{-2\sqrt{2\lambda}d})},
    \label{eq:J2}
  \end{align}
  and
  \begin{align}
    J_{II} &= \int_{d_2}^{\infty}dx_0\,\big(f_{12}(\vect{x}_0)-f_{12}\sup0\big)\nonumber\\
    &=\int_{d_2}^\infty dx_0\,e^{2\sqrt{2(\lambda+\chi_2)}(d_2-x_0)}\dfrac{(u_1 e^{-2\sqrt{2\lambda}d}-u_2)}
    {\sqrt{2(\lambda+\chi_2)}(1-u_1u_2 e^{-2\sqrt{2\lambda}d})}    \\
    &=\dfrac{(u_1 e^{-2\sqrt{2\lambda}d}-u_2)}
    {4(\lambda+\chi_2)(1-u_1u_2 e^{-2\sqrt{2\lambda}d})}    \label{eq:J3}
  \end{align}
  The total spatial integral for the fully renormalized two-body solution is found by adding together Eqs.(\ref{eq:J1})-(\ref{eq:J3}),
  and subtracting off the one-body integrals.  The result is
  \begin{align}
    &\int_{-\infty}^\infty dx_0\bigg[\big(f_{12}(\vect{x}_0)-f_{12}\sup0\big) -\big(f_{1}(\vect{x}_0)-f_{1}\sup0\big)
    -\big(f_{2}(\vect{x}_0)-f_{2}\sup0\big)\bigg]\\
    % 
   =&\frac{u_2 e^{-2\sqrt{2\lambda}d} - u_1}{4(\lambda+\chi_1)(1-u_1u_2 e^{-2\sqrt{2\lambda}d})} 
    +\frac{u_1}{4(\lambda+\chi_1)}- \frac{u_2 e^{-2\sqrt{2\lambda}d}}{4\lambda} 
    \nonumber\\
    &+\frac{2d\,u_1u_2 e^{-2\sqrt{2\lambda}d}}{\sqrt{2\lambda}(1-u_1u_2 e^{-2\sqrt{2\lambda}d})}
    +\frac{(u_1 +u_2)(1-e^{-2\sqrt{2\lambda}d})}
    {4\lambda(1-u_1u_2 e^{-2\sqrt{2\lambda}d})}\nonumber\\
    & -\frac{(u_1+u_2) (1-e^{-2\sqrt{2\lambda}d})}{4\lambda}\nonumber\\
    &+\dfrac{u_1 e^{-2\sqrt{2\lambda}d}-u_2}{4(\lambda+\chi_2)(1-u_1u_2 e^{-2\sqrt{2\lambda}d})}
    -\dfrac{u_1 e^{-2\sqrt{2\lambda}d}}{4\lambda}    +\dfrac{u_2}{4(\lambda+\chi_2)}
  \end{align}
  Then simplifying a little using $a/(1-x) -a = ax/(1-x)$.
\begin{align*}
  J=&+\frac{2d\,u_1u_2 e^{-2\sqrt{2\lambda}d}}{\sqrt{2\lambda}(1-u_1u_2 e^{-2\sqrt{2\lambda}d})}\nonumber\\
  &\frac{u_2 e^{-2\sqrt{2\lambda}d} - u_1}{4(\lambda+\chi_1)(1-u_1u_2 e^{-2\sqrt{2\lambda}d})} 
    +\frac{u_1}{4(\lambda+\chi_1)}   \\
    &   +\frac{(u_1 +u_2)(1-e^{-2\sqrt{2\lambda}d})}{4\lambda(1-u_1u_2 e^{-2\sqrt{2\lambda}d})}
    -\frac{(u_1+u_2)}{4\lambda}\\
    &+\dfrac{u_1 e^{-2\sqrt{2\lambda}d}-u_2}{4(\lambda+\chi_2)(1-u_1u_2 e^{-2\sqrt{2\lambda}d})}
       +\dfrac{u_2}{4(\lambda+\chi_2)}
  \end{align*}

\begin{align*}
  J=&+\frac{2d\,u_1u_2 e^{-2\sqrt{2\lambda}d}}{\sqrt{2\lambda}(1-u_1u_2 e^{-2\sqrt{2\lambda}d})}\nonumber\\
  &+\frac{u_2 e^{-2\sqrt{2\lambda}d}[1 - (u_1)^2]}{4(\lambda+\chi_1)(1-u_1u_2 e^{-2\sqrt{2\lambda}d})} 
    \\
    & +\frac{(u_1 +u_2)e^{-2\sqrt{2\lambda}d}[-1+u_1u_2]}
    {4\lambda(1-u_1u_2 e^{-2\sqrt{2\lambda}d})}\\
    &+\frac{u_1 e^{-2\sqrt{2\lambda}d}[1-(u_2)^2]}{4(\lambda+\chi_2)(1-u_1u_2 e^{-2\sqrt{2\lambda}d})}
  \end{align*}
Now factor out common terms.  Momentarily suppress $u$.
\begin{align*}
  J=\frac{e^{-2\sqrt{2\lambda}d}}{(1-u_1u_2 e^{-2\sqrt{2\lambda}d})}
    \bigg[\frac{2d\,u_1u_2 }{\sqrt{2\lambda}}
  +\frac{u_2 [1 - (u_1)^2]}{4(\lambda+\chi_1)} 
     +\frac{(u_1 +u_2)[-1+u_1u_2]}
    {4\lambda}
    +\frac{u_1[1-(u_2)^2]}{4(\lambda+\chi_2)}\bigg]
  \end{align*}
Then regroup across terms.  
\begin{align*}
  J=\frac{e^{-2\sqrt{2\lambda}d}}{(1-u_1u_2 e^{-2\sqrt{2\lambda}d})}
    \bigg[\frac{2d\,u_1u_2 }{\sqrt{2\lambda}}
    +u_2 [1 - (u_1)^2]\left(\frac{1}{4(\lambda+\chi_1)}-\frac{1}{4\lambda} \right)
    +u_1[1-(u_2)^2]\left(\frac{1}{4(\lambda+\chi_2)}-\frac{1}{4\lambda}\right)\bigg]
  \end{align*}
Now factor out $u_1u_2$.  
\begin{align*}
  J=\frac{u_1u_2e^{-2\sqrt{2\lambda}d}}{(1-u_1u_2 e^{-2\sqrt{2\lambda}d})}
    \bigg[\frac{2d}{\sqrt{2\lambda}}
    -[u_1^{-1} - (u_1)]\frac{\chi_1}{4\lambda(\lambda+\chi_1)}
    -[u_2^{-1}-(u_2)]\frac{\chi_2}{4\lambda(\lambda+\chi_2)}\bigg]
  \end{align*}
This is as far as you can get without using the exact form of the reflection coefficients.  
\subsection{TE reflection coefficients}
For TE reflection coefficents, get
% \begin{align*}
%   J=\frac{u\supTE_1u\supTE_2e^{-2\sqrt{2\lambda}d}}{(1-u\supTE_1u\supTE_2 e^{-2\sqrt{2\lambda}d})}
%     \bigg[\frac{2d}{\sqrt{2\lambda}}
%     -[\frac{\sqrt{\lambda}+\sqrt{\lambda+\chi_1}}{\sqrt{\lambda}-\sqrt{\lambda+\chi_1}}
%       -\frac{\sqrt{\lambda}-\sqrt{\lambda+\chi_1}}{\sqrt{\lambda}+\sqrt{\lambda+\chi_1}}]\frac{\chi_1}{4\lambda(\lambda+\chi_1)}
%     +\{1\leftrightarrow 2\}\bigg]
%   \end{align*}

% \begin{align*}
%   J=\frac{u\supTE_1u\supTE_2e^{-2\sqrt{2\lambda}d}}{(1-u\supTE_1u\supTE_2 e^{-2\sqrt{2\lambda}d})}
%     \bigg[\frac{2d}{\sqrt{2\lambda}}
%     -\bigg(\frac{(\sqrt{\lambda}+\sqrt{\lambda+\chi_1})^2-(\sqrt{\lambda}-\sqrt{\lambda+\chi_1})^2}
%     {\lambda-(\lambda+\chi_1)}\bigg)
%     \frac{\chi_1}{4\lambda(\lambda+\chi_1)}    +\{1\leftrightarrow 2\}\bigg]
%   \end{align*}

\begin{align*}
  J&=\frac{u\supTE_1u\supTE_2e^{-2\sqrt{2\lambda}d}}{(1-u\supTE_1u\supTE_2 e^{-2\sqrt{2\lambda}d})}
    \bigg[\frac{2d}{\sqrt{2\lambda}}
    -\bigg(\frac{4\sqrt{\lambda}\sqrt{\lambda+\chi_1}}{\lambda-(\lambda+\chi_1)}\bigg)
    \frac{\chi_1}{4\lambda(\lambda+\chi_1)}    +\{1\leftrightarrow 2\}\bigg]\\
%   \end{align*}
% \begin{align*}
  &=\frac{u\supTE_1u\supTE_2e^{-2\sqrt{2\lambda}d}}{\sqrt{2\lambda}(1-u\supTE_1u\supTE_2 e^{-2\sqrt{2\lambda}d})}
    \bigg[2d
    +\frac{\sqrt{2}}{\sqrt{\lambda+\chi_1}} 
    +\frac{\sqrt{2}}{\sqrt{\lambda+\chi_2}} 
 \bigg].
  \end{align*}
This is the result substituted into Eq.~\ref{


\subsection{TM Reflection Coefficients}
In contrast, for the TM polarization, the $\sqrt{\lambda}\rightarrow e^{2\Xi}\sqrt{\lambda}$, but $\sqrt{\lambda+\chi}$ is unchanged.
Note that the post-factor of $\chi_i/[4\lambda(\lambda+\chi_i)]$ came from the integrating the  
the exponentials, rather than the reflection coefficients.
This means the answer is
\begin{align*}
  J&=\frac{u\supTM_1u\supTM_2e^{-2\sqrt{2\lambda}d}}{(1-u\supTM_1u\supTM_2 e^{-2\sqrt{2\lambda}d})}
  \bigg[\frac{2d}{\sqrt{2\lambda}}
  -\sum_{i=1}^2\bigg(\frac{4e^{2\Xi_i}\sqrt{\lambda}\sqrt{\lambda+\chi_i}}{\lambda\,e^{4\Xi_i}-(\lambda+\chi_i)}
  \frac{\chi_i}{4\lambda (\lambda+\chi_i)}\bigg)\bigg]\\
% \end{align*}
% \begin{align*}
  &=\frac{u\supTM_1u\supTM_2e^{-2\sqrt{2\lambda}d}}{\sqrt{2\lambda}(1-u\supTM_1u\supTM_2 e^{-2\sqrt{2\lambda}d})}
  \bigg[2d
  -\sum_{i=1}^2\frac{\sqrt{2}e^{2\Xi_i}\chi_i}{\sqrt{\lambda+\chi_i}[\lambda\,e^{4\Xi_i}-(\lambda+\chi_i)]}
 \bigg]
\end{align*}



% \section{Integrated Renormalized TM Casimir Interaction Energy}

% The integrated, renormalized Casimir interaction energy is
% Suppress all TM superscripts.
% \begin{align}
% I_{tot}% -I_1-I_2 + I_0 
% =& 
%   \frac{2u_1u_2 e^{-2\sqrt{2\lambda}d}d}{\sqrt{2\lambda}(1-u_1u_2 e^{-2\sqrt{2\lambda}d})} 
%  - (u_1+u_2)\frac{e^{-2\sqrt{2\lambda}d}}{4\lambda(1-u_1u_2e^{-2\sqrt{2\lambda}d})}
% \nonumber\\
% &+\dfrac{u_2 e^{-2\sqrt{2\lambda}d}}{4(\lambda+\chi_1)(1-u_1u_2 e^{-2\sqrt{2\lambda}d})} 
% +\frac{u_1 e^{-2\sqrt{2\lambda}d}}{4(\lambda+\chi_2)(1-u_1u_2 e^{-2\sqrt{2\lambda}d})}  \nonumber\\
% & -   \left(\frac{u_1}{4(\lambda+\chi_1)}-\frac{(u_1+u_2)}{4\lambda}  
% + \frac{u_2}{4(\lambda+\chi_2)}\right)\frac{u_1u_2 e^{-2\sqrt{2\lambda}d}}{(1-u_1u_2 e^{-2\sqrt{2\lambda}d})}.
% \end{align}
% \comment{I know this works, so let's try it out}

% First factor out the obvious common denominator. 
% \begin{align}
% I_{tot}% -I_1-I_2 + I_0 
% =& \frac{e^{-2\sqrt{2\lambda}d}}{(1-u_1u_2 e^{-2\sqrt{2\lambda}d})} \bigg[
%   \frac{2u_1u_2d}{\sqrt{2\lambda}} 
%  - (u_1+u_2)\frac{1}{4\lambda}
% \nonumber\\
% &+\dfrac{u_2 }{4(\lambda+\chi_1)} 
%  +\frac{u_1}{4(\lambda+\chi_2)}  \nonumber\\
% & -   \left(\frac{u_1}{4(\lambda+\chi_1)}-\frac{(u_1+u_2)}{4\lambda}  
% + \frac{u_2}{4(\lambda+\chi_2)}\right)u_1u_2 \bigg]
% \end{align}

% Now try to group terms in $u_i$.  Also just focus on the square-bracketed term.
% \begin{align}
%  J&=%   \frac{2u_1u_2d}{\sqrt{2\lambda}} 
% %  +\left(\frac{1}{\lambda+\chi_2}-\frac{1}{\lambda}\right)\frac{u_1}{4}
% % +\left(\frac{1}{\lambda+\chi_1}-\frac{1}{\lambda}\right)\frac{u_2}{4}
% %   \nonumber\\
% % & -   \left(\frac{u_1}{4(\lambda+\chi_1)}-\frac{(u_1+u_2)}{4\lambda}  
% % + \frac{u_2}{4(\lambda+\chi_2)}\right)u_1u_2 \\
%  &=  \frac{2u_1u_2d}{\sqrt{2\lambda}} 
%  +\left(\frac{1}{\lambda+\chi_2}-\frac{1}{\lambda}\right)\frac{(u_1-u_1u_2)}{4}
% +\left(\frac{1}{\lambda+\chi_1}-\frac{1}{\lambda}\right)\frac{(u_2-u_1u_2^2)}{4}
% %   \nonumber\\
% % & -   \left(\frac{u_1}{4(\lambda+\chi_1)}-\frac{(u_1+u_2)}{4\lambda}  
% % + \frac{u_2}{4(\lambda+\chi_2)}\right)u_1u_2 
% \end{align}

\subsection{Working backwards}
Can also use
\begin{align}
  \frac{d}{dp}\log u &= \frac{d}{dp}\log(e^{2\Xi}p-\sqrt{p^2+\chi})-\frac{d}{dp}\log(e^{2\Xi}p+\sqrt{p^2+\chi}) \\
  &= \frac{2\chi e^{2\Xi}}{\sqrt{p^2+\chi}[e^{4\Xi}p^2-(p^2+\chi)]}.
\end{align}
Also know
\begin{align}
  \frac{d}{dp}(1 - u_1u_2 e^{-2p t d}) & = u_1u_2e^{-2p t d}(2 t d - \frac{d}{dp}\log u_1-\frac{d}{dp}\log u_2)
\end{align}
So 
\begin{align}
  \frac{d}{dp}\log(1-u_1u_2 e^{-2p t d}) &= \frac{u_1u_2 e^{-2p t d}}{1-u_1u_2 e^{-2 p t d}}\left( 
    2 t d -  \frac{2\chi_1 e^{2\Xi_1}}{\sqrt{p^2+\chi_1}[e^{4\Xi_1}p^2-(p^2+\chi_1)]}
    -\frac{2\chi_2 e^{2\Xi_2}}{\sqrt{p^2+\chi_2}[e^{4\Xi_2}p^2-(p^2+\chi_2)]}.\right)
\end{align}




%%% Local Variables: 
%%% mode: latex
%%% TeX-master: "thesis_master"
%%% End: 
