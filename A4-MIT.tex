\chapter{Review of Scattering Method}

This appendix briefly reviews the derivation of the fluctuating surface current method,
and comments on the simplifications that enforcing the boundary conditions offer.
% This numerical approach is related to prior multipole scattering approach~\cite{Rahi2009}.
% Earlier work chose to expand the scattering amplitudes in terms of mode function
% solutions.  In contrast, this work expands the fields at the surfaces using 
% boundary elements - in particular triangles.  

Freely available as ScuffEM package maintained by Homer Reid~\cite{ScuffEM2016}.  

\section{Path Integral Derivation}

This derivation is interesting from a couple points of view.  First, it is a general 
path integral for the electromagnetic field in media, which uses a general gauge. Earlier work 
used Weyl or temporal gauge where $A_0=0$~\cite{Rahi2009,Bechler1999}.
This has the nice effect that $\vect{E}=-\partial_t\vect{A}$,
but temporal gauge is not a complete gauge fixing, since it does not remove gauge transformations that are 
constant in time.  
The key step is enforcing electromagnetic boundary conditions at the surfaces.
As a result of that explicit boundary-matching, the gauge fixing is simple,
 and this computation can show that the results are gauge-invariant.
 In constrast, our attempts at the worldline method rely on these boundary conditions emerging 
from the partition function, since they must also be present in the resulting worldline path integral. 

The classical action is 
\begin{equation}
  S = \frac{1}{2}\int\frac{d\omega}{2\pi}\int d^3x\,\left[ 
    \epsilon(\omega,\vect{x})|\vect{E}(\omega,\vect{x})|^2 - \mu(\omega,\vect{x})|\vect{H}(\omega,\vect{x})|^2\right],
\end{equation}
where the fields have already been written in their Fourier components to better handle dispersion.  
The fields $\vect{E},\vect{H}$ can be rewritten in terms of the standard potentials, $\phi,\vect{A}$
where the initial Lagrangian density is 
\begin{equation}
 \mathcal{L}= \int\frac{1}{2}\int\frac{d\omega}{2\pi}\,\left[ 
    \epsilon(\omega,\vect{x})(\partial_i A^0+i\omega A^i)(\partial_i {A^0}^*-i\omega A^i)
- \frac{1}{\mu}(\epsilon_{ijk}\partial_jA^k\epsilon_{imn}\partial_mA^n)\right]
\end{equation}
% \begin{shaded}
% This follows since if we Fourier expand the real-valued time version of this we have 
% \begin{equation}
%   \int_{0}^T dt e^{i(\omega_n+\omega_mt)}A_nA_m = A_nA_m^*\delta_{n,-m}
% \end{equation}
% Since the fields are real, $A_n^* = A_{-n}$.  If we take the limit $T\rightarrow\infty$,
% the Fourier series is replaced by a sum.  (This will recur again when we confront finite temperature.)
% \end{shaded}

Note that this derivation proceeds under the assumption that we will  explicitly enforce 
EM boundary conditions at all surfaces.  Thus we can neglect and ignore any terms involving
$\partial_i\epsilon$, which only occur on the surfaces. 
This stands in contrast to our attempts at this path integral for the worldline method would retain these
terms to enforce boundary conditions, and to scale the gradients to be in flat space.  

The fields and Lagrangian can be written in their Euclidean form by Wick rotation
$\{\omega, A^0, {A^0}^*\}\rightarrow \{i\xi,iA_0,iA_0^*\}$.
% \begin{align}
%  \mathcal{L}&= \int\frac{1}{2}\int\frac{d\omega}{2\pi}\,\big[ 
%     \epsilon(\omega,\vect{x})(\partial_i A^0\partial_i {A^0}^*+i\omega A^i\partial_i {A^0}^*
%     -i\omega\partial_iA^0{A^i}^* +\omega^2A^i {A^i}^*)\nonumber\\
%     &- \frac{1}{\mu}(\partial_jA^k\partial_jA^k-\partial_jA^k\partial_kA^j)\big]\\
% % &= \int\frac{1}{2}\int\frac{d\omega}{2\pi}\,\big[ 
% %     \epsilon(\omega,\vect{x})( A^0(-\partial_i\partial^i) {A^0}^*+i\omega A^i\partial_i {A^0}^*
% %     -i\omega\partial_iA^0{A^i}^*+\omega^2A^i {A^i}^*)\nonumber\\
% %     &+ \frac{1}{\mu}(A^k\partial_j\partial_jA^k-A^k\partial_j\partial_kA^j) 
% %     -A^0(\partial_i\epsilon)\partial_iA^{0*} +A^k(\partial_j\frac{1}{\mu})(\partial_jA^k-\partial_kA^j)\big]
% &= \int\frac{1}{2}\int\frac{d\omega}{2\pi}\,\big[ 
%     \epsilon(\omega,\vect{x})( A^{0,*}(-\partial_i\partial^i) A^0-i\omega {A^0}^*(\partial_iA^i+A^i\partial_i\epsilon)
%     -i\omega\partial_iA^0{A^i}^*+\omega^2A^i {A^i}^*)\nonumber\\
%     &+ \frac{1}{\mu}(A^k\partial_j\partial_jA^k-A^k\partial_j\partial_kA^j) 
%     -A^0(\partial_i\epsilon)\partial_iA^{0*} +A^k(\partial_j\frac{1}{\mu})(\partial_jA^k-\partial_kA^j)\big]
% \end{align}
After the Wick rotation, and integration by parts, the Lagrangian density can be written as
\begin{equation}
  \mathcal{L}=\tilde{A}^{\mu,*} \bigg[\left(-\epsilon\xi^2+\frac{1}{\mu}\nabla^2\right)\delta_{\mu\nu} 
  - \alpha_\mu\alpha_\nu\bigg]\tilde{A}^\nu,
\end{equation}
where $(\tilde{A}^0,\tilde{A}^j) = (\sqrt{\epsilon\mu}A^0,\vect{A})$, 
and $\alpha_\mu=(i\sqrt{\epsilon}\xi,\frac{1}{\sqrt{\mu}}\nabla)$.
The terms in $(\alpha_\mu\tilde{A}^\mu)^2$ correspond to pure gauge degrees of freedom, and thus 
are irrelevant to physical states.  Since the boundary conditions will be explicitly
enforced elsewhere, the spatial dependence of $\epsilon, \mu$ can be ignored and these can be commuted
past differential operators with impunity.   As a result, the Lagrangian is
\begin{equation}
  \mathcal{L}=\tilde{A}^{\mu,*} [\alpha_\rho\alpha^\rho\delta_{\mu\nu} - \alpha_\mu\alpha_\nu]\tilde{A}^\nu,
\end{equation}
where the matrix has the form of a transverse projector in $\alpha^\mu$.  Note that this is proceeding
in exact analogy with path integral quantization of the electromagnetic field in free space.  
The matrix has a null eigenvalue along $\alpha_\mu$, which corresponds to gauge-transformations.  
The null eigenvalue is normally removed within path integrals by using the Fadeev-Popov method for gauge fixing. 
The essence of this method (adapted by t'Hooft) is to introduce a functional $\delta$-function that counts physically
distinct states only once, while keeping the correct weight for each state.  
%We will discuss gauge-fixing in greater detail later.  
% While Fadeev-Popov gauge fixing runs into issues in non-Abelian theories at very large field strengths,
% we are only going to use it for perturbative QED computations, well inside its range of validity.
% They then link up with the Fadeev-Popov method by noting that the Fadeev-Popov method handles gauge-fixing by introducing 
This is done by introducing 
\begin{equation}
  1 = \int d\omega \exp\left[-\int_0^{\hbar\beta}d\tau\,\frac{\alpha_{FP}\omega^2}{2} \right]
  \det\bigg[-\frac{\delta G}{\delta f}\bigg]\delta[G(a,f)-\omega]
\end{equation}
to the path integral, which is true up to divergent constants which cancel out under renormalization.
There are two factors of unity in this expression. First, the $\delta$-function and functional
 determinant restrict the integrals to fields with $G(a)=\omega$, while giving each of them equal weight.
Second, there is a Gaussian integral over $\omega$, which is constructed to put the gauge-fixing 
terms in the action in the exponential.

The functional derivative can be read off by making a gauge transformation on the gauge-condition.
In this computation a generalized Lorentz gauge $G[\tilde{A}^0,\tilde{A}^j] = i\omega \tilde{A}^0 + \partial_j \tilde{A}^j$
is fixed.  Under a gauge transformation
  $(\tilde{A}^0,A^j)\rightarrow  (\tilde{A}^0,A^j) + (i\omega,\partial^j)f$,
the functional derivative is, 
\begin{equation}
  \frac{\delta G}{\delta f} = \frac{G[A^0+i\omega f,A^j+\partial^jf]  -[A^0,A^j]}{f}
  = (-\omega^2 +\partial_j^2).
\end{equation}
This functional determinant is constant, and does not depend on the geometry at all, so it will cancel 
out in renormalization and can be dropped.
This is one of the simplifications from explicitly enforcing the boundary conditions.  

After gauge-fixing, the Lagrangian is given by
\begin{equation}
  \mathcal{L}=\tilde{A}^{\mu,*} [\alpha_\rho\alpha^\rho\delta_{\mu\nu}
  -(1-\alpha^{-1}_{FP})\alpha_\mu\alpha_\nu]\tilde{A}^\nu
=\tilde{\vect{A}}^{*}[\mathfrak{D}_1-\mathfrak{D}_2]\tilde{\vect{A}},
\end{equation}
where $\alpha_{FP}$ is an arbitrary parameter related to the width of the Gaussian.  
% Some common choices are $\alpha_{FP}^{-1}=1$ or Feynman gauge, and $\alpha^{-1}_{FP}=\infty$ is Landau gauge.
% Feynman gauge is simple since it cancels off the gauge dependent terms
The gauge-fixed partition function is
\begin{equation}
  Z_{EM} = \int [D\tilde{A}]_C e^{-\frac{\beta}{2}\int d\vect{x} \mathcal{L}},
\end{equation}
where $[D\tilde{A}]_C$ integrates over all field conditions that satisfy the boundary conditions.
The boundary conditions will be explicitly enforced via functional delta-functions.  

\subsection{EM Boundary Conditions}

% Recall that Maxwell's equations in media are:
% \begin{align}
%   \nabla\cdot\vect{D} =\rho  &\qquad  \nabla\times \vect{E} =-\partial_t \vect{B}\\
%  \nabla\cdot\vect{B}=0&\qquad   \nabla\times\vect{H} = -\partial_t \vect{D} +\vect{j} 
% \end{align}
In the absence of any free charges or currents, the electromagnetic boundary conditions are given by 
\begin{gather}
  \vect{E}^\|_1 =\vect{E}^\|_2 \qquad  \vect{H}^\|_1 =\vect{H}^\|_2 \\
  B_1^\perp = B_2^\perp\qquad   D_1^\perp = D_2^\perp,
\end{gather}
where the $1,2$ subscripts denote the media the fields are evaluated in.  
These boundary conditions can be enforced by introducing functional $\delta$-functions,
% We can enforce boundary conditions in path integrals via functional delta functions using
% the Fourier representation of the delta function 
% (Cites Bordag, and Kardar and Li)
\begin{equation}
  \delta(\vect{E}^\|_1-\vect{E}^\|_2) = \int D\vect{K} e^{i\int_{\partial \mathcal{O}} d^3\vect{x}\,
    \vect{K}\cdot(\vect{E}^\|_1-\vect{E}^\|_2)},
\end{equation}
where the integral takes place over the interface/boundary of the object~\cite{Bordag1985, Li1992}.
There is a similar expression for the magnetic fields $\vect{H}$, with $\vect{N}$ as the multiplier.  
These constraining fields can be thought of as Lagrange multiplier in the action.  
They can be physically thought of as fluctuating surface currents, as they 
are precisely the currents one must have to enforce the boundary conditions.  

The boundary conditions can be recast in terms of the gauge fields by introducing an appropriate
derivative operators $L^{E,r}$ and $L^{M,r}$.  These operators represent the usual 
relations between the physical fields $\vect{E},\vect{H}$ and the gauge potentials.  
In piece-wise constant media the operators are given by
\begin{equation}
  E_j:=L^{E,r}_{j\mu}\tilde{A}^\mu = -\frac{1}{\sqrt{\epsilon^r\mu^r}}\partial_j \tilde{A}^0+i\xi A^j,\qquad
  H_j:=L^{M,r}_{j\mu}\tilde{A}^\mu = \frac{1}{\mu^r}\epsilon_{jkl}\partial_k A^l
\end{equation}
In matrix form
\begin{gather}
  L^{E,r} = \left( \begin{array}{cccc} 
      -\frac{1}{\sqrt{\epsilon^r\mu^r}}\partial_x & i\xi & 0 & 0\\
      -\frac{1}{\sqrt{\epsilon^r\mu^r}}\partial_y & 0 &i\xi & 0\\
      -\frac{1}{\sqrt{\epsilon^r\mu^r}}\partial_z & 0 & 0 & i\xi
    \end{array}
  \right) \qquad
  L^{M,r} = \frac{1}{\mu_r}\left( \begin{array}{cccc} 
      0 & 0 & -\partial_z & \partial_y\\
      0 & \partial_z & 0 & -\partial_x\\
      0 & -\partial_y & \partial_x & 0
    \end{array}
  \right)
  \end{gather}
Note that the piece-wise continuous media assumption lets you commute derivatives and 
material functions with impunity. 


% The constrained path integral is meant to only include fields which satisfy the boundary conditions.
% This restriction is carried out by inserting the functional delta functions.  
%  resulting path integral over the potential $A^\mu$ is then Gaussian, the integral can be formally carried 
% out readily.  This will leave us with a picture of interacting surface currents.  

The partition function is restricted by introducing the functional $\delta$-functions on all surfaces,
and splitting the fields into regions internal to each body,
% \begin{align}
%   Z_{TE} &= \int D\tilde{A}^\mu \prod_rD\vect{K}_rD\vect{N}_r\exp\bigg\{-\frac{\beta}{2}\int d^3\vect{x}\,
%     \tilde{\vect{A}}^{*} [\mathfrak{D}_1 -\mathfrak{D}_2]\tilde{\vect{A}}\nonumber\\
%     &\hspace{1cm}  +i\sum_{r} \int_{\partial\mathcal{O}_r}d\vect{x}\,\vect{K}\cdot(
%     \vect{L}^{E,r}\tilde{\vect{A}}^{r}-\vect{L}^{E,e}\tilde{\vect{A}}^e)\nonumber\\
%     &\hspace{1cm}  +i\sum_{r} \int_{\partial\mathcal{O}_r}d\vect{x}\,\vect{N}\cdot(
%     \vect{L}^{M,r}\tilde{\vect{A}}^{r}-\vect{L}^{M,e}\tilde{\vect{A}}^e)\bigg\}
% \end{align}
%The partition function can be slightly re-arranged to a more suitable form,
\begin{align}
  Z_{TE} &= \int D\tilde{A}^\mu \prod_rD\vect{K}_rD\vect{N}_re^{-S_{A,e}-\sum_rS_{A,r}+i\sum_rC_r-iC_e}
\end{align}
where 
\begin{align}
  S_{A,e}&=\frac{\beta}{2}\int_e d^3\vect{x}\,  \tilde{\vect{A}}^{e,*} \mathfrak{D}^e\tilde{\vect{A}}^e \\
 S_{A,r}&=\frac{\beta}{2}\int_r d^3\vect{x}\, \tilde{\vect{A}}^{r,*} \mathfrak{D}^r\tilde{\vect{A}}^r\\
 C_r &=\int_{\partial\mathcal{O}_r}d\vect{x}\,
 (\vect{K}^r\cdot\vect{L}^{E,r} + \vect{N}^r\cdot\vect{L}^{M,r})\cdot\tilde{\vect{A}}^r\\
 C_e&=\sum_{r}\int_{\partial\mathcal{O}_r}d\vect{x}\,
 (\vect{K}^r\cdot\vect{L}^{E,e} + \vect{N}^r\cdot\vect{L}^{M,e})\cdot\tilde{\vect{A}}^e
\end{align}
where we use $r$ to denote the body at position $\vect{x}$, and $e$ to denote the exterior.
Note that both the operators, and fields depend on material constants, resulting in a somewhat
overburdened notation.  We have introduced bold-face to denote both vectors and matrices.

At this point, the integral over fields internal to the bodies $\tilde{\vect{A}}^r$, and fields 
external to all bodies, $\tilde{\vect{A}}^e$ is carried out.  The internal integral couples the currents
on the same body to one another, while the external integral couples currents on different bodies together.
The Gaussian integral over $\vect{A}$ in each region can be carried out using
\begin{equation}
  \int D\phi e^{-\frac{\beta}{2}\int d^3\vect{x}\, [\phi^i \mathcal{L}_{ij}\phi^j +i+\psi^k\phi_k]}
= e^{-\frac{1}{2\beta}\int d^3\vect{x}\,\psi^k G_{km} \psi^m},
\end{equation}
where $\mathcal{L}_{ij}G^{jm} = \delta_{i}^m$, and $G$ is the Green function corresponding to $L$.

The resulting path integral is
\begin{align}
  Z_{TE} &= \int \prod_rD\vect{K}_rD\vect{N}_re^{-S_e[\vect{K}_r,\vect{N}_r] - S_r[\vect{K}_r,\vect{N}_r] }
\end{align}
where the external and internal effective actions are given by 
\begin{align}
  S_e&:=\frac{1}{2\beta}\sum_{r,r'}
  \int\limits_{\partial\mathcal{O}_r}d\vect{x}\int\limits_{\partial\mathcal{O}_{r'}}d\vect{x'}\, 
  (\vect{K}^r\cdot\vect{L}^{E,e} + \vect{N}^r\cdot\vect{L}^{M,e})(x)\cdot\vect{G}^e\cdot
  (\vect{K}^{r'}\cdot\vect{L}^{E,e} + \vect{N}^{r'}\cdot\vect{L}^{M,e})(x')\\
  S_r&:= \frac{1}{2\beta}\sum_{r}\int_{\partial\mathcal{O}_r}d\vect{x}\int_{\partial\mathcal{O}_r}d\vect{x'}\,
    [\vect{K}^r\cdot\vect{L}^{E,r} + \vect{N}^r\cdot\vect{L}^{M,r}](x)\cdot
    \vect{G}^r\cdot[\vect{K}^r\cdot\vect{L}^{E,r} + \vect{N}^r\cdot\vect{L}^{M,r}](x')\bigg\}.
\end{align}
Note that the $S_r$ correspond to body self-energies, as do terms with $r=r'$ from $S_e$.
The renormalization happens at the end of the computation, so these should not be removed yet. 
% (Note you don't carry out renormalization yet - this is added to interaction terms.  When we compute
% the determinant these do not cleanly decouple, thus you can't just throw this away.)

% Both of these correspond to photon exchange from patches on the same body.  
% These terms renormalize out if we consider energy differences between configurations where the 
% bodies are removed to spatial infinity.  These terms are independent of all other positioning
% of the bodies, and amount to additive constants to the action.  In the partition function
% they are multiplicative constants, $Z_{TE,r}$, where we can also use
% $ Z_{TE,\infty} = \prod_r Z_{TE,r}$, since there are no interactions between the bodies in this limit.
Focusing on the interaction terms, the effective interaction integrand of the path integral 
can be written as 
\begin{align}
  S_{e}[\vect{K},\vect{N}] &= \frac{1}{2\beta}\sum_{r\ne r'}
  \int\limits_{\partial\mathcal{O}_r}d\vect{x}\int\limits_{\partial\mathcal{O}_{r'}}d\vect{x'}\, 
\left(\hspace{-0.1cm} \begin{array}{cc} 
    \vect{K}^r(x) &   \vect{N}^r(x)
  \end{array}
\hspace{-0.1cm}
\right)
\left(\begin{array}{cc} 
    \vect{L}^{E,e}\vect{G}\vect{L}^{E,e} &\vect{L}^{E,e}\vect{G}\vect{L}^{M,e}\\
    \vect{L}^{M,e}\vect{G}\vect{L}^{E,e} &\vect{L}^{M,e}\vect{G}\vect{L}^{M,e}\\
  \end{array}
  \right)
\left(\hspace{-0.1cm} \begin{array}{c} 
    \vect{K}^{r'}(x') \\   \vect{N}^{r'}(x')
  \end{array}
\hspace{-0.1cm}
\right)
\end{align}
The corresponding interior term is replaces $L$ and $G$ with their interior equivalents.  

It turns out that the combined matrices $L^P G L^Q$ correspond exactly to the usual homogenous Green function
dyadics, $\Gamma^{PQ}$ for $P, Q\in\{E,M\}$.  In addition, the gauge-pieces do not contribute at all.  

First, we need the Green function $\mathbf{G}$ corresponding to $\mathcal{D} =\mathcal{D}_1-\mathcal{D}_2$.
Fortunately, the projector structure makes this is easy to derive in Fourier space.  Recall that
\begin{align}
  \mathfrak{D} = \alpha^2[P_T +\alpha^{-1}_{FP}P_L]
\end{align}
where $\alpha_\mu=(i\sqrt{\epsilon}\xi,\frac{1}{\sqrt{\mu}}\nabla)$, and we introduced
the projection operators which the following properties
\begin{gather}
  P_L+P_T = I\\
  P_L = \frac{\alpha^\mu\alpha^\nu}{\alpha^2}\\
  P_T = \delta_{\mu\nu}-\frac{\alpha^\mu\alpha^\nu}{\alpha^2}.
\end{gather}
(Reid points out that manipulating these things is actually easiest in momentum space.  We can use
\begin{equation}
  \int d\vect{x} F(\vect{x})D(\partial_i)G(\vect{x}) = \int \frac{d\vect{q}}{(2\pi)^d}
  F(\vect{q})D(i q_j)G(-\vect{q}).
\end{equation}
Then we can find the Fourier-space green function using 
$\alpha(q) = i(\sqrt{\epsilon}\xi,\frac{\vect{q}}{\sqrt{\mu}})$
\begin{align}
  I &=\mathfrak{D}G \\
  &=\alpha^2[P_T -\alpha^{-1}_{FP}P_L](a G_T+bG_L)\\
  \rightarrow \vect{G}& = \frac{1}{\alpha^2}(P_T - \alpha_{FP}P_L)
   = -\frac{\mu_r}{\kappa^2+\vect{q}^2}[\delta^{\mu\nu}
  + \frac{\mu_r(1-\alpha_{FP})}{\kappa^2+\vect{q}^2}\alpha^\mu\alpha^\nu],
\end{align}
where we used $\alpha^2 = -\epsilon_r\mu_r\xi^2-\vect{q}^2$,  and defined 
$\kappa = \sqrt{\epsilon\mu}\xi$.
None of the longitudinal components of $\alpha^\mu\alpha^\nu$ contribute to the path 
integral, since under the appropriate differentiation, they all vanish.  


%%% Local Variables: 
%%% mode: latex
%%% TeX-master: "thesis_master"
%%% End: 
