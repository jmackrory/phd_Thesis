\abstract{

The Casimir effect refers to an attractive force between material bodies due to quantum fluctuations
in the electromagnetic field.  The Casimir effect is difficult to calculate in general, since it 
is sensitive to the exact shapes of the bodies.
 Calculating the Casimir effect between general bodies usually requires a numerical approach.
The worldline method computes Casimir energies by creating an ensemble of space-time paths corresponding
to a virtual particle interacting with the bodies.  This was originally developed for interacting scalar
fields, rather than the vector electromagnetic field.  

This thesis presents results extending the worldline method to account for the material properties 
and polarizations of electromagnetism.  The method will be initially developed  for the case of planar geometries 
of bodies, with the goal of developing a method that holds in arbitrary arrangements of bodies.  

This thesis starts by covering background material on path integrals, 
and quantizing the electromagnetic field in media.
The electromagnetic field is decomposed in terms
of two scalar fields for planar bodies. The scalar fields correspond to the 
transverse-electric and transverse-magnetic polarizations of the electromagnetic field.
The worldline path integrals are developed for both polarizations, and solved analytically.
Next, numerical methods are developed and tested in the context of planar bodies.  
The numerical results show agreement with the known analytical solutions.  The convergence
properties of the method are also discussed.  
Finally, progress towards a general worldline method based on coupling the two scalar polarizations is discussed.
}


