\abstract{

% This thesis presents results from two projects. 
% In the first project, we extend a method of computing Casimir energies towards more realistic.
The Casimir effect refers to an attractive force between material bodies due to fluctuations
in the electromagnetic field.  The Casimir effect is difficult to calculate in general, since it 
is sensitive to the exact shapes of the bodies, which generally necessitates a numerical approach.
The worldline method computes Casimir energies by creating an ensemble of space-time paths corresponding
to a virtual particle interacting with the bodies.  This was originally developed for interacting scalar
fields, rather than the vector electromagnetic field.  

%We present results that extend the worldline method of computing Casimir energies to electromagnetism.
This thesis presents results extending the worldline method to account for the material properties 
and polarizations of electromagnetism.  Initially we focus on the case of planar geometries 
of bodies, with the goal of developing a method that holds in arbitrary arrangements of bodies.  

We cover the requisite background on path integrals,
quantizing the EM field in media, and exploiting the Feynman-Kac formula to find analytical
solutions to the path integrals.  We develop and discuss the numerical methods, and show agreement
with prior results and discuss the convergence properties of the method.  Finally we discuss progress
towards a fully general method.

% The second project discusses continuous position measurements of atoms.  The position of an atom
% can be measured by illuminating the atom with resonant light, and imaging the resulting light.  
% Current experiments are pushing towards realizing such experiments where the crossover between
% an atom's quantum mechanical motion, and classical motion can be observed.  
% We discuss the quantum trajectory formulation for constructing an atom's quantum trajectory
%  based on measurements from an electron-multiplying charge-coupled device (EMCCD) camera.  
% We will also discuss the reflection of atoms from a strong non-uniform position measurement.  
}


