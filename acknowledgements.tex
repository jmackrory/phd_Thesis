\acknowledge{
  First, I would like to acknowledge my advisor, Daniel Steck.  
  All of the work in this thesis was undertaken in very close collaboration with him.
  Dan's indefatigable optimism kept me from giving up on this approach, 
  and any time I thought we were permanently stuck Dan would find another way of looking at the problem.
  The big ideas that allowed the numerical methods for the TM polarization to work at all were his.  
  Much of my understanding of stochastic processes, quantum trajectories and the Casimir effect
  has come from studying his notes on ``Quantum and Atom Optics'', which is reflected in the number of times they are
  referenced.
    
  Next I would like to acknowledge our collaborators, Tanmoy Bhattacharya and Kurt Jacobs.
  In particular Tanmoy has provided both the basic ground for us starting on the worldline 
  Casimir project in explaining path integrals and the novel techniques involved in the worldline path integrals.
  In addition he has proven a great oracle for new ideas in Monte-Carlo methods,
  and skeptical tests of some of our more outlandish ideas.  
  Kurt collaborated with us on a paper on non-uniform quantum position measurements, 
  which unfortunately was left out of this thesis.

  I would like to thank my lab-mates, Jeremy Thorn, Elizabeth Schoene, Eryn Cook, Paul Martin,
  Wes Erickson, and Richard Wagner. %  In particular, Jeremy developed the camera model used in the chapter 
  % on quantum measurement.
  We have had many discussions about my work over the years, and I've appreciated learning  
  a little about how much work goes into a building a cold trapped-atom experiment.
  I also appreciate the hours of companionship we've spent together outside the lab, be it hiking, 
  gaming, or watching terrible movies.

  Finally, I would like to thank my wife Erin Mondloch for her emotional support throughout
 this project and particularly during the writing phase.  You are by far the best thing 
 I have found in my time here.

 I would also like to acknowledge the financial support of the University of Oregon Physics Department
 throughout the years.

}


%%% Local Variables: 
%%% mode: latex
%%% TeX-master: "thesis_master"
%%% End: 
