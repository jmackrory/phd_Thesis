\chapter{Conclusion}

The goal of this thesis was to develop a general purpose numerical method
employing the worldline method to calculate electromagnetic Casimir energies. 
We have been partially successful in those aims.   

Following~\citet{Bordag1998,Bordag1999}, we developed a full vector path integral~(\ref{eq:vector_path_integral}) for
the EM field.  So far it has not be implemented as a numerical method.
Instead, we developed an approximate worldline description for the EM field in terms of two independent scalar fields, corresponding 
to the TE and TM polarizations.   
Although the decoupled scalars are adapted to a planar geometry, 
they share some similarities with the potentials in the full vector
path integral, and are a useful test case in their own right.  

We showed analytically and numerically that the polarization worldline path integrals recover the known expressions for the 
Casimir--Polder and Casimir energies in planar geometries, at zero and high temperature.  
Doing so involved regularizing singular TM potentials, and finding analytical solutions to the path integral
in certain geometries.  The analytical expressions for the path average of the TM potential are 
 essential for numerical computations with this method.  

Even with regularized solutions, it was necessary to develop techniques to efficiently
sample the worldline path integral.  The TE integrand was relatively simple to evaluate, while the TM
integrand is much more challenging and still under study.
The birth-death method for sampling paths was essential for bringing the statistical errors under control, 
and the partial averaging method also allowed us to evaluate the derivatives required for the TM method.
The numerical methods we developed are in agreement with the expected analytical results.
% where previously, the finite difference method had been previously plagued by convergence issues.

The methods that were developed could be used as an (uncontrolled) approximation to the Casimir effect in a general geometry.
They will also probably be useful in handling the vector path integral.    
In cases where path integrals can be analytically solved for open Brownian bridges [such as
Dirichlet~(\ref{eq:Dirichlet}) and TM boundary conditions~(\ref{eq:TM_potential})], 
those expressions can be applied locally at each step of the path.  
At each step, the potential could be computed using a local planar approximation to the exact solution.
The local solutions joined together along the path, could form a basis for solving a path integral
in general, based on the local approximations throughout the path.  

Another possible approach to leveraging the results contained here into a general method is to consider 
how the two scalar polarizations are coupled.  
At each point along the path, the EM field could be split into the TE and TM polarizations based 
on the nearest surface normal.
The weights for the polarizations are the components of an auxiliary two component vector that travels along the path.
At each step, the terms acquire the appropriate TE or TM potential, 
and are then coupled together via a rotation matrix where the rotation
angle depends on the change in the surface normal.  

% In the introduction we noted that the scattering method is currently the only general
% method for computing EM Casimir energies in arbitrary geometries, and that is still true.  
The worldline method has not yet been generalized to full electromagnetism.   
%The work presented here brings that prospect closer.  
However, the worldline has a number of attractive features such as its simple parallelism, and the possibility
for superior performance in very complicated geometries.  Given the progress thus far,  
I believe that this method is worth developing further, where it could complement existing methods
and may have uses in electromagnetism beyond just Casimir physics.




%%% Local Variables: 
%%% mode: latex
%%% TeX-master: "thesis_master"
%%% End: 
