\chapter{Conclusion}

Broad summary text.  Note of future directions and applications.

\begin{itemize}
  \item Developed scalar worldlines.  
  \item Goal was for Gaussian, geometry independent, method for computing Casimir energies taking
    into account dielectric properties.
    Check against known calculations.
    
  \item Really missing on geometry independent, since polarization decomposition is restricted to
    particular geometries.  Have idea of coupling on each step, with rotation/coupling based on how
    nearest normal changes direction.
    Ad-hoc.  Hard to definitively check and debug.
  \item Carefully studied convergence properties, and developed numerical methods, that while 
    only tested for planes could generalize as uncontrolled approximations.  (Add fields in quadrature)
  \item In comparison with Scattering method - that is mature code, that works quickly.
    It is currently limited to surface integrals, but apparently, can use effective surface currents
    to capture slower variation.  State of the art actually employed in research groups looking to 
    compute their Casimir-Polder potential.
  \item Worldline may be useful companion for very different underpinnings, and convergence properties.
    Worldline path resolution may be analogous to fine enough tesselation of surfaces.  
    As yet, neither method can handle anisotropic media.  
    
  \item Also developed model for quantum trajectories accounting for camera noise.  
    May one day be useful to lab in reconstructing atomic trajectories.

  \item Maybe: 
    Developed CUDA code for both worldlines (path generation, branching) and quantum trajectories
    (weighting outcomes, FFT for split-operator)
\end{itemize}



%%% Local Variables: 
%%% mode: latex
%%% TeX-master: "thesis_master"
%%% End: 
