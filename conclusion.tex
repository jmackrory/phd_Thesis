\chapter{Conclusion}

The goal for this thesis was to develop a general purpose numerical method
employing the worldline method to calculate electromagnetic Casimir effects. 
We have been partially successful in those aims.   % Although we have not verified that the full vector
% path integral~(\ref{eq:vector_path_integral}) gives the correct answers.

We developed a full vector path integral~(\ref{eq:vector_path_integral}), but have not 
been able to solve it, or verify that it gives the correct answers.
Instead, we developed a worldline description for the EM field in terms of two independent scalar fields, corresponding 
to the TE and TM polarizations.   
Although the decoupled scalars are adapted to a planar geometry (although they may describe
a spherical geometry as well), they share some similarities with the potentials in the full vector
path integral, and are a useful test case in their own right.    

We were able to show that the polarization worldline path integrals recover the known expressions for the 
Casimir--Polder and Casimir energies in planar geometries, at zero and high temperature.  
Doing so involved regularizing the singular TM potentials, and finding analytical solutions to the path integral
in certain geometries.  The analytical expressions for the path average of the TM potential are 
absolutely essential for numerical computations with this method method.  

Even with the regularized solutions, it was necessary to develop a number of techniques to efficiently
sample the worldline path integral.  The TE integrand was relatively simple to evaluate and explore, while the TM
integrand was much more challenging and is still under study.
The birth-death method for sampling paths was essential for bringing the numerical fluctuations under control.
The partial averaging method also allowed us to evaluate the derivatives required for the TM method.
The numerical methods we developed were able to reproduce the expected analytical results.
% where previously, the finite difference method had been previously plagued by convergence issues.

The methods that were developed could be used as an uncontrolled approximation to the Casimir effect in a general geometry.
and will probably be useful in handling the full vector path integral.    
In cases where path integrals could be found for open Brownian bridges [such as
Dirichlet~(\ref{eq:Dirichlet}) and TM boundary conditions~(\ref{eq:TM_potential})], 
those expressions can be applied locally as a path propagates.  
At each step, the potential could be computed using a local planar approximation to the exact solution,
which would extend them to a more general method.  

Another possible approach to leveraging the results contained here into a general method is to consider coupling the two 
scalar polarizations.  
At each point along the path, the EM field could be split into the TE and TM polarizations based 
on the nearest surface normal.
As the path propagates through space, that nearest surface normal would change direction, which would couple the fields 
together.  

The simulations carried out here were written for a typical desktop CPU.
The worldline is also a natural fit for simulations on a graphics processing unit (GPU), 
which contains a large number of processors suited to parallel programming. 
It may be highly effective to port the TE and TM worldline methods over to a GPU.
However, programming a GPU is a somewhat involved endeavor, and much of the code would have to be 
rewritten to take advantage of this new architecture.  

In the introduction we noted that the scattering method is currently the only general
method for computing Casimir energies in arbitrary geometries.  That is still true.  The worldline method,
while it has a number of attractive features, has not yet been generalized to full electromagnetism.   
Hopefully, the work presented here brings that prospect closer.  

%\begin{itemize}
%  \item Developed scalar worldlines.  
  % \item Goal was for Gaussian, geometry independent, method for computing Casimir energies taking
  %   into account dielectric properties.
  %   Check against known calculations.
    
  % \item Really missing on geometry independent, since polarization decomposition is restricted to
  %   particular geometries.  Have idea of coupling on each step, with rotation/coupling based on how
  %   nearest normal changes direction.
  %   Ad-hoc.  Hard to definitively check and debug.
  % \item Carefully studied convergence properties, and developed numerical methods, that while 
  %   only tested for planes could generalize as uncontrolled approximations.  (Add fields in quadrature)
  % \item Need markov-chain methods to temper fluctuations.  Introduced monte carlo sampling for both 
  %   positino and time.  
  %   Effective boundary conditions, and Feynman-Kac formulae for accelerating convergence.
  % \item In comparison with Scattering method - that is mature code, that works quickly.
  %   It is currently limited to surface integrals, but apparently, can use effective surface currents
  %   to capture slower variation.  State of the art actually employed in research groups looking to 
  %   compute their Casimir-Polder potential.
  % \item Worldline may be useful companion for very different underpinnings, and convergence properties.
  %   Worldline path resolution may be analogous to fine enough tesselation of surfaces.  
  %   As yet, neither method can handle anisotropic media.  
    
  % \item Also developed model for quantum trajectories accounting for camera noise.  
  %   May one day be useful to lab in reconstructing atomic trajectories.

  % \item Maybe: 
  %   Developed CUDA code for both worldlines (path generation, branching) and quantum trajectories
  %   (weighting outcomes, FFT for split-operator)
%\end{itemize}



%%% Local Variables: 
%%% mode: latex
%%% TeX-master: "thesis_master"
%%% End: 
