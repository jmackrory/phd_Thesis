\subsection{Worst prediction in physics}

Let us briefly note that the Casimir energy is sometimes related to the cosmological constant in general relativity.
  ``one of the worst predictions in physics.''  
In Einstein's General Theory of Relativity, energy is coupled to the metric of spacetime.
  The cosmological constant acts to drive accelerating expansion~\cite{Carroll2004}.
  Note that the cosmological constant can be provided by the vacuum energy of the quantum fields of matter on spacetime.
  Note however that it is the energy itself that shows up, not energy differences. 
If we attempt to identify the vacuum energy with the cosmological constant, we estimate the energy from 
\begin{equation}
E \sim \int_0^\Lambda dk\,k^2 (\hbar c k),
\end{equation}
where $\Lambda$ is a high-frequency cutoff, which denotes the short wavelength
 below which our effective description of the world in terms of quantum 
field theory breaks down.
  If we pick $\Lambda=[c^3/(\hbar G)]^{1/3}$ to be the Plank wavelength, 
where $G$ is Newton's gravitation constant, at which point quantum gravity
 effects are expected to become important, then we predict, 
\begin{equation}
E \sim \hbar c \Lambda^4 
\end{equation}
Unfortunately, this is around $10^{120}$ times too large relative to the 
measured value.  
This discrepency is known as the cosmological constant problem.  
Fortunately, we will always be considering experiments on a more terrestrial scale,
 where it makes much more physical sense to only consider energy differences.
 In which case, these issues do not arise, and experiments and theory are in much closer accord.    


\subsubsection{Calculation for perfect conductors}

\comment{Give simple derivation of force? Cite Bordag/Dalvit?}
Here we will briefly reprise Casimir's original calculation~\cite{Casimir1948}.
  Consider a perfectly conducting box of length $L$.
  We will place another perfectly conducting plate, of area $L^2$, a distance $a$ from the $xy$ wall.   
The energy for the ground state of the quantized electromagnetic field is   
\begin{equation}
E = \sum_{\text{modes}}\frac{\hbar\omega_\alpha}{2}.
\end{equation}
Since the energies for each mode are positive, and the sum extends over all possible modes,
 this energy is badly divergent.
  However, the energy \emph{differences} between two configurations
  are finite.  This subtraction is a form of renormalization.  

We can write down the allowed modes for a rectangular cavity defined by, 
\begin{equation}
0 \le x \le L, \quad 0 \le y\le L, \quad 0 \le z \le a.
\end{equation}
From the boundary conditions, and transversality of the field,
 the mode functions for a perfectly conducting box 
are\footnote{Eq. (8.62) in `` Quantum and Atom Optics'', by Daniel A. Steck 
  \cite{SteckNotes}}:
\begin{align}
\mathbf{f}_{\mathbf{k},\zeta}(\mathbf{r}) =& \sqrt{\frac{8}{V}}\bigg[ 
\hat{x}(\hat{\epsilon}_{\mathbf{k},\zeta}\cdot\hat{x})\cos k_xx\sin k_yy\sin k_z z\nonumber\\
&+\hat{y}(\hat{\epsilon}_{\mathbf{k},\zeta}\cdot\hat{y})\sin k_xx\cos k_yy\sin k_z z\nonumber\\
&+\hat{z}(\hat{\epsilon}_{\mathbf{k},\zeta}\cdot\hat{z})\sin k_xx\sin k_yy\cos k_z z\bigg],
\end{align}
where $\hat{\epsilon}_{\mathbf{k},\zeta}$ is the polarization vector associated with
 the mode with wavevector $\vect{k}$ and the polarization index $\zeta$ takes on two values.
  The conducting boundary conditions mean that the fields must vanish on the surfaces, which requires that
\begin{equation}
k_x = \frac{pi n_x}{L},\quad  k_y = \frac{\pi n_y}{L}, \quad k_z=\frac{\pi n_z}{a},
\end{equation}
where $n_x,n_y,n_z$ are positive integers. 
 The requirement that $\nabla\cdot\vect{E}=0$, then requires that 
$\vect{k}\cdot\vect{E}=0$, which limits the polarizations to two values.
  We can write the frequency for a particular mode as
\begin{equation}
  \omega = c|\vect{k}| = c\sqrt{k_x^2+k_y^2+k_z^2}.
\end{equation}
Then if we take the limit $L\rightarrow\infty$, we can replace the sums over
 these modes with integrals over a continuum, 
\begin{equation}
E = \hbar c\bigg(\frac{L}{\pi}\bigg)^2{\sum_{n_z=0}^\infty}'\int_0^\infty dk_x
\int_0^\infty dk_y \left(\sqrt{k_x^2+k_y^2+\frac{n_z^2\pi^2}{a^2}}\right),
\end{equation}
where ${\sum_n}'$ includes a factor of $1/2$ for the $n=0$ term.
  (At $n_z=0$, there is only one polarization, while for larger $n$ there are 
two polarizations.  
When we come to consider non-zero temperature effects later, 
we will encounter similar summations.)

We can introduce polar coordinates to evaluate the integral over $k_x,k_y$.  
If we define $\kappa=\sqrt{k_x^2+k_y^2}$, and carry out the angular integral in 
the upper quarter plane, we find
\begin{equation}
E = \hbar c\bigg(\frac{L}{\pi}\bigg)^2\frac{\pi}{2}{\sum_{n_z=0}^\infty}'
\int_0^\infty d\kappa\,\kappa \sqrt{\kappa^2+\frac{n_z^2\pi^2}{a^2}}.
\end{equation}

On its own this energy is highly divergent.  
We will then subtract off the energy when the plates are moved farther apart.
In the limit of large $a$ we can also convert the sum over integers into 
another integral.  
The result of this subtraction is
\begin{equation}
E = \hbar c\bigg(\frac{L}{\pi}\bigg)^2\frac{\pi}{2}\int_0^\infty d\kappa\,\kappa
 \left({\sum_{n_z=0}^\infty}'\sqrt{\kappa^2+\frac{n_z^2\pi^2}{a^2}}
-\frac{\pi}{a}\int_0^\infty dk_z\sqrt{\kappa^2+k_z^2}\right).
\end{equation}
At this point Casimir introduces a function $f(k/k_m)$ to regularize the calculation.
This physically amounts to including some dispersion to model an 
ultra-violet (UV) cutoff.  
It is known \comment{Kramers-Kr\"onig} that the metal becomes transparent to 
photons at high enough energies, or short enough wavelengths,
 where $k_m$ indicates the cutoff.
  $f(k/k_m)$ is one for $k\ll 1$, and approaches zero for $k\gg k_m$.  

If we change variable again to $u=(a\kappa/\pi)^2$, then
\begin{equation}
E = \hbar c\bigg(\frac{L}{\pi}\bigg)^2\frac{\pi}{2}\frac{1}{2}
\left(\frac{\pi}{a}\right)^3\int_0^\infty du 
\left({\sum_{n_z=0}^\infty}'\sqrt{u+n_z^2}-\frac{\pi}{a}\int_0^\infty dn\sqrt{u+n^2}\right)
f\left(\frac{\pi\sqrt{n^2+u}}{a k_m}\right).
\end{equation}
%(\pi/a)^2 \frac{1}{2} from change of variable
If we now use the Euler-Maclaurin formula,
\begin{equation}
{\sum_{n=0}^\infty}' F(n)-\int_0^\infty dn\, F(n)
=-\frac{1}{12}F'(0)+\frac{1}{24\times 30}F'''(0),
\end{equation}
where primes denote differentiation w.r.t the argument, 
\begin{equation}
F(n)=\int_{n^2}^\infty dw\, \sqrt{w}f\left(\frac{\pi w}{ak_m}\right),
\end{equation}
where we have changed variables in the $u$ integral to $w = \sqrt{n^2+u}$
Then taking 
\begin{align}
F'(n) & = -\frac{2n^2}f\left(\frac{\pi n^2}{a k_m}\right)\\
F'(0) & = 0\\
F'''(0) & = -4
\end{align}
\comment{huh?  Introduced $f(k)$ to regularize Euler-Maclaurin}.

The end result of this, this that the renormalied vacuum energy per unit area is
\begin{equation}
\frac{E}{L^2} = -\frac{\hbar c\pi^2}{720 a^3}
\end{equation}

